\documentclass[10pt, oneside]{article} 
\usepackage{amsmath, amsthm, amssymb, calrsfs, wasysym, verbatim, bbm, color, graphics, geometry, esint, float}


\geometry{tmargin=.75in, bmargin=.75in, lmargin=.75in, rmargin = .75in}  

\newcommand{\bbR}{\mathbb{R}}
\newcommand{\bbC}{\mathbb{C}}
\newcommand{\bbZ}{\mathbb{Z}}
\newcommand{\bbN}{\mathbb{N}}
\newcommand{\bbQ}{\mathbb{Q}}
\newcommand{\Cdot}{\boldsymbol{\cdot}}
\newcommand{\scA}{\mathscr{A}}
\newcommand{\curl}{\text{curl}}

\theoremstyle{definition}
\newtheorem{exmp}{Example}[section]
\newtheorem{thm}{Theorem}
\newtheorem{defn}{Definition}
\newtheorem{prop}{Proposition}
\newtheorem{conv}{Convention}
\newtheorem{rem}{Remark}
\newtheorem{lem}{Lemma}
\newtheorem{cor}{Corollary}
\input{paolo-pset.tex}



\title{UChicago Measure and Integration Notes}
\author{Notes by Agustín Esteva, Lectures by Kenig, Books by Stein and Sakarchi, }
\date{Academic Year 2024-2025}

\begin{document}

\maketitle
\tableofcontents

\vspace{.25in}

\section{Lectures}

\subsection{Monday, Mar 24: What is a Measure}
\begin{defn}
    Let $X$ be a set and suppose $A\subseteq X,$ then we say that $\mu(A)$ is a measure of $A$ if:
    \begin{enumerate}
        \item $\mu(A) \geq 0$ (it is fine if $\mu (A) = \infty$) 
        \item $\mu(A) \in \bbR$ (but we don't allow $\mu(A) = \pm \infty$, only one)
        \item ($\sigma-$additivity ) $\mu(\bigsqcup_{i=1}^\infty A_i) = \sum_{i=1}^\infty\mu(A_i)$
    \end{enumerate}
\end{defn}
\begin{exmp}
    The counting measure is the trivial example:
    \[\mu(A) = |A|.\]
    The length measure is 
    \[\mu(A) = V(A).\]
    In $\bbR,$ we could have that 
    \[\mu([a,b]) = b-a.\]
\end{exmp}
\begin{exmp} [H]
    The $4-$corner Cantor set has measure $0.$
    \begin{figure}
        \centering
        \includegraphics[width=0.1\linewidth]{Images/4-corner Cantor.png}
        \caption{$4-$corner Cantor Set}
    \end{figure}
\end{exmp}
\begin{prop}
    In $\bbR,$ a set $C\subset \bbR$ is a Cantor set if it is compact, nonempty, has empty interior, and has no isolated points. On a topological space, $C$ is a Cantor set if if it is non-empty, compact, Hausdorff, has no isolated points, and has a countable base of clopen sets.In a nonempty, Hausdotff, topological space, then if a set is compact, metrizable, and has no isolated points, then it is the continuous image of a Cantor set. In every complete metric space without isolated points, there is a Cantor set.
\end{prop}

The following few proofs and remarks deal with the Banach-Tarski Paradox. 
\begin{rem}
    
\end{rem}


\end{document}