\documentclass[10pt, oneside]{article} 
\usepackage{amsmath, amsthm, amssymb, calrsfs, wasysym, verbatim, bbm, color, graphics, geometry, esint}


\geometry{tmargin=.75in, bmargin=.75in, lmargin=.75in, rmargin = .75in}  

\newcommand{\bbR}{\mathbb{R}}
\newcommand{\bbC}{\mathbb{C}}
\newcommand{\bbZ}{\mathbb{Z}}
\newcommand{\bbN}{\mathbb{N}}
\newcommand{\bbQ}{\mathbb{Q}}
\newcommand{\Cdot}{\boldsymbol{\cdot}}
\newcommand{\scA}{\mathscr{A}}
\newcommand{\curl}{\text{curl}}

\theoremstyle{definition}
\newtheorem{exmp}{Example}[section]
\newtheorem{thm}{Theorem}
\newtheorem{defn}{Definition}
\newtheorem{prop}{Proposition}
\newtheorem{conv}{Convention}
\newtheorem{rem}{Remark}
\newtheorem{lem}{Lemma}
\newtheorem{cor}{Corollary}
% Copyright 2021 Paolo Adajar (padajar.com, paoloadajar@mit.edu)
% 
% Permission is hereby granted, free of charge, to any person obtaining a copy of this software and associated documentation files (the "Software"), to deal in the Software without restriction, including without limitation the rights to use, copy, modify, merge, publish, distribute, sublicense, and/or sell copies of the Software, and to permit persons to whom the Software is furnished to do so, subject to the following conditions:
%
% The above copyright notice and this permission notice shall be included in all copies or substantial portions of the Software.
% 
% THE SOFTWARE IS PROVIDED "AS IS", WITHOUT WARRANTY OF ANY KIND, EXPRESS OR IMPLIED, INCLUDING BUT NOT LIMITED TO THE WARRANTIES OF MERCHANTABILITY, FITNESS FOR A PARTICULAR PURPOSE AND NONINFRINGEMENT. IN NO EVENT SHALL THE AUTHORS OR COPYRIGHT HOLDERS BE LIABLE FOR ANY CLAIM, DAMAGES OR OTHER LIABILITY, WHETHER IN AN ACTION OF CONTRACT, TORT OR OTHERWISE, ARISING FROM, OUT OF OR IN CONNECTION WITH THE SOFTWARE OR THE USE OR OTHER DEALINGS IN THE SOFTWARE.

\usepackage{fullpage}
\usepackage{enumitem}
\usepackage{amsfonts, amssymb, amsmath,amsthm}
\usepackage{mathtools}
\usepackage[pdftex, pdfauthor={\name}, pdftitle={\classnum~\assignment}]{hyperref}
\usepackage[dvipsnames]{xcolor}
\usepackage{bbm}
\usepackage{graphicx}
\usepackage{mathrsfs}
\usepackage{pdfpages}
\usepackage{tabularx}
\usepackage{pdflscape}
\usepackage{makecell}
\usepackage{booktabs}
\usepackage{natbib}
\usepackage{caption}
\usepackage{subcaption}
\usepackage{physics}
\usepackage[many]{tcolorbox}
\usepackage{version}
\usepackage{ifthen}
\usepackage{cancel}
\usepackage{listings}
\usepackage{courier}

\usepackage{tikz}
\usepackage{istgame}

\hypersetup{
	colorlinks=true,
	linkcolor=blue,
	filecolor=magenta,
	urlcolor=blue,
}

\setlength{\parindent}{0mm}
\setlength{\parskip}{2mm}

\setlist[enumerate]{label=({\alph*})}
\setlist[enumerate, 2]{label=({\roman*})}

\allowdisplaybreaks[1]

\newcommand{\psetheader}{
	\ifthenelse{\isundefined{\collaborators}}{
		\begin{center}
			{\setlength{\parindent}{0cm} \setlength{\parskip}{0mm}
				
				{\textbf{\classnum~\semester:~\assignment} \hfill \name}
				
				\subject \hfill \href{mailto:\email}{\tt \email}
				
				Instructor(s):~\instructors \hfill Due Date:~\duedate	
				
				\hrulefill}
		\end{center}
	}{
		\begin{center}
			{\setlength{\parindent}{0cm} \setlength{\parskip}{0mm}
				
				{\textbf{\classnum~\semester:~\assignment} \hfill \name\footnote{Collaborator(s): \collaborators}}
				
				\subject \hfill \href{mailto:\email}{\tt \email}
				
				Instructor(s):~\instructors \hfill Due Date:~\duedate	
				
				\hrulefill}
		\end{center}
	}
}

\renewcommand{\thepage}{\classnum~\assignment \hfill \arabic{page}}

\makeatletter
\def\points{\@ifnextchar[{\@with}{\@without}}
\def\@with[#1]#2{{\ifthenelse{\equal{#2}{1}}{{[1 point, #1]}}{{[#2 points, #1]}}}}
\def\@without#1{\ifthenelse{\equal{#1}{1}}{{[1 point]}}{{[#1 points]}}}
\makeatother

\newtheoremstyle{theorem-custom}%
{}{}%
{}{}%
{\itshape}{.}%
{ }%
{\thmname{#1}\thmnumber{ #2}\thmnote{ (#3)}}

\theoremstyle{theorem-custom}

\newtheorem{theorem}{Theorem}
\newtheorem{lemma}[theorem]{Lemma}
\newtheorem{example}[theorem]{Example}

\newenvironment{problem}[1]{\color{black} #1}{}

\newenvironment{solution}{%
	\leavevmode\begin{tcolorbox}[breakable, colback=green!5!white,colframe=green!75!black, enhanced jigsaw] \proof[\scshape Solution:] \setlength{\parskip}{2mm}%
	}{\renewcommand{\qedsymbol}{$\blacksquare$} \endproof \end{tcolorbox}}

\newenvironment{reflection}{\begin{tcolorbox}[breakable, colback=black!8!white,colframe=black!60!white, enhanced jigsaw, parbox = false]\textsc{Reflections:}}{\end{tcolorbox}}

\newcommand{\qedh}{\renewcommand{\qedsymbol}{$\blacksquare$}\qedhere}

\definecolor{mygreen}{rgb}{0,0.6,0}
\definecolor{mygray}{rgb}{0.5,0.5,0.5}
\definecolor{mymauve}{rgb}{0.58,0,0.82}

% from https://github.com/satejsoman/stata-lstlisting
% language definition
\lstdefinelanguage{Stata}{
	% System commands
	morekeywords=[1]{regress, reg, summarize, sum, display, di, generate, gen, bysort, use, import, delimited, predict, quietly, probit, margins, test},
	% Reserved words
	morekeywords=[2]{aggregate, array, boolean, break, byte, case, catch, class, colvector, complex, const, continue, default, delegate, delete, do, double, else, eltypedef, end, enum, explicit, export, external, float, for, friend, function, global, goto, if, inline, int, local, long, mata, matrix, namespace, new, numeric, NULL, operator, orgtypedef, pointer, polymorphic, pragma, private, protected, public, quad, real, return, rowvector, scalar, short, signed, static, strL, string, struct, super, switch, template, this, throw, transmorphic, try, typedef, typename, union, unsigned, using, vector, version, virtual, void, volatile, while,},
	% Keywords
	morekeywords=[3]{forvalues, foreach, set},
	% Date and time functions
	morekeywords=[4]{bofd, Cdhms, Chms, Clock, clock, Cmdyhms, Cofc, cofC, Cofd, cofd, daily, date, day, dhms, dofb, dofC, dofc, dofh, dofm, dofq, dofw, dofy, dow, doy, halfyear, halfyearly, hh, hhC, hms, hofd, hours, mdy, mdyhms, minutes, mm, mmC, mofd, month, monthly, msofhours, msofminutes, msofseconds, qofd, quarter, quarterly, seconds, ss, ssC, tC, tc, td, th, tm, tq, tw, week, weekly, wofd, year, yearly, yh, ym, yofd, yq, yw,},
	% Mathematical functions
	morekeywords=[5]{abs, ceil, cloglog, comb, digamma, exp, expm1, floor, int, invcloglog, invlogit, ln, ln1m, ln, ln1p, ln, lnfactorial, lngamma, log, log10, log1m, log1p, logit, max, min, mod, reldif, round, sign, sqrt, sum, trigamma, trunc,},
	% Matrix functions
	morekeywords=[6]{cholesky, coleqnumb, colnfreeparms, colnumb, colsof, corr, det, diag, diag0cnt, el, get, hadamard, I, inv, invsym, issymmetric, J, matmissing, matuniform, mreldif, nullmat, roweqnumb, rownfreeparms, rownumb, rowsof, sweep, trace, vec, vecdiag, },
	% Programming functions
	morekeywords=[7]{autocode, byteorder, c, _caller, chop, abs, clip, cond, e, fileexists, fileread, filereaderror, filewrite, float, fmtwidth, has_eprop, inlist, inrange, irecode, matrix, maxbyte, maxdouble, maxfloat, maxint, maxlong, mi, minbyte, mindouble, minfloat, minint, minlong, missing, r, recode, replay, return, s, scalar, smallestdouble,},
	% Random-number functions
	morekeywords=[8]{rbeta, rbinomial, rcauchy, rchi2, rexponential, rgamma, rhypergeometric, rigaussian, rlaplace, rlogistic, rnbinomial, rnormal, rpoisson, rt, runiform, runiformint, rweibull, rweibullph,},
	% Selecting time-span functions
	morekeywords=[9]{tin, twithin,},
	% Statistical functions
	morekeywords=[10]{betaden, binomial, binomialp, binomialtail, binormal, cauchy, cauchyden, cauchytail, chi2, chi2den, chi2tail, dgammapda, dgammapdada, dgammapdadx, dgammapdx, dgammapdxdx, dunnettprob, exponential, exponentialden, exponentialtail, F, Fden, Ftail, gammaden, gammap, gammaptail, hypergeometric, hypergeometricp, ibeta, ibetatail, igaussian, igaussianden, igaussiantail, invbinomial, invbinomialtail, invcauchy, invcauchytail, invchi2, invchi2tail, invdunnettprob, invexponential, invexponentialtail, invF, invFtail, invgammap, invgammaptail, invibeta, invibetatail, invigaussian, invigaussiantail, invlaplace, invlaplacetail, invlogistic, invlogistictail, invnbinomial, invnbinomialtail, invnchi2, invnF, invnFtail, invnibeta, invnormal, invnt, invnttail, invpoisson, invpoissontail, invt, invttail, invtukeyprob, invweibull, invweibullph, invweibullphtail, invweibulltail, laplace, laplaceden, laplacetail, lncauchyden, lnigammaden, lnigaussianden, lniwishartden, lnlaplaceden, lnmvnormalden, lnnormal, lnnormalden, lnwishartden, logistic, logisticden, logistictail, nbetaden, nbinomial, nbinomialp, nbinomialtail, nchi2, nchi2den, nchi2tail, nF, nFden, nFtail, nibeta, normal, normalden, npnchi2, npnF, npnt, nt, ntden, nttail, poisson, poissonp, poissontail, t, tden, ttail, tukeyprob, weibull, weibullden, weibullph, weibullphden, weibullphtail, weibulltail,},
	% String functions 
	morekeywords=[11]{abbrev, char, collatorlocale, collatorversion, indexnot, plural, plural, real, regexm, regexr, regexs, soundex, soundex_nara, strcat, strdup, string, strofreal, string, strofreal, stritrim, strlen, strlower, strltrim, strmatch, strofreal, strofreal, strpos, strproper, strreverse, strrpos, strrtrim, strtoname, strtrim, strupper, subinstr, subinword, substr, tobytes, uchar, udstrlen, udsubstr, uisdigit, uisletter, ustrcompare, ustrcompareex, ustrfix, ustrfrom, ustrinvalidcnt, ustrleft, ustrlen, ustrlower, ustrltrim, ustrnormalize, ustrpos, ustrregexm, ustrregexra, ustrregexrf, ustrregexs, ustrreverse, ustrright, ustrrpos, ustrrtrim, ustrsortkey, ustrsortkeyex, ustrtitle, ustrto, ustrtohex, ustrtoname, ustrtrim, ustrunescape, ustrupper, ustrword, ustrwordcount, usubinstr, usubstr, word, wordbreaklocale, worcount,},
	% Trig functions
	morekeywords=[12]{acos, acosh, asin, asinh, atan, atanh, cos, cosh, sin, sinh, tan, tanh,},
	morecomment=[l]{//},
	% morecomment=[l]{*},  // `*` maybe used as multiply operator. So use `//` as line comment.
	morecomment=[s]{/*}{*/},
	% The following is used by macros, like `lags'.
	morestring=[b]{`}{'},
	% morestring=[d]{'},
	morestring=[b]",
	morestring=[d]",
	% morestring=[d]{\\`},
	% morestring=[b]{'},
	sensitive=true,
}

\lstset{ 
	backgroundcolor=\color{white},   % choose the background color; you must add \usepackage{color} or \usepackage{xcolor}; should come as last argument
	basicstyle=\footnotesize\ttfamily,        % the size of the fonts that are used for the code
	breakatwhitespace=false,         % sets if automatic breaks should only happen at whitespace
	breaklines=true,                 % sets automatic line breaking
	captionpos=b,                    % sets the caption-position to bottom
	commentstyle=\color{mygreen},    % comment style
	deletekeywords={...},            % if you want to delete keywords from the given language
	escapeinside={\%*}{*)},          % if you want to add LaTeX within your code
	extendedchars=true,              % lets you use non-ASCII characters; for 8-bits encodings only, does not work with UTF-8
	firstnumber=0,                % start line enumeration with line 1000
	frame=single,	                   % adds a frame around the code
	keepspaces=true,                 % keeps spaces in text, useful for keeping indentation of code (possibly needs columns=flexible)
	keywordstyle=\color{blue},       % keyword style
	language=Octave,                 % the language of the code
	morekeywords={*,...},            % if you want to add more keywords to the set
	numbers=left,                    % where to put the line-numbers; possible values are (none, left, right)
	numbersep=5pt,                   % how far the line-numbers are from the code
	numberstyle=\tiny\color{mygray}, % the style that is used for the line-numbers
	rulecolor=\color{black},         % if not set, the frame-color may be changed on line-breaks within not-black text (e.g. comments (green here))
	showspaces=false,                % show spaces everywhere adding particular underscores; it overrides 'showstringspaces'
	showstringspaces=false,          % underline spaces within strings only
	showtabs=false,                  % show tabs within strings adding particular underscores
	stepnumber=2,                    % the step between two line-numbers. If it's 1, each line will be numbered
	stringstyle=\color{mymauve},     % string literal style
	tabsize=2,	                   % sets default tabsize to 2 spaces
%	title=\lstname,                   % show the filename of files included with \lstinputlisting; also try caption instead of title
	xleftmargin=0.25cm
}



\title{UChicago Measure and Integration Notes}
\author{Notes by Agustín Esteva, Lectures by Kenig, Books by Stein and Sakarchi, }
\date{Academic Year 2024-2025}

\begin{document}

\maketitle
\tableofcontents

\vspace{.25in}

\section{Lectures}

\subsection{Tuesday, Jan 21: Measurable Sets}
\begin{defn}
    We say that $E \subset \bbR^d$ is \textit{measurable} if, given $\epsilon>0,$ there exists an open set $O$ such that $E \subset O$ and 
    \[m_\ast(O - E)< \epsilon\]
\end{defn}
This is almost equivalent to saying that 
\[m_\ast(O) \leq m_\ast(E) + \epsilon\]
\begin{prop}
    The following are properties of measurable sets:
    \begin{enumerate}
        \item If $O$ is open, then $O$ is measurable (they cover themselves)
        \item If $m_\ast(E) = 0,$ then $E$ is measurable.
        \item A countable union of measurable sets is measurable.
        \item Closed sets are measurable
        \item Complements are measurable
    \end{enumerate}
\end{prop}
\begin{prop}
    A countable intersection of measurable sets is measurable.
\end{prop}
\begin{proof}
    Let 
    \[E = \bigcap E_j,\] where $E_j$ is measurable for all $j.$ Then we get that by DeMorgan's law:
    \[\bigcap E_j = \left(\bigcup E_j\right)^c,\] so we are done by the previous proposition.
\end{proof}
\begin{thm}
    Suppose $E_1, E_2, \dots$ are measurable and mutually disjoint, then 
    \[m\left(\bigcup E_j\right) = \sum_j m(E_j).\]
\end{thm}
\begin{proof}
    We claim that if $E$ is measurable, and $\epsilon>0,$ then there exists some closed $F \subset E$ such that $m(E - F) < \epsilon.$ To see this, consider that $E^c$ is measurable, and thus there exists an open set $O$ such that $m_\ast(O - E^c) < \epsilon.$ We let $F = O^c,$ and then $F \subset E,$ and $F$ is closed. Then we have that $O - E^c = O\cap E = F\cap E.$ We know that $m(O\cap E) < \epsilon.$ Note that we have that 
    \[E - F = E \cap F^c = E \cap O,\] and thus $m(E - F) < \epsilon.$

    Assume $E_j$ are bounded for all $j.$ By the previous claim, there exists a closed $F_j \subset E_j$ such that 
    \[m(E_j - F_j) < \frac{\epsilon}{2^j}\] for all $j.$ Note that by boundedness of $E_j$ and by closedness of $F_j,$ the $F_j$ are compact and disjoint. Fix $N\in \bbN,$ then we have that 
    \[m\left(\bigcup_1^N F_j\right) = \sum_1^Nm(F_j)\] Let $E = \bigcup E_j,$ then we have that $\bigcup F_j \subset E_j.$ Thus, we have that 
    \[m(E)\geq m(\bigcup_1^N E_j) = \sum_1^N m(E_j).\] Thus, we get that 
    \[m(E) \geq \sum_{j=1}^Nm(F_j) \geq \sum_{j=1}^N m(E_j) - \frac{\epsilon}{2^j}\geq \sum_{j=1}^N m(E_j) - \epsilon.\] Letting $N\to \infty,$ we get that 
    \[m(E)\geq \sum_{j=1}^\infty m(E_j).\] By subadditivity, we also have that 
    \[m(E)\leq \sum_{j=1}^\infty m(E_j).\]

    For the general case, find cubes $Q_k \subset Q_{k+1}$ such that $\bigcup Q_k = \bbR^d.$ Then let $E_{j,k} = E_j \cap (Q_k - Q_{k-1}).$ Then we have that $E = \bigcup_{k,j} E_{k,j},$ where the $E_{k,j}$ are disjoint and bounded. Then we have by the work done above that 
    \[m(E_j) = \sum_{j}\sum_k m(E_{k,j}) = \sum_j m(E_j)\]
\end{proof}
\begin{cor}
    Suppose $\{E_j\}$ is a countable collection of measurable subsets of $\bbR^d.$ Assume further that $E_j\subset E_{j+1},$ and that $E  = \bigcup_j E_j.$ Then we have that 
    \[m(E) = \lim_{n\to \infty} m(E_n).\] If, on the other hand, the $E_j$ decrease to $E,$ that is, $E = \bigcap_j E_j,$ then if $m(E_k)< \infty,$ then 
    \[m(E) = \lim_{n\to \infty} m(E_n)\]
\end{cor}
\begin{proof}
    For the first result, let $G_1 = E_1, $ $G_2 = E_2 - E_1,$ and so on until 
    \[G_n = E_n - E_{n-1}.\] Obviously, the $G_k$ are measurable and disjoint and $E = \bigcup_k G_k.$ Then 
    \[m(E) = \sum_{k=1}^\infty m(G_k) = \lim_{n\to \infty} \sum_{k=1}^n m(G_k) = m\left(\lim_{n\to \infty}\bigcup_{k=1}^N G_k\right) = m\left(\lim_{n\to \infty} E_n\right)\]

    For the second result, we assume without loss of generality that $E_1 <\infty,$ and we set $G_1 = E_1 - E_2,$ $G_2 = E_2 - E_3.$ Again, the $G_k$ are mutually disjoint and measurable and 
    \[E_1 = E \cup \bigcup_{k=1}^\infty G_k.\]
    Thus, we get that (letting $N\to \infty$)
    \[m(E_1) = m(E) + \sum_{k=1}^N m(G_k) = m(E) + \sum_{k=1}^\infty m(G_{k}) - m(G_{k+1}) = m(E) + m(E_1) - m(E_N).\]
\end{proof}
\begin{thm}
    Suppose $E \subset \bbR^d,$ $E$ is measurable. Then for all $\epsilon >0:$
    \begin{enumerate}
        \item There exists an open $O$ with $E \subset O$ such that $m(O - E)< \epsilon.$
        \item There exists a closed $F$ with $F\subset E$ such that $m(E - F)< \epsilon.$
        \item If $m(E) < \infty,$ then there exists a compact set $K$ such that 
        $K \subset F$ and $m(E - K)< \epsilon.$
        \item If $m(E) < \infty,$ then there exists $F = \bigcup Q_j$ such that 
        $m(E \triangle F) < \epsilon.$ (note that the symmetric difference refers to the points belonging to only one of the sets).
    \end{enumerate}
\end{thm}
\begin{proof}
    (i) and (ii) have already been proved. For (iii), pick $F$ closed such that $F\subset E$ and $m(E - F) < \frac{\epsilon}{2}.$ Let $K_m = F \cap \overline{B_m(0)}.$ Evidently, each $K_m$ is compact and we have that 
    \[E - K_m \downarrow E - F.\] We know by the previous corollary that 
    \[m(E - F) = \lim_{m\to \infty} (E - K_m),\] and so for $m$ large, $m(E - K_m)< \epsilon.$

    For (iv), we pick closed $\{Q_j\}$ such that $E \subset \bigcup Q_j$ and $\sum Q_j \leq m(E) + \frac{\epsilon}{2}.$ Thus, the series is convergent, and there exists some large $N$ such that $\sum_{N+1}^\infty |Q_j| \leq \frac{\epsilon}{2}.$ Now we let $F = \bigcup_{j=1}^N Q_j.$ We thus get that 
    \begin{align*}
        m(E \triangle F) = &m(E - F) +  m(F - E)\\
        &= \sum_{N+1}^\infty m(Q_j) + (\sum m(Q_j) - m(E))\\
        &< \epsilon.
    \end{align*}
\end{proof}

\newpage
\subsection{Thursday, Jan 23: Constructing a Non-Measurable Set and Measurable Functions}
\begin{lemma}
    Suppose $a\in \bbR,$ then $m^\ast (E + a) = m^\ast(E).$
\end{lemma}
\begin{proof}
    Note that $E + a  = \{x + a \; ; \; x\in E\}.$ Pick $E \subset^\infty \bigcup Q_i$ such that $\sum |Q_i| \leq m^\ast(E) + \epsilon,$ then 
    \[E \subseteq \bigcup_{i=1}^\infty (Q_i + a) \implies m^\ast(E + a)  \geq \sum |Q_i + a| + \epsilon.\]
    For the reverse direction, consider that 
    \[m^\ast(E + a) \leq m^\ast((E + a) - a)  = \m^\ast (E).\]
\end{proof}
\begin{thm}
    (Axiom of Choice) If $\mathcal{E} = \{S_\alpha \; : \;  \alpha \in \mathcal{A}, S_\alpha \neq \emptyset \; \forall \alpha \in \mathcal{A}\},$ then there exists a function $f: \mathcal{E} \to \bigcup S_\alpha$ such that $f(S_\alpha) \in S_\alpha.$
\end{thm}
Define $\sim$ from $[0,1]$ such that $x\sim y$ if $x-y \in \bbQ.$ Thus, 
\[[0,1] = \bigcup_{x\in [0,1]} [x].\] As an example, \[[0] = \bbQ \cap [0,1], \qquad [\frac{1}{\sqrt{2}}] = \{\frac{1}{\sqrt{2}}, \frac{1}{\sqrt{2}} + \frac{1}{4}, \dots\}\] Using the Axiom of Choice, pick $\alpha \in [x]$ such that if $[x] \neq [y],$ then $\alpha_x \neq \alpha_y.$ Define 
\[N := \{\alpha_{[x]}\}.\]
\begin{thm}
    $N$ is not measurable.
\end{thm}
\begin{proof}
    Suppose it is. Define $\{r_k\} = \bbQ \cap [-1, 1],$ and define $N_k = N + r_k.$ We claim that $N_k \cap N_j = \emptyset$ when $k\neq j.$ To see this, let $x$ be in the intersection, then 
    \[x = \alpha_1 + r_k = \alpha_2 + r_j \implies \alpha_1 - \alpha_2 = r_j - r_k \in \bbQ \implies \alpha_1 = \alpha_2 \implies r_j = r_k,\] contradiction the fact that $k\neq j.$ We further claim that 
    \[[0,1]\subseteq \bigcup N_k \subseteq [-1, 2].\] The second inclusion is by construction. Let $x\in [0,1],$ then there exists some $\alpha \in [x],$ and thus 
    \[\alpha \in N \implies x-a \in \bbQ \implies x = \alpha + r_k \implies x\in N_k\] By the first claim, we have that by translation invariance, that
    \[m(\bigcup N_k) = \sum m(N_k) = \sum m(N) = \{0, \infty\}.\] But then 
    \[1 = m([0,1])\leq m(\bigcup N_k) \leq m([-1,2]) = 3,\] which is a contradiction.
\end{proof}
\begin{defn}
    A collection of sets $S\subset P(X)$ is a \textbf{$\sigma-$algebra} if:
    \begin{enumerate}
        \item $X \in S;$
        \item If $\{E_i\} \in S,$ then $\bigcup E_i \in S.$
        \item If $E \in S,$ then $E^c \in S.$
    \end{enumerate}
\end{defn}
\begin{exmp}
Examples of $\sigma-$algebras:
    \begin{enumerate}
        \item $S = \{\emptyset, X\}$
        \item Set of measurable sets.
        \item $2^\bbR$
    \end{enumerate}
\end{exmp}
\begin{lemma}
    Suppose $\mathcal{F}_\alpha,$ $\alpha \in \mathcal{A}$  is a collection of $\sigma-$algebras on $X,$ then 
    \[\mathcal{F} = \bigcap_\alpha \mathcal{F}_\alpha\] is a $\sigma-$algebra.
\end{lemma}
\begin{proof}
    Since $X \in \mathcal{F}_\alpha$ for all $\alpha,$ then $X\in \mathcal{F}.$ Suppose $\{E_i\}\in \mathcal{F},$ then for all $\alpha,$ $\{E_i\} \in \mathcal{F}_\alpha,$ and thus
    \[\bigcup E_i \in \mathcal{F_\alpha}\] for all $\alpha,$ and because this holds for all $\alpha,$ then we are done. The proof for the third property is the same as the previous one.
\end{proof}
\begin{defn}
    If $\mathcal{E} \subseteq P(X),$ then we define
    \[\sigma(\mathcal{E}) := \bigcap_{\mathcal{E}\subseteq \mathcal{F}_\alpha} \mathcal{F}_\alpha\]
\end{defn}
\begin{defn}
    The \textbf{Borel $\sigma-$algebra}, $\mathcal{B}(\bbR^n),$ is defined to be $\sigma(O),$ where $O \subseteq \bbR^n$ are open.
\end{defn}
\begin{prop}
$\mathcal{B}$ satisfies the following properties:
    \begin{enumerate}
        \item All open sets are in the $\mathcal{B}$
        \item All closed sets are in $\mathcal{B}$
        \item If $E \in \mathcal{B},$ then $E$ is measurable.
    \end{enumerate}
\end{prop}

\begin{defn}
    A set is $G_\delta$ if $E = \bigcap O_i,$ where $O_i$ is open.

    A set is $F_\sigma$ if $E = \bigcup F_i,$ where $F_i$ is closed.
\end{defn}
\begin{thm}
    $A\subseteq \bbR^n$ is measurable if and only there exists a $G \in G_\delta$ set such that $A\subseteq G$ and $m^\ast(G - A) = 0.$ The symmetric claim holds for $F_\sigma$ sets as well.
\end{thm}
\begin{proof}
    $(\implies)$ If $A$ is measurable, then there exists $O_n \supset A$ open such that $m^\ast(O_n - A)< \frac{1}{n}.$ We let $G = \bigcap O_n.$ 

    $(\impliedby)$ Since null sets are measurable, then since $A = G - (G - A),$ so $A$ is measurable.
\end{proof}

\begin{rem}
    Let $E\subseq \bbR^d,$ then 
    \[\chi_E(x) = \begin{cases}
        1, \quad x\in E\\
        0, \quad x\notin E
    \end{cases}.\]
\end{rem}
It would make sense for $\chi_E(x)$ to be measurable when $E$ is measurable!
\begin{defn}
    A \textbf{simple function} is defined to be 
    \[f(x) = \sum_{k=1}^n a_k \chi_{E_k}(x), \qquad m(E) < \infty.\]
\end{defn}
\begin{defn}
    A \textbf{step function} is defined to be 
    \[f = \sum_{k=1}^n a_k \chi_{R_k}, \qquad R_k \; \text{rect}\]
\end{defn}
\begin{defn}
Let $f: \bbR^d \to \bbR \cup \{\pm \infty\}.$ We say $f$ is \textbf{finite value} if $f(x) \neq \{\pm \infty\}$ for all $x\in \bbR^d$    
\end{defn}

\begin{defn}
    We say $f$ is \textbf{measurable} if for all $a\in \bbR,$ $f^{-1}([-\infty, a))$ is measurable. In other words, 
    \[m(\{x \; | \; f(x)< a\})< \infty.\]
\end{defn}

\begin{thm}
    $f$ is measurable if and only if for all $a\in \bbR,$ $\{x \; | \; f(x)\leq a\}$ is measurable.
\end{thm}
\begin{proof}
    We can write 
    \[\{f \leq a\} = \bigcap \{ f < a + \frac{1}{k}\},\] and since the countable intersection of measurable sets is measurable, we are done. For the other direction, consider that
    \[\{f < a\} = \bigcup \{f \leq a - \frac{1}{k}\}.\]
\end{proof}

\newpage
\subsection{Tuesday, Jan 28: Measurable Functions}
\begin{prop}
    If $f$ is finite valued, then $f$ is measurable if and only if $f^{-1}(O)$ is measurable for all $O$ open if and only if $f^{-1}(F)$ is measurable for all $F$ closed. 
\end{prop}
This It follows immediately that
\begin{prop}
    If $f$ is continuous on $\bbR^d,$ then $f$ is measurable. Moreover, if $f$ is measurable, finite valued, and $\Phi$ is continuous on $\bbR,$ then $\Phi \circ f$ is measurable.
\end{prop}
\begin{proof}
    We prove the second statement. Since $\Phi$ is continuous, we have that $O_a =  \Phi^{-1}((-\infty, a))$ is open. Since $f$ is measurable, we have that $f^{-1}(O_a)$ is measurable, and thus $(\Phi\circ f) ((-\infty, a )) = f^{-1}(\Phi^{-1}((-\infty, a)))$ 
\end{proof}

\begin{prop}
    Suppose $\{f_n\}$ is a sequence of measurable functions. Then $\sup_n f_n(x)$ and $\inf_n f_n(x),$ $\limsup_n f_n(x)$ and $\liminf_n f_n(x)$ are all measurable.
\end{prop}
\begin{proof}
    Consider that \[ \{\sup_n f_n(x) > a\} = \bigcup_n \{f_n(x) > a\}\] We also have that $-\inf_n(-f_n(x)) = \sup_n f_n(x)$ and that 
    \[\limsup_n f_n(x) = \inf_n(\sup_k\geq n f_k(x))\]
\end{proof}
As a quick consequence, we have that if the $f_n$ are measurable, then $\lim_{n\to \infty} f_n(x) = f(x)$ is measurable.
\begin{prop}
    \begin{enumerate}
        \item Suppose that $f$ is measurable, then $f^k$ is measurable.
        \item If $f$ and $g$ are measurable, then $f + g$ and $fg$ are measurable (given that $f$ and $g$ are finite valued).
    \end{enumerate}
\end{prop}
\begin{proof}
    The first is a simple consequence of Proposition 5. We claim that 
    \[\{f + g > a\} = \bigcup_{r\in \bbR}(f > a-r) \cap \{g > r\}.,\] which proves the second. Consider now that 
    \[fg = \frac{1}{4}\left[(f + g)^2 - (f - g)^2\right]\]
\end{proof}

\begin{defn}
    We say that $f$ and $g$ are equal \textbf{almost everywhere}, or equal \textbf{a.e.} if $m\{f(x) \neq g(x)\} = 0.$
\end{defn}
\begin{prop}
    If $f$ is measurable and $f = g$ a.e, then $g$ is measurable.
\end{prop}
\begin{rem}
    If $f_n$ are measurable and $\lim_{n\to \infty}f_n(x) = g(x)$ a.e, then $g$ is measurable. 
\end{rem}

\begin{thm}
    Suppose $f\geq 0$ and measurable on $\bbR^d.$ Then there exists an increasing sequence of non-negative simple functions $(\varphi_k)$ such that 
    \[\varphi_k(x) \leq \varphi_{k+1}(x), \qquad \lim_{k\to \infty} \varphi_k(x) = f(x). \qquad \forall x\]
\end{thm}
\begin{proof}
    For each $k,$ subdivide the values of $f$ which fall in $[0,k].$ Partition $[0,k]$ into subintervals $[\frac{j-1}{2^k}, \frac{j}{2^k}].$ Define 
    \[\tilde{\varphi_k}(x) = \begin{cases}
        \frac{j-1}{2^k}, \qquad f(x)\in [\frac{j-1}{2^k}, \frac{j}{2^k}]\\
        k, \qquad f(x) \geq k
    \end{cases},\]
Obviously, $\tilde{\phi_k}$ is measurable and $\tilde{\varphi_k} \leq \tilde{\vaprhi_{k+1}}.$ 

When $f(x) = \infty,$ we have that $\tilde{\varphi_k}(x) = k$ for any $k,$ so then $\varphi_k \uparrow f.$ On the other hand, suppose $f(x) < \infty,$ then there exists some $k_0$ such that for all $k>k_0,$ $f(x)< k,$ and 
\[0 \leq f(x) - \tilde{\varphi_k} \leq \frac{1}{2^k} \implies \tilde{\varphi_k} \to f.\] Define
\[\varphi_k(x) = \tilde{\varphi_k}(x)\cdot \chi_{B_k(0)}.\] Because $\chi_{B_k(0)}\leq \chi_{B_{k+1}(0)},$ we have that $\varphi_k(x) \leq \varphi_{k+1}(x).$

Let $x$ be fixed, and choose $k$ such that $|x| < k,$ then $\varphi_k(x) = \tilde{\varphi_k(x)}.$
\end{proof}

\newpage
\subsection{Thursday, Jan 30: Constructing the Lebesgue Integral}
\begin{rem}
    To handle negative functions, then consider the general case when $f$ is measurable. Then let 
    \[f_+ = \max(f(x), 0), \qquad f_- = \max(- f(x), 0).\] Both are nonnegative, and we have that 
    \[f(x) = f_+(x) - f_-(x).\]
    Thus, when $f\leq 0,$ we have that 
    \[f_+(x)  =0, \qquad f_-(x) = -f(x).\] 
\end{rem}

\begin{thm}
    Suppose $f$ is measurable. Then there exists a sequence $(\varphi_k) \to f$ such that $\varphi_k$ are simple functions with $|\varphi_k| \leq |\varphi_{k+1}|.$ In particular, we have that 
    \[|\varphi_k(x)| \leq |f(x)|, \qquad \forall k, x\]
\end{thm}
\begin{proof}
    From Theorem 7, we can pick a sequence $(\varphi^{(i)}_k),$ where $i = \{1,2\}$ such that 
    \[0 \leq \varphi_{k+1}^{(1)}(x)\leq \varphi_{k+1}^{(1)}, \varphi_k^{(1)}\uparrow f_+(x),\] and similarly for $\varphi_k^{(2)}\uparrow f(x).$ 
    \[\varphi_k := \varphi_k^{1} - \varphi_k^{2} \to f(x).\] It remains to be seen that 
    \[|\varphi_k(x)| = \varphi_k^{1} + \varphi_k^{2}.\]

    If $f_+ = 0,$ then $\varphi_k^{(1)} = 0,$ and so 
    \[\varphi_k(x) = -\varphi_k^{(2)},\] and so our claim is proven.

    If $f_- = 0,$ then $\varphi_k^{(2)} = 0,$ and so 
    \[\varphi_k(x) = \varphi_k^{(1)},\] and so our claim is proven.

    When $f_+(x) >0,$ then $f_-(x) = 0,$ and so $\varphi-k^{(2)} = 0,$ and $|f(x)| = f(x)$ and thus our claim is proven. Similarly for when $f_-(x) < 0.$
\end{proof}

\begin{thm}
    Suppose $f$ is measurable. Then there exists a sequence of step functions $(\psi_k(x))$ such that 
    \[\psi_k(x) \to f(x) \qquad a.e.\]
\end{thm}
\begin{proof}
    Suppose that 
    \[f(x) = \chi_E(x),\] where $E$ is a measurable set of finite measure, then aply the previous theorem. Let $\epsilon > 0.$ There exist cubes $Q_j$ such that $E \subset \bigcup_{j=1}^N Q_j$ and 
    \[m(\Delta \bigcup_{j=1}^N Q_j)< \epsilon.\] Thus, there exist almost disjoint rectangles $\tilde{R_1}, \dots, \tilde{R_n}$ such that 
    \[\bigcup_{j=1}^N Q_j = \bigcup_{j=1}^M \tilde{R_j},\] then find $R_j \subset \tilde{R_j}.$ We have that $R_j$ are disjoint and 
    \[m(E \Delta \bigcup_{j=1}^M R_j) \leq 2\epsilon.\] Thus, except for a set of measure $\leq 2\epsilon,$ we have that 
    \[f(x) = \sum_{j=1}^M \chi_{R_j}(x)\]

    For all $k\geq 1,$ there exists a step funcntion $\psi_k(x)$ such that if 
    \[E_k = \{x \: : \: f(x) \neq \psi_k(x)\},\] then $m(E_k) \leq \frac{1}{2^k}.$ Thus, 
    \[F_K = \bigcup_{j=K+1}^\infty E_j, \qquad F = \bigcap_{k=1}^\infty F_K.\] By the Borel-Cantelli lemma, we have that $m(F) = 0.$ To see that $\psi_k \to f$ for $x\in F^c$ consider that 
    \[F^c = \bigcup_{j=1}^\infty \bigcap_{k\geq j}F^c_k.\] Thus, if $x\in F^c,$ then there exists a $j$ such that than $x\in \bigcap_{k\geq j}F_k^c,$ and then $\psi_k(x) = f(x).$
\end{proof}

\newpage
\subsection{Tuesday, Feb 4: Littlewood's Three Principles}
\begin{enumerate}
    \item Every measurable set is a finite union of intervals. (For intuition, look at Theorem 2.d)
    \item Every measurable function is continuous almost everywhere. (Lusin)
    \item Every pointwise convergence sequence of measurable functions is uniformly convergence. (Egorov)
\end{enumerate}
We begin with (c).
\begin{thm}
    (Egorov) Let $\{f_k(x)\}$ be a sequence on measurable functions on a measurable set $E$ such that $m(E) < \infty.$ Assume that $f_n(x) \to f$ almost everywhere. Then for all $\epsilon>0,$ there exists $A_\epsilon \subset E$ such that $A_\epsilon$ is closed and $m(E \setminus A_\epsilon) < \epsilon.$ Moreover, $f_n(x) \to f(x)$ uniformly for $x\in A_\epsilon.$
\end{thm}
\begin{proof}
    Without loss of generality, we suppose 
    \[f_n(x) \to f(x) \qquad \forall \; x\in E.\] For each $k,n \in \bbN,$ define 
    \[E_k^n = \{x \in E \; ; \; |f_j(x) - f(x)| < \frac{1}{n}, \quad \forall j\geq k\}.\]
    Obviously, we have that 
    \[E_k^n \subset E_{k+1}^n, \qquad E_{k}^n \uparrow E \implies E = \bigcup_{k=1}^\infty E_{k}^n = \lim_{n\to \infty}E_k^n,\] and so 
    \[\lim_{t\to \infty}m(E_k^n) = m(E).\] There exists some $k_m$ such that $m(E) - m(E^k^n_{k_m}) = m(E \setminus E^n_{k_m}) < \frac{1}{2^n}.$ Thus, 
    \[|f_j(x) - f(x)| < \frac{1}{2^n} \qquad j\geq k_m, \; x\in E^n_{n_m}.\] Let $\epsilon>0.$ Choose $N$ such that 
    \[\sum_{N}^\infty \frac{1}{2^n} < \frac{\epsilon}{2},\] and so if we define 
    \[\tilde{A}_\epsilon = \bigcap_{m\geq N} E^n_{k_m}.\] Thus, 
    \[m(E \setminus \tilde{A}_\epsilon) =m(E \cap \bigcup_{m\geq N} (E^n_{k_m})^c) = m(\bigcup E\cap (E^n_{k_m})^c) = m(\bigcup E\setminus E^n_{k_m}) \leq \sum_{N}^\infty \frac{1}{2^n} < \frac{\epsilon}{2}\]

    Let $\delta>0$ and let $\frac{1}{n}< \delta,$ then for any $x\in \tilde{A}_\epsilon,$ we have that $x\in E^n_{k_m},$ and thus if $j\geq n,$ we have that 
    \[|f_j(x) - f(x)|< \frac{1}{n}< \delta,\] and so $f$ is uniformly continuous on $\tilde{A}_\epsilon.$ 

    By Theorem 2, there exists some closed $A_\epsilon \subset \tilde{A}_\epsilon$ such that 
    $m(\tilde{A}_\epsilon \setminus A_\epsilon)< \frac{\epsilon}{2}.$ 
\end{proof}

\begin{thm}
    (Luzin) Suppose $f$ is measurable and finite valued on $E,$ where $E$ is of finite measure. For all $\epsilon >0,$ there exists a closed $F_\epsilon$ such that $m(F \setminus F_\epsilon) < \epsilon$ and $f|_{F_\epsilon}$ is continuous.
\end{thm}
\begin{proof}
    Let $(f_n)$ be a sequence of step functions such that $f_n \to f$ almost everywhere. Find $E_n \subset E$ such that $m(E_n)< \frac{1}{2^n}$ and $f_n$ are continuous on $E_n^c$ (since we would be taking only the value $1$ or $0$ on this set). By Egorov, there exists a closed set $A_\frac{\epsilon}{3}$ such that $f_n \to f$ uniformly on $A_\frac{\epsilon}{3}$ and $m(A_\frac{\epsilon}{3})< \frac{\epsilon}{3}.$
    Let 
    \[F:= A_\frac{\epsilon}{3} \setminus \bigcup_{n\geq N} E_n = A_\frac{\epsilon}{3} \cap \bigcup_{n\geq N}E_n^c.\] Thus, we have that 
    \[m(E \setminus F') = m(E \cap (\bigcup_{n\geq N}A_\frac{\epsilon}{3}\cap E_n^c)^c) = m(E \cap \bigcap_{n\geq N} A_\frac{\epsilon}{3}^c \cup E_n) \leq m(E \setminus A_\frac{\epsilon}{3}) + m(E \setminus E \bigcap_{m\geq n} E_n) < \frac{2\epsilon}{3}.\] We know that $f_n \to f$ uniformly on $F',$ and since each $f_n$ is continuous on $F',$ then $f$ is continuous on $F'.$ 

    There exists a closed $F_\epsilon\subset F'$ closed such that $m(F' \setminus F_\epsilon)< \frac{\epsilon   }{3}$ 
\end{proof}

\newpage
\subsection{Tuesday, Feb 11: The Lebesgue Integral}
We will build the integral in four steps
\begin{enumerate}
    \item Simple functions
    \item Bounded functions
    \item Supported on sets with final measure
    \item Non-negative functions
\end{enumerate}
Then we will generalize.
\begin{rem}
    Let $\varphi(x)$ be simple, then 
    \[\varphi(x) := \sum_{i=k}^n a_k \chi_{E_k}(x).\] 
    We run into a problem!
    \[0 = \chi_E(x) - \chi_E(x),\] so if we define the integral as simply the $\sum a_k \times m(E_k),$ we run into a uniqueness problem.
\end{rem}
\begin{defn}
    The \textbf{canonical form} of $\varphi,$ a simple function, is such that $\varphi$ takes finitely many different values on disjoint sets.
\end{defn}
\begin{rem}
    (Existence) Suppose $\varphi$ takes the values $c_1, \dots, c_m$ which are all distinct (we throw out repeats). Let \[F_k := \{x \; ; \; \varphi(x) = c_k\}.\] Note first that $F_k$ is measurable. Then note that for $k \neq k',$ then $F_k \cap F_{k'} = \emptyset.$ Evidently
    \[\varphi(x) = \sum_{i=1}^M c_k \chi_{F_k}(x).\] Becase $m(E_j) < \infty$ for each $j,$ and $F_k \subset \bigcup_{j=1}^N E_j,$ and thus $m(F_k) < \infty.$
\end{rem}
\begin{defn}
    Let $\varphi(x) = \sum_{k=1}^M c_k \chi_{F_k}$ be the canonical representation of a simple function $\varphi,$ then the \textbf{Lebesgue integral} of $\varphi$ is
    \[\int \varphi = \sum_{k=1}^M c_k m(F_k)\]
\end{defn}
\begin{rem}
    If $E$ is measurable, then $\chi_E(\varphi)$ is simple. Then by definition,
    \[\int_E\varphi = \int \varphi \cdot \chi_E\]
\end{rem}
\begin{prop}
The following hold
\begin{enumerate}
    \item If $\varphi = \sum_{i=1}^N a_k \chi_{E_k}$ is any representation of $\varphi$ as a simple function, then
    \[\int \varphi = \sum_{k=1}^N a_k m(E_k).\]
    \item If $\varphi$ and $\psi$ are simple, and $a,b \in \bbR,$ then 
    \[\int (a\varphi + b\psi) = a\int \varphi + b\int\psi\]
    \item Suppose $E, F$ are disjoint measurable sets with finite measure, then 
    \[\int_{E\cup F} \varphi = \int_E \varphi + \int_F \varphi\]
    \item Suppose $\varphi \leq \psi$ where both are simple. Then 
    \[\int \varphi \leq \int \psi\]
    \item Let $\varphi$ be a simple function, then $|\varphi|$ is a simple function and 
    \[\left|\int \varphi\right|\leq \int |\varphi|\]
\end{enumerate}
\end{prop}

\begin{proof}
    (a) Consider first the case when 
    \[\varphi(x) = \sum_{k=1}^N a_k \chi_{E_k},\] such that $E_k$ are mutually disjoint but the $a_k$ are not necessarily distinct. For all $a\neq 0$ such that $a\in \{a_k\},$ we define $E_a' = \bigcup E_{k_a},$ such that $a_k = a.$ Evidently, $E_a'$ are disjoint and measurable. Moreover, $m(E_a') = \sum m(E_{k_{a}}).$ Thus, $\phi = \sum a \chi_{E_a'},$ which is the canonical representationof $\varphi.$ Thus, we find that 
    \[\int \varphi = \sum a m(E_a') = \sum a \sum m(E_{k_a}) = \sum a_k m(E_k)\]

    Now for the general case, suppose $\varphi = \sum a_k \chi_{E_k}.$ Refine 
    \[\bigcup_{k=1}^N E_k = \bigcup_{j=1}^m E_j^*,\] where the $E_j^*$ are mutually disjoint, and for each $k,$ 
    \[E_k = \bigcup E_k^* \cdot E_j^* \subset E_k.\] For each $j,$ we let $a_j^* = \sum a_k,$ where $E_k \supset E_j^*.$ Thus, we have decomposed it to the first case.

    (b) Obvious

    (c) Obvious with $\chi_{E \cup F} = \chi_{E} + \chi_F$

    (d) Obvious with $\varphi \geq 0$ case.

    (d) Obvious from triangle inequality of summations of the canonical representation
\end{proof}

\newpage
\subsection{Thursday, Feb 18: The Lebesgue Integral, Bounded Convergence Theorem, Fatou}
I missed the class where the integral was constructed with (b) bounded functions and (d) None-negative functions. Srry.

\begin{thm}
    (Bounded Convergence Theorem) Suppose $(f_n)$ is a sequence of measurable functions, all uniformly bounded by some $M>0,$ all suppose on some finitely measurable $E,$ with $f_n(x) \to f(x)$  almost everywhere. Then 
    \[\lim_{n\to \infty}\int f_n = \int f\]
\end{thm}
The proof pretty much uses Ergorov's Theorem (Look at Tuesday, Feb 4: Littlewood's Three Principles)

\begin{thm}
    Suppose $f$ is Riemann integrable on $[a,b],$ then $f$ is measurable and 
    \[\int_{[a,b]}^\mathcal{R} f = \int_{[a,b]}^\mathcal{L}f\]
\end{thm}
\begin{proof}
    By definition of the Riemann integral, $f$ is bounded by some $M>0.$ There exist $(\varphi_k), (\psi_k)$ step functions uniformly bounded by $M,$ and the $\varphi_k$ are increasing and the $\psi_k$ are decreasing. Here, $\varphi_k$ and $\psi_k$ are the infemum and supremum over the sub intervals of partitions. 
    \[\lim_{k\to \infty}\int_{[a,b]}^{\cal R} \varphi_k = \lim_{k\to \infty}\int_{[a,b]}^\cal R \psi_k = \int_{[a,b]}^\cal R f.\]  For step functions, we obviously have that since the Riemann integral of the step function is the Lebesgue integral of the step function, then
    \begin{align}
    \lim_{k\to \infty}\int_{[a,b]}^{\cal R} \varphi_k  = \lim_{k\to \infty}\int_{[a,b]}^{\cal L} \varphi_k, \qquad \lim_{k\to \infty}\int_{[a,b]}^{\cal R} \psi_k  = \lim_{k\to \infty}\int_{[a,b]}^{\cal L} \psi_k.    
    \end{align}
    Since both sequences are monotonic and bounded, we define
    \[\tilde{\varphi} := \lim_{k\to \infty} \varphi_k, \qquad \tilde{\psi} := \lim_{k\to \infty} \psi_k,\] and clearly, 
    \[\tilde{\varphi} \leq f \leq \tilde{\psi}.\] Limits of measurable functions are measurable, and the bounded convergence theorem yields that 
    \[\lim_{k\to \infty}\int_{[a,b]}^{\cal L} \varphi_k = \int_{[a,b]}^{\cal L} \widetilde{\varphi}\]
    \[\lim_{k\to \infty}\int_{[a,b]}^{\cal L} \psi_k = \int_{[a,b]}^{\cal L} \widetilde{\psi}\] Using, (1), we see that 
    \[\int_{[a,b]}^ {\cal L} \widetilde{\varphi} = \int_{[a,b]}^ {\cal L} \widetilde{\psi}  = \int_{[a,b]}^ {\cal L} f,\] and so 
    \[\int_{[a,b]}^ {\cal L} (\widetilde{\psi} - \widetilde{\varphi}) = 0.\] Since $\widetilde{\psi}\geq \widetilde{\varphi},$ then by a classic result, we have that $\widetilde{\psi} = \widetilde{\varphi}$ a.e., and so they also equal $f$ a.e. Moreover, $\psi_k \to f$ a.e., and so by the BCT,
    \[\lim_{k\to \infty}\int_{[a,b]}^{\cal L} \psi_k = \int_{[a,b]}^{\cal L} f,\] and we are done by (1) and the results above.
\end{proof}

If $f\geq 0,$ then recall we construct its Lebesgue integral by 
\[\int f = \sup \int g,\] where $0\leq g \leq f,$ $g$ are bounded by $M,$ and $\sup g$ has finite measure.

\begin{prop}
    If $f$ is integrable, then $f(x)< \infty$ a.e. 
\end{prop}
\begin{proof}
    Let $E_k = \{x \; : \; f(x) > k\},$ then $k \chi_{E_k} \leq k,$ and so 
    \[k m(E_k) \leq \int f < \infty,\] and so $m(E_k) \to 0$ by moving the $k$ to the other side. Thus, since $E_k \downarrow E_\infty,$ then
    \[m(E_\infty) = 0.\]
\end{proof}
\begin{prop}
    If $\int f = 0,$ and $f\geq 0,$ then $f = 0$ a.e.
\end{prop}
\begin{proof}
    If $g$ is bounded w support of finite measure, then 
    \[0 \leq \int g \leq f,\] but $g = 0,$ and so $\sup g = f = 0.$ Define 
    \[\widetilde{E}_k = \{x \; : \; f(x) > \frac{1}{k}\},\] and thus 
    \[\frac{1}{k}m(\tilde{E}_k) \leq \int f = 0,\] and so $m(\widetilde{E}_k) = 0,$ and so 
    \[\{x \; : \; f(x) >0\} = \bigcup_k \widetilde{E}_k \implies m(x \; : \; f(x)>0) = 0.\]
\end{proof}


\begin{lemma}
    (Fatou's Lemma) Suppose $(f_n)$ is non-negative and measurable, and $f(x) = \lim_{n\to \infty}f_n(x)$ a.e. Then 
    \[\int f \leq \liminf_{n\to \infty}\int f_n.\]
\end{lemma}
\begin{proof}
    Let $0\leq g \leq f,$ with $g$ bounded and has finitely measured support $E$. Let $g_n = \min(g, f_n).$ Evidently, $g_n$ is bounded and support of $g_n$ is contained in $E.$ Notice that $g_n \to g$ a.e. Using the bounded convergence theorem, we have that 
    \[\lim_{n\to \infty} \int g_n = \int g\]
    \[\int g_n \leq \int f_n \implies \int g \leq \liminf_{n\to \infty}\int f_n,\] and so taking the supremum over all such $g$ yields the result.
\end{proof}

\begin{cor} (Monotone Convergence Theorem)
    Suppose $f_n\geq 0,$ $f_n$ is measurable with $0 \leq f_n\leq f$ such that $f_n \to f$ a.e. Then 
    \[\lim_{n\to \infty} \int f_n = \int f\]
\end{cor}
\begin{proof}
    We have that 
    \[f_n \leq f \implies \int f_n \leq \int f \implies \limsup_{n\to \infty} \int f_n \leq \int f.\] By Fatou's Lemma, we have that 
    \[\int f \leq \liminf_{n\to \infty}\int f_n.\] Thus, the limit exists and it equals the integral of the limit. 
\end{proof}

\begin{cor}
    Consider $\sum_{k\geq 0} a_k(x),$ with $a_k \geq 0$ and measurable. Then 
    \[\int \sum a_k(x) = \sum \int a_k\] and 
    if 
    \[\sum^\infty \int a_k < \infty,\] then $\sum a_k(x)$ converges a.e.
\end{cor}
\begin{proof}
    Let $f_n(x) = \sum_{k=1}^n a_k(x)$ and $f(x) = \sum^\infty a_k(x).$ Then $f_n(x)\geq 0,$ and $f_n \uparrow f.$ By the monotone convergence theorem, we have that 
    \[\int\sum_{k=1}^\infty a_k(x) = \lim_{n\to \infty}\int f_n(x) = \lim_{n\to \infty}\int \sum_{k=1}^n a_k(x) = \lim_{n\to \infty}\sum_{k=1}^n \int a_k(x) = \sum_{k=1}^\infty \int a_k(x).\] 

    For the second claim, we have that if $\sum \int a_k < \infty,$ then $\int \sum a_k < \infty,$ and so $\sum a_k < \infty$ by Proposition 10.
\end{proof}

\begin{lemma}
    (Borel-Cantelli) Suppose $(E_k).$ Define 
    \[E := \{x \; : \: x\in \bigcup E_k \text{ i.o.}\}.\] That is, 
    \[E = \bigcap_{n=1}^\infty \bigcup_{k\geq m}^\infty E_k.\] If $m(E_k)< \infty,$ then $m(E) = 0.$
\end{lemma}
\begin{proof}
    Let $a_k(x) = \chi_{E_k},$ since $x\in E,$ then 
    \[\sum a_k(x) = \infty \iff \sum m(E_k) = \sum \int a_k < \infty,\] and so $\sum a_k < \infty$ a.e, and so we are done.
\end{proof}

We will begin our final stage of constructing the integral. If $f$ is real valued, then $f$ is Lebesgue integrable if $|f|$ is.

\begin{defn}
    Suppose $f$ is integrable. Then 
    \[f^+ = \max(0, f), \qquad f^- = \max (0, -f).\] Thus, 
    \[|f|= f^+ - f^-.\] Moreover, we say that $f$ is \textbf{Lebesgue Integrable} if $f^+$ and $f^-$ are integrable. If $f$ is integrable, then 
    \[\int f = \int f^+ - \int f^-\]
\end{defn}
\begin{prop}
    Suppose $f = f_1 - f_2,$ where $f_i \geq 0$ is measurable and integrable. Then 
    \[\int f = \int f_1 - \int f_2.\]
\end{prop}
\begin{proof}
    We have that $f = f^+ - f^-,$ and so 
    \[f^+ - f^- = f_1 - f_2.\] Thus, 
    \[f^+ + f_2 = f_1 + f^- \implies \int f^+ + \int f_2 = \int f_1 + \int f^- \implies \int f = \int f^+ - \int f^- = \int f^1 - \int f^2\]
\end{proof}

\begin{prop}
    If $f$ is integrable, then $|f(x)|< \infty$ a.e.
\end{prop}

\begin{prop}
    The integral of Lebesgue measurable functions is linear, additive, monotonic, and satisfies the triangle inequality.
\end{prop}

\begin{prop}
    Let $f$ be integrable on $\bbR^d,$ $\epsilon>0.$ Then 
    \begin{itemize}
        \item there exists a set $B$ of finite measure such that 
        \[\int_{B^c} |f| < \epsilon\]
        \item (absolute continuity) There exists a $\delta>0$ such that if $m(E) \leq \delta,$ then $\int_E |f|\leq \epsilon.$
    \end{itemize}
\end{prop}
\begin{proof}
    Without loss of generality, let $f\geq 0.$ Let 
    \[B_N := \{\|x\| < N\}, \quad f_N(x) = f(x)\chi_{B_N}(x).\] Evidently, $f_N \geq 0$ and $f_N \uparrow f.$ Thus, we use the monotone convergence theorem:
    \[\int f = \lim_{N\to \infty}\int f_N = \lim_{N\to \infty}\int f(x)\chi_{B_N} = \lim_{N\to \infty}\int_{B_N}f(x).\] Thus, for large $N:$
    \[|\int f- \int_{B_N} f| = |\int_{B_N^c} f| <\epsilon\]

    For (ii), let 
    \[E_N := \{x \in \; : \; f(x) < N\}, \qquad f_N = \chi_{E_N}f.\] Then $f_N \geq 0,$ $f_N \uparrow f.$ Let $\epsilon>0.$ Again, by the monotone convergence therem, for large $n \geq N,$
    \[\int f - f_n < \frac{\epsilon}{2}.\]  Let $\delta>0$ such that $N\delta \leq \frac{\epsilon}{2}.$ Thus, if $m(E) < \delta,$ then 
    \[\int_E f = \int_E f - f_N  + \int_E f_N \leq \frac{\epsilon}{2} + \int_{E_N \cap E}f < \frac{\epsilon}{2} + m(E_N) N < \epsilon\]
\end{proof}

\begin{thm}
    (Dominated Convergence Theorem) Suppose $(f_N)$ is measurable, $f_n(x)\to f(x)$ a.e. If $|f_x(x)|\leq g(x),$ where $g$ is integrable, then 
    \[\int |f_n - f|\to 0 \implies \int f_n \to \int f\]
\end{thm}

\begin{proof}
    Let $E_N = \{x\; : \: |x| \leq N, |g(x)| \leq N\}.$ Let $\epsilon>0.$ By the previous proposition, letting $g_n = g \chi_{E_n}$ there exists some $N$ such that $\int_{E_N^c}g < \epsilon.$ 

    By the bounded convergence theorem, since $f_n \chi_{E_n} \leq g$ for all $f,$ $E_N \subset B_N$ (finite measure), then 
    \[\int_{E_n} f_n \to f.\] Thus, we find that 
    \begin{align*}
        \int |f_n - f| &= \int_{E_N}|f_n - f| + \int_{E_N^c}|f_n - f|\\
        &< \epsilon + 2\int_{E_N^c}g\\
        &< \epsilon
    \end{align*}
\end{proof}

\newpage
\subsection{Tuesday, Feb 25: The Completeness of $L^1$}
\begin{defn}
    A \textbf{complex valued function} is a function such that 
    $f(x) = u(x) + iv(x),$ where $u(x), v(x)\in \bbR$ for all $x.$
\end{defn}
\begin{rem}
    We note that 
    \[|f(x)| = (u^2(x) + v^2(x))^\frac{1}{2},\] and $f$ is measurable if and only if $u,v$ are measurable. We say that $f$ is integrable if and only if $|f(x)|$ is integrable. Moreover, $f$ is integrable on $E$ measurable if $\chi_E f$ is integrable. 
\end{rem}
\begin{prop}
    If $a\in \bbC$ and $f$ is integrable, then $af$ is integrable.
\end{prop}
\begin{proof}
    We can express $a$ as $a = \alpha + i\beta$ and so 
    \[af = (\alpha + i \beta)(u + iv) = (au - \beta v) + i(\alpha v + \beta u).\] It remains to check that 
    $|a f| = |a||f|.$
\end{proof}

\begin{defn}
    We say that 
    \[L^1 := \{f \; : \; |f| \text{ is integrable}\}\] and $L^1$ is equipped with 
    \[\|f\|_{L^1} = \int |f |\]
\end{defn}
\begin{prop}
    Suppose $f,g \in L^1(\bbR^d),$ then 
    \begin{enumerate}
        \item $\|af\|_{L^1} = |a|\|f\|_{L^1}$
        \item $\|f + g\|_{L^1}\leq \|f\|_{L^1} + \|g\|_{L^1}$
        \item $\|f\|_{L^1} = 0$ if and only if $f \equiv 0$
        \item $d(f,g) = \|f - g\|_{L^1}$
    \end{enumerate}
\end{prop}
Thus, $L^1$ is a normed space.
\begin{thm}
    (Riesz-Fischer) $L^1$ is complete. 
\end{thm}
\begin{proof}
    Suppose $(f_n)\in L^1$ is Cauchy, then for large $n,m > N_k$ we have that 
    \[\|f_n - f_m\|_{L^1} = \int |f_n(x) - f_m(x)|dx < \frac{1}{2^k}.\] Thus, if there exists a subsequence where $n_k = \sum N_K$ such that
    \[\|f_{n_{k+1}} - f_{n_k}\| < \frac{1}{2^k}.\] Thus, let 
    \[f(x):= f_{n_1}(x) + \sum_{k=1}^\infty f_{n_{k+1}}(x) - f_{n_k}(x).\] To show this $f$ is well defined, first we have that 
    \[g(x) = |f_{n_1}(x)| + \sum_{k=1}^\infty |f_{n_k+1}(x) - f_{n_k}(x)|.\]
    First, $g\geq 0$ and 
    \[\int |f_{n_1}(x)| + \sum_{k=1}^\infty |f_{n_k+1}(x) - f_{n_k}(x)|dx= \int f_{n_1}(x) + \sum_{k=1}^\infty \int |f_{n_k+1}(x) - f_{n_k}(x)| \leq \int |f_{n_1}| + \sum_{k=1}^\infty \frac{1}{2^k} < \infty,\] and so $g\in L^1.$ By the dominated convergence theorem, we have that $f_j\in L^1,$ where 
    \[|f_j(x)|:=  \left|f_{n_1}(x) +\sum_{k=1}^j f_{n_{k+1}}(x) - f_{n_k}(x)\right|,\] and since $f_j \to f,$ and $|f_j| \in L^1,$ then by the dominated convergence theorem,
    \[\int |f_j| \leq \int g \implies \int |f| \leq \int g.\] Thus, $f\in L^1,$ and so $|f|< \infty$ almost everywhere. Consider that by telescoping the sum, we have that
    \[f_{j}(x) = f_{n_{j+1}},\] and so $f_{n_{k+1}}(x)\to f$ a.e. Thus, since our subsequence converges to $f$ and $(f_n)$ is Cauchy, then the series converges to $f$ pointwise almost everywhere. By Egorov's theorem, we are done. 
\end{proof}

\begin{defn}
    A family $\cal F$ is dense in $L^1$ if for any $\epsilon>0$ and for any $f\in L^1,$ there exists some $g\in \cal F$ such that $\|f - g\|< \epsilon.$
\end{defn}
\begin{thm}
    The following families $\cal F$ are dense in $L^1:$
    \begin{enumerate}
        \item Simple functions
        \item Step functions
        \item Continuous functions with compact support
    \end{enumerate}
\end{thm}

\newpage
\subsection{Thursday Feb 27: Invariance Properties}
For a function on $\bbR^d,$ $h \in \bbR^d,$ we denote $f_h(x) = f(x-h).$
\begin{prop}
    Suppose $f$ is integrable. Then $f_h$ is integrable and 
    \[\int f_h = \int f\]
\end{prop}
\begin{proof}
    Suppose in the degenerate case that $f = \chi_E,$ where $E$ is measurable. Then of course, $f_h = \chi_{E_h}$ where $E_h$ is $E$ translated by $h,$ and so $f$ is integrable since $m(E + h) = m(E).$ Thus we know this is true for simple functions, and we can approximate all $L^1$ functions by simple functions and so we conclude.
\end{proof}
The following two proofs are similar
\begin{prop}
    Suppose $f$ is integrable, then if $\delta>0,$ then $f_\delta(x) = f(\delta x)$ is integrable. Moreover,
    \[\int f_\delta = \delta^d \int f\]
\end{prop}
\begin{prop}
    Suppose $f(x)$ is integrable, then $f(-x)$ is integrable and 
    \[\int f(x) = \int f(-x)\]
\end{prop}

\begin{prop}
    Suppose $f \in L^1,$ then $\|f_h - f\|_{L^1} \xrightarrow[]{h \to 0} 0.$
\end{prop}
\begin{proof}
    If $g$ is continuous w compact support, then $g$ is uniformly continuous. Thus, 
    \[\int |g_h (x) - g(x)|dx \xrightarrow[]{|h|\to 0}0\]

    Thus, if $f \in L^1,$ then 
    \[\|f_h - f\| \leq \|f_h - g_h\| + \|g_h - g\| + \|g - f\| = 2\|f - g\| + \|g_h - g\|,\] where the first term on the RHS is less than $\epsilon$ since $C([a,b])$ is dense in $L^1.$
\end{proof}

\begin{thm}
    (Fubini) Let $x\in \bbR^{d_1}$ and $y\in \bbR^{d_2},$ where $\bbR^d = \bbR^{d_1} \times \bbR^{d_2}.$ Suppose $f$ is a function on $\bbR^d,$ then the slice of $f$ corresponding on $\bbR^{d_1}$ is (we hold $y$ constant)
    \[f^y(x):=f(x,y)\] and similarly for the slice for the $y'$s. Then 
    \begin{enumerate}
        \item $f^y$ is integrable on $\bbR^{d_1}$
        \item $F_y(x) = \int_{\bbR^{d_1}}f^y(x)$ is integrable on $\bbR^{d_2}.$
        \item 
        \[\int_{\bbR^{d_2}}\left(\int_{\bbR^{d_1}}f^y(x)\right) = \int_{\bbR^d}f\]
    \end{enumerate}
\end{thm}

\begin{proof}
(Step 1) Define 
\[\mathcal{F}:= \{\text{integrable functions on $\bbR^d$ such that $a,b,c$ hold}\}\] Let $\{f_n\} \subset \cal F.$ For all $k \in [n],$ there exists some $A_k \subset \bbR^{d_2}$ such that $m(A_k) = 0$ and $f_k^y$ is integrable on $\bbR^d,$ where $y\notin A_k.$ Let $A = \bigcup_{k=1}^N A_k.$ Then $m(A) = 0$ and $y\in A^c,$ and $f_k^y$ is measurable and integrable for all $k \leq n.$  By linearity, we have that $\text{span}\bigcup f_n\subset \cal F.$

(Step 2) Suppose $\{f_n\} \subset \cal F$ and either $f_n\uparrow f$ or $f_n \downarrow f.$ We claim that either $f\in \cal F.$ Consider $-f_n,$ then we can assume WLOG that $f_n \uparrow f.$ Now consider $f_n - f_1 \geq 0,$ then WLOG, $f_n \geq 0.$ Let $A = \bigcup A_n.$ the $m(A) = 0$ and $y\notin A$ and $f_n^y$ is integrable on $\bbR^{d_1}.$ Thus, by monotone convergence theorem on $x$,
\[g_n(y) = \int_{\bbR^{d_1}} f_n^y(x) \uparrow g(y) = \int_{\bbR^{d_1}}f^y(x)dx\] By the monotone convergence in $y,$ we have that 
\[\int f_k(x,y)dxdy=\int g_k(y)\uparrow \int g(y)  = \int  \left(\int f(x,y)dx\right)dy\]

(Step 3) Suppose $f = \chi_E,$ $E = G_\delta,$ then $f \in \cal F$ (if $m(E)< \infty.$)
\begin{enumerate}
    \item $E_1 = Q_1 \times Q_2,$ where $Q_1 \subset \bbR^{d_1}$ and $Q_2 \subset \bbR^{d_2}$ are both open. Then 
    \[\chi_{E}(x,y) = \chi_{Q_1 \times Q_2}(x,y) = \chi_{Q_1}(x) \chi_{Q_2}(y),\] and so for each $y,$ and so for each $y,$ $\chi_E(x,y)$ is measurable in $x,$ and 
    \[\int \chi_E(x,y)dx = \chi_{Q_2}(y)\int \chi_{Q_1}(x)dx = \chi_{Q_2}(y)|Q_1|\] and 
    \[\int \left(\int \chi_E(x,y)dx\right)dy = |Q_2||Q_1| = m(E),\] and so $\chi_E \in \cal F.$
    \item Suppose $E \subset \partial Q,$ where $Q\subset \bbR^d$ is a closed cube. Evidently, $m(E) = 0,$ and so $\int \chi_E(x,y) dx dy = 0.$ For almost every $y$ (except for bottom and top lines), $E^y$ has measure $0$ on $\bbR^{d_1},$ and so $\int\chi_E^y(x)dx = 0$ almost everywhere, and thus $\int\left(\int\chi_E^y(x)dx \right)dy = 0,$ and thus $\chi_E \in \cal F.$
    \item Suppose $E = \bigcup_{k=1}^K Q_k,$ where $Q_k$ are closed cubes with disjoint interiors. Let $\text{int}Q_k = \tilde{Q}_k.$ And let $\chi_E$ be a finite linear combination of $\chi_{\tilde{Q}_k}$ and $\chi_{A_k},$ where $A_k\subset \partial Q_k.$ That is,
    \[\chi_E = \sum a_k\left(\chi_{\tilde{Q}_k} + \chi_{A_k}\right),\] and thus by previous steps we know that $\chi_E \in \cal F.$
    \item Suppose $E$ is open and has finite measure. Then 
    \[E = \bigcup_{k=1}^\infty Q_k,\] $Q_k$ closed cubes with disjoint interior. Let $f_k = \sum_{j=1}^k \chi_{Q_j} \in \cal F$ from stage 3. But $f_k \uparrow \chi_E,$ and so then by step 2 we are done.
    \item Suppose $E$ is $G_\delta$ of finite measure. Since $E \subset G_\delta,$ then there exists $(\tilde{O}_n)$ such that 
    \[E = \bigcap \tilde{ O}_k\] $E$ is measurable, and thus $E\subset O,$ where $O$ is open and measurable, and so take 
    \[O_k = O \cap \bigcap_{n=1}^k \tilde{O}_k,\] and we see that $(O_k)$ is non increasing with 
    \[E = \bigcap O_k.\] Let $f_k = \chi_{O_k},$ then $f_k \in \cal F$ and $\chi_E  \in L^1$ and $f_k \downarrow \chi_E,$ and thus $\chi_E \in \cal F.$
\end{enumerate}
(Step 4) Suppose $E \subset \bbR^d$ with $m(E) = 0$. There exists some $G \in G_\delta$ such that $E \subset G$ with $m(G) = 0.$ But by the previous step, $\chi_G \in \cal F$ with 
\[\int \left(\int \chi_G(x,y)dx\right)dy = 0.\] Thus, for almost every $y,$ 
\[\int \chi_G(x,y)dx = 0 = \int_{\bbR^{d_1}}\chi^y_{G}x dx.\] But $E^y \subset G^y,$ and so $m_\ast(E^y)\leq m_\ast(G^y) = 0,$ and thus $m(E^y) = 0$ and $\chi_{E}^y(x)$ is measurable with integral $0$ and thus $\chi_E \in \cal F.$

(Step 5) Suppose $f = \chi_E$ with $m(E)< \infty,$ then $E = G \cup Z,$ where $G\in G_\delta$ with $m(G)< \infty$ and $m(Z) = 0.$ Then $\chi_E \in \cal F.$ 

(Step 6) Let $f\in L^1.$ WLOG, $f\geq 0.$ There exist $(\varphi_n)$ simple functions such that $\varphi_n \uparrow f,$ where $\varphi \in \cal F$ by like step 1 or something. Thus, $f\in \cal F.$
\end{proof}

\begin{thm}
    Assume $f(x,y) \geq 0$ is measurable on $\bbR^d.$ Then for almost every $y \in \bbR^{d_2},$ 
    \begin{itemize}
        \item $f^y$ is measurable on $\bbR^{d_1}$
        \item $F^y = \int f^y(x) dx$ is measurable on $\bbR^{d_2}$
        \item $\int F^y  = \int f(x,y)dxdy.$
    \end{itemize}
\end{thm}



\end{document}