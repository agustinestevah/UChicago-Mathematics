\documentclass[11pt]{article}

% NOTE: Add in the relevant information to the commands below; or, if you'll be using the same information frequently, add these commands at the top of paolo-pset.tex file. 
\newcommand{\name}{Agustín Esteva}
\newcommand{\email}{aesteva@uchicago.edu}
\newcommand{\classnum}{208}
\newcommand{\subject}{Accelerated Analysis in $\bbR^n$ III}
\newcommand{\instructors}{Don Stull}
\newcommand{\assignment}{Problem Set 1}
\newcommand{\semester}{Spring 2025}
\newcommand{\duedate}{\today}
\newcommand{\bA}{\mathbf{A}}
\newcommand{\bB}{\mathbf{B}}
\newcommand{\bC}{\mathbf{C}}
\newcommand{\bD}{\mathbf{D}}
\newcommand{\bE}{\mathbf{E}}
\newcommand{\bF}{\mathbf{F}}
\newcommand{\bG}{\mathbf{G}}
\newcommand{\bH}{\mathbf{H}}
\newcommand{\bI}{\mathbf{I}}
\newcommand{\bJ}{\mathbf{J}}
\newcommand{\bK}{\mathbf{K}}
\newcommand{\bL}{\mathbf{L}}
\newcommand{\bM}{\mathbf{M}}
\newcommand{\bN}{\mathbf{N}}
\newcommand{\bO}{\mathbf{O}}
\newcommand{\bP}{\mathbf{P}}
\newcommand{\bQ}{\mathbf{Q}}
\newcommand{\bR}{\mathbf{R}}
\newcommand{\bS}{\mathbf{S}}
\newcommand{\bT}{\mathbf{T}}
\newcommand{\bU}{\mathbf{U}}
\newcommand{\bV}{\mathbf{V}}
\newcommand{\bW}{\mathbf{W}}
\newcommand{\bX}{\mathbf{X}}
\newcommand{\bY}{\mathbf{Y}}
\newcommand{\bZ}{\mathbf{Z}}
\newcommand{\Vol}{\text{Vol}}

%% blackboard bold math capitals
\newcommand{\bbA}{\mathbb{A}}
\newcommand{\bbB}{\mathbb{B}}
\newcommand{\bbC}{\mathbb{C}}
\newcommand{\bbD}{\mathbb{D}}
\newcommand{\bbE}{\mathbb{E}}
\newcommand{\bbF}{\mathbb{F}}
\newcommand{\bbG}{\mathbb{G}}
\newcommand{\bbH}{\mathbb{H}}
\newcommand{\bbI}{\mathbb{I}}
\newcommand{\bbJ}{\mathbb{J}}
\newcommand{\bbK}{\mathbb{K}}
\newcommand{\bbL}{\mathbb{L}}
\newcommand{\bbM}{\mathbb{M}}
\newcommand{\bbN}{\mathbb{N}}
\newcommand{\bbO}{\mathbb{O}}
\newcommand{\bbP}{\mathbb{P}}
\newcommand{\bbQ}{\mathbb{Q}}
\newcommand{\bbR}{\mathbb{R}}
\newcommand{\bbS}{\mathbb{S}}
\newcommand{\bbT}{\mathbb{T}}
\newcommand{\bbU}{\mathbb{U}}
\newcommand{\bbV}{\mathbb{V}}
\newcommand{\bbW}{\mathbb{W}}
\newcommand{\bbX}{\mathbb{X}}
\newcommand{\bbY}{\mathbb{Y}}
\newcommand{\bbZ}{\mathbb{Z}}

%% script math capitals
\newcommand{\sA}{\mathscr{A}}
\newcommand{\sB}{\mathscr{B}}
\newcommand{\sC}{\mathscr{C}}
\newcommand{\sD}{\mathscr{D}}
\newcommand{\sE}{\mathscr{E}}
\newcommand{\sF}{\mathscr{F}}
\newcommand{\sG}{\mathscr{G}}
\newcommand{\sH}{\mathscr{H}}
\newcommand{\sI}{\mathscr{I}}
\newcommand{\sJ}{\mathscr{J}}
\newcommand{\sK}{\mathscr{K}}
\newcommand{\sL}{\mathscr{L}}
\newcommand{\sM}{\mathscr{M}}
\newcommand{\sN}{\mathscr{N}}
\newcommand{\sO}{\mathscr{O}}
\newcommand{\sP}{\mathscr{P}}
\newcommand{\sQ}{\mathscr{Q}}
\newcommand{\sR}{\mathscr{R}}
\newcommand{\sS}{\mathscr{S}}
\newcommand{\sT}{\mathscr{T}}
\newcommand{\sU}{\mathscr{U}}
\newcommand{\sV}{\mathscr{V}}
\newcommand{\sW}{\mathscr{W}}
\newcommand{\sX}{\mathscr{X}}
\newcommand{\sY}{\mathscr{Y}}
\newcommand{\sZ}{\mathscr{Z}}


\renewcommand{\emptyset}{\O}

\newcommand{\abs}[1]{\lvert #1 \rvert}
\newcommand{\norm}[1]{\lVert #1 \rVert}
\newcommand{\sm}{\setminus}


\newcommand{\sarr}{\rightarrow}
\newcommand{\arr}{\longrightarrow}

% NOTE: Defining collaborators is optional; to not list collaborators, comment out the line below.
%\newcommand{\collaborators}{Alyssa P. Hacker (\texttt{aphacker}), Ben Bitdiddle (\texttt{bitdiddle})}

% Copyright 2021 Paolo Adajar (padajar.com, paoloadajar@mit.edu)
% 
% Permission is hereby granted, free of charge, to any person obtaining a copy of this software and associated documentation files (the "Software"), to deal in the Software without restriction, including without limitation the rights to use, copy, modify, merge, publish, distribute, sublicense, and/or sell copies of the Software, and to permit persons to whom the Software is furnished to do so, subject to the following conditions:
%
% The above copyright notice and this permission notice shall be included in all copies or substantial portions of the Software.
% 
% THE SOFTWARE IS PROVIDED "AS IS", WITHOUT WARRANTY OF ANY KIND, EXPRESS OR IMPLIED, INCLUDING BUT NOT LIMITED TO THE WARRANTIES OF MERCHANTABILITY, FITNESS FOR A PARTICULAR PURPOSE AND NONINFRINGEMENT. IN NO EVENT SHALL THE AUTHORS OR COPYRIGHT HOLDERS BE LIABLE FOR ANY CLAIM, DAMAGES OR OTHER LIABILITY, WHETHER IN AN ACTION OF CONTRACT, TORT OR OTHERWISE, ARISING FROM, OUT OF OR IN CONNECTION WITH THE SOFTWARE OR THE USE OR OTHER DEALINGS IN THE SOFTWARE.

\usepackage{fullpage}
\usepackage{enumitem}
\usepackage{amsfonts, amssymb, amsmath,amsthm}
\usepackage{mathtools}
\usepackage[pdftex, pdfauthor={\name}, pdftitle={\classnum~\assignment}]{hyperref}
\usepackage[dvipsnames]{xcolor}
\usepackage{bbm}
\usepackage{graphicx}
\usepackage{mathrsfs}
\usepackage{pdfpages}
\usepackage{tabularx}
\usepackage{pdflscape}
\usepackage{makecell}
\usepackage{booktabs}
\usepackage{natbib}
\usepackage{caption}
\usepackage{subcaption}
\usepackage{physics}
\usepackage[many]{tcolorbox}
\usepackage{version}
\usepackage{ifthen}
\usepackage{cancel}
\usepackage{listings}
\usepackage{courier}

\usepackage{tikz}
\usepackage{istgame}

\hypersetup{
	colorlinks=true,
	linkcolor=blue,
	filecolor=magenta,
	urlcolor=blue,
}

\setlength{\parindent}{0mm}
\setlength{\parskip}{2mm}

\setlist[enumerate]{label=({\alph*})}
\setlist[enumerate, 2]{label=({\roman*})}

\allowdisplaybreaks[1]

\newcommand{\psetheader}{
	\ifthenelse{\isundefined{\collaborators}}{
		\begin{center}
			{\setlength{\parindent}{0cm} \setlength{\parskip}{0mm}
				
				{\textbf{\classnum~\semester:~\assignment} \hfill \name}
				
				\subject \hfill \href{mailto:\email}{\tt \email}
				
				Instructor(s):~\instructors \hfill Due Date:~\duedate	
				
				\hrulefill}
		\end{center}
	}{
		\begin{center}
			{\setlength{\parindent}{0cm} \setlength{\parskip}{0mm}
				
				{\textbf{\classnum~\semester:~\assignment} \hfill \name\footnote{Collaborator(s): \collaborators}}
				
				\subject \hfill \href{mailto:\email}{\tt \email}
				
				Instructor(s):~\instructors \hfill Due Date:~\duedate	
				
				\hrulefill}
		\end{center}
	}
}

\renewcommand{\thepage}{\classnum~\assignment \hfill \arabic{page}}

\makeatletter
\def\points{\@ifnextchar[{\@with}{\@without}}
\def\@with[#1]#2{{\ifthenelse{\equal{#2}{1}}{{[1 point, #1]}}{{[#2 points, #1]}}}}
\def\@without#1{\ifthenelse{\equal{#1}{1}}{{[1 point]}}{{[#1 points]}}}
\makeatother

\newtheoremstyle{theorem-custom}%
{}{}%
{}{}%
{\itshape}{.}%
{ }%
{\thmname{#1}\thmnumber{ #2}\thmnote{ (#3)}}

\theoremstyle{theorem-custom}

\newtheorem{theorem}{Theorem}
\newtheorem{lemma}[theorem]{Lemma}
\newtheorem{example}[theorem]{Example}

\newenvironment{problem}[1]{\color{black} #1}{}

\newenvironment{solution}{%
	\leavevmode\begin{tcolorbox}[breakable, colback=green!5!white,colframe=green!75!black, enhanced jigsaw] \proof[\scshape Solution:] \setlength{\parskip}{2mm}%
	}{\renewcommand{\qedsymbol}{$\blacksquare$} \endproof \end{tcolorbox}}

\newenvironment{reflection}{\begin{tcolorbox}[breakable, colback=black!8!white,colframe=black!60!white, enhanced jigsaw, parbox = false]\textsc{Reflections:}}{\end{tcolorbox}}

\newcommand{\qedh}{\renewcommand{\qedsymbol}{$\blacksquare$}\qedhere}

\definecolor{mygreen}{rgb}{0,0.6,0}
\definecolor{mygray}{rgb}{0.5,0.5,0.5}
\definecolor{mymauve}{rgb}{0.58,0,0.82}

% from https://github.com/satejsoman/stata-lstlisting
% language definition
\lstdefinelanguage{Stata}{
	% System commands
	morekeywords=[1]{regress, reg, summarize, sum, display, di, generate, gen, bysort, use, import, delimited, predict, quietly, probit, margins, test},
	% Reserved words
	morekeywords=[2]{aggregate, array, boolean, break, byte, case, catch, class, colvector, complex, const, continue, default, delegate, delete, do, double, else, eltypedef, end, enum, explicit, export, external, float, for, friend, function, global, goto, if, inline, int, local, long, mata, matrix, namespace, new, numeric, NULL, operator, orgtypedef, pointer, polymorphic, pragma, private, protected, public, quad, real, return, rowvector, scalar, short, signed, static, strL, string, struct, super, switch, template, this, throw, transmorphic, try, typedef, typename, union, unsigned, using, vector, version, virtual, void, volatile, while,},
	% Keywords
	morekeywords=[3]{forvalues, foreach, set},
	% Date and time functions
	morekeywords=[4]{bofd, Cdhms, Chms, Clock, clock, Cmdyhms, Cofc, cofC, Cofd, cofd, daily, date, day, dhms, dofb, dofC, dofc, dofh, dofm, dofq, dofw, dofy, dow, doy, halfyear, halfyearly, hh, hhC, hms, hofd, hours, mdy, mdyhms, minutes, mm, mmC, mofd, month, monthly, msofhours, msofminutes, msofseconds, qofd, quarter, quarterly, seconds, ss, ssC, tC, tc, td, th, tm, tq, tw, week, weekly, wofd, year, yearly, yh, ym, yofd, yq, yw,},
	% Mathematical functions
	morekeywords=[5]{abs, ceil, cloglog, comb, digamma, exp, expm1, floor, int, invcloglog, invlogit, ln, ln1m, ln, ln1p, ln, lnfactorial, lngamma, log, log10, log1m, log1p, logit, max, min, mod, reldif, round, sign, sqrt, sum, trigamma, trunc,},
	% Matrix functions
	morekeywords=[6]{cholesky, coleqnumb, colnfreeparms, colnumb, colsof, corr, det, diag, diag0cnt, el, get, hadamard, I, inv, invsym, issymmetric, J, matmissing, matuniform, mreldif, nullmat, roweqnumb, rownfreeparms, rownumb, rowsof, sweep, trace, vec, vecdiag, },
	% Programming functions
	morekeywords=[7]{autocode, byteorder, c, _caller, chop, abs, clip, cond, e, fileexists, fileread, filereaderror, filewrite, float, fmtwidth, has_eprop, inlist, inrange, irecode, matrix, maxbyte, maxdouble, maxfloat, maxint, maxlong, mi, minbyte, mindouble, minfloat, minint, minlong, missing, r, recode, replay, return, s, scalar, smallestdouble,},
	% Random-number functions
	morekeywords=[8]{rbeta, rbinomial, rcauchy, rchi2, rexponential, rgamma, rhypergeometric, rigaussian, rlaplace, rlogistic, rnbinomial, rnormal, rpoisson, rt, runiform, runiformint, rweibull, rweibullph,},
	% Selecting time-span functions
	morekeywords=[9]{tin, twithin,},
	% Statistical functions
	morekeywords=[10]{betaden, binomial, binomialp, binomialtail, binormal, cauchy, cauchyden, cauchytail, chi2, chi2den, chi2tail, dgammapda, dgammapdada, dgammapdadx, dgammapdx, dgammapdxdx, dunnettprob, exponential, exponentialden, exponentialtail, F, Fden, Ftail, gammaden, gammap, gammaptail, hypergeometric, hypergeometricp, ibeta, ibetatail, igaussian, igaussianden, igaussiantail, invbinomial, invbinomialtail, invcauchy, invcauchytail, invchi2, invchi2tail, invdunnettprob, invexponential, invexponentialtail, invF, invFtail, invgammap, invgammaptail, invibeta, invibetatail, invigaussian, invigaussiantail, invlaplace, invlaplacetail, invlogistic, invlogistictail, invnbinomial, invnbinomialtail, invnchi2, invnF, invnFtail, invnibeta, invnormal, invnt, invnttail, invpoisson, invpoissontail, invt, invttail, invtukeyprob, invweibull, invweibullph, invweibullphtail, invweibulltail, laplace, laplaceden, laplacetail, lncauchyden, lnigammaden, lnigaussianden, lniwishartden, lnlaplaceden, lnmvnormalden, lnnormal, lnnormalden, lnwishartden, logistic, logisticden, logistictail, nbetaden, nbinomial, nbinomialp, nbinomialtail, nchi2, nchi2den, nchi2tail, nF, nFden, nFtail, nibeta, normal, normalden, npnchi2, npnF, npnt, nt, ntden, nttail, poisson, poissonp, poissontail, t, tden, ttail, tukeyprob, weibull, weibullden, weibullph, weibullphden, weibullphtail, weibulltail,},
	% String functions 
	morekeywords=[11]{abbrev, char, collatorlocale, collatorversion, indexnot, plural, plural, real, regexm, regexr, regexs, soundex, soundex_nara, strcat, strdup, string, strofreal, string, strofreal, stritrim, strlen, strlower, strltrim, strmatch, strofreal, strofreal, strpos, strproper, strreverse, strrpos, strrtrim, strtoname, strtrim, strupper, subinstr, subinword, substr, tobytes, uchar, udstrlen, udsubstr, uisdigit, uisletter, ustrcompare, ustrcompareex, ustrfix, ustrfrom, ustrinvalidcnt, ustrleft, ustrlen, ustrlower, ustrltrim, ustrnormalize, ustrpos, ustrregexm, ustrregexra, ustrregexrf, ustrregexs, ustrreverse, ustrright, ustrrpos, ustrrtrim, ustrsortkey, ustrsortkeyex, ustrtitle, ustrto, ustrtohex, ustrtoname, ustrtrim, ustrunescape, ustrupper, ustrword, ustrwordcount, usubinstr, usubstr, word, wordbreaklocale, worcount,},
	% Trig functions
	morekeywords=[12]{acos, acosh, asin, asinh, atan, atanh, cos, cosh, sin, sinh, tan, tanh,},
	morecomment=[l]{//},
	% morecomment=[l]{*},  // `*` maybe used as multiply operator. So use `//` as line comment.
	morecomment=[s]{/*}{*/},
	% The following is used by macros, like `lags'.
	morestring=[b]{`}{'},
	% morestring=[d]{'},
	morestring=[b]",
	morestring=[d]",
	% morestring=[d]{\\`},
	% morestring=[b]{'},
	sensitive=true,
}

\lstset{ 
	backgroundcolor=\color{white},   % choose the background color; you must add \usepackage{color} or \usepackage{xcolor}; should come as last argument
	basicstyle=\footnotesize\ttfamily,        % the size of the fonts that are used for the code
	breakatwhitespace=false,         % sets if automatic breaks should only happen at whitespace
	breaklines=true,                 % sets automatic line breaking
	captionpos=b,                    % sets the caption-position to bottom
	commentstyle=\color{mygreen},    % comment style
	deletekeywords={...},            % if you want to delete keywords from the given language
	escapeinside={\%*}{*)},          % if you want to add LaTeX within your code
	extendedchars=true,              % lets you use non-ASCII characters; for 8-bits encodings only, does not work with UTF-8
	firstnumber=0,                % start line enumeration with line 1000
	frame=single,	                   % adds a frame around the code
	keepspaces=true,                 % keeps spaces in text, useful for keeping indentation of code (possibly needs columns=flexible)
	keywordstyle=\color{blue},       % keyword style
	language=Octave,                 % the language of the code
	morekeywords={*,...},            % if you want to add more keywords to the set
	numbers=left,                    % where to put the line-numbers; possible values are (none, left, right)
	numbersep=5pt,                   % how far the line-numbers are from the code
	numberstyle=\tiny\color{mygray}, % the style that is used for the line-numbers
	rulecolor=\color{black},         % if not set, the frame-color may be changed on line-breaks within not-black text (e.g. comments (green here))
	showspaces=false,                % show spaces everywhere adding particular underscores; it overrides 'showstringspaces'
	showstringspaces=false,          % underline spaces within strings only
	showtabs=false,                  % show tabs within strings adding particular underscores
	stepnumber=2,                    % the step between two line-numbers. If it's 1, each line will be numbered
	stringstyle=\color{mymauve},     % string literal style
	tabsize=2,	                   % sets default tabsize to 2 spaces
%	title=\lstname,                   % show the filename of files included with \lstinputlisting; also try caption instead of title
	xleftmargin=0.25cm
}

% NOTE: To compile a version of this pset without problems, solutions, or reflections, uncomment the relevant line below.

%\excludeversion{problem}
%\excludeversion{solution}
%\excludeversion{reflection}

\begin{document}	
	
	% Use the \psetheader command at the beginning of a pset. 
	\psetheader

\section*{Problem 1}


\newpage

\section*{Problem 2}
\begin{problem}
    Let $x \in \mathbb{R}^n$, and $A \subseteq \mathbb{R}^n$. Prove that
\[
\mathbf{m}^*(x + A) = \mathbf{m}^*(A),
\]
where $x + A = \{x + y \mid y \in A\}$. Prove that, if $A \in \mathcal{M}(\mathbf{m})$, then $x + A \in \mathcal{M}(\mathbf{m})$.
\end{problem}
\begin{solution}
    \begin{lemma}
    Let $X$ be a metric space and $A \subset X$ be open. Let $x\in X.$ Then $x + A$ is open.
    \end{lemma}
    \begin{proof}
        Let $z \in x + A.$ Then $z = x + a$ for some $a \in A.$ Since $a \in A$ and $A$ is open, there exists some $r>0$ such that $B_r(a)\subset A.$ We claim that $x + B_r(a) \subset x + A.$ Indeed, since $x + B_r(a) = B_r(a + x) = B_r(z),$ then it suffices to show that $x + B_r(a) \subset x + A.$ Let $z' \in x + B_r(a),$ then $z' = x + a'$ for some $a' \in B_r(a) \subset A,$ and so $z' \in x + A.$ Thus, we have found some $r>0$ such that $B_r(z)\subset x + A,$ and thus $x + A$ is open.
    \end{proof}
    \begin{lemma}
        Let $\{A_n\}_{n =1}^\infty$ be a countable open cover of $A \subset X,$ where $X$ is a topological space. If $x \in X,$ then $\{x + A_n\}$ is a countable open cover of $x + A.$
    \end{lemma}
    \begin{proof}
        Let $z \in x + A.$ Then $z = x + a$ for some $a\in A$ and thus there exists some $k \in \bbN$ such that $A_k \in \{A_n\}$ and $a \in A_k.$ Thus, $z\in x + A_k.$ By Lemma 1, we know that $x + A_k$ is open, and thus $\{x + A_n\}_{n =1}^\infty$ is a countable open cover of $x + A.$
    \end{proof}
    \begin{lemma}
     Suppose $E \in \cal E,$ and $E \subset \bbR^n.$ Then for any $x\in \bbR^n,$ $x + E \in \cal E.$
    \end{lemma}
    \begin{proof}
        It suffices to show that $x + E$ is the union of finite intervals. Since $E \in \cal E,$ then $E = \bigcup_{k=1}^n I_k.$ Thus, we claim that
        \[x + E = x + \bigcup_{k=1}^N I_k = \bigcup_{k=1}^N(x + I_k).\] The only equality that needs explaining is the second one. To see it, let $z \in x + \bigcup_{k=1}^N I_k,$ then there exists some $j \in [N]$ and some $x_j \in I_j$ such that $z = x + x_j.$ Thus, $z \in x + I_j,$ and so $x \in \bigcup_{k=1}^N (x + I_k).$ For the other inclusion, let $z\in \bigcup_{k=1}^N (x + I_k),$ then there exists some $j \in [N]$ such that $z\in x + I_j.$ Thus, $z = x + x_j$ for some $x_j \in I_j.$ Since $x_j \in \bigcup I_k,$ then $z \in x + \bigcup I_k.$
        
        We claim that $x + I_k$ is an interval for any $k \in [n].$ It should be obvious that if $I_k$ is the interval made of points $\textbf{x} = (x_1, \dots, x_n) \in \bbR^n$ such that $a_i \leq x_i \leq b_i$ (where the $\leq$ can be replaced with $<$), then $x + I_k$ is the set of points $\textbf{x}' = (x_1', \dots, x_n')$ such that $a_i  + x \leq x_i' \leq b_i + x,$ and thus $x + I_k$ is an interval. 
    \end{proof}
    
    \begin{lemma}
        Suppose $E \in \cal E,$ and $E \subset \bbR^n.$ Then for any $x\in \bbR^n,$ we have that $\Vol(E) = \Vol(x + E).$ 
    \end{lemma}
    \begin{proof}
        First, we know that $\Vol(x + E)$ is well defined since by Lemma 3, $x + E \in \cal E.$ By work in class, we know that we can decompose $E$ into $E = \bigsqcup_{k=1}^N I_k,$ where each the $I_k$ are disjoint intervals. By work done in the Lemma 3, we know that $x + E = \bigsqcup_{k=1}^N (x + I_k),$ where again, $(x + I_k)$ are disjoint nonempty intervals made of points such that $\textbf{x} \in (x + I_k)$ if and only if 
        \[a_i  + x \leq x_i \leq b_i + x,\] where $(a_i, b_i)$ are the endpoints of the edges of each $I_i.$ Thus, we have that 
        \begin{align*}
            \Vol(E) &= \Vol(\bigsqcup_{k=1}^N I_k)\\
            &= \sum_{k=1}^N \Vol(I_k)\\
            &= \sum_{k=1}^N \prod_{i=1}^n(b_i^k - a_i^k)\\
            &= \sum_{k=1}^N \prod_{i=1}^n(b_i^k - a_i^k + x - x)\\
            &= \sum_{k=1}^N \prod_{i=1}^n((b_i^k + x) - (a_i^k + x))\\
            &= \sum_{k=1}^N (x + I_k)\\
            &= \Vol(x + E)
        \end{align*}
    \end{proof}
Finally, we are ready to prove the statement. Let $\epsilon>0.$ Let $\{A_n\}_{n=1}^\infty$ be a countable open cover of $A$ such that each $A_n \in \cal E$ and 
\[\sum_{n=1}^\infty \Vol(A_n) \leq m^*(A) + \epsilon.\] By Lemma 4, we have that 
\[\sum_{n=1}^\infty \Vol(A_n) = \sum_{n=1}^\infty\Vol(x + A_n).\] By Lemma 2, we know that $\{x + A_n\}_{n=1}^\infty$ is a countably open cover of $x + A.$ By Lemma 3, $x + A_n \in \cal E$ for all $n.$ Thus, 
\[m^*(x + A) \leq\sum_{n=1}^\infty\Vol(x + A_n) =\sum_{n=1}^\infty \Vol(A_n) \leq m^*(A) + \epsilon,\] and so because this holds for any $\epsilon>0,$ we have that.
\begin{align}
    m^*(x + A) \leq m^*(A)
\end{align}

Let $\{x + A_n\}_{n=1}^\infty$ be a countable open cover of $x + A$ such that each $x + A_n \in \cal E$ and 
\[\sum_{n=1}^\infty \Vol(x + A_n) \leq m^*(x + A) + \epsilon\]
We can apply all our lemmas to $A_n$ since $A_n = (-x) + x + A_n.$ Thus, $\{A_n\}_{n=1}^\infty$ is a countable cover of $A$ with $A_n \in \cal E$ for all $n.$  Thus, 
\[m^*(A) \leq \sum_{n=1}^\infty \Vol(A_n) \leq \sum_{n=1}^\infty \Vol(x + A_n) \leq m^*(x + A) + \epsilon.\] Because this holds for all $\epsilon>0,$ we have that 
\begin{align}
    m^*(A) \leq m^*(x + a)
\end{align}
Putting together (1) and (2), we see that $m^*(A) = m^*(x + A).$
\end{solution}

\newpage
\section*{Problem 3}
\begin{problem}
    Consider the real line in $\mathbb{R}^2$, 
\[
X = \{(x,0) \mid x \in \mathbb{R}\}.
\]
What is $m^*(X)$ (where $m^*$ is the Lebesgue outer measure defined on $\mathbb{R}^2$)? Show that $X \in \mathcal{M}(m)$.
\end{problem}
\begin{solution}
Define 
\[X_n := \{(x,0) \mid x \in (-n, n)\}\]
Evidently, $X = \bigcup_{n=1}^\infty X_n.$ Thus, we know by Theorem 11.8 in Rudin that
\begin{align}
m^*(X) \leq \sum_{n=1}^\infty m^*(X_n)    
\end{align}
Let $\epsilon>0.$ For each $n,$ define 
\[E_\epsilon^n = \{(x,y) \in \bbR^2 \mid x \in (-n, n), y \in (-\frac{\epsilon}{2^n \cdot 4n}, \frac{\epsilon}{2^n \cdot 4n})\}.\] Clearly, we know that 
\begin{enumerate}
    \item $X_n \subset E_\epsilon^n$ for any $n,$ for any $\epsilon>0.$ Thus, $E_\epsilon^n$ is an open cover of $X_n,$ and so 
    \begin{align}
        m^*(X_n) \leq \Vol(E_\epsilon^n), \qquad \forall \;n \in \bbN
    \end{align}
    \item $E_\epsilon^n$ is an interval, and thus 
    \begin{align}\Vol(E_\epsilon^n) = (2n)(\frac{\epsilon}{2^n \cdot 2n}) = \frac{\epsilon}{2^n}    
    \end{align}
By (3), (4), and (5):
\[m^*(X) \leq \sum_{n=1}^\infty m^*(X_n)\leq \sum_{n=1}^\infty \Vol(E_\epsilon^n) \leq \sum_{n=1}^\infty \frac{\epsilon}{2^n} = \epsilon.\] Because this holds for all $\epsilon>0$, $m^*(X) = 0.$
\end{enumerate}
\end{solution}

\newpage
\section*{Problem 4}
\begin{problem}
    Let $A, B \subseteq \mathbb{R}^n$, and suppose that
\[
d := \inf_{x \in A, y \in B} |x - y| > 0.
\]
Prove that
\[
m^*(A \cup B) = m^*(A) + m^*(B).
\]
\end{problem}
\begin{solution}
    From Theorem 11.8 in Rudin, we have that 
    \[m^*(A \cup B) \leq m^*(A) + m^*(B),\] so it suffices to show the other inequality.

    Let $\epsilon>0.$ There exists $\{X_n\}_{n=1}^\infty$ countable open cover of $A\cup B$ such that
    \[\sum_{n=1}^\infty \Vol(X_n) \leq m^*(A \cup B) + \epsilon.\] Si
\end{solution}
\end{document}