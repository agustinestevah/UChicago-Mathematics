\documentclass[11pt]{article}

% NOTE: Add in the relevant information to the commands below; or, if you'll be using the same information frequently, add these commands at the top of paolo-pset.tex file. 
\newcommand{\name}{Agustín Esteva}
\newcommand{\email}{aesteva@uchicago.edu}
\newcommand{\classnum}{208}
\newcommand{\subject}{Accelerated Analysis in $\bbR^n$ III}
\newcommand{\instructors}{Don Stull}
\newcommand{\assignment}{Problem Set 1}
\newcommand{\semester}{Spring 2025}
\newcommand{\duedate}{\today}
\newcommand{\bA}{\mathbf{A}}
\newcommand{\bB}{\mathbf{B}}
\newcommand{\bC}{\mathbf{C}}
\newcommand{\bD}{\mathbf{D}}
\newcommand{\bE}{\mathbf{E}}
\newcommand{\bF}{\mathbf{F}}
\newcommand{\bG}{\mathbf{G}}
\newcommand{\bH}{\mathbf{H}}
\newcommand{\bI}{\mathbf{I}}
\newcommand{\bJ}{\mathbf{J}}
\newcommand{\bK}{\mathbf{K}}
\newcommand{\bL}{\mathbf{L}}
\newcommand{\bM}{\mathbf{M}}
\newcommand{\bN}{\mathbf{N}}
\newcommand{\bO}{\mathbf{O}}
\newcommand{\bP}{\mathbf{P}}
\newcommand{\bQ}{\mathbf{Q}}
\newcommand{\bR}{\mathbf{R}}
\newcommand{\bS}{\mathbf{S}}
\newcommand{\bT}{\mathbf{T}}
\newcommand{\bU}{\mathbf{U}}
\newcommand{\bV}{\mathbf{V}}
\newcommand{\bW}{\mathbf{W}}
\newcommand{\bX}{\mathbf{X}}
\newcommand{\bY}{\mathbf{Y}}
\newcommand{\bZ}{\mathbf{Z}}
\newcommand{\Vol}{\text{Vol}}

%% blackboard bold math capitals
\newcommand{\bbA}{\mathbb{A}}
\newcommand{\bbB}{\mathbb{B}}
\newcommand{\bbC}{\mathbb{C}}
\newcommand{\bbD}{\mathbb{D}}
\newcommand{\bbE}{\mathbb{E}}
\newcommand{\bbF}{\mathbb{F}}
\newcommand{\bbG}{\mathbb{G}}
\newcommand{\bbH}{\mathbb{H}}
\newcommand{\bbI}{\mathbb{I}}
\newcommand{\bbJ}{\mathbb{J}}
\newcommand{\bbK}{\mathbb{K}}
\newcommand{\bbL}{\mathbb{L}}
\newcommand{\bbM}{\mathbb{M}}
\newcommand{\bbN}{\mathbb{N}}
\newcommand{\bbO}{\mathbb{O}}
\newcommand{\bbP}{\mathbb{P}}
\newcommand{\bbQ}{\mathbb{Q}}
\newcommand{\bbR}{\mathbb{R}}
\newcommand{\bbS}{\mathbb{S}}
\newcommand{\bbT}{\mathbb{T}}
\newcommand{\bbU}{\mathbb{U}}
\newcommand{\bbV}{\mathbb{V}}
\newcommand{\bbW}{\mathbb{W}}
\newcommand{\bbX}{\mathbb{X}}
\newcommand{\bbY}{\mathbb{Y}}
\newcommand{\bbZ}{\mathbb{Z}}

%% script math capitals
\newcommand{\sA}{\mathscr{A}}
\newcommand{\sB}{\mathscr{B}}
\newcommand{\sC}{\mathscr{C}}
\newcommand{\sD}{\mathscr{D}}
\newcommand{\sE}{\mathscr{E}}
\newcommand{\sF}{\mathscr{F}}
\newcommand{\sG}{\mathscr{G}}
\newcommand{\sH}{\mathscr{H}}
\newcommand{\sI}{\mathscr{I}}
\newcommand{\sJ}{\mathscr{J}}
\newcommand{\sK}{\mathscr{K}}
\newcommand{\sL}{\mathscr{L}}
\newcommand{\sM}{\mathscr{M}}
\newcommand{\sN}{\mathscr{N}}
\newcommand{\sO}{\mathscr{O}}
\newcommand{\sP}{\mathscr{P}}
\newcommand{\sQ}{\mathscr{Q}}
\newcommand{\sR}{\mathscr{R}}
\newcommand{\sS}{\mathscr{S}}
\newcommand{\sT}{\mathscr{T}}
\newcommand{\sU}{\mathscr{U}}
\newcommand{\sV}{\mathscr{V}}
\newcommand{\sW}{\mathscr{W}}
\newcommand{\sX}{\mathscr{X}}
\newcommand{\sY}{\mathscr{Y}}
\newcommand{\sZ}{\mathscr{Z}}


\renewcommand{\emptyset}{\O}

\newcommand{\abs}[1]{\lvert #1 \rvert}
\newcommand{\norm}[1]{\lVert #1 \rVert}
\newcommand{\sm}{\setminus}


\newcommand{\sarr}{\rightarrow}
\newcommand{\arr}{\longrightarrow}

% NOTE: Defining collaborators is optional; to not list collaborators, comment out the line below.
%\newcommand{\collaborators}{Alyssa P. Hacker (\texttt{aphacker}), Ben Bitdiddle (\texttt{bitdiddle})}

\input{paolo-pset.tex}

% NOTE: To compile a version of this pset without problems, solutions, or reflections, uncomment the relevant line below.

%\excludeversion{problem}
%\excludeversion{solution}
%\excludeversion{reflection}

\begin{document}	
	
	% Use the \psetheader command at the beginning of a pset. 
	\psetheader

\section*{Problem 1}
\begin{problem}
    Let $\mathcal{R}$ be the ring of all elementary subsets of $(0, 1]$. If $0 < a \leq b \leq 1$, define
\[
\phi([a, b]) = \phi([a, b)) = \phi((a, b]) = \phi((a, b)) = b - a,
\]
but define
\[
\phi((0, b)) = \phi((0, b]) = 1 + b
\]
if $0 < b \leq 1$. Show that this gives an additive set function $\phi$ on $\mathcal{R}$, which is not regular and which cannot be extended to a countably additive set function on a $\sigma$-ring.
\end{problem}
\begin{solution}
    (Additive) Let $A, B \in \cal R$ disjoint. Suppose neither $A$ nor $B$ contain intervals of the form $(0,b]$ or $(0,b).$ Since both $A$ and $B$ are elementary, then we can write $A$ and $B$ as a disjoint union of finite intervals. That is \[A = \bigcup_{i=1}^\infty (a_i, b_i), \qquad B = \bigcup_{i=1}^\infty (c_i, d_i), \qquad \text{The parenthesis may be replaced by $[\cdot,\cdot], [\cdot,\cdot),$ or $(\cdot,\cdot]$}\] Since $A \cap B  = \emptyset,$ $(a_i, b_i) \cap (c_j, d_j) = \emptyset$ for all $i,j.$ Thus we can write 
    \[A \cup B = \bigcup_{i=1}^\infty (e_i, d_i)\] where each interval is disjoint and moreover, 
    \[\phi(A \cup B) =\phi\left( \bigcup_{i=1}^\infty (e_i, d_i)\right) = \sum_{n=1}^\infty \phi((e_i, d_i)) = \sum_{n=1}^\infty \phi(a_i, b_i) + \sum_{n=1}^\infty \phi(c_i, d_i) = \phi(A) + \phi(B)\]
    Suppose $A$ contains an interval $(a_k,b_k)$ such that $a_k = 0.$ Then since $A$ and $B$ are disjoint, it is clear that $B$ cannot also contain some interval $(c_{k'}, d_{k'})$ such that $c_{k'} = 0,$ as then the sets would not be disjoint. But then the above formula still holds, noting that we can separate the infinite sum by singling out the $(a_k, b_k)$ interval and adding $1$ on both sides. 

    (Not Regular) Suppose $\phi$ were regular. Let $A = (0, 1].$ Then $A \in \cal R$ and $\phi(A) = 1 + 1 = 2.$ Let $\epsilon = \frac{1}{2}$ Since $\phi$ is regular, there exists some closed $F \in \cal R$ such that $F \subset A$ and \[\phi(A) \leq \phi(F) + \frac{1}{2}.\] However, since $F$ is closed and $F\subset A,$ then $F$ cannot contain an interval of the form $(0,b)$ or $(0,b],$ as otherwise, it would not be closed. So then $\phi(F) \leq \phi((\frac{1}{n}, 1]) = 1-\frac{1}{n}$ for any $n,$ and we arrive at a contradiction, since \[\phi(F) + \frac{1}{2} \leq 1.5 < 2 = \phi(A).\]

    (Not countably additive) Suppose that $\phi$ is countably additive, then consider that 
    \[(0,1] = \bigcup_{n=1}^\infty (\frac{1}{2^n},\frac{1}{2^{n-1}}].\] We know that $\phi(0,1] = 2,$ but 
    \[\phi\left(\bigcup_{n=1}^\infty \right) = \sum_{n=1}^\infty \phi\left((\frac{1}{2^n},\frac{1}{2^{n-1}}]\right) = \sum_{n=1}^\infty \frac{1}{2^n} = 1.\] Since $1\neq 2,$ we arrive at a contradiction and so $\phi$ is not countably additive on $\cal R.$
\end{solution}

\newpage

\newpage
\section*{Problem 2}
\begin{problem}
    \begin{enumerate}
        \item If $A$ is open, then $A \in \mathcal{M}(m).$ If $B$ is closed, then $B \in \mathcal{M}(m).$
        \begin{solution}
            \begin{lemma}
                Suppose $U\subset \bbR^d$ is open. Then it can be uniquely expressed as a countable union of disjoint open intervals, where the endpoints of the intervals don't belong to $U.$ 
            \end{lemma} 
            \begin{proof}
            (We were shown this proof in Honors Analysis 1, so it might resemble the one found in Pugh's Real Analysis). 
                Let $x\in U.$ Define then 
                \[a_x = \inf\{a \mid (a,x) \subseteq U\}, \qquad b_x = \sup\{b \mid (x,b)\subseteq U\}, \qquad I_x = (a_x, b_x).\]  Suppose, for the sake of contradiction, that $b_x \in U.$ There exists some open $J_x \subseteq U$ such that $J \ni b_x$ (by the above construction), so then $A_x = J_x \cup I_x$ is an open interval containing $b_x$ and $x.$ Since $A_x$ is open, then $b_x \in B_r(b_x) \subseteq A_x,$ and thus there exists some $b' >b_x$ such that $b' \in A_x \subset U,$ and so $b_x \neq \sup\{b \mid (x,b) \subseteq U\}.$ Thus, neither $b_x$ nor $a_x$ are in $U.$ Let $y\in U.$ Either $I_x, I_y$ are disjoint, or by the above argument, since $I_x \cup I_y$ is an open interval containing $x,$ and $y,$ $I_x = I_y.$ Let $q_x \in I_x$ for each disjoint interval in $U,$ where $q_x \in \bbQ,$ then we can count the $I_x$ by the rationals within them. 

                As a side note, it is valid for $a_x = -\infty$ and $b_x = \infty.$
            \end{proof}
            By the above lemma, $A = \bigcup_{k=1}^\infty I_k,$ where each $I_k$ is a disjoint open interval. 
            We have shown in class that if $I_k \in \cal E,$ then $I_k \in \mathcal{M}(m).$ Since $\mathcal{M}(m)$ is a $\sigma-$algebra, then it is closed under countable unions, and thus $U \in \mathcal{M}(m).$ 

            We know that since $B$ is closed, then $B^c$ is open, and thus $B^c \in \mathcal{M}(m).$ Since it is a $\sigma-$algebra closed under complements, then $(B^c)^c = B \in \mathcal{M}(m).$
        \end{solution}
        \item Let $\epsilon>0$ and suppose $E \in \mathcal{M}(m).$ Then there exist closed $F$ and open $G$ such that $F \subset E \subset G$ and
        \[m(G\setminus E) < \epsilon, \qquad m(E \sm A)< \epsilon.\]
    \begin{solution}
        \begin{lemma}
            Suppose $A, B$ are measurable, then $A \setminus B$ is measurable. 
        \end{lemma}
        \begin{proof}
            It is easy to show that 
            \[A \setminus B = A \cap B^c.\] Thus, since $B$ is measurable, then $B^c$ is measurable. Since $\mathcal{M}(m)$ is closed under finite (indeed countable) intersections, and $A$ is measurable, then $A \cap B^c \in \mathcal{M}(m).$ 
        \end{proof}
    \begin{lemma}
        Suppose $G$ and $A$ are measurable, then 
        \[m(G) - m(A) = m(G\setminus A).\]
    \end{lemma}
    \begin{proof}
        We have that $G\setminus A$ is measurable by Lemma 1. We can write
        \[G = (G\setminus A) \cup A.\] Since the decomposition is disjoint and $m$ is countably additive, then 
        \[m(G) = m(G\setminus A) + m(A),\] and we are done.
    \end{proof}
        Since $E \in \mathcal{M}(m),$ then $m(E) = m^*(E).$ There exist $\{E_n\}$ countable open cover of elementary sets such that\footnote{Here we recall that if $E\in \cal E,$ then $m^*(E) = \Vol(E)$} 
        \[E \subset \bigcup_{n=1}^\infty E_n, \qquad \sum_{n=1}^\infty m^*(E_n) \leq m^*(E) + \epsilon.\] Define $G = \bigcup E_n,$ then since $m^* = m$ on elementary sets and $G$ is an open set and thus measurable by part (a), we have that
        \[m(G) = m(\bigcup_{n=1}^\infty E_n) \leq \sum_{n=1}^\infty m(E_n) = \sum_{n=1}^\infty m^*(E_n) \leq m^*(E) + \epsilon = m(E) + \epsilon.\] By the previous lemma, we have that subtracting the $m(E)$ from the right hand side,
        \[m(G \setminus A) = m(G) - m(A) < \epsilon.\]

        Since $E \in \mathcal{M}(m),$ then $E^c \in \mathcal{M}(m),$ and thus by the above, there exists some open $G\supset E^c$ such that $m(G \setminus E^c) < \epsilon.$ Because $G^c$ is closed and clearly $G^c \subset E,$ it suffices to show that $m(E \setminus G^c) < \epsilon.$ Clearly, $E\setminus G^c = G\setminus E^c$ and so $m(E\sm G^c) = m(G \sm E^c) < \epsilon.$ Call $F = G^c$ and you are done.
    \end{solution}
    \item 
    $\cal B$ are the Borel sets, where $\mathcal{B} = \sigma(\cal G),$ where $\cal G$ is the collection of open sets in $\bbR.$ Show that if $A \in \cal B,$ then $A\in \mathcal{M}(m).$
    \begin{solution}
        Suppose $A \in \cal B,$ then $A$ is made up of countable unions, intersection, and/or complements of open sets. But since open sets are measurable by part (a) and $\mathcal{M}(m)$ is closed under all of these operations, then $A \in \mathcal{M}(m).$ 
    \end{solution}
    \item  Suppose $E \in \mathcal{M}(m),$ then there exist $F,G \in \cal B$ such that $F \subset E \subset G.$
    \[m(G\sm B) = m(B\sm F) = 0\]
    \begin{solution}
        Let $E \in \mathcal{M}(m),$ let $n \in \bbN.$ By part (b), there exists an open $G_n\supset E$ such that \[m(G_n \sm E) < \frac{1}{n}.\] Since $G_n$ is open, then $G_n \in \cal B.$ Define $G: = \bigcap_{n=1}^\infty G_n.$ Since $\cal B$ is a $\sigma-$algebra, then $G \in \cal B.$ Moreover, we have that for any $n,$ since $G \in \mathcal{B} \subset \mathcal{M}(m)$
        \[G \sm E \subseteq G_n \sm E \implies m(G \sm E) \leq m(G_n \sm E)< \frac{1}{n},\] and so $m(G \sm E) = 0.$ 

        We have that $E^c \in \mathcal{M}(m),$ and so there exists some $G\in \cal B$ containing $E^c$ such that $m(G\sm E^c) = 0.$ But then $G\sm E^c = E\sm G^c.$ $G^c \in \cal B,$ so call $F = G^c,$ and we get that $m(E \sm F) = 0.$
    \end{solution}
    \item Let $\mathcal{N}(m) = \{A \in \mathcal{M}(m) \mid m^*(A) = 0\}.$ Then $\mathcal{N}(m)$ is a $\sigma-$algebra.
    \begin{solution}
        Since $m^*(\emptyset) = 0,$ then $\emptyset \in \mathcal{N}(m).$

        Suppose $A_1, \dots \in \mathcal{N}(m).$ Then since $m^*$ is countably subadditive, 
        \[m(\bigcup_{n=1}^\infty A_n) \leq \sum_{n=1}^\infty m(A_n) = 0,\] and so $\bigcup_{n=1}^\infty A_n \in \mathcal{N}(m).$

        Suppose $A,B \in \mathcal{N}(m),$ then $A\sm B \subset A$ and thus by monotonicity of $m^*,$ we have that 
        \[m^*(A \sm B) \leq m^*(A) = 0,\] and so $m^*(A\sm B) =  0$ and thus $A\sm B \in \mathcal{N}(m)$
    \end{solution}
    \end{enumerate}

\end{problem}


\newpage
\section*{Problem 3}
\begin{problem}
    Let $x \in \mathbb{R}^n$, and $A \subseteq \mathbb{R}^n$. Prove that
\[
\mathbf{m}^*(x + A) = \mathbf{m}^*(A),
\]
where $x + A = \{x + y \mid y \in A\}$. Prove that, if $A \in \mathcal{M}(\mathbf{m})$, then $x + A \in \mathcal{M}(\mathbf{m})$.
\end{problem}
\begin{solution}
    \begin{lemma}
    Let $X$ be a metric space and $A \subset X$ be open. Let $x\in X.$ Then $x + A$ is open.
    \end{lemma}
    \begin{proof}
        Let $z \in x + A.$ Then $z = x + a$ for some $a \in A.$ Since $a \in A$ and $A$ is open, there exists some $r>0$ such that $B_r(a)\subset A.$ We claim that $x + B_r(a) \subset x + A.$ Indeed, it is not hard to see that $x + B_r(a) = B_r(a + x) = B_r(z),$ so then it suffices to show that $x + B_r(a) \subset x + A.$ Let $z' \in x + B_r(a),$ then $z' = x + a'$ for some $a' \in B_r(a) \subset A,$ and so $z' \in x + A.$ Thus, we have found some $r>0$ such that $B_r(z)\subset x + A,$ and thus $x + A$ is open.
    \end{proof}
    \begin{lemma}
        Let $\{A_n\}_{n =1}^\infty$ be a countable open cover of $A \subset X,$ where $X$ is a topological space. If $x \in X,$ then $\{x + A_n\}$ is a countable open cover of $x + A.$
    \end{lemma}
    \begin{proof}
        Let $z \in x + A.$ Then $z = x + a$ for some $a\in A$ and thus there exists some $k \in \bbN$ such that $A_k \in \{A_n\}$ and $a \in A_k.$ Thus, $z\in x + A_k.$ By Lemma 4, we know that $x + A_k$ is open, and thus $\{x + A_n\}_{n =1}^\infty$ is a countable open cover of $x + A.$
    \end{proof}
    Note that the converse holds as well by identical logic.
    \begin{lemma}
     Suppose $E \in \cal E,$ and $E \subset \bbR^n.$ Then for any $x\in \bbR^n,$ $x + E \in \cal E.$
    \end{lemma}
    \begin{proof}
        It suffices to show that $x + E$ is the union of finite intervals. Since $E \in \cal E,$ then $E = \bigcup_{k=1}^n I_k.$ Thus, we claim that
        \[x + E = x + \bigcup_{k=1}^N I_k = \bigcup_{k=1}^N(x + I_k).\] The only equality that needs explaining is the second one. To see it, let $z \in x + \bigcup_{k=1}^N I_k,$ then there exists some $j \in [N]$ and some $x_j \in I_j$ such that $z = x + x_j.$ Thus, $z \in x + I_j,$ and so $x \in \bigcup_{k=1}^N (x + I_k).$ For the other inclusion, let $z\in \bigcup_{k=1}^N (x + I_k),$ then there exists some $j \in [N]$ such that $z\in x + I_j.$ Thus, $z = x + x_j$ for some $x_j \in I_j.$ Since $x_j \in \bigcup I_k,$ then $z \in x + \bigcup I_k.$
        
        We claim that $x + I_k$ is an interval for any $k \in [n].$ It should be obvious that if $I_k$ is the interval made of points $\textbf{x} = (x_1, \dots, x_n) \in \bbR^n$ such that $a_i \leq x_i \leq b_i$ (where the $\leq$ can be replaced with $<$), then $x + I_k$ is the set of points $\textbf{x}' = (x_1', \dots, x_n')$ such that $a_i  + x \leq x_i' \leq b_i + x,$ and thus $x + I_k$ is an interval. 
    \end{proof}
    
    \begin{lemma}
        Suppose $E \in \cal E,$ and $E \subset \bbR^n.$ Then for any $x\in \bbR^n,$ we have that $\Vol(E) = \Vol(x + E).$ 
    \end{lemma}
    \begin{proof}
        First, we know that $\Vol(x + E)$ is well defined since by Lemma 6, $x + E \in \cal E.$ By work in class, we know that we can decompose $E$ into $E = \bigsqcup_{k=1}^N I_k,$ where each the $I_k$ are disjoint intervals. By work done in the Lemma 6, we know that $x + E = \bigsqcup_{k=1}^N (x + I_k),$ where again, $(x + I_k)$ are disjoint nonempty intervals made of points such that $\textbf{x} \in (x + I_k)$ if and only if 
        \[a_i  + x \leq x_i \leq b_i + x,\] where $(a_i, b_i)$ are the endpoints of the edges of each $I_i.$ Thus, we have that 
        \begin{align*}
            \Vol(E) &= \Vol(\bigsqcup_{k=1}^N I_k)\\
            &= \sum_{k=1}^N \Vol(I_k)\\
            &= \sum_{k=1}^N \prod_{i=1}^n(b_i^k - a_i^k)\\
            &= \sum_{k=1}^N \prod_{i=1}^n(b_i^k - a_i^k + x - x)\\
            &= \sum_{k=1}^N \prod_{i=1}^n((b_i^k + x) - (a_i^k + x))\\
            &= \sum_{k=1}^N (x + I_k)\\
            &= \Vol(x + E)
        \end{align*}
    \end{proof}
Finally, we are ready to prove the statement. Let $\epsilon>0.$ Let $\{A_n\}_{n=1}^\infty$ be a countable open cover of $A$ such that each $A_n \in \cal E$ and 
\[\sum_{n=1}^\infty \Vol(A_n) \leq m^*(A) + \epsilon.\] By Lemma 7, we have that 
\[\sum_{n=1}^\infty \Vol(A_n) = \sum_{n=1}^\infty\Vol(x + A_n).\] By Lemma 5, we know that $\{x + A_n\}_{n=1}^\infty$ is a countably open cover of $x + A.$ By Lemma 6, $x + A_n \in \cal E$ for all $n.$ Thus, 
\[m^*(x + A) \leq\sum_{n=1}^\infty\Vol(x + A_n) =\sum_{n=1}^\infty \Vol(A_n) \leq m^*(A) + \epsilon,\] and so because this holds for any $\epsilon>0,$ we have that.
\begin{align}
    m^*(x + A) \leq m^*(A)
\end{align}

Let $\{x + A_n\}_{n=1}^\infty$ be a countable open cover of $x + A$ such that each $x + A_n \in \cal E$ and 
\[\sum_{n=1}^\infty \Vol(x + A_n) \leq m^*(x + A) + \epsilon\]
We can apply all our lemmas to $A_n$ since $A_n = (-x) + x + A_n.$ Thus, $\{A_n\}_{n=1}^\infty$ is a countable cover of $A$ with $A_n \in \cal E$ for all $n.$  Thus, 
\[m^*(A) \leq \sum_{n=1}^\infty \Vol(A_n) \leq \sum_{n=1}^\infty \Vol(x + A_n) \leq m^*(x + A) + \epsilon.\] Because this holds for all $\epsilon>0,$ we have that 
\begin{align}
    m^*(A) \leq m^*(x + a)
\end{align}
Putting together (1) and (2), we see that $m^*(A) = m^*(x + A).$

\begin{lemma}
    Suppose $A, B \subset \bbR^d.$ Then for any $x\in \bbR^d,$ 
    \[m^*(A \triangle B) = m^*\big[(x + A)\triangle (x + B)\big]\]
\end{lemma}
\begin{proof}
    It should be easy to see that under some set manipulation, we get that 
    \[(x + A)\triangle (x + B) = x + A\triangle B.\] So then by the translation invariance we proved above, 
    \[m^*\big[(x + A)\triangle (x + B)\big] = m^*(x + A\triangle B) = m^*(A\triangle B).\]
\end{proof}

Suppose $A \in \mathcal{M}(m),$ then there exists some $(A_n)\in \mathcal{M}_F(m)$ such that $A = \bigcup_{n=1}^\infty A_n$. We claim that (1) $x + A = \bigcup_{n=1}^\infty (x + A_n)$ where (2) $x + A_n \in \mathcal{M}_F(m).$ To see the second claim, consider that since $A_n \in \mathcal{M}_F(m),$ then for each $n,$ there exists a sequence $(E^{(n)}_k) \in \cal E$ such that $E^{(n)}_k \to A_n$ (in exterior measure) as $k\to \infty.$ Thus, for every $n,$ we have by Lemma 8 that for large $k,$ 
\[m^*(E^{(n)}_k \triangle A_n) = m^*\big[(x + E^{(n)}_k) \triangle (x + A_n)\big] < \epsilon\]

By lemma 6, we know that $x + E_k^{(n)} \in \cal E,$ so then $x + A_n \in \mathcal{M}_F(m)$ for every $n.$ It remains to show that 
\[x + A = \bigcup_{n=1}^\infty (x + A_n),\] but the logic of this proof is identical to that of Lemma 5's.
\end{solution}

\newpage
\section*{Problem 4}
\begin{problem}
    Consider the real line in $\mathbb{R}^2$, 
\[
X = \{(x,0) \mid x \in \mathbb{R}\}.
\]
What is $m^*(X)$ (where $m^*$ is the Lebesgue outer measure defined on $\mathbb{R}^2$)? Show that $X \in \mathcal{M}(m)$.
\end{problem}
\begin{solution}
Define 
\[X_n := \{(x,0) \mid x \in (-n, n)\}\]
Evidently, $X = \bigcup_{n=1}^\infty X_n.$ Thus, we know by Theorem 11.8 in Rudin that
\begin{align}
m^*(X) \leq \sum_{n=1}^\infty m^*(X_n)    
\end{align}
Let $\epsilon>0.$ For each $n,$ define 
\[E_\epsilon^n = \{(x,y) \in \bbR^2 \mid x \in (-n, n), y \in (-\frac{\epsilon}{2^n \cdot 4n}, \frac{\epsilon}{2^n \cdot 4n})\}.\] Clearly, we know that 
\begin{enumerate}
    \item $X_n \subset E_\epsilon^n$ for any $n,$ for any $\epsilon>0.$ Thus, $E_\epsilon^n$ is an open cover of $X_n,$ and so since $E_\epsilon^n$ is an interval,
    \begin{align}
        m^*(X_n) \leq \Vol(E_\epsilon^n), \qquad \forall \;n \in \bbN
    \end{align}
    \item $E_\epsilon^n$ is an interval, and thus 
    \begin{align}\Vol(E_\epsilon^n) = (2n)(\frac{\epsilon}{2^n \cdot 2n}) = \frac{\epsilon}{2^n}    
    \end{align}
By (3), (4), and (5):
\[m^*(X) \leq \sum_{n=1}^\infty m^*(X_n)\leq \sum_{n=1}^\infty \Vol(E_\epsilon^n) \leq \sum_{n=1}^\infty \frac{\epsilon}{2^n} = \epsilon.\] Because this holds for all $\epsilon>0$, $m^*(X) = 0.$ 

\end{enumerate}
Since $X_n \in \cal E$ for all $n$ because they are all intervals, then $X_n \in \mathcal{M}(m)$ for all $n.$ Thus, since $\mathcal{M}(m)$ is closed under countable unions and $X = \bigcup_{n=1}^\infty X_n,$ then $X \in \mathcal{M}(m).$
\end{solution}

\newpage
\section*{Problem 5}
\begin{problem}
    Let $A, B \subseteq \mathbb{R}^n$, and suppose that
\[
d := \inf_{x \in A, y \in B} |x - y| > 0.
\]
Prove that
\[
m^*(A \cup B) = m^*(A) + m^*(B).
\]
\end{problem}
\begin{solution}
    From Theorem 11.8 in Rudin, we have that 
    \[m^*(A \cup B) \leq m^*(A) + m^*(B),\] so it suffices to show the other inequality.

    Let $\epsilon>0.$ There exists $\{X_n\}_{n=1}^\infty$ countable open cover of $A\cup B$ such that
    \begin{align}
    \sum_{n=1}^\infty \Vol(X_n) \leq m^*(A \cup B) + \epsilon.    
    \end{align}
    Each $X_n \in \cal E,$ so we can subdivide the $X_n$ into $\tilde{X}_n$ so that each $\title{X}_n$ has diameter less than $\frac{d}{2}$ and each $\tilde{X}_n$ intersects at least one $A$ or $B$ (we can throw out the ones that don't without any consequence). Thus, there exists some index set $I$ such that if $i \in I,$ then $X_i \cap B = \emptyset.$ Since $\text{diam}(X_n) <\frac{d}{2},$ then  there exists if $j \in I^c,$ then $X_j \cap A = \emptyset.$ Thus, $\{X_i\}_{i \in I}$ is a countable open cover of $A$ disjoint completely from $\{X_j\}_{j \in I^c},$ which is a countable open cover of $B.$ Thus by (6), 
    \[m^*(A) + m^*(B) \leq \sum_{i \in I} \Vol(X_i) + \sum_{j \in I^c} \Vol(X_i) \leq \sum_{n=1}^\infty \Vol(X_n) \leq m^*(A \cup B) + \epsilon.\] Because this holds for any $\epsilon>0,$ we are done.
\end{solution}

\newpage
\section*{Problem 6}
\begin{problem}
    Let $f: \bbR \to \bbR.$ We define the \textbf{graph} of $f$ to be the set 
    \[G(f)= \{(x,y) \in \bbR^2 \mid f(x) = y\}\] Show that if $f$ is continuous, then $m\left(G(f)\right) = 0.$
\end{problem}
\begin{solution}
    First, we will show that $G(f)$ is measurable. Let $\epsilon>0.$ If we define 
    \[A_n = \{(x, f(x))\in \bbR^2 \mid x \in [-n, n]\},\] then 
    \[G(f) = \bigcup_{n=1}^\infty A_n.\] We want to show that $A_n \in \mathcal{M}_F(m)$ for each $n.$ Thus, it suffices to find a sequence $E^{(n)}_k \in \cal E$ such that $E^{(n)}_k \to A_n$ (in outer measure) as $k\to \infty.$ 
    Consider that since $f$ is continuous and $[-n, n]\subset \bbR$ is compact, then $f$ is uniformly continuous on $[-n,n].$  There exists some $\delta_n >0$ such that if $x,y \in [-n, n]$ and $|x-y| < \delta_\epsilon,$ then $|f(x) - f(y)|< \frac{\epsilon}{2^{{n+1}}}.$ 
    
    
    Partition $[-n, n]$ into a partition $P,$ $-n = t_0< \dots< t_K = n$ such that $\|P\| < \min\{1, \delta_n\}.$ Thus if $x,y \in [t_i, t_{i+1}],$ then $|f(x) - f(y)|< \frac{\epsilon}{2^{n+1}}.$ Call $E^{(n)}_i$ the boxes of height $2 \cdot \frac{\epsilon}{2^{n+1}} = \frac{\epsilon}{2^n}$ for each subinterval $[t_i, t_{i+1}]$ of $P.$ Notice that each $E_{i}^{(n)}$ is an interval and thus 
    \[m^*(E_{i}^{(n)}) = \Vol(E_{i}^{(n)}) = (t_{i} - t_{i-1}) \cdot \frac{\epsilon}{2^{n+1}}\]
    
    Call $E^{(n)}_K = \bigcup_{i \in [K]} E_i^{(n)}$ the union of all such boxes. Note that again, $E^{(n)}_K \in \cal E$ since it is the finite union of intervals.
    
    Since $A^n \subset E_K^{(n)}$ by the construction of $E_K^{(n)},$ we have that $E_K^{(n)} \triangle A_n = A_n.$ Therefore, using the finite additivity of $m^*,$ we see that by telescoping the sum:
    \begin{align*}
      m^*(E_K^{(n)} \triangle A_n) &= m^*(A_n) \leq m^*(E_K^{(n)})\\ 
      &= m^*\left(\bigcup_{i\in [K]} E_i^{(n)}\right)\\
      &= \sum_{i =1}^{K}m^*(E_i^{(n)})\\
      &= \sum_{i=1}^K (t_{i} - t_{i-1}) \cdot \frac{\epsilon}{2^{n+1}}\\ 
      &= (2n)\cdot \frac{\epsilon}{2^{n+1}}\\
      &= \frac{\epsilon}{2^n}\\
      &< \epsilon
    \end{align*}
    Thus, we see that $A_n \in \mathcal{M}_F(m),$ and since $G(f) = \bigcup_{n=1}^\infty A_n,$ then $G(f)$ is measurable. Using the countable additivity of the Lebesgue measure, and the above calculation:
    \[m(G(f)) = \sum_{n=1}^\infty m(A_n) < \sum_{n=1}^\infty \frac{\epsilon}{2^n} = \epsilon.\] Because this holds for any $\epsilon>0,$ we see that $m(G(f)) = 0.$
\end{solution}


\end{document}