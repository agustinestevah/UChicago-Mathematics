\documentclass[11pt]{article}

% NOTE: Add in the relevant information to the commands below; or, if you'll be using the same information frequently, add these commands at the top of paolo-pset.tex file. 
\newcommand{\name}{Agustín Esteva}
\newcommand{\email}{aesteva@uchicago.edu}
\newcommand{\classnum}{208}
\newcommand{\subject}{Accelerated Analysis in $\bbR^n$ III}
\newcommand{\instructors}{Don Stull}
\newcommand{\assignment}{Problem Set 2}
\newcommand{\semester}{Spring 2025}
\newcommand{\duedate}{04-09-2025}
\newcommand{\bA}{\mathbf{A}}
\newcommand{\bB}{\mathbf{B}}
\newcommand{\bC}{\mathbf{C}}
\newcommand{\bD}{\mathbf{D}}
\newcommand{\bE}{\mathbf{E}}
\newcommand{\bF}{\mathbf{F}}
\newcommand{\bG}{\mathbf{G}}
\newcommand{\bH}{\mathbf{H}}
\newcommand{\bI}{\mathbf{I}}
\newcommand{\bJ}{\mathbf{J}}
\newcommand{\bK}{\mathbf{K}}
\newcommand{\bL}{\mathbf{L}}
\newcommand{\bM}{\mathbf{M}}
\newcommand{\bN}{\mathbf{N}}
\newcommand{\bO}{\mathbf{O}}
\newcommand{\bP}{\mathbf{P}}
\newcommand{\bQ}{\mathbf{Q}}
\newcommand{\bR}{\mathbf{R}}
\newcommand{\bS}{\mathbf{S}}
\newcommand{\bT}{\mathbf{T}}
\newcommand{\bU}{\mathbf{U}}
\newcommand{\bV}{\mathbf{V}}
\newcommand{\bW}{\mathbf{W}}
\newcommand{\bX}{\mathbf{X}}
\newcommand{\bY}{\mathbf{Y}}
\newcommand{\bZ}{\mathbf{Z}}
\newcommand{\Vol}{\text{Vol}}

%% blackboard bold math capitals
\newcommand{\bbA}{\mathbb{A}}
\newcommand{\bbB}{\mathbb{B}}
\newcommand{\bbC}{\mathbb{C}}
\newcommand{\bbD}{\mathbb{D}}
\newcommand{\bbE}{\mathbb{E}}
\newcommand{\bbF}{\mathbb{F}}
\newcommand{\bbG}{\mathbb{G}}
\newcommand{\bbH}{\mathbb{H}}
\newcommand{\bbI}{\mathbb{I}}
\newcommand{\bbJ}{\mathbb{J}}
\newcommand{\bbK}{\mathbb{K}}
\newcommand{\bbL}{\mathbb{L}}
\newcommand{\bbM}{\mathbb{M}}
\newcommand{\bbN}{\mathbb{N}}
\newcommand{\bbO}{\mathbb{O}}
\newcommand{\bbP}{\mathbb{P}}
\newcommand{\bbQ}{\mathbb{Q}}
\newcommand{\bbR}{\mathbb{R}}
\newcommand{\bbS}{\mathbb{S}}
\newcommand{\bbT}{\mathbb{T}}
\newcommand{\bbU}{\mathbb{U}}
\newcommand{\bbV}{\mathbb{V}}
\newcommand{\bbW}{\mathbb{W}}
\newcommand{\bbX}{\mathbb{X}}
\newcommand{\bbY}{\mathbb{Y}}
\newcommand{\bbZ}{\mathbb{Z}}

%% script math capitals
\newcommand{\sA}{\mathscr{A}}
\newcommand{\sB}{\mathscr{B}}
\newcommand{\sC}{\mathscr{C}}
\newcommand{\sD}{\mathscr{D}}
\newcommand{\sE}{\mathscr{E}}
\newcommand{\sF}{\mathscr{F}}
\newcommand{\sG}{\mathscr{G}}
\newcommand{\sH}{\mathscr{H}}
\newcommand{\sI}{\mathscr{I}}
\newcommand{\sJ}{\mathscr{J}}
\newcommand{\sK}{\mathscr{K}}
\newcommand{\sL}{\mathscr{L}}
\newcommand{\sM}{\mathscr{M}}
\newcommand{\sN}{\mathscr{N}}
\newcommand{\sO}{\mathscr{O}}
\newcommand{\sP}{\mathscr{P}}
\newcommand{\sQ}{\mathscr{Q}}
\newcommand{\sR}{\mathscr{R}}
\newcommand{\sS}{\mathscr{S}}
\newcommand{\sT}{\mathscr{T}}
\newcommand{\sU}{\mathscr{U}}
\newcommand{\sV}{\mathscr{V}}
\newcommand{\sW}{\mathscr{W}}
\newcommand{\sX}{\mathscr{X}}
\newcommand{\sY}{\mathscr{Y}}
\newcommand{\sZ}{\mathscr{Z}}


\renewcommand{\emptyset}{\O}

\newcommand{\abs}[1]{\lvert #1 \rvert}
\newcommand{\norm}[1]{\lVert #1 \rVert}
\newcommand{\sm}{\setminus}


\newcommand{\sarr}{\rightarrow}
\newcommand{\arr}{\longrightarrow}

% NOTE: Defining collaborators is optional; to not list collaborators, comment out the line below.
%\newcommand{\collaborators}{Alyssa P. Hacker (\texttt{aphacker}), Ben Bitdiddle (\texttt{bitdiddle})}

\input{paolo-pset.tex}

% NOTE: To compile a version of this pset without problems, solutions, or reflections, uncomment the relevant line below.

%\excludeversion{problem}
%\excludeversion{solution}
%\excludeversion{reflection}

\begin{document}	
	
	% Use the \psetheader command at the beginning of a pset. 
	\psetheader

\section*{Problem 1}
\begin{problem}
    If $f\geq 0$ and $\int_E f dm = 0,$ prove that $f(x) = 0$ almost everywhere on $E.$
\end{problem}
\begin{solution}
Suppose not.  Define 
    \[X_n : = \{x \in E \mid f(x) \geq \frac{1}{n}\}.\] There exists some $n$ such that $m(X_n) >0.$  Since $\frac{1}{n} \leq f(x)$ for all $x \in X_n$ and $f \in \mathcal{L}(m)$ by the remarks in Problem 7, we get that by a few more results in Problem 7:
\[ 0 = \int_E f \geq \int_{X_n}f \geq \frac{1}{n}m(X_n) >0,\] which is a contradiction.
\end{solution}

\newpage
\section*{Problem 2}
\begin{problem}
If $\int_A f dm  = 0$ for every $A \in \cal M$ such that $A \subseteq E$ where $E\in \cal M,$ then $f(x) = 0$ almost everywhere on $E.$    
\end{problem}
\begin{solution}
Define \[X_n^+:= \{x \in E \mid f(x) \geq \frac{1}{n}\}\] as in the previous problem. Since $f$ is measurable, then by definition, $X_n^+\subseteq E$ is measurable. Then by Problem 7
\[0 = \int_{X_n^+} f \geq \frac{1}{n}m(X_n^+),\] and so $m(X_n^+) = 0$ for each $n.$ 
We can write $X^+ = \displaystyle\bigcup_\bbN X_n^+.$ Then 
\[m(X^+) \leq \sum_{n=1}^\infty m(X_n) = 0,\] and so $m(X^+) = 0.$ Define
\[X_n^- ;= \{x \in E \mid f(x) < \frac{1}{n}\}.\] Again, $m(X_n^-) = 0$ for the same reason, and so if $X^- = \bigcup X_n^-,$ then
\[m(X^-) \leq \sum_{n=1}^\infty m(X_n^-) = 0.\] Thus, 
\[\{x \in E \mid f(x) \neq 0\} = X^+ \sqcup X^-,\] then 
\[m(\{x \in E \mid f(x) \neq 0\}) = 0\]
\end{solution}

\newpage
\section*{Problem 3}
\begin{problem}
    If $(f_n)$ is a sequence of measurable functions, prove that the set of points $x$ at which $(f_n(x))$ converges is measurable.
\end{problem}
\begin{solution}
    Recall that since $f_n$ is measurable for each $n,$ then 
    $\{x \mid f_n(x) < \frac{1}{k}\}$ is measurable for each $n$ and for each $k.$ 

    We claim that if $f_n(x) \to L_x,$ then for any $\epsilon>0,$ for any $n>0,$
    \[\{x \mid |f_n(x) - L_x| < \epsilon\}\] is measurable. That is, the set of $x$ such that $f_n(x)$ is close to a limit is measurable. We can check that this is measurable because 
    \[\{x \mid |f_n(x) - L_x| < \epsilon\} = \{x \mid f_n(x) > L_x - \epsilon\} \cap \{x \mid f_n(x) < L_x + \epsilon\},\] where both are measurable since $f_n$ is measurable.

    Note that if $f_n(x)$ is not Cauchy, then it does not converge. Thus, we must necessary have the condition that for all $\epsilon>0,$ $|f_n(x) - f_m(x)| < \epsilon.$

    By all the reasons stated above, we can write the set of points for which $x$ converges as 
    \[\bigcap_{k=1}^\infty \bigcup_{j = 1}^\infty\bigcap_{n=\geq k} \bigcap_{m \geq k}\{x \in E \mid |f_n(x) - f_m(x)| < \frac{1}{k}\},\] which is measurable since $\cal M$ is closed under countable intersections.
\end{solution}

\newpage
\section*{Problem 4}
\begin{problem}
    Define $g: [0,1] \to \bbR$ and $f_n: [0, 1] \to \bbR$ such that
    \[g(x) := \begin{cases}
        0, \qquad x \in [0, \frac{1}{2}]\\
        1, \qquad x \in (\frac{1}{2}, 1]
    \end{cases},\]
    \[f_{n}(x) = 
    \begin{cases}
        g(x), \quad \:\;\:\qquad n = 2k\\
        g(1-x), \qquad n = 2k + 1
    \end{cases}\]
    Show that 
    \[\liminf_{n\to \infty}f_n(x) = 0\] but 
    \[\int_0^1 f_n(x)dx = \frac{1}{2}\]
\end{problem}
\begin{solution}
    We have that 
    \[\lim_{n\to \infty}f_n(x) = \sup_{n}\inf_{k\geq n}f_n(x).\] It is clear that for any $n,$ for any $x \in [0,\frac{1}{2}]$ 
    \[\inf_{k \geq n} f_n(x) = f_{2k}(x) = g(x) = 0\] For any $x\in (\frac{1}{2}, 1],$ we have that 
    \[\inf_{k\geq n}f_n(x) = f_{2k+1} = g(1-x) = 0\]
    
    and so 
    \[\sup_{n\to \infty } 0 = 0.\] Thus, \[\liminf_{n\to \infty}f_n(x) = 0\] for any $x\in [0,1].$ 

    Meanwhile, 
    \begin{align*}
    \int_0^1 f_n(x)dx &= \begin{cases}
        \int_0^1 g(x)dx, \quad n = 2k\\
        \int_0^1 g(1-x), \quad n = 2k+1
    \end{cases}\\ &=
    \begin{cases}
        \int_0^\frac{1}{2} g(x)dx + \int_\frac{1}{2}^1 g(x)dx\\
        \int_0^\frac{1}{2} g(1-x)dx + \int_\frac{1}{2}^1{2} g(1-x)dx
    \end{cases}\\ &= \begin{cases}
        \int_\frac{1}{2}^1 dx\\
        \int_0^\frac{1}{2} dx
    \end{cases}\\ &=m([0,\frac{1}{2}])  \\
    &= \frac{1}{2}
    \end{align*}
Comparing with $(77),$ we indeed see that if $\liminf_{n\to \infty}f_n = f = 0$ as shown above, then 
\[\int_0^1 f(x) dx = 0 \leq \frac{1}{2} = \liminf_{n\to \infty}\int_0^1 f_n(x) dx,\] as per Fatou's lemma.
\end{solution}



\newpage
\section*{Problem 5}
\begin{problem}
    Let 
    \[f_n(x) = \begin{cases}
        \frac{1}{n}, \quad |x| \leq n\\
        0, \quad \;|x| >n.
    \end{cases}\]
    Then $f_n(x) \to 0$ uniformly on $\bbR,$ but 
    \[\int_\bbR f_n dx = 2.\]
\end{problem}
\begin{solution}
    Let $\epsilon>0$ and $x\in \bbR.$ Let $N \in \bbN$ such that $\frac{1}{N}< \epsilon.$ Let $n \geq N.$ If $|x| > n,$ then 
    \[|f_n(x) - 0| = |0| < \epsilon.\]
    If $|x|\leq n,$ then 
    \[|f_n(x) - 0| = |\frac{1}{n}| < \epsilon.\] Thus, 
    $f_n{\rightrightarrows} \;0.$ However, 
    for any $n \in \bbN,$ 
    \[F_n = \int_{-n}^n f_n(x) = 2\int_0^n \frac{1}{n}dx = 2\frac{1}{n}m([0,n]) = 2\] and so 
    \[\int_\bbR f_n dx = \lim_{n\to \infty}F_n = 2.\]

    The dominated convergence theorem fails because for any function $g$ that dominates $f_n,$ its integral is not finite over $\bbR.$ 
\end{solution}



\newpage
\section*{Problem 6}
\begin{problem}
    Suppose $f: \bbR^n \to \bbR.$ There exists a sequence $\varphi_n$ of simple functions such that $\varphi_n \to f$ pointwise. Moreover, 
    \begin{enumerate}
        \item If $f \geq 0,$ then  $\varphi_n \uparrow f$
        \item If $f $ is measurable, then $\varphi_n$ are all measurable.
    \end{enumerate}
\end{problem}
\begin{solution}
    Let $f\geq 0.$ Define
    \[E_{i}^{(n)} := \{x \mid f(x) \in [\frac{i-1}{2^n}, \frac{i}{2^n})\} = f^{-1}\big([\frac{i-1}{2^n}, \frac{i}{2^n})\big), \quad F_n := \{x\mid f(x) \geq n\} = f^{-1}\big([n, \infty)\big).\] For each $n \in \bbN$ and for each $i \in [n2^n],$ define 
    \[\varphi_n = \sum_{i=1}^{n2^n} \frac{i-1}{2^n}\chi_{E_i^{(n)}} + n\chi_{F_n}.\] 

We claim that for any fixed $n \in \bbN,$ the $E_{i}^{(n)}, F_n$ are mutually disjoint  and
    \[\left(\bigsqcup_{i=1}^{n2^n} E_{i}^{(n)} \right)\cup F_n = \bbR^n.\] To see this, fix $n \in \bbN.$ Clearly, since $f$ is a function, the sets are disjoint. For any $x\in \bbR^n,$ we have that $f(x) > 0.$ Thus, either $f(x) \geq n,$ in which case $x\in F_n,$ or $f(x) \leq n,$ in which case there is some $i$ for which $f(x)\in [\frac{i-1}{2^n}, \frac{i}{2^n})$ and thus $x\in E_{i}^{(n)}.$  
    
    Let $\epsilon>0,$ let $x \in X.$ There is some $N_1 \in \bbN$ such that $f(x) < N_1.$ Thus, for all $n \geq N_1,$ we have that there is some $j$ for which $x \in E_j^{(n)}.$ Thus, $\chi_{E_j^{(n)}} = 1$ and by the mutual disjoint-ness of all the sets, 
    \[\chi_{E_i^{(n)}} = \chi_{F_n} = 0\]
    Let $N_2 \in \bbN$ such that $\frac{1}{2^{N_2}} < \frac{\epsilon}{2}.$  Thus, if $n\geq N_2$ and $x\in E_{i   }^{(n)},$ then $f(x) \in [\frac{i-1}{2^n}, \frac{i}{2^n})$ and \[|f(x) - \frac{i-1}{2^n}| \leq \frac{1}{2^n} \leq \frac{1}{2^{N_2}} < \frac{\epsilon}{2}\]
    
    Then for any $n \geq N_x = \max\{N_1, N_2\},$ we have that
    \begin{align*}
        |\varphi_n(x) - f(x)| &= \left|\sum_{i=1}^{n2^n} \frac{i-1}{2^n}\chi_{E_i^{(n)}}(x) + n\chi_{F_n}(x) - f(x)\right|\\
        &= \left|\frac{i-1}{2^n} - f(x)\right|\\
        &\leq \left|\frac{i-1}{2^n}  - \frac{i}{2^n} \right| + \left|\frac{i}{2^n} - f(x) \right|\\
        &< \epsilon
    \end{align*}
    Thus, $\varphi_n \to f$ pointwise. 

    Since $f \geq 0,$ we can show that $\varphi_n \uparrow f.$ It suffices to show that 
    \begin{enumerate}
        \item $0 \leq \varphi_n \leq f$ for all $n;$
        \item $\varphi_{n-1} \leq \varphi_{n}$ for all $n.$
    \end{enumerate}
    To prove (a), first fix $n$ and let $x\in X.$ We have two cases:
    \begin{itemize}
        \item If $f(x) \geq n,$ then 
        \[\varphi_n(x) = \sum_{i=1}^{n2^n} \frac{i-1}{2^n}\chi_{E_i^{(n)}}(x) + n\chi_{F_n}(x) = n \leq f(x)\]
        \item If $f(x) < n,$ then there is some $j$ such that $\chi_{E^{(n)}_j} = 1$ and is $0$ if $i \neq j.$ Thus, 
        \[\varphi_n(x) = \sum_{i=1}^{n2^n} \frac{i-1}{2^n}\chi_{E_i^{(n)}}(x) + n\chi_{F_n}(x) = \frac{j-1}{2^n} \leq f(x)\]
    \end{itemize}

    To prove (b), let $n \in \bbN$ and $x\in X.$ We have a couple of cases:
    \begin{itemize}
        \item If $f(x) \geq n >n-1,$ then 
        \[\varphi_n(x) = \chi_{F_n} = n \geq n-1 =\chi_{F_{n-1}}(x)= \varphi_{n-1}(x).\]
        \item If $n > f(x) \geq n-1,$ then $f(x) \in [\frac{(j-1)}{2^n}, \frac{j}{2^n}),$ for some $j$ such that $\frac{(j-1)}{2^n} \geq n-1$ and so
        \[\varphi_n(x) = \frac{(j-1)}{2^n}\chi_{E_{(n-1)2^n}^{(n)}}(x) \geq n-1= F_{n-1}(x) = \varphi_{n-1}(x)\]
        \item If $f(x) < n-1 < n,$ then we claim that for any $i \in (n-1)2^{n-1},$
        \[E_{i}^{(n-1)}= E_{2i -1}^{(n)}\sqcup E_{2i}^{(n)}.\] To see this, 
        \begin{align*}
        E_{i}^{(n-1)} &=[\frac{i-1}{2^{n-1}}, \frac{i}{n^{n-1}})\\ &= [\frac{2(i -1)}{2^{n}}, \frac{2i}{2^n})\\ &=  [\frac{(2i -1) -1}{2^{n}}, \frac{2i}{2^n})\\ &= [\frac{(2i-1) -1}{2^{n}}, \frac{2i-1}{2^{n}}) \sqcup [\frac{2i-1}{2^{n}}, \frac{2i}{2^n})\\ &= E_{2i -1}^{(n)}\sqcup E_{2i}^{(n)}    
        \end{align*}
        Thus, we can see that
        \[\chi_{E_{i}^{(n-1)}}  = \chi_{E_{2i -1}^{(n)}\sqcup E_{2i}^{(n)}} = \chi_{E_{2i -1}^{(n)}} + \chi_{ E_{2i}^{(n)}}\]
        Thus, suppose $x\in E_i^{n-1},$ then either 
        \begin{itemize}
            \item $x\in E_{2i -1}^{(n)},$ in which case:
            \begin{align*}
                \varphi_{n-1}(x) &= \frac{i-1}{2^{n-1}}\chi_{E_{i}^{(n-1)}}(x)\\
                &= \frac{2(i-1)}{2^{n}}(\chi_{E_{2i -1}^{(n)}}(x) + \chi_{ E_{2i}^{(n)}}(x))\\
                &= \frac{2(i-1)}{2^{n}}\chi_{E_{2i -1}^{(n)}}(x)\\
                &= \varphi_n(x)
            \end{align*}
            \item $x\in E_{2i}^{(n)},$ then in this case
            \begin{align*}
                \varphi_{n-1}(x) &= \frac{i-1}{2^{n-1}}\chi_{E_{i}^{(n-1)}}(x)\\
                &= \frac{2(i-1)}{2^{n}}(\chi_{E_{2i -1}^{(n)}}(x) + \chi_{ E_{2i}^{(n)}}(x))\\
                &= \frac{2(i-1)}{2^{n}}\chi_{E_{2i}^{(n)}}(x)\\
                &\leq 
                \frac{2i-1}{2^{n}}\chi_{E_{2i}^{(n)}}(x)\\
                &= \varphi_n(x)
            \end{align*}
        \end{itemize}
Thus, $\varphi_n \uparrow f$ and we have proved (a)

Suppose now that $f\geq 0$ and $f$ is measurable. Then by the definition, $E_i^{(n)}$ and $F_n$ are measurable for any $n$ and any $i.$ Thus, by a theorem done in class, since $\varphi_n$ is simple and made up of $E_i^{(n)}$ and $F_n$ measurable, then $\varphi_n$ is measurable for all $n.$   
        
For the general case, decompose $f$ by $f = f^+ - f^-$ where both components are non-negative. By what we just showed, there exist $0 \leq \varphi_n^+ \uparrow f^+$ simple and $0 \leq \varphi_n^- \uparrow f^-.$ We show in the following problem that the difference of simple functions is a simple function, so then 
\[\varphi_n = \varphi_n^+ - \varphi_n^-\] is simple. Moreover, 
\[\lim_{n\to \infty} \varphi_n = \lim_{n\to \infty}\varphi_n^+ - \lim_{n\to \infty} \varphi_n^- = f^+ - f^- = f,\] and so $\varphi_n$ is a sequence of simple functions which converge pointwise to $f.$ We have proved the main claim. 

To prove (b), not that if $f$ is measurable, then $f^+$ and $f^-$ are non-negative and measurable, and so we have shown that $\varphi_n^+$ and $\varphi_n^-$ are measurable. The difference of measurable functions is measurable, and so $\varphi_n$ is measurable for all $n.$ We have now proved (b).
    \end{itemize}
\end{solution}




\newpage
\section*{Problem 7}
\begin{problem}
    \begin{enumerate}
        \item If $f$ is measurable and bounded on $E$ and if $m(E) < \infty,$ then $f \in \mathcal{L}(m)$ on $E.$
        \item If $a \leq f(x) \leq b$ for $x \in E$ and $m(E) < \infty,$ then 
        \[a m(E) \leq \int_E f dm \leq bm(E)\]
        \item If $f, g \in \mathcal{L}(m)$ on $E$ and if $f(x) \leq g(x)$ for $x\in E,$ then \[\int_E f dm \leq \int_E g dm\]
        \item If $f \in \mathcal{L}(m)$ on $E,$ then $cf \in \mathcal{L}(m)$ on $E,$ and for every $c\in \bbR,$ 
        \[\int_E cf dm = c\int_E fdm.\]
        \item If $m(E) = 0$ and $f$ is measurable, then 
        \[\int_E f dm = 0\]
        \item If $f \in \mathcal{L}(m)$ on $E,$ $A\in \cal M,$ and $A\subset E,$ then $f \in \mathcal{L}(m)$ on $A.$ 
    \end{enumerate}
\end{problem}
\begin{solution}
\begin{lemma}
    Suppose $\varphi, \psi$ are measurable simple functions such that $ 0 \leq \varphi \leq \psi.$ Then  for any $E \in \cal M,$ 
    \[\int_E \varphi  \leq \int_E \psi.\]
\end{lemma}
\begin{proof}
   We claim that simple functions are closed under function addition. Consider that if $\varphi$ and $\psi$ are simple, then each has finite range. Thus, any linear combination of the two must also have finite range. Thus, $\psi - \varphi$ is a simple function. By a result proven in class,
   \[\int_E \psi - \varphi = I_E(\psi - \varphi) = \sum_{k=1}^K c_k m(E_k \cap E).\] Since $\psi - \varphi \geq 0,$ then $c_k \geq 0$ for all $k,$ and thus by linearity 
   \[0 \leq \int_E \psi - \varphi = \int_E \psi - \int_E \varphi.\]  Thus, $\int_E \varphi \leq \int_E \psi.$ 
\end{proof}

\rule{\linewidth}{0.4pt}

(c) Suppose $f \leq g$ where both are Lebesgue integrable. Assume first that $ f\geq 0.$ For any simple $0\leq \varphi\leq f$
\[\int_E \varphi \leq \sup_{0\leq \psi \leq g}\int_E \psi = \int_E g \implies \int_E f = \sup_{0 \leq \varphi \leq f}\int_E\varphi \leq \int_E g\] 
\[f^+ - f^- \leq g^+ - g^-\] and so $f^+ \leq g^+$ and $f^- \geq g^-.$ Since both are nonnegative, then 
\[\int_E f^+dm \leq \int_E g^+dm, \qquad \int_E f^-dm \geq \int_E g^- dm,\] proving that (we exclude the $dm$ differential on each integral for readability):
\[\int_E f = \int_E f^+ - \int_E f^- \leq \int_E g^+ - \int_E g^- = \int_E g.\]

\rule{\linewidth}{0.4pt}

(a) Suppose $f$ is measurable and bounded on $E$ and $m(E) < \infty.$ First, assume $f\geq 0.$ Since $f$ is bounded on $E,$ there exists some $M >0$ such that $f(x) \leq M$ for all $x\in E.$ Let $\varphi$ be a simple measurable function such that $0\leq \varphi \leq f.$ By part (c) above, if we define $g = M \chi_{E},$ then $g$ is a simple measurable function and so since $ \varphi, g \in \mathcal{L}(m)$  and $\varphi \leq g$ for all $n,$ then for all $n,$ we have by (c) that
\[\int_E \varphi \leq \int_E g \implies \sup_{0 \leq \varphi \leq f}\int_E \varphi \leq \int_Eg \implies \int_E f \leq M m(E \cap E) = Mm(E) < \infty.\] Thus, $\int_E f < \infty$ and so $f \in \mathcal{L}(m).$ For the general case, we don't assume $f$ to be nonnegative. Then $f = f^+ - f^-,$ where both $f^+$ and $f^-$ are measurable and nonnegative, and thus $f^+, f^- \in \mathcal{L}(m)$ by what we just proved, and so both integrals are finite, proving that $f \in \mathcal{L}(m).$

\rule{\linewidth}{0.4pt}

(b) Suppose $a\leq f(x) \leq b$ for all $x \in E$ and $m(E) < \infty.$ By part (a), $f \in \mathcal{L}(m).$ By part (c), we get that 
\[\int_E a \leq \int_E f \leq \int_E b.\] Since $a$ and $b$ are simple functions, then $\int_E a = a m(E)$ and $\int_E b = bm(E),$ and we get our result.

\rule{\linewidth}{0.4pt}

(d) Suppose first that $f, c \geq 0.$ Since $f \in \mathcal{L}(m),$ then $f$ is measurable and nonnegative. We claim that: 
\begin{enumerate}
    \item [(1)] If $\varphi$ is a simple measurable function, then $c \varphi$ is a simple measurable function. To see this, let $\varphi = \sum_{k=1}^n a_k \chi_{E_k}.$ Then $c \varphi = \sum_{k=1}^n ca_k \chi_{E_k}.$ Since each $E_k$ is measurable since $\varphi$ is measurable, then $c\varphi$ is a simple measurable function.
    \item [(2)] If $\varphi$ is a simple measurable function, then 
    \[\int_E c \varphi = c \int_E \varphi.\] First, we have that $c \varphi \in \mathcal{L}(m)$ since by the above claim it is a simple measurable function. Moreover, 
    \[\int_E c\varphi = \sum_{k=1}^n ca_k m(E \cap E_k) = c\sum_{k=1}^n a_k m(E \cap E_k) = c\int_E \varphi.\]
    \item [(3)] If $0\leq \varphi\leq f$ is simple, then $0\leq c\varphi \leq cf$ is simple as well. The claim is pretty obvious by what we have showed. 
\end{enumerate}
By the above claims, we have that for any such simple $\varphi,$ $c\varphi \in \mathcal{L}(m)$ and thus $cf \in \mathcal{L}(m)$ is measurable Moreover,
\[\int_E c\varphi = c\int_E \varphi \implies \sup_{\varphi}\int_E c\varphi  =  c\sup_{\varphi} \int_E \varphi  \implies \int_E cf  = c\int_E f.\]


Now suppose $f\geq 0$ and $c \leq 0.$ By what we just showed,  we have that $-c \geq 0$ and so $-cf \in \mathcal{L}(m)$ with 
\[\int_E -cf = -c\int_E f.\] It suffices to show that \[-\int_E f = \int_E -f.\] To see this, it suffices to show it for simple functions. Let $\varphi$ be a simple measurable function, then $\varphi$ is bounded on $E$ and so $-\varphi$ is bounded on $E.$ Then we get by (a) that $-\varphi \in \mathcal{L}(m).$ Then we claim that $\int_E -\varphi = -\int_E \varphi.$ Consider that 
\[\int_E -\varphi = I_E(-\varphi ) = \sum_{k=1}^n -c_k m(E_k \cap E) = -\sum_{k=1}^n c_k m(E_k \cap E) = -I_E(\varphi) = -\int_E \varphi.\] Thus, taking supremums over the simple functions yields the required result. Thus, since 
\[\int_E -cf = -c \int_E f \implies c\int_E f = (-)(-c)\int_E f = -\int_E -cf = \int_E (-)-cf = \int_E cf \]

For any $f \in \cal L,$ take $f = f^+ - f^-$ and $c\in \bbR$ Then by linearity and everything we have shown above:
\[c\int_E f = c(\int_E f^+ - \int_E f^+) = \int_E cf^+ - \int_E cf^- = \int_E c(f^+ - f^-) = \int_E cf\]

\rule{\linewidth}{0.4pt}

(e) Suppose first that $f\geq 0.$ for any simple $0 \leq \varphi \leq f,$ we have that
\[I_E(\varphi) = \sum_{k=1}^{N}c_k m(E \cap E_k) \leq \sum_{k=1}^{N}c_k m(E) = 0.\] And thus 
\[\sup_{0 \leq \varphi\leq f}I_E(\varphi) = \int_E f = 0\]

For any $f$ measurable, take $f = f^+ - f^-.$ Then by what we just showed, $\int_E f^+ = 0$ and $\int_E f^- = 0.$

\rule{\linewidth}{0.4pt}

(f) We prove the result first for simple functions and then for non-negative functions. Let $\varphi = \sum_{k=1}^N c_k \chi_{E_k}$ be a simple non-negative measurable function on $E$ and $A\in \cal M$ with $A\subset E.$ Then since for any $k$ we have that $E_k \cap A \subset E_k \cap E,$ then
\[\int_A \varphi = I_A(\varphi) = \sum_{k=1}^N c_k m(E_k \cap A) \leq \sum_{k=1}^N c_k m(E_k \cap E) = I_E(\varphi) = \int_E \varphi < \infty,\] and so $\int_A \varphi$ is finite and thus $\varphi \in \mathcal{L}(m)$ on $A.$ 

Let $f \geq 0$ be Lebesgue integrable. Then take some simple $0\leq \varphi \leq f.$ By the above case, we have that
\[\int_A \varphi \leq \int_E \varphi \implies \int_A f = \sup_\varphi\int_A \varphi \leq \sup_\varphi\int_E \varphi = \int_E f< \infty,\] and so $\int_A f < \infty$ and thus $f \in \mathcal{L}(m)$ on $A.$

Let $f \in \mathcal{L}(m)$ on $E.$ Then split it up into $f^+$ and $f^-.$ Both are integrals over $A$ are finite by the above case, and so $f \in \mathcal{L}(m)$ on $A.$
\end{solution}

\newpage
\section*{Problem 8}
\begin{problem}
    Suppose that $A,B \subseteq \bbR^n$ and $A \in \mathcal{M}(m)$ and $m^*(A \triangle B) = 0.$ Show that $B \in \mathcal{M}(m)$ and determine $m(B).$
\end{problem}
\begin{solution}
\begin{lemma}
    Let $E\subset \bbR^n.$ If $m^*(E) = 0,$ then $E\in \cal M.$
\end{lemma}
\begin{proof}
Let $\epsilon>0.$ 
    Since $\emptyset \in \cal E$ and 
    \[m^*(E \triangle \emptyset) = m^*(E) = 0 < \epsilon,\] then $\emptyset \to E$ in outer measure and so $E \in \cal M_F\subset \cal M.$   
\end{proof}

By sub-additivity, since 
\[A \triangle B= (A\sm B) \sqcup (B\sm A) \implies 0  = m^*(A\triangle B)\geq \begin{cases}
    m^*(A \sm B)\\
    m^*(B\sm A)
\end{cases}\]
Thus, $m^*(A\sm B) = m^*(B\sm A) = 0.$ By Lemma 3, $A\sm B, B\sm A, A\triangle B \in \cal M.$ Since $\cal M$ is a ring and $A\in \cal M,$ then 
\begin{align}
A\sm (A\sm B) = A\cap B \in \cal M.    
\end{align}
 Thus, 
 \begin{align}
(A\cap B) \sqcup (B\sm A) = B \in \cal M.     
 \end{align}
 

To see the second claim, we note that by (1) and by additivity of $m:$:
\[m(A \cap B) = m(A) - m(A \sm B) = m(A) - m^*(A\sm B) = m(A).\]
By (2):
\[m(B) = m(A\cap B) + m(B\sm A) = m(A) + m^*(B\sm A) = m(A)\]

\end{solution}
\end{document}