\documentclass[11pt]{article}

% NOTE: Add in the relevant information to the commands below; or, if you'll be using the same information frequently, add these commands at the top of paolo-pset.tex file. 
\newcommand{\name}{Agustín Esteva}
\newcommand{\email}{aesteva@uchicago.edu}
\newcommand{\classnum}{208}
\newcommand{\subject}{Accelerated Analysis in $\bbR^n$ III}
\newcommand{\instructors}{Don Stull}
\newcommand{\assignment}{Problem Set 2}
\newcommand{\semester}{Spring 2025}
\newcommand{\duedate}{04-09-2025}
\newcommand{\bA}{\mathbf{A}}
\newcommand{\bB}{\mathbf{B}}
\newcommand{\bC}{\mathbf{C}}
\newcommand{\bD}{\mathbf{D}}
\newcommand{\bE}{\mathbf{E}}
\newcommand{\bF}{\mathbf{F}}
\newcommand{\bG}{\mathbf{G}}
\newcommand{\bH}{\mathbf{H}}
\newcommand{\bI}{\mathbf{I}}
\newcommand{\bJ}{\mathbf{J}}
\newcommand{\bK}{\mathbf{K}}
\newcommand{\bL}{\mathbf{L}}
\newcommand{\bM}{\mathbf{M}}
\newcommand{\bN}{\mathbf{N}}
\newcommand{\bO}{\mathbf{O}}
\newcommand{\bP}{\mathbf{P}}
\newcommand{\bQ}{\mathbf{Q}}
\newcommand{\bR}{\mathbf{R}}
\newcommand{\bS}{\mathbf{S}}
\newcommand{\bT}{\mathbf{T}}
\newcommand{\bU}{\mathbf{U}}
\newcommand{\bV}{\mathbf{V}}
\newcommand{\bW}{\mathbf{W}}
\newcommand{\bX}{\mathbf{X}}
\newcommand{\bY}{\mathbf{Y}}
\newcommand{\bZ}{\mathbf{Z}}
\newcommand{\Vol}{\text{Vol}}

%% blackboard bold math capitals
\newcommand{\bbA}{\mathbb{A}}
\newcommand{\bbB}{\mathbb{B}}
\newcommand{\bbC}{\mathbb{C}}
\newcommand{\bbD}{\mathbb{D}}
\newcommand{\bbE}{\mathbb{E}}
\newcommand{\bbF}{\mathbb{F}}
\newcommand{\bbG}{\mathbb{G}}
\newcommand{\bbH}{\mathbb{H}}
\newcommand{\bbI}{\mathbb{I}}
\newcommand{\bbJ}{\mathbb{J}}
\newcommand{\bbK}{\mathbb{K}}
\newcommand{\bbL}{\mathbb{L}}
\newcommand{\bbM}{\mathbb{M}}
\newcommand{\bbN}{\mathbb{N}}
\newcommand{\bbO}{\mathbb{O}}
\newcommand{\bbP}{\mathbb{P}}
\newcommand{\bbQ}{\mathbb{Q}}
\newcommand{\bbR}{\mathbb{R}}
\newcommand{\bbS}{\mathbb{S}}
\newcommand{\bbT}{\mathbb{T}}
\newcommand{\bbU}{\mathbb{U}}
\newcommand{\bbV}{\mathbb{V}}
\newcommand{\bbW}{\mathbb{W}}
\newcommand{\bbX}{\mathbb{X}}
\newcommand{\bbY}{\mathbb{Y}}
\newcommand{\bbZ}{\mathbb{Z}}

%% script math capitals
\newcommand{\sA}{\mathscr{A}}
\newcommand{\sB}{\mathscr{B}}
\newcommand{\sC}{\mathscr{C}}
\newcommand{\sD}{\mathscr{D}}
\newcommand{\sE}{\mathscr{E}}
\newcommand{\sF}{\mathscr{F}}
\newcommand{\sG}{\mathscr{G}}
\newcommand{\sH}{\mathscr{H}}
\newcommand{\sI}{\mathscr{I}}
\newcommand{\sJ}{\mathscr{J}}
\newcommand{\sK}{\mathscr{K}}
\newcommand{\sL}{\mathscr{L}}
\newcommand{\sM}{\mathscr{M}}
\newcommand{\sN}{\mathscr{N}}
\newcommand{\sO}{\mathscr{O}}
\newcommand{\sP}{\mathscr{P}}
\newcommand{\sQ}{\mathscr{Q}}
\newcommand{\sR}{\mathscr{R}}
\newcommand{\sS}{\mathscr{S}}
\newcommand{\sT}{\mathscr{T}}
\newcommand{\sU}{\mathscr{U}}
\newcommand{\sV}{\mathscr{V}}
\newcommand{\sW}{\mathscr{W}}
\newcommand{\sX}{\mathscr{X}}
\newcommand{\sY}{\mathscr{Y}}
\newcommand{\sZ}{\mathscr{Z}}


\renewcommand{\emptyset}{\O}

\newcommand{\abs}[1]{\lvert #1 \rvert}
\newcommand{\norm}[1]{\lVert #1 \rVert}
\newcommand{\sm}{\setminus}


\newcommand{\sarr}{\rightarrow}
\newcommand{\arr}{\longrightarrow}

% NOTE: Defining collaborators is optional; to not list collaborators, comment out the line below.
%\newcommand{\collaborators}{Alyssa P. Hacker (\texttt{aphacker}), Ben Bitdiddle (\texttt{bitdiddle})}

% Copyright 2021 Paolo Adajar (padajar.com, paoloadajar@mit.edu)
% 
% Permission is hereby granted, free of charge, to any person obtaining a copy of this software and associated documentation files (the "Software"), to deal in the Software without restriction, including without limitation the rights to use, copy, modify, merge, publish, distribute, sublicense, and/or sell copies of the Software, and to permit persons to whom the Software is furnished to do so, subject to the following conditions:
%
% The above copyright notice and this permission notice shall be included in all copies or substantial portions of the Software.
% 
% THE SOFTWARE IS PROVIDED "AS IS", WITHOUT WARRANTY OF ANY KIND, EXPRESS OR IMPLIED, INCLUDING BUT NOT LIMITED TO THE WARRANTIES OF MERCHANTABILITY, FITNESS FOR A PARTICULAR PURPOSE AND NONINFRINGEMENT. IN NO EVENT SHALL THE AUTHORS OR COPYRIGHT HOLDERS BE LIABLE FOR ANY CLAIM, DAMAGES OR OTHER LIABILITY, WHETHER IN AN ACTION OF CONTRACT, TORT OR OTHERWISE, ARISING FROM, OUT OF OR IN CONNECTION WITH THE SOFTWARE OR THE USE OR OTHER DEALINGS IN THE SOFTWARE.

\usepackage{fullpage}
\usepackage{enumitem}
\usepackage{amsfonts, amssymb, amsmath,amsthm}
\usepackage{mathtools}
\usepackage[pdftex, pdfauthor={\name}, pdftitle={\classnum~\assignment}]{hyperref}
\usepackage[dvipsnames]{xcolor}
\usepackage{bbm}
\usepackage{graphicx}
\usepackage{mathrsfs}
\usepackage{pdfpages}
\usepackage{tabularx}
\usepackage{pdflscape}
\usepackage{makecell}
\usepackage{booktabs}
\usepackage{natbib}
\usepackage{caption}
\usepackage{subcaption}
\usepackage{physics}
\usepackage[many]{tcolorbox}
\usepackage{version}
\usepackage{ifthen}
\usepackage{cancel}
\usepackage{listings}
\usepackage{courier}

\usepackage{tikz}
\usepackage{istgame}

\hypersetup{
	colorlinks=true,
	linkcolor=blue,
	filecolor=magenta,
	urlcolor=blue,
}

\setlength{\parindent}{0mm}
\setlength{\parskip}{2mm}

\setlist[enumerate]{label=({\alph*})}
\setlist[enumerate, 2]{label=({\roman*})}

\allowdisplaybreaks[1]

\newcommand{\psetheader}{
	\ifthenelse{\isundefined{\collaborators}}{
		\begin{center}
			{\setlength{\parindent}{0cm} \setlength{\parskip}{0mm}
				
				{\textbf{\classnum~\semester:~\assignment} \hfill \name}
				
				\subject \hfill \href{mailto:\email}{\tt \email}
				
				Instructor(s):~\instructors \hfill Due Date:~\duedate	
				
				\hrulefill}
		\end{center}
	}{
		\begin{center}
			{\setlength{\parindent}{0cm} \setlength{\parskip}{0mm}
				
				{\textbf{\classnum~\semester:~\assignment} \hfill \name\footnote{Collaborator(s): \collaborators}}
				
				\subject \hfill \href{mailto:\email}{\tt \email}
				
				Instructor(s):~\instructors \hfill Due Date:~\duedate	
				
				\hrulefill}
		\end{center}
	}
}

\renewcommand{\thepage}{\classnum~\assignment \hfill \arabic{page}}

\makeatletter
\def\points{\@ifnextchar[{\@with}{\@without}}
\def\@with[#1]#2{{\ifthenelse{\equal{#2}{1}}{{[1 point, #1]}}{{[#2 points, #1]}}}}
\def\@without#1{\ifthenelse{\equal{#1}{1}}{{[1 point]}}{{[#1 points]}}}
\makeatother

\newtheoremstyle{theorem-custom}%
{}{}%
{}{}%
{\itshape}{.}%
{ }%
{\thmname{#1}\thmnumber{ #2}\thmnote{ (#3)}}

\theoremstyle{theorem-custom}

\newtheorem{theorem}{Theorem}
\newtheorem{lemma}[theorem]{Lemma}
\newtheorem{example}[theorem]{Example}

\newenvironment{problem}[1]{\color{black} #1}{}

\newenvironment{solution}{%
	\leavevmode\begin{tcolorbox}[breakable, colback=green!5!white,colframe=green!75!black, enhanced jigsaw] \proof[\scshape Solution:] \setlength{\parskip}{2mm}%
	}{\renewcommand{\qedsymbol}{$\blacksquare$} \endproof \end{tcolorbox}}

\newenvironment{reflection}{\begin{tcolorbox}[breakable, colback=black!8!white,colframe=black!60!white, enhanced jigsaw, parbox = false]\textsc{Reflections:}}{\end{tcolorbox}}

\newcommand{\qedh}{\renewcommand{\qedsymbol}{$\blacksquare$}\qedhere}

\definecolor{mygreen}{rgb}{0,0.6,0}
\definecolor{mygray}{rgb}{0.5,0.5,0.5}
\definecolor{mymauve}{rgb}{0.58,0,0.82}

% from https://github.com/satejsoman/stata-lstlisting
% language definition
\lstdefinelanguage{Stata}{
	% System commands
	morekeywords=[1]{regress, reg, summarize, sum, display, di, generate, gen, bysort, use, import, delimited, predict, quietly, probit, margins, test},
	% Reserved words
	morekeywords=[2]{aggregate, array, boolean, break, byte, case, catch, class, colvector, complex, const, continue, default, delegate, delete, do, double, else, eltypedef, end, enum, explicit, export, external, float, for, friend, function, global, goto, if, inline, int, local, long, mata, matrix, namespace, new, numeric, NULL, operator, orgtypedef, pointer, polymorphic, pragma, private, protected, public, quad, real, return, rowvector, scalar, short, signed, static, strL, string, struct, super, switch, template, this, throw, transmorphic, try, typedef, typename, union, unsigned, using, vector, version, virtual, void, volatile, while,},
	% Keywords
	morekeywords=[3]{forvalues, foreach, set},
	% Date and time functions
	morekeywords=[4]{bofd, Cdhms, Chms, Clock, clock, Cmdyhms, Cofc, cofC, Cofd, cofd, daily, date, day, dhms, dofb, dofC, dofc, dofh, dofm, dofq, dofw, dofy, dow, doy, halfyear, halfyearly, hh, hhC, hms, hofd, hours, mdy, mdyhms, minutes, mm, mmC, mofd, month, monthly, msofhours, msofminutes, msofseconds, qofd, quarter, quarterly, seconds, ss, ssC, tC, tc, td, th, tm, tq, tw, week, weekly, wofd, year, yearly, yh, ym, yofd, yq, yw,},
	% Mathematical functions
	morekeywords=[5]{abs, ceil, cloglog, comb, digamma, exp, expm1, floor, int, invcloglog, invlogit, ln, ln1m, ln, ln1p, ln, lnfactorial, lngamma, log, log10, log1m, log1p, logit, max, min, mod, reldif, round, sign, sqrt, sum, trigamma, trunc,},
	% Matrix functions
	morekeywords=[6]{cholesky, coleqnumb, colnfreeparms, colnumb, colsof, corr, det, diag, diag0cnt, el, get, hadamard, I, inv, invsym, issymmetric, J, matmissing, matuniform, mreldif, nullmat, roweqnumb, rownfreeparms, rownumb, rowsof, sweep, trace, vec, vecdiag, },
	% Programming functions
	morekeywords=[7]{autocode, byteorder, c, _caller, chop, abs, clip, cond, e, fileexists, fileread, filereaderror, filewrite, float, fmtwidth, has_eprop, inlist, inrange, irecode, matrix, maxbyte, maxdouble, maxfloat, maxint, maxlong, mi, minbyte, mindouble, minfloat, minint, minlong, missing, r, recode, replay, return, s, scalar, smallestdouble,},
	% Random-number functions
	morekeywords=[8]{rbeta, rbinomial, rcauchy, rchi2, rexponential, rgamma, rhypergeometric, rigaussian, rlaplace, rlogistic, rnbinomial, rnormal, rpoisson, rt, runiform, runiformint, rweibull, rweibullph,},
	% Selecting time-span functions
	morekeywords=[9]{tin, twithin,},
	% Statistical functions
	morekeywords=[10]{betaden, binomial, binomialp, binomialtail, binormal, cauchy, cauchyden, cauchytail, chi2, chi2den, chi2tail, dgammapda, dgammapdada, dgammapdadx, dgammapdx, dgammapdxdx, dunnettprob, exponential, exponentialden, exponentialtail, F, Fden, Ftail, gammaden, gammap, gammaptail, hypergeometric, hypergeometricp, ibeta, ibetatail, igaussian, igaussianden, igaussiantail, invbinomial, invbinomialtail, invcauchy, invcauchytail, invchi2, invchi2tail, invdunnettprob, invexponential, invexponentialtail, invF, invFtail, invgammap, invgammaptail, invibeta, invibetatail, invigaussian, invigaussiantail, invlaplace, invlaplacetail, invlogistic, invlogistictail, invnbinomial, invnbinomialtail, invnchi2, invnF, invnFtail, invnibeta, invnormal, invnt, invnttail, invpoisson, invpoissontail, invt, invttail, invtukeyprob, invweibull, invweibullph, invweibullphtail, invweibulltail, laplace, laplaceden, laplacetail, lncauchyden, lnigammaden, lnigaussianden, lniwishartden, lnlaplaceden, lnmvnormalden, lnnormal, lnnormalden, lnwishartden, logistic, logisticden, logistictail, nbetaden, nbinomial, nbinomialp, nbinomialtail, nchi2, nchi2den, nchi2tail, nF, nFden, nFtail, nibeta, normal, normalden, npnchi2, npnF, npnt, nt, ntden, nttail, poisson, poissonp, poissontail, t, tden, ttail, tukeyprob, weibull, weibullden, weibullph, weibullphden, weibullphtail, weibulltail,},
	% String functions 
	morekeywords=[11]{abbrev, char, collatorlocale, collatorversion, indexnot, plural, plural, real, regexm, regexr, regexs, soundex, soundex_nara, strcat, strdup, string, strofreal, string, strofreal, stritrim, strlen, strlower, strltrim, strmatch, strofreal, strofreal, strpos, strproper, strreverse, strrpos, strrtrim, strtoname, strtrim, strupper, subinstr, subinword, substr, tobytes, uchar, udstrlen, udsubstr, uisdigit, uisletter, ustrcompare, ustrcompareex, ustrfix, ustrfrom, ustrinvalidcnt, ustrleft, ustrlen, ustrlower, ustrltrim, ustrnormalize, ustrpos, ustrregexm, ustrregexra, ustrregexrf, ustrregexs, ustrreverse, ustrright, ustrrpos, ustrrtrim, ustrsortkey, ustrsortkeyex, ustrtitle, ustrto, ustrtohex, ustrtoname, ustrtrim, ustrunescape, ustrupper, ustrword, ustrwordcount, usubinstr, usubstr, word, wordbreaklocale, worcount,},
	% Trig functions
	morekeywords=[12]{acos, acosh, asin, asinh, atan, atanh, cos, cosh, sin, sinh, tan, tanh,},
	morecomment=[l]{//},
	% morecomment=[l]{*},  // `*` maybe used as multiply operator. So use `//` as line comment.
	morecomment=[s]{/*}{*/},
	% The following is used by macros, like `lags'.
	morestring=[b]{`}{'},
	% morestring=[d]{'},
	morestring=[b]",
	morestring=[d]",
	% morestring=[d]{\\`},
	% morestring=[b]{'},
	sensitive=true,
}

\lstset{ 
	backgroundcolor=\color{white},   % choose the background color; you must add \usepackage{color} or \usepackage{xcolor}; should come as last argument
	basicstyle=\footnotesize\ttfamily,        % the size of the fonts that are used for the code
	breakatwhitespace=false,         % sets if automatic breaks should only happen at whitespace
	breaklines=true,                 % sets automatic line breaking
	captionpos=b,                    % sets the caption-position to bottom
	commentstyle=\color{mygreen},    % comment style
	deletekeywords={...},            % if you want to delete keywords from the given language
	escapeinside={\%*}{*)},          % if you want to add LaTeX within your code
	extendedchars=true,              % lets you use non-ASCII characters; for 8-bits encodings only, does not work with UTF-8
	firstnumber=0,                % start line enumeration with line 1000
	frame=single,	                   % adds a frame around the code
	keepspaces=true,                 % keeps spaces in text, useful for keeping indentation of code (possibly needs columns=flexible)
	keywordstyle=\color{blue},       % keyword style
	language=Octave,                 % the language of the code
	morekeywords={*,...},            % if you want to add more keywords to the set
	numbers=left,                    % where to put the line-numbers; possible values are (none, left, right)
	numbersep=5pt,                   % how far the line-numbers are from the code
	numberstyle=\tiny\color{mygray}, % the style that is used for the line-numbers
	rulecolor=\color{black},         % if not set, the frame-color may be changed on line-breaks within not-black text (e.g. comments (green here))
	showspaces=false,                % show spaces everywhere adding particular underscores; it overrides 'showstringspaces'
	showstringspaces=false,          % underline spaces within strings only
	showtabs=false,                  % show tabs within strings adding particular underscores
	stepnumber=2,                    % the step between two line-numbers. If it's 1, each line will be numbered
	stringstyle=\color{mymauve},     % string literal style
	tabsize=2,	                   % sets default tabsize to 2 spaces
%	title=\lstname,                   % show the filename of files included with \lstinputlisting; also try caption instead of title
	xleftmargin=0.25cm
}

% NOTE: To compile a version of this pset without problems, solutions, or reflections, uncomment the relevant line below.

%\excludeversion{problem}
%\excludeversion{solution}
%\excludeversion{reflection}

\begin{document}	
	
	% Use the \psetheader command at the beginning of a pset. 
	\psetheader

\section*{Problem 1}
\begin{problem}
    Suppose $f \in \mathcal{L}(m)$ on $E$ and $g$ is measurable and bounded. Then $fg \in \mathcal{L}(m).$
\end{problem}
\begin{solution}
Since $f,g$ are measurable then $fg$ is measurable. Since $f \in \mathcal{L}(m),$ then $|f| \in \mathcal{L}(m).$
    Since $g$ is bounded, then $|g(E)| \leq C$ for some $C>0.$ Decompose $fg$ into 
    \[fg = (fg)^+ - (fg)^-.\] It suffices to show that both terms are integrable. For all $x\in E,$ we have that 
    \[|(fg)(E)|  =  |f(E)||g(E)| \leq C|f(E)|.\] Thus, since $(fg)^+, (fg)^- \leq |fg|,$ then by a remark proved in the previous homework, since $C|f|\in \mathcal{L}(m)$ 
    \[(fg)^+ \leq |fg| \leq C|f| \implies \int_E (fg)^+ \leq C\int |f| < \infty.\] Similarly, 
    \[(fg)^- \leq |fg|\leq C|f| \implies \int_E (fg)^- \leq C\int |f| < \infty.\]
    Thus, $fg \in \mathcal{L}(m).$
\end{solution}

\newpage
\section*{Problem 2}
\begin{problem} (Egorov)
    Let $E\subseteq \bbR$ be measurable with $m(E) < \infty.$ Let $(f_n)$ be measurable such that $f_n : E\to \bbR$ for all $n$ and $f_n \to f$ pointwise to some function $f: E\to \bbR.$ Then, for all $\epsilon>0,$ there exists a closed $F\subset E$ such that 
    \[f_n(x)\uconv f(x) \quad \forall x\in F, \quad m(E\sm F) < \epsilon\]
\end{problem}
\begin{enumerate}
    \item 
    \begin{problem}
        Show that, under these assumptions, for every $\eta >0$ and $\delta>0,$ there is a measurable subset $A\subset E$ and $N \in \bbN$ such that 
        \[|f_n(x) - f(x)| < \eta \quad \text{for all $x\in A$ and $n \geq N$ and} \quad m(E\sm A) < \delta.\]
    \end{problem}
    \begin{solution}
        Let $\delta >0,$ let $\eta = \frac{1}{k}.$ For any $k \in \bbN,$ define for each $N \in \bbN:$
        \[A_N^{(k)} := \{x \in E \mid |f_n(x) - f(x)| < \frac{1}{k}, \quad \forall n \geq N\}.\] Since $\lim_{n\to \infty}f_n(x) = f(x)$ for each $x\in E,$ then $f$ is measurable, and thus $A_N^{(k)}$ is measurable for each $N$ and each $k.$ 

        We claim that 
        \begin{align}
        \lim_{N\to \infty}m(A_N^{(k)}) = m(E)    
        \end{align}
        To see this, it suffices to show that
        \begin{enumerate}
            \item $A_{N}^{k}$ is ascending with respect to $N;$
            \item $\bigcup_{N=1}^\infty E_N^{(k)} = E.$
        \end{enumerate}
        To see (i), let $x\in A_N^{(k)}.$ By definition, since $N+1 \geq N,$ then $|f_{N+1}(x) - f(x)| < \frac{1}{k}.$ Thus, $x\in A_{N+1}^{(k)}.$ One inclusion of (ii) is obvious. To see the other, let $x\in E.$ Since $f_n(x) \to f(x),$ then there exists some $N_x \in \bbN$ such that if $n\geq N,$ then $|f_n(x) - f(x)| < \frac{1}{k}.$ Thus, $x\in E^k_{N_x}$ (and in fact, $x\in A_n^{(k)}$) for each $n \geq N_x$) and so $E\subseteq \bigcup_{N=1}^\infty A_N^{(k)}.$  

        By (i) and (ii), and the fact that each $A_N^{(k)}$ is measurable, we have (1) by a theorem in class; so for each $k\in \bbN,$ there is some $N_{k}  \in \bbN$ such that
        \[m(E\sm A_{N_{k}}^{(k)}) < \frac{1}{2^k}.\] Define 
        \[A:= \bigcap_{k \geq K} A_{N_k}^{(k)},\] where $K \in \bbN$ is chosen such that 
        \[\sum_{i = K}^\infty \frac{1}{2^i} < \frac{\delta}{2}.\] Since each $A_{N_k}^{(k)}$ has already been shown to be measurable and this is a countable intersection, $A$ is measurable.
        Let $\eta >0$ and $x\in A.$ There is some $k>0$ such that $\frac{1}{k}< \eta.$ Thus, since $x\in A,$ then by definition, $x \in A_{N_k}^{(k)},$ and thus if $n\geq N_k,$ we have that 
        \[|f_n(x) - f(x)| < \frac{1}{k}< \eta.\] It suffices to show that $m(E \sm A) < \delta.$ 

        \begin{align*}
            m(E\sm A) &= m\left(E\sm \bigcap_{k\geq K}A_{N_k}^{(k)}\right)\\
            &= m\left(E \cap \left(\bigcap_{k\geq K}A_{N_k}^{(k)}\right)^c\right)\\
            &= m\left(E \cap \bigcup_{k\geq K}(A_{N_k}^{(k)})^c\right)\\
            &= m\left(\bigcup_{k\geq K}E \cap (A_{N_k}^{(k)})^c \right)\\
            &= m\left(\bigcup_{k\geq K} E\sm A_{N_k}^{(k)}\right)\\
            &\leq \sum_{k\geq K} m(A_{N_k}^{(k)})\\
            &= \sum_{k\geq K} \frac{1}{2^k}\\
            &< \delta
        \end{align*}
    \end{solution}
    \item Prove Egorov's theorem.
    \begin{solution}
        Let $\epsilon>0.$ By part (a), there exists some $A\subset E$ such that  $f_n \uconv f$ on $A$ and $m(E\sm A) < \frac{\epsilon}{2}.$ Since $A$ is measurable, we have proved that there is a closed $F \subset A\subset E$ such that $m(A\sm F) < \frac{\epsilon}{2}.$ Thus, 
        \[m(E\sm F)= m\big((E \sm A) \cup (A \sm F)\big) \leq m(E\sm A) + m(A\sm F) < \frac{\epsilon}{2} + \frac{\epsilon}{2} = \epsilon.\] 

        Since $F\subset A,$ and the convergence on $A$ is uniform, then the convergence on $F$ must be uniform, since we can just use the $N \in \bbN$ from $A$ for any $x\in F.$
    \end{solution}
\end{enumerate}

\newpage
\section*{Problem 3}
\begin{problem}
    (Luzin) Let $f: E \to \bbR$ be a measurable function with $E \in \cal M$ such that $m(E) < \infty.$ For every $\epsilon>0,$ there exists a closed set $F\subseteq E$ with $m(E\sm F) < \epsilon$ such that $f|_F$ is continuous.
\end{problem}
\begin{enumerate}
    \item Prove this when $f$ is a simple function on $E.$
    \begin{solution}
        Let $\epsilon>0.$ Let $\varphi$ be a simple function on $E,$ we claim there exists some $F\subseteq E$ closed with $m(E\sm F) < \epsilon$ such that $\varphi|_F$ is continuous. Note that since $E$ is measurable, then $\varphi$ is a measurable simple function.
        
        There exists some $n \in \bbN$ such that $\frac{1}{n} < \frac{\epsilon}{2}.$ Since $\varphi$ is simple on $E,$ then we can take it, without loss of generality, to be defined as 
        \[\varphi = \sum_{k=1}^N c_k \chi_{E_k}, \quad \bigsqcup_{k=1}^N E_k. = E,\] where each $E_k$ is a measurable interval. Choose $x_k:=\sup E_k$ for $k \in \{1,2,\dots N-1\}.$  Define 
        \[X_k = [x_k - \frac{1}{2Nn}, x_k + \frac{1}{2N n}] \cap E, \quad X = \bigcup_{k=1}^N X_k.\] 
        Note that each $X_k$ is an interval (and thus measurable) with $m(X_k) = \frac{1}{Nn}$ for each $k.$ Moreover, note that $X$ is measurable since it is the countable union of measurable sets. We claim that 
        \begin{enumerate}
            \item $\varphi$ is continuous on $X^c \cap E$
            \item $m(X) \leq \frac{1}{n}.$
        \end{enumerate}
        To see (i), let $x\in X^c \cap E,$ let $\eta >0.$ Then $x \in E_k \sm X$ for some $k$ and $\varphi(x) = c_k.$ Choose $\delta = \frac{1}{2N n}.$ If $y\in (x - \delta, x + \delta),$ then by definition, $y \in E_k \sm X,$ and so $\varphi(y) = c_k.$ Thus, \[|\varphi(y) - \varphi(x)| = 0 < \eta.\] 
        To see (ii), consider that 
        \[m(X) = m\left(\bigcup_{k=1}^N X_k\right) \leq \sum_{k=1}^N m(X_k) = \sum_{k=1}^N \frac{1}{Nn} = \frac{1}{n}< \epsilon\]

        Thus, since $m(E) < \infty,$ we have that since $X,$ $E$ are measurable, then $X^c$ is measurable. Then
        \begin{align*}
            m\big(E \sm (X^c\cap E)\big) &= m\big(E \cap (X^c \cap E)^c\big)\\
            &= m\big(E \cap (X \cup E^c)\big)\\
            &= m(E\cap X)\\
            &\leq m(X)\\
            & = \frac{1}{n}\\
            &< \frac{\epsilon}{2}
        \end{align*}
        Define $A = X^c \cap E.$ There exists some closed $F\subseteq A$ such that $m(A\sm F) < \frac{\epsilon}{2}.$ Since $\varphi$ is continuous on $A$ and $F\subset A,$ then $\varphi$ is continuous on $F.$ That is, $\varphi|_F$ is continuous. Moreover, 
                \[m(E\sm F)= m\big((E \sm A) \cup (A \sm F)\big) \leq m(E\sm A) + m(A\sm F) < \frac{\epsilon}{2} + \frac{\epsilon}{2} = \epsilon.\] 
    \end{solution}
    \item Prove Luzin's Theorem.
    \begin{solution}
        Let $f$ be a measurable function and let $\epsilon>0$. By a theorem proved in the previous problem set, there exists a sequence $(\varphi_n): E\to \bbR$ of measurable simple functions such that $\varphi_n \to f$ pointwise. By Egorov's theorem in the previous problem, there exists a closed $A\subseteq E$ such that $\varphi_n \uconv f$ on $A$ and $m(E\sm A)< \frac{\epsilon}{3}.$ By the above problem for each $n,$ there exists some closed $F_n\subset E$ such that $\varphi_n|_{F_n}$ is continuous and $m(E\sm F_n) < \frac{1}{2^n}.$ There exists some $N >0$ such that
        \[\sum_{n\geq N}\frac{1}{2^n} < \frac{\epsilon}{3}.\]

        Define 
        \[F':= A \sm \left(\bigcup_{n\geq N}(F_n^c \cap E)\right)\] Since $F'\subset A,$ then $\varphi_n \uconv f$ on $F'.$ We claim that $\varphi_n$ is continuous on $A$ for any $n\geq N.$ To see this, suppose not, then $\varphi_n$ is discontinuous for some $x\in A.$ Thus, $x\in F_n^c$ due to how the $F_n$s were defined in the above problem which is a contradiction. Then we see that since continuous functions uniformly converge to continuous functions, $f$ is continuous on $F'.$ Moreover, note that 
        \begin{align*}
            m(E\sm F') &= m\bigg(E \sm \left(A \sm \left(\bigcup_{n\geq N}(F_n^c \cap E)\right)\right)\bigg)\\
            &= m\left(E\sm \left(A \cap \left(\bigcup_{n\geq N}(F_n^c \cap E)\right)^c\right)\right)\\
            &= m\left(E \sm \left(A \cap \bigcap_{n\geq N}F_n\right)\right)\\
            &= m\left(E \cap \left(A \cap \bigcap_{n\geq N}F_n\right)^c\right)\\
            &= m\left(E \cap \bigg(A^c \cup \bigcup_{n\geq N} F_n^c\bigg)\right)\\
            &= m\left((E\sm A) \cup (E \cap \bigcup_{n\geq N}F_n ^c) \right)\\
            &= m\left((E\sm A) \cup \bigcup_{n\geq N}(F_n ^c \cap E)\right)\\
            &\leq m(E\sm A) + m(\bigcup_{n\geq N}F_n ^c \cap E)\\
            &< \frac{\epsilon}{3} + \sum_{n=N}^\infty \frac{1}{2^n}\\
            &< \frac{2\epsilon}{3},
        \end{align*}
    where the second to last inequality is from work on the previous part. Thus, we have found a set $F'$ such that $m(E\sm F') < \frac{2\epsilon}{3}$ and $f$ is continuous on $F'.$ There exists some closed set $F\subseteq F' \subseteq E$ such that $m(F' \sm F) < \frac{\epsilon}{3}.$ Then since $F \subseteq F'$ and $f$ is continuous on $F',$ then $f$ is continuous on $F.$ We conclude by noting that 
    \[m(E\sm F) = m\big(E\sm F') \sqcup (F'\sm F)\big) \leq m(E\sm F') + m(F'\sm F) < \frac{2\epsilon}{3} + \frac{\epsilon}{3}= \epsilon\]
    \end{solution}
\end{enumerate}


\newpage
\section*{Problem 4}
\begin{problem}
    Use Fatou's Lemma to prove the Monotone Convergence Theorem.
\end{problem}
\begin{solution}
    Let $E\in \cal M,$ let $(f_n): E\to \bbR$ be a sequence of non-negative measurable functions such that $f_n \uparrow f$ pointwise. We claim that 
    \[\lim_{n\to \infty} \int_E f_n = \int_E f.\] 
    Since each $f_n$ is measurable, then $\displaystyle\lim_{n\to \infty}f_n = f$ is measurable. Moreover, since $\displaystyle\liminf_{n\to \infty}f_n = f,$ and each $f_n$ is non-negative, then Fatou's lemma states that 
    \begin{align}
    \int_E f dm \leq \liminf_{n\to \infty} \int_E f_n    dm
    \end{align}
    
    
    
    Define 
    \[g_n(x):= f(x)- f_n(x), \quad x\in E\] Then $g_n \geq 0$ for each $n$ and $g_n \to 0$ pointwise, and thus \[\liminf_{n\to \infty} g_n = 0\]  and then it follows that $g_n$ is measurable. By Fatou's lemma, we have that 
    \begin{align}
    \int_E 0dm \leq \liminf_{n\to \infty}\int_E g_n = \liminf_{n\to \infty}\int_E f - f_n dm \implies \limsup_{n\to \infty}\int_E f_n dm \leq \int_Ef dm    
    \end{align}

    (2) and (3) prove the MCT.    
\end{solution}


\newpage
\section*{Problem 5}
\begin{problem}
    Suppose that $E\subseteq \bbR^n$ is measurable. Let $(f_n)$ be a sequence of non-negative functions such that $0 \leq f_1(x)$ for almost every $x\in E,$ and for $n\geq 1,$ $f_n(x) \leq f_{n + 1}(x)$ for almost every $x\in E.$ Prove that 
    \[\lim_{n\to \infty} \int_E f_ndm = \int_E f dm,\] where, for every $x,$ we have that
    \[f(x):= \begin{cases}
        \displaystyle\lim_{n\to \infty}f_n(x), \quad \text{if the limit exists}\\
        0, \qquad\quad\quad\;\;\: \text{else}
    \end{cases}\]
\end{problem}
\begin{solution}
    For every $n,$ define
    \[X_1 := \{x \in E \mid f_1(x) < 0\}, \quad X_n:=\{x\in E \mid f_n(x) < f_{n+1}(x)\} \quad n >1.\] Let $X = \bigcup_{\bbN}X_n,$ then by the assumption of the problem and sub-additivity (and the fact that each $X_n$ is measurable since $f_n$ is measurable),
    \[m(X)\leq \sum_{n=1}^\infty m(X_n) = 0.\] Thus, $m(X) = 0.$ Take 
    \[g_n:= f_n \chi_{E\sm X}.\] Let $x\in E.$ Either $x\in X \cap E$ or $x\in X^c \cap E.$ If $x\in X\cap E,$ then $g_n(x) = 0$ for any $n,$ and thus 
    \[0 \leq g_1(x) \leq g_2(x)\cdots.\] If $x\in X^c\cap E,$ then $g_n = f_n$ and thus
    \[0 \leq g_1(x) \leq g_2(x) \leq \cdots.\] Either way since we are in the extended reals and the sequence is monotonic, the limit function $g(x) = \displaystyle\lim_{n\to \infty} g_n(x)$ exists for every $x\in E.$ By the normal monotone convergence theorem, we have that 
    \[\lim_{n\to \infty}\int_E g_n dm = \int_E g dm.\] Which by definition implies that 
    \[\lim_{n\to \infty} \int_E f_n \chi_{E\sm X}dm = \int_E f_n \chi_{E\sm X} dm = \int_{E\sm X},\] which in turn directly implies that 
    \[\lim_{n\to \infty} \int_{E\sm X} f_n dm = \int_{E\sm X} f_n dm.\] Since $E = (E\sm X) \sqcup X$ and $m(X) = 0,$ we have that since the integrals are countably additive, then 
    \begin{align*}
        \lim_{n\to \infty}\int_E f_n dm &= \lim_{n\to \infty} \left(\int_{E\sm X} f_n dm + \int_X f dm\right)\\
        &= \lim_{n\to \infty} \int_{E\sm X}f_n dm\\
        &=\int_{E\sm X} f dm\\
        &= \int_{E\sm X} f dm + \int_X fdm\\
        &= \int_E fdm
    \end{align*}
\end{solution}



\end{document}