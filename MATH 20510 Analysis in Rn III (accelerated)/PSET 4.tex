\documentclass[11pt]{article}

% NOTE: Add in the relevant information to the commands below; or, if you'll be using the same information frequently, add these commands at the top of paolo-pset.tex file. 
\newcommand{\name}{Agustín Esteva}
\newcommand{\email}{aesteva@uchicago.edu}
\newcommand{\classnum}{20510}
\newcommand{\subject}{Accelerated Analysis in $\bbR^n$ III}
\newcommand{\instructors}{Don Stull}
\newcommand{\assignment}{Problem Set 4}
\newcommand{\semester}{Spring 2025}
\newcommand{\duedate}{04-23-2025}
\newcommand{\bA}{\mathbf{A}}
\newcommand{\bB}{\mathbf{B}}
\newcommand{\bC}{\mathbf{C}}
\newcommand{\bD}{\mathbf{D}}
\newcommand{\bE}{\mathbf{E}}
\newcommand{\bF}{\mathbf{F}}
\newcommand{\bG}{\mathbf{G}}
\newcommand{\bH}{\mathbf{H}}
\newcommand{\bI}{\mathbf{I}}
\newcommand{\bJ}{\mathbf{J}}
\newcommand{\bK}{\mathbf{K}}
\newcommand{\bL}{\mathbf{L}}
\newcommand{\bM}{\mathbf{M}}
\newcommand{\bN}{\mathbf{N}}
\newcommand{\bO}{\mathbf{O}}
\newcommand{\bP}{\mathbf{P}}
\newcommand{\bQ}{\mathbf{Q}}
\newcommand{\bR}{\mathbf{R}}
\newcommand{\bS}{\mathbf{S}}
\newcommand{\bT}{\mathbf{T}}
\newcommand{\bU}{\mathbf{U}}
\newcommand{\bV}{\mathbf{V}}
\newcommand{\bW}{\mathbf{W}}
\newcommand{\bX}{\mathbf{X}}
\newcommand{\bY}{\mathbf{Y}}
\newcommand{\bZ}{\mathbf{Z}}
\newcommand{\Vol}{\text{Vol}}

%% blackboard bold math capitals
\newcommand{\bbA}{\mathbb{A}}
\newcommand{\bbB}{\mathbb{B}}
\newcommand{\bbC}{\mathbb{C}}
\newcommand{\bbD}{\mathbb{D}}
\newcommand{\bbE}{\mathbb{E}}
\newcommand{\bbF}{\mathbb{F}}
\newcommand{\bbG}{\mathbb{G}}
\newcommand{\bbH}{\mathbb{H}}
\newcommand{\bbI}{\mathbb{I}}
\newcommand{\bbJ}{\mathbb{J}}
\newcommand{\bbK}{\mathbb{K}}
\newcommand{\bbL}{\mathbb{L}}
\newcommand{\bbM}{\mathbb{M}}
\newcommand{\bbN}{\mathbb{N}}
\newcommand{\bbO}{\mathbb{O}}
\newcommand{\bbP}{\mathbb{P}}
\newcommand{\bbQ}{\mathbb{Q}}
\newcommand{\bbR}{\mathbb{R}}
\newcommand{\bbS}{\mathbb{S}}
\newcommand{\bbT}{\mathbb{T}}
\newcommand{\bbU}{\mathbb{U}}
\newcommand{\bbV}{\mathbb{V}}
\newcommand{\bbW}{\mathbb{W}}
\newcommand{\bbX}{\mathbb{X}}
\newcommand{\bbY}{\mathbb{Y}}
\newcommand{\bbZ}{\mathbb{Z}}

%% script math capitals
\newcommand{\sA}{\mathscr{A}}
\newcommand{\sB}{\mathscr{B}}
\newcommand{\sC}{\mathscr{C}}
\newcommand{\sD}{\mathscr{D}}
\newcommand{\sE}{\mathscr{E}}
\newcommand{\sF}{\mathscr{F}}
\newcommand{\sG}{\mathscr{G}}
\newcommand{\sH}{\mathscr{H}}
\newcommand{\sI}{\mathscr{I}}
\newcommand{\sJ}{\mathscr{J}}
\newcommand{\sK}{\mathscr{K}}
\newcommand{\sL}{\mathscr{L}}
\newcommand{\sM}{\mathscr{M}}
\newcommand{\sN}{\mathscr{N}}
\newcommand{\sO}{\mathscr{O}}
\newcommand{\sP}{\mathscr{P}}
\newcommand{\sQ}{\mathscr{Q}}
\newcommand{\sR}{\mathscr{R}}
\newcommand{\sS}{\mathscr{S}}
\newcommand{\sT}{\mathscr{T}}
\newcommand{\sU}{\mathscr{U}}
\newcommand{\sV}{\mathscr{V}}
\newcommand{\sW}{\mathscr{W}}
\newcommand{\sX}{\mathscr{X}}
\newcommand{\sY}{\mathscr{Y}}
\newcommand{\sZ}{\mathscr{Z}}


\renewcommand{\emptyset}{\O}

\newcommand{\abs}[1]{\lvert #1 \rvert}
\newcommand{\norm}[1]{\lVert #1 \rVert}
\newcommand{\sm}{\setminus}


\newcommand{\sarr}{\rightarrow}
\newcommand{\arr}{\longrightarrow}

% NOTE: Defining collaborators is optional; to not list collaborators, comment out the line below.
%\newcommand{\collaborators}{Alyssa P. Hacker (\texttt{aphacker}), Ben Bitdiddle (\texttt{bitdiddle})}

\input{paolo-pset.tex}

% NOTE: To compile a version of this pset without problems, solutions, or reflections, uncomment the relevant line below.

%\excludeversion{problem}
%\excludeversion{solution}
%\excludeversion{reflection}

\begin{document}	
	
	% Use the \psetheader command at the beginning of a pset. 
	\psetheader

\section*{Problem 1}
Let \( f: \mathbb{R} \to \mathbb{R} \) be a nonnegative measurable function such that \( 0 \leq f(x) \leq 1 \) for every \( x \in \mathbb{R} \). Suppose that 
$
\int_{\mathbb{R}} f \, dm < \infty.$
Prove that 
\[
\lim_{k \to \infty} \int_{\mathbb{R}} (f(x))^k \, dm = m(f^{-1}(1)).
\]
\begin{solution}
    Define 
    \[A:= \{x \in \bbR \mid f(x) = 1\}.\] $A$ is measurable since $f$ is measurable. Define the measurable simple function
    \[g = f\chi_A = \begin{cases}
        1, \quad f(x) = 1\\
        0, \quad f(x) <1
    \end{cases}.\] 
    Define the sequence $f_k(x) = (f(x))^k.$
    
    Let $x\in \bbR.$ If $x\in A,$ then $f_k(x) = 1 = g(x)$ for each $k.$ If $x\notin A,$ then $0<f(x)<1,$ and thus $f_k(x) \to 0 = g(x).$ Thus, $f \to g$ pointwise. Moreover, $|f_k(x)| \leq 1,$ for all $x,$ and so by the dominated convergence theorem, 
    \[\lim_{k\to \infty}\int_\bbR f_n dm = \int_\bbR g dm = m(A) = m(f^{-1}(\{1\})\]
\end{solution}


\newpage
\section*{Problem 2}
Let 
\[
f(x) = \frac{1}{\sqrt{x}} \quad \text{for } x \in (0,1], \quad f(0) = 0.
\]
Prove that 
\[
\int_{0}^{1} f \, dm = 2.
\]
(Hint: Use the Monotone Convergence Theorem.)
\begin{solution}
    Define the sequence of functions
    \[f_n(x) = f(x)\chi_{[\frac{1}{n}, 1]}.\] We know that $f_n$ is Riemann integrable on $[\frac{1}{n}, 1]$ for each $n,$ and so 
    \[\int_{[0,1]} f_n \, dm = \int_\frac{1}{n}^1 f \, dm = \int_\frac{1}{n}^1 \frac{1}{\sqrt{x}}\,dx =  2\sqrt{x}\bigg|_\frac{1}{n}^1 = 2  - \sqrt{\frac{1}{n}} \to 2.\] But by the monotone convergence theorem, we have that since $f_n \uparrow f$ pointwise, then 
    \[\int_{[0,1]} f \, dm = \lim_{n\to \infty}\int_{[0,1]} {f_n}\]
\end{solution}


\newpage

\section*{Problem 3}
Let 
\[
f(x) = \frac{1}{1 + x^2}.
\]
Prove that 
\[
\int_{\mathbb{R}} f \, dm = \pi.
\]

\begin{solution}
    Define $f_n := f(x)\chi_{[-n, n]}.$ Then $f_n \uparrow f.$ By the MCT:
    \[\int_\bbR f\, dm= \lim_{n\to \infty} \int_\bbR f_n \, dm= \lim_{n\to \infty} \int_{[-n,n]} f \, dm = \lim_{n\to \infty} \int_{[-n, n]}  \frac{1}{1 + x^2} \, dm = \lim_{n\to \infty} \int_{[-n, n]}  \frac{1}{1 + x^2} \, dx.\] We compute the Riemann integral:
    \[\int_{[-n, n]}  \frac{1}{1 + x^2} \, dx = \arctan(\theta)\bigg|_{-n}^n= 2 \arctan(n) \to 2\frac{\pi}{2} = \pi.\]
\end{solution}

\newpage

\section*{Problem 4}
Suppose \( f \in L^1 \) on \( E \). Prove that, for every \( \epsilon > 0 \), there exists a simple function \( g \) such that 
\[
\int_{E} |f - g| \, dm < \epsilon.
\]
Prove that, if \( f \in L^1 \) on \( E \subseteq \mathbb{R} \), then, for every \( \epsilon > 0 \), there exists a step function \( s \) such that 
\[
\int_{E} |f - s| \, dm < \epsilon.
\]
Recall that a step function \( s \) is a simple function such that, for every \( c \in \text{range}(s) \), \( s^{-1}(c) \) is an interval.
\begin{solution}
     Let $\epsilon>0.$ Since $f\in \mathcal{L}^1(E),$ we have that $\int_E |f|< \infty.$ Let $A = \{x \in E \mid f(x) \geq 0\}$ and $B = \{x \in E \mid f(x) < 0\}.$ Then 
    \[\infty >\int_E |f(x)|  = \int_A |f(x)| + \int_B |f(x)| = \int_E f^+ + \int_E f^-.\] Thus, $\int_E f^+, \int_E f^- < \infty.$ By definition, there exists a simple function $\varphi^+, \varphi^-$ such that 
    \[I_E(\varphi^+) > \int_E f^+ -\frac{\epsilon}{2}, \quad I_E(\varphi^-) > \int_E f^- - \frac{\epsilon}{2}.\]
    Thus, 
    \[\int_E f^+ - \varphi^+ < \frac{\epsilon}{2}, \quad \int_E f^-  - \varphi^- < \frac{\epsilon}{2}.\] Define $\varphi = \varphi^+ - \varphi^-.$ Then
    \begin{align*}
        \int_E |f  - \varphi| &= \int_A |f - \varphi| +  \int_B |f - \varphi|\\
        &= \int_A f^+ - \varphi^+ + \int_B f^- - \varphi^-\\
        &< \epsilon.
    \end{align*}

    For the second part, it suffices to show that we can approximate $f= \chi_E$ with step functions and then use dominated convergence theorem.
\end{solution}


\newpage
\section*{Problem 5}
Prove that, if \( f: \mathbb{R} \to \mathbb{R} \), and \( f \in \cal L \), then
\[
\lim_{n \to \infty} \int_{0}^{2\pi} f(x) \cos(nx) \, dm = 0.
\]
(Hint: First prove this when \( f \) is simple.)
\begin{solution}
By the previous problem, we can approximate $f$ by step functions. Suppose first $f$ is a step function. That is, we can write 
\[f = \sum_{k= 1}^N c_k \chi_{R_k},\] where each $R_k$ is a disjoint interval. Then for any $n,$ we have that
\begin{align*}
    \int_{0}^{2\pi} f(x)\cos(nx) \, dm &= \int_{0}^{2\pi} \sum_{k=1}^N c_k \chi_{R_k}\cos(nx) \, dm\\
    &= \sum_{k=1}^N c_k\int_{R_k} \cos(nx)\, dm\\
    &= \sum_{k=1}^N c_k \bigg[\frac{1}{n}\sin(nx)\bigg]_{a_k}^{b_k}\\
    &\leq 2 Nc_{(N)} \frac{1}{n}\\
    &\to 0,
\end{align*}
where $c_{(N)} = \max\{c_1, \dots, c_N\}.$ 

For a general $f,$ we use the previous problem. There exists a simple function $g$ such that 
\[\int_{0}^{2\pi} |f - g|\, dm < \epsilon.\] Then 
\[\left|\int_0^{2\pi} (f - g)\cos(nx)\, dm\right| \leq \int_0^{2\pi} |f - g||\cos(nx)| \leq \int_0^{2\pi} |f-g| < \epsilon.\] And so because this is true for any $\epsilon,$ then
\begin{align*}
\epsilon &>\left|\int_0^{2\pi} (f - g)\cos(nx)\, dm\right|\\
& \geq \int_0^{2\pi} (f - g)\cos(nx)\\
&= \int_0^{2\pi} f \cos(nx)\, dm - \int_0^{2\pi} g \cos(nx)\, dm,
\end{align*}
and so \[\int_0^{2\pi} g \cos(nx)\, dm = \int_0^{2\pi} f \cos(nx)\, dm.\] Thus, by what we just showed, as $n\to \infty,$ we have that 
\[\int_0^{2\pi} f \cos(nx)\, dm \to 0.\]


    
\end{solution}


\newpage
\section*{Problem 6}
Suppose that \( f: \mathbb{R}^d \to \mathbb{R} \) is measurable and nonnegative, and, for every \( n, k \in \mathbb{N} \), define
\[
E_{n,k} = \{x \mid \frac{k}{2^n} \leq f(x) < \frac{k+1}{2^n}\}.
\]
Show that, as \( n \to \infty \),
\[
\sum_{k=1}^{\infty} \frac{k}{2^n}m(E_{n,k}) \to \int_{\mathbb{R}^d} f \, dm.
\]
\begin{solution}
    Define the function $f_{n}(x) = \sum_{k=1}^\infty \frac{k}{2^n}\chi_{E_{n,k}}.$ Note that $f_n \uparrow f.$ By the MCT, we have that
    \[\lim_{n\to \infty}\int_{\bbR^d} f_n \, dm = \int_{\bbR^d} f \, dm,\] where 
    \[\int_{\bbR^d} f_{n} \, dm = \sum_{k=1}^\infty \frac{k}{2^n}m(E_{n,k} \cap \bbR^d) = \sum_{k=1}^\infty \frac{k}{2^n}m(E_{n,k}) \]
\end{solution}



\end{document}