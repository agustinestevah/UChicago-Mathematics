\documentclass[11pt]{article}

% NOTE: Add in the relevant information to the commands below; or, if you'll be using the same information frequently, add these commands at the top of paolo-pset.tex file. 
\newcommand{\name}{Agustín Esteva}
\newcommand{\email}{aesteva@uchicago.edu}
\newcommand{\classnum}{20510}
\newcommand{\subject}{Accelerated Analysis in $\bbR^n$ III}
\newcommand{\instructors}{Don Stull}
\newcommand{\assignment}{Problem Set 5}
\newcommand{\semester}{Spring 2025}
\newcommand{\duedate}{04-30-2025}
\newcommand{\bA}{\mathbf{A}}
\newcommand{\bB}{\mathbf{B}}
\newcommand{\bC}{\mathbf{C}}
\newcommand{\bD}{\mathbf{D}}
\newcommand{\bE}{\mathbf{E}}
\newcommand{\bF}{\mathbf{F}}
\newcommand{\bG}{\mathbf{G}}
\newcommand{\bH}{\mathbf{H}}
\newcommand{\bI}{\mathbf{I}}
\newcommand{\bJ}{\mathbf{J}}
\newcommand{\bK}{\mathbf{K}}
\newcommand{\bL}{\mathbf{L}}
\newcommand{\bM}{\mathbf{M}}
\newcommand{\bN}{\mathbf{N}}
\newcommand{\bO}{\mathbf{O}}
\newcommand{\bP}{\mathbf{P}}
\newcommand{\bQ}{\mathbf{Q}}
\newcommand{\bR}{\mathbf{R}}
\newcommand{\bS}{\mathbf{S}}
\newcommand{\bT}{\mathbf{T}}
\newcommand{\bU}{\mathbf{U}}
\newcommand{\bV}{\mathbf{V}}
\newcommand{\bW}{\mathbf{W}}
\newcommand{\bX}{\mathbf{X}}
\newcommand{\bY}{\mathbf{Y}}
\newcommand{\bZ}{\mathbf{Z}}
\newcommand{\Vol}{\text{Vol}}

%% blackboard bold math capitals
\newcommand{\bbA}{\mathbb{A}}
\newcommand{\bbB}{\mathbb{B}}
\newcommand{\bbC}{\mathbb{C}}
\newcommand{\bbD}{\mathbb{D}}
\newcommand{\bbE}{\mathbb{E}}
\newcommand{\bbF}{\mathbb{F}}
\newcommand{\bbG}{\mathbb{G}}
\newcommand{\bbH}{\mathbb{H}}
\newcommand{\bbI}{\mathbb{I}}
\newcommand{\bbJ}{\mathbb{J}}
\newcommand{\bbK}{\mathbb{K}}
\newcommand{\bbL}{\mathbb{L}}
\newcommand{\bbM}{\mathbb{M}}
\newcommand{\bbN}{\mathbb{N}}
\newcommand{\bbO}{\mathbb{O}}
\newcommand{\bbP}{\mathbb{P}}
\newcommand{\bbQ}{\mathbb{Q}}
\newcommand{\bbR}{\mathbb{R}}
\newcommand{\bbS}{\mathbb{S}}
\newcommand{\bbT}{\mathbb{T}}
\newcommand{\bbU}{\mathbb{U}}
\newcommand{\bbV}{\mathbb{V}}
\newcommand{\bbW}{\mathbb{W}}
\newcommand{\bbX}{\mathbb{X}}
\newcommand{\bbY}{\mathbb{Y}}
\newcommand{\bbZ}{\mathbb{Z}}

%% script math capitals
\newcommand{\sA}{\mathscr{A}}
\newcommand{\sB}{\mathscr{B}}
\newcommand{\sC}{\mathscr{C}}
\newcommand{\sD}{\mathscr{D}}
\newcommand{\sE}{\mathscr{E}}
\newcommand{\sF}{\mathscr{F}}
\newcommand{\sG}{\mathscr{G}}
\newcommand{\sH}{\mathscr{H}}
\newcommand{\sI}{\mathscr{I}}
\newcommand{\sJ}{\mathscr{J}}
\newcommand{\sK}{\mathscr{K}}
\newcommand{\sL}{\mathscr{L}}
\newcommand{\sM}{\mathscr{M}}
\newcommand{\sN}{\mathscr{N}}
\newcommand{\sO}{\mathscr{O}}
\newcommand{\sP}{\mathscr{P}}
\newcommand{\sQ}{\mathscr{Q}}
\newcommand{\sR}{\mathscr{R}}
\newcommand{\sS}{\mathscr{S}}
\newcommand{\sT}{\mathscr{T}}
\newcommand{\sU}{\mathscr{U}}
\newcommand{\sV}{\mathscr{V}}
\newcommand{\sW}{\mathscr{W}}
\newcommand{\sX}{\mathscr{X}}
\newcommand{\sY}{\mathscr{Y}}
\newcommand{\sZ}{\mathscr{Z}}


\renewcommand{\emptyset}{\O}

\newcommand{\abs}[1]{\lvert #1 \rvert}
\newcommand{\norm}[1]{\lVert #1 \rVert}
\newcommand{\sm}{\setminus}


\newcommand{\sarr}{\rightarrow}
\newcommand{\arr}{\longrightarrow}

% NOTE: Defining collaborators is optional; to not list collaborators, comment out the line below.
%\newcommand{\collaborators}{Alyssa P. Hacker (\texttt{aphacker}), Ben Bitdiddle (\texttt{bitdiddle})}

% Copyright 2021 Paolo Adajar (padajar.com, paoloadajar@mit.edu)
% 
% Permission is hereby granted, free of charge, to any person obtaining a copy of this software and associated documentation files (the "Software"), to deal in the Software without restriction, including without limitation the rights to use, copy, modify, merge, publish, distribute, sublicense, and/or sell copies of the Software, and to permit persons to whom the Software is furnished to do so, subject to the following conditions:
%
% The above copyright notice and this permission notice shall be included in all copies or substantial portions of the Software.
% 
% THE SOFTWARE IS PROVIDED "AS IS", WITHOUT WARRANTY OF ANY KIND, EXPRESS OR IMPLIED, INCLUDING BUT NOT LIMITED TO THE WARRANTIES OF MERCHANTABILITY, FITNESS FOR A PARTICULAR PURPOSE AND NONINFRINGEMENT. IN NO EVENT SHALL THE AUTHORS OR COPYRIGHT HOLDERS BE LIABLE FOR ANY CLAIM, DAMAGES OR OTHER LIABILITY, WHETHER IN AN ACTION OF CONTRACT, TORT OR OTHERWISE, ARISING FROM, OUT OF OR IN CONNECTION WITH THE SOFTWARE OR THE USE OR OTHER DEALINGS IN THE SOFTWARE.

\usepackage{fullpage}
\usepackage{enumitem}
\usepackage{amsfonts, amssymb, amsmath,amsthm}
\usepackage{mathtools}
\usepackage[pdftex, pdfauthor={\name}, pdftitle={\classnum~\assignment}]{hyperref}
\usepackage[dvipsnames]{xcolor}
\usepackage{bbm}
\usepackage{graphicx}
\usepackage{mathrsfs}
\usepackage{pdfpages}
\usepackage{tabularx}
\usepackage{pdflscape}
\usepackage{makecell}
\usepackage{booktabs}
\usepackage{natbib}
\usepackage{caption}
\usepackage{subcaption}
\usepackage{physics}
\usepackage[many]{tcolorbox}
\usepackage{version}
\usepackage{ifthen}
\usepackage{cancel}
\usepackage{listings}
\usepackage{courier}

\usepackage{tikz}
\usepackage{istgame}

\hypersetup{
	colorlinks=true,
	linkcolor=blue,
	filecolor=magenta,
	urlcolor=blue,
}

\setlength{\parindent}{0mm}
\setlength{\parskip}{2mm}

\setlist[enumerate]{label=({\alph*})}
\setlist[enumerate, 2]{label=({\roman*})}

\allowdisplaybreaks[1]

\newcommand{\psetheader}{
	\ifthenelse{\isundefined{\collaborators}}{
		\begin{center}
			{\setlength{\parindent}{0cm} \setlength{\parskip}{0mm}
				
				{\textbf{\classnum~\semester:~\assignment} \hfill \name}
				
				\subject \hfill \href{mailto:\email}{\tt \email}
				
				Instructor(s):~\instructors \hfill Due Date:~\duedate	
				
				\hrulefill}
		\end{center}
	}{
		\begin{center}
			{\setlength{\parindent}{0cm} \setlength{\parskip}{0mm}
				
				{\textbf{\classnum~\semester:~\assignment} \hfill \name\footnote{Collaborator(s): \collaborators}}
				
				\subject \hfill \href{mailto:\email}{\tt \email}
				
				Instructor(s):~\instructors \hfill Due Date:~\duedate	
				
				\hrulefill}
		\end{center}
	}
}

\renewcommand{\thepage}{\classnum~\assignment \hfill \arabic{page}}

\makeatletter
\def\points{\@ifnextchar[{\@with}{\@without}}
\def\@with[#1]#2{{\ifthenelse{\equal{#2}{1}}{{[1 point, #1]}}{{[#2 points, #1]}}}}
\def\@without#1{\ifthenelse{\equal{#1}{1}}{{[1 point]}}{{[#1 points]}}}
\makeatother

\newtheoremstyle{theorem-custom}%
{}{}%
{}{}%
{\itshape}{.}%
{ }%
{\thmname{#1}\thmnumber{ #2}\thmnote{ (#3)}}

\theoremstyle{theorem-custom}

\newtheorem{theorem}{Theorem}
\newtheorem{lemma}[theorem]{Lemma}
\newtheorem{example}[theorem]{Example}

\newenvironment{problem}[1]{\color{black} #1}{}

\newenvironment{solution}{%
	\leavevmode\begin{tcolorbox}[breakable, colback=green!5!white,colframe=green!75!black, enhanced jigsaw] \proof[\scshape Solution:] \setlength{\parskip}{2mm}%
	}{\renewcommand{\qedsymbol}{$\blacksquare$} \endproof \end{tcolorbox}}

\newenvironment{reflection}{\begin{tcolorbox}[breakable, colback=black!8!white,colframe=black!60!white, enhanced jigsaw, parbox = false]\textsc{Reflections:}}{\end{tcolorbox}}

\newcommand{\qedh}{\renewcommand{\qedsymbol}{$\blacksquare$}\qedhere}

\definecolor{mygreen}{rgb}{0,0.6,0}
\definecolor{mygray}{rgb}{0.5,0.5,0.5}
\definecolor{mymauve}{rgb}{0.58,0,0.82}

% from https://github.com/satejsoman/stata-lstlisting
% language definition
\lstdefinelanguage{Stata}{
	% System commands
	morekeywords=[1]{regress, reg, summarize, sum, display, di, generate, gen, bysort, use, import, delimited, predict, quietly, probit, margins, test},
	% Reserved words
	morekeywords=[2]{aggregate, array, boolean, break, byte, case, catch, class, colvector, complex, const, continue, default, delegate, delete, do, double, else, eltypedef, end, enum, explicit, export, external, float, for, friend, function, global, goto, if, inline, int, local, long, mata, matrix, namespace, new, numeric, NULL, operator, orgtypedef, pointer, polymorphic, pragma, private, protected, public, quad, real, return, rowvector, scalar, short, signed, static, strL, string, struct, super, switch, template, this, throw, transmorphic, try, typedef, typename, union, unsigned, using, vector, version, virtual, void, volatile, while,},
	% Keywords
	morekeywords=[3]{forvalues, foreach, set},
	% Date and time functions
	morekeywords=[4]{bofd, Cdhms, Chms, Clock, clock, Cmdyhms, Cofc, cofC, Cofd, cofd, daily, date, day, dhms, dofb, dofC, dofc, dofh, dofm, dofq, dofw, dofy, dow, doy, halfyear, halfyearly, hh, hhC, hms, hofd, hours, mdy, mdyhms, minutes, mm, mmC, mofd, month, monthly, msofhours, msofminutes, msofseconds, qofd, quarter, quarterly, seconds, ss, ssC, tC, tc, td, th, tm, tq, tw, week, weekly, wofd, year, yearly, yh, ym, yofd, yq, yw,},
	% Mathematical functions
	morekeywords=[5]{abs, ceil, cloglog, comb, digamma, exp, expm1, floor, int, invcloglog, invlogit, ln, ln1m, ln, ln1p, ln, lnfactorial, lngamma, log, log10, log1m, log1p, logit, max, min, mod, reldif, round, sign, sqrt, sum, trigamma, trunc,},
	% Matrix functions
	morekeywords=[6]{cholesky, coleqnumb, colnfreeparms, colnumb, colsof, corr, det, diag, diag0cnt, el, get, hadamard, I, inv, invsym, issymmetric, J, matmissing, matuniform, mreldif, nullmat, roweqnumb, rownfreeparms, rownumb, rowsof, sweep, trace, vec, vecdiag, },
	% Programming functions
	morekeywords=[7]{autocode, byteorder, c, _caller, chop, abs, clip, cond, e, fileexists, fileread, filereaderror, filewrite, float, fmtwidth, has_eprop, inlist, inrange, irecode, matrix, maxbyte, maxdouble, maxfloat, maxint, maxlong, mi, minbyte, mindouble, minfloat, minint, minlong, missing, r, recode, replay, return, s, scalar, smallestdouble,},
	% Random-number functions
	morekeywords=[8]{rbeta, rbinomial, rcauchy, rchi2, rexponential, rgamma, rhypergeometric, rigaussian, rlaplace, rlogistic, rnbinomial, rnormal, rpoisson, rt, runiform, runiformint, rweibull, rweibullph,},
	% Selecting time-span functions
	morekeywords=[9]{tin, twithin,},
	% Statistical functions
	morekeywords=[10]{betaden, binomial, binomialp, binomialtail, binormal, cauchy, cauchyden, cauchytail, chi2, chi2den, chi2tail, dgammapda, dgammapdada, dgammapdadx, dgammapdx, dgammapdxdx, dunnettprob, exponential, exponentialden, exponentialtail, F, Fden, Ftail, gammaden, gammap, gammaptail, hypergeometric, hypergeometricp, ibeta, ibetatail, igaussian, igaussianden, igaussiantail, invbinomial, invbinomialtail, invcauchy, invcauchytail, invchi2, invchi2tail, invdunnettprob, invexponential, invexponentialtail, invF, invFtail, invgammap, invgammaptail, invibeta, invibetatail, invigaussian, invigaussiantail, invlaplace, invlaplacetail, invlogistic, invlogistictail, invnbinomial, invnbinomialtail, invnchi2, invnF, invnFtail, invnibeta, invnormal, invnt, invnttail, invpoisson, invpoissontail, invt, invttail, invtukeyprob, invweibull, invweibullph, invweibullphtail, invweibulltail, laplace, laplaceden, laplacetail, lncauchyden, lnigammaden, lnigaussianden, lniwishartden, lnlaplaceden, lnmvnormalden, lnnormal, lnnormalden, lnwishartden, logistic, logisticden, logistictail, nbetaden, nbinomial, nbinomialp, nbinomialtail, nchi2, nchi2den, nchi2tail, nF, nFden, nFtail, nibeta, normal, normalden, npnchi2, npnF, npnt, nt, ntden, nttail, poisson, poissonp, poissontail, t, tden, ttail, tukeyprob, weibull, weibullden, weibullph, weibullphden, weibullphtail, weibulltail,},
	% String functions 
	morekeywords=[11]{abbrev, char, collatorlocale, collatorversion, indexnot, plural, plural, real, regexm, regexr, regexs, soundex, soundex_nara, strcat, strdup, string, strofreal, string, strofreal, stritrim, strlen, strlower, strltrim, strmatch, strofreal, strofreal, strpos, strproper, strreverse, strrpos, strrtrim, strtoname, strtrim, strupper, subinstr, subinword, substr, tobytes, uchar, udstrlen, udsubstr, uisdigit, uisletter, ustrcompare, ustrcompareex, ustrfix, ustrfrom, ustrinvalidcnt, ustrleft, ustrlen, ustrlower, ustrltrim, ustrnormalize, ustrpos, ustrregexm, ustrregexra, ustrregexrf, ustrregexs, ustrreverse, ustrright, ustrrpos, ustrrtrim, ustrsortkey, ustrsortkeyex, ustrtitle, ustrto, ustrtohex, ustrtoname, ustrtrim, ustrunescape, ustrupper, ustrword, ustrwordcount, usubinstr, usubstr, word, wordbreaklocale, worcount,},
	% Trig functions
	morekeywords=[12]{acos, acosh, asin, asinh, atan, atanh, cos, cosh, sin, sinh, tan, tanh,},
	morecomment=[l]{//},
	% morecomment=[l]{*},  // `*` maybe used as multiply operator. So use `//` as line comment.
	morecomment=[s]{/*}{*/},
	% The following is used by macros, like `lags'.
	morestring=[b]{`}{'},
	% morestring=[d]{'},
	morestring=[b]",
	morestring=[d]",
	% morestring=[d]{\\`},
	% morestring=[b]{'},
	sensitive=true,
}

\lstset{ 
	backgroundcolor=\color{white},   % choose the background color; you must add \usepackage{color} or \usepackage{xcolor}; should come as last argument
	basicstyle=\footnotesize\ttfamily,        % the size of the fonts that are used for the code
	breakatwhitespace=false,         % sets if automatic breaks should only happen at whitespace
	breaklines=true,                 % sets automatic line breaking
	captionpos=b,                    % sets the caption-position to bottom
	commentstyle=\color{mygreen},    % comment style
	deletekeywords={...},            % if you want to delete keywords from the given language
	escapeinside={\%*}{*)},          % if you want to add LaTeX within your code
	extendedchars=true,              % lets you use non-ASCII characters; for 8-bits encodings only, does not work with UTF-8
	firstnumber=0,                % start line enumeration with line 1000
	frame=single,	                   % adds a frame around the code
	keepspaces=true,                 % keeps spaces in text, useful for keeping indentation of code (possibly needs columns=flexible)
	keywordstyle=\color{blue},       % keyword style
	language=Octave,                 % the language of the code
	morekeywords={*,...},            % if you want to add more keywords to the set
	numbers=left,                    % where to put the line-numbers; possible values are (none, left, right)
	numbersep=5pt,                   % how far the line-numbers are from the code
	numberstyle=\tiny\color{mygray}, % the style that is used for the line-numbers
	rulecolor=\color{black},         % if not set, the frame-color may be changed on line-breaks within not-black text (e.g. comments (green here))
	showspaces=false,                % show spaces everywhere adding particular underscores; it overrides 'showstringspaces'
	showstringspaces=false,          % underline spaces within strings only
	showtabs=false,                  % show tabs within strings adding particular underscores
	stepnumber=2,                    % the step between two line-numbers. If it's 1, each line will be numbered
	stringstyle=\color{mymauve},     % string literal style
	tabsize=2,	                   % sets default tabsize to 2 spaces
%	title=\lstname,                   % show the filename of files included with \lstinputlisting; also try caption instead of title
	xleftmargin=0.25cm
}

% NOTE: To compile a version of this pset without problems, solutions, or reflections, uncomment the relevant line below.

%\excludeversion{problem}
%\excludeversion{solution}
%\excludeversion{reflection}

\begin{document}	
	
	% Use the \psetheader command at the beginning of a pset. 
	\psetheader

\section*{Problem 1}
\begin{problem}
    Let $f \in \cal R$ be $2-\pi$ periodic. 
    \begin{enumerate}
        \item Show that the Fourier Series of $f$ can be written as 
        \[\hat{f}(0) + \sum_{n=1}^\infty (\hat{f}(n) + \hat{f}(-n))\cos(nx) + i(\hat{f}(n) - \hat{f}(-n))\sin(nx) \]
        \begin{solution}
        By definition, we have that for any $N \in \bbN,$
\begin{align*}
    S_N(f) &= \sum_{-N}^N (f,e_n)e_n\\
    &= \sum_{-N}^N \hat{f}(n)e_n\\
    &= \sum_{-N}^N\hat{f}(n)e^{inx}\\
    &= \sum_{-N}^N \hat{f}(n)(\cos(nx) + i\sin(nx))\\
    &= \hat{f}(0) + \sum_{n=1}^N \hat{f}(n)\cos(nx) + i\hat{f}(n)\sin(nx) + \sum_{n= -N}^{-1}\hat{f}(n)\cos(nx) + i\hat{f}(n)\sin(nx)\\
    &= \hat{f}(0) + \sum_{n=1}^N \hat{f}(n)\cos(nx) + i\hat{f}(n)\sin(nx) + \sum_{n= 1}^{N}\hat{f}(-n)\cos(-nx) + i\hat{f}(-n)\sin(-nx)\\
    &= \hat{f}(0) + \sum_{n=1}^N \hat{f}(n)\cos(nx) + i\hat{f}(n)\sin(nx) + \sum_{n= 1}^{N}\hat{f}(-n)\cos(nx) - i\hat{f}(-n)\sin(nx)\\
    &=\hat{f}(0) + \sum_{n=1}^N (\hat{f}(n) + \hat{f}(-n))\cos(nx) + i(\hat{f} (n) - \hat{f}(-n)\sin(nx)\\
\end{align*}
        \end{solution}
    \item Prove that if $f$ is even, then $\hat{f}(n) = \hat{f}(-n)$ and we get a cosine series.
    \begin{solution}
        Let $f$ be even so that $f(x) = f(-x).$ Then 
        \begin{align*}
    \hat{f}(-n) &= (f,e_{-n})\\
    &= \frac{1}{2\pi}\int_{-\pi}^{\pi} f(x)e^{-i(-n)x}\,dx\\
    &= \frac{1}{2\pi}\int_{-\pi}^{\pi} f(x)e^{inx}\,dx\\
    &= \frac{1}{2\pi}\int_{-\pi}^{\pi} f(-x)e^{inx}\,dx\\
    &= -\frac{1}{2\pi}\int_{\pi}^{-\pi} f(u)e^{-inu}\,du\\
    &= \frac{1}{2\pi}\int_{-\pi}^{\pi} f(u)e^{-inu}\,du\\
    &= (f,e_n)\\
    &= \hat{f}(n)
        \end{align*}
Moreover, we use the identity derived in part (a) to notice that
\begin{align*}
    S_N(f) &= \hat{f}(0) + \sum_{n=1}^N (\hat{f}(n) + \hat{f}(-n))\cos(nx) + i(\hat{f} (n) - \hat{f}(-n)\sin(nx)\\
    &= \hat{f}(0) +\sum_{n=1}^N (\hat{f}(n) + \hat{f}(n))\cos(nx) + i(\hat{f} (n) - \hat{f}(n)\sin(nx)\\
    &= \hat{f}(0) +2\sum_{n=1}^N \hat{f}(n)\cos(nx)
\end{align*}
as desired. 
    \end{solution}
\item Prove that if $f$ is odd, then $\hat{f}(n) = -\hat{f}(-n)$ and we get a sine series.
\begin{solution}
    Let $f$ be odd such that $f(x)  = -f(-x).$ Then 
    \begin{align*}
        -\hat{f}(-n) &= -(f,e_{-n})\\
        &= - \frac{1}{2\pi}\int_0^{2\pi} f(x)e^{inx}\,dx\\
        &= \frac{1}{2\pi}\int_0^{2\pi} -f(x)e^{inx}dx\\
        &= \frac{1}{2\pi}\int_0^{-2\pi} -(-f(-u)e^{-inu})\,du\\
        &= -\frac{1}{2\pi}\int_0^{-2\pi}f(u)e^{-inu}\,du\\
        &= \frac{1}{2\pi}\int_0^{2\pi}f(u)e^{-inu}\,du\\
        &= (f,e_n)\\
        &= \hat{f}(n).
    \end{align*}
Moreover, we use the identity derived in part (a) to show that 
\begin{align*}
    S_N(f) &= \hat{f}(0) + \sum_{n=1}^N (\hat{f}(n) + \hat{f}(-n))\cos(nx) + i(\hat{f} (n) - \hat{f}(-n)\sin(nx)\\
    &= \hat{f}(0) + \sum_{n=1}^N (-\hat{f}(-n) + \hat{f}(-n))\cos(nx) + i(\hat{f} (n) + \hat{f}(n)\sin(nx)\\
    &= \hat{f}(0) + \sum_{n=1}^N i(2\hat{f} (n))\sin(nx)\\
    &= \hat{f}(0) + 2i\sum_{n=1}^N \hat{f} (n)\sin(nx)
\end{align*}
as desired
\end{solution}
\item
Suppose that $f(x + \pi) = f(x)$ for all $x\in \bbR.$ Show that $\hat{f}(n) = 0$ for all odd $n.$
\begin{solution}
    Let $n$ be odd, then 
    \begin{align*}
        \hat{f}(n) &= (f,e_{n})\\
        &= \frac{1}{2\pi}\int_{-\pi}^{\pi} f(x)e^{-inx}dx\\
        &= \frac{1}{2\pi}\left[\int_{-\pi}^0 f(x)e^{-inx}\,dx + \int_0^{\pi} f(x)e^{-inx}\, dx\right]\\
        &= \frac{1}{2\pi}\left[\int_{-\pi}^0 f(x)e^{-inx}\,dx + \int_{-\pi}^{0} f(u+\pi)e^{-in(u + \pi)}\,dx\right]\\
        &= \frac{1}{2\pi}\left[\int_{-\pi}^0 f(x)e^{-inx}\,dx + \int_{-\pi}^{0} f(u)\frac{e^{-inu}}{e^{in\pi}}\,dx\right]
    \end{align*}
Since we have that $n$ is odd, then
\begin{align*}
    e^{in\pi} = \cos(n\pi) + i\sin(n\pi)
    = -1 + 0 = -1,
\end{align*}
then 
\[\hat{f}(n) = \frac{1}{2\pi}\left[\int_{-\pi}^0 f(x)e^{-inx}\,dx - \int_{-\pi}^{0} f(u)e^{-inu}\,dx\right] = 0.\]
\end{solution}
\item 
Show that $f$ is real valued if, and only if, $\overline{\hat{f}(n)} = \hat{f}(-n)$ for all $n \in \bbN.$
\begin{solution}
    ($\implies$) Suppose $f$ is real valued. Then $\overline{f(x)} = f(x)$ for all $x\in \bbR.$ Thus, we note that since the conjugate of the integral is the integral of the conjugate and similarly for products, we can compute
    \begin{align*}
        \overline{\hat{f}(n)} &= \overline{\frac{1}{2\pi}\int_0^{2\pi}f(x)e^{-inx}\,dx}\\
        &=  \frac{1}{2\pi}\int_0^{2\pi}\overline{f(x)}\,\overline{e^{-inx}}\,dx\\
        &= \frac{1}{2\pi}\int_0^{2\pi}f(x)e^{inx}\,dx\\
        &= \frac{1}{2\pi}\int_0^{2\pi}f(x)e^{-i(-n)x}\,dx\\
        &= \hat{f}(-n)
    \end{align*}

($\impliedby$) Suppose $f$ is continuous and $\overline{\hat{f}(n)} = \hat{f}(-n).$ To see that $f$ is real value, it suffices to show that $\overline{f(x)} = f(x)$ for any $x\in \bbR.$ By a corollary in class, it suffices to show that the Fourier coefficients of $\overline{f}$ and $\hat{f}(n)$ are equal.
\begin{align*}(\overline{f}, e_n) &= \frac{1}{2\pi}\int_{-\pi}^\pi \overline{f(x)} e^{-inx} \, dx\\
&= \frac{1}{2\pi}\int_{-\pi}^\pi \overline{f(x)e^{inx}}\, dx\\
&= \overline{\frac{1}{2\pi}\int_{-\pi}^\pi {f(x)e^{inx}}\, dx}\\
&= \overline{(f,e_{-n})}\\
&= \overline{\hat{f}(-n)}\\
&= \overline{\overline{\hat{f}(n)}}\\
&= \hat{f}(n)
\end{align*}
\end{solution}
    \end{enumerate}
\end{problem}

\newpage
\section*{Problem 2}
\begin{problem}
    Let $f(x) = |x|$ be defined on $[-\pi, \pi].$
    \begin{enumerate}
        \item Calculate $\hat{f}(0).$
        \begin{solution}
            \item We have that 
            \[\hat{f}(0) = (f,e_0)= \frac{1}{2\pi}\int_{-\pi}^\pi |x|\,dx = \frac{1}{\pi}\int_0^\pi x \, dx = \frac{\pi}{2}\]
        \end{solution}
        \item Calculate $\hat{f}(n)$ when $n \neq 0.$
        \begin{solution}
    We integrate by parts
\begin{align*}
    \hat{f}(n) &= (f,e_n)\\
    &= \frac{1}{2\pi}\int_{-\pi}^\pi |x|e^{-inx}\,dx\\
    &= \frac{1}{2\pi}\left[\int_0^\pi x e^{-inx}\,dx - \int_{-\pi}^0 x e^{-inx}\, dx\right].
\end{align*}
First term:
\begin{align*}
    \int_0^\pi x e^{-inx}\,dx &= \frac{-1}{in}e^{-inx}x\bigg|_0^{\pi} + \frac{1}{in}\int_0^\pi e^{-inx}\,dx\\
    &=\frac{-1}{in}e^{-i\pi n}\pi + \frac{1}{in}\frac{-1}{in}\left[e^{-in\pi} - 1\right]\\
    &= \frac{-e^{-i\pi n}\pi}{in} + \frac{e^{-in \pi} -1}{n^2}\\
    &= \frac{e^{-i\pi n} i \pi n}{n^2} + \frac{e^{-in \pi} -1}{n^2}\\
    &= \frac{-1 + e^{-i\pi n}(1 + i\pi n)}{n^2}
\end{align*}
With similar algebra, we see that 
\[\int_{-\pi}^0 x e^{-inx} = \frac{1 + e^{i\pi n}(-1 + i\pi n)}{n^2}\]

Thus, 
\begin{align*}
    \hat{f}(n) &= \frac{1}{2\pi}\left[\int_0^\pi x e^{-inx}\,dx - \int_{-\pi}^0 x e^{-inx}\, dx\right].\\
    &= \frac{1}{2\pi}\left[\frac{-1 + e^{-i\pi n}(1 + i\pi n)}{n^2} - \frac{1 + e^{i\pi n}(-1 + i\pi n)}{n^2}\, dx\right]\\
    &= \frac{1}{2\pi}\left[\frac{-2 + e^{-i\pi n}(1 + i\pi n) - e^{i\pi n}(-1 + i\pi n)}{n^2}\right]\\
    &= \frac{1}{2\pi}\left[\frac{-2 + (e^{-i\pi n} + e^{ i\pi n}) + i\pi n(e^{-i\pi n} - e^{i\pi n})}{n^2}\right]
\end{align*}
        \end{solution}
\item Calculate the Fourier Series in terms of sines and cosines.
\begin{solution}
    By the first question, since $|x|$ is even, we will have a Fourier series of the form 
    \[S_N(f) = \hat{f}(0) + 2\sum_{n=1}^N \hat{f}(n)\cos(nx).\] From part (b), we have that for $n \neq 0,$
    \begin{align*}
        \hat{f}(n) &= \frac{1}{2\pi}\left[\frac{-2 + (e^{-i\pi n} + e^{ i\pi n}) + i\pi n(e^{-i\pi n} - e^{i\pi n})}{n^2}\right]
    \end{align*}
    We have that using properties of $\sin$ and $\cos,$ 
    \begin{align*}
       e^{-i\pi n} + e^{ i\pi n} = \cos(-\pi n) + i\sin(-\pi n) +  \cos(\pi n) + i\sin(\pi n) = 2\cos(\pi n) 
    \end{align*}
    \begin{align*}
        e^{-i\pi n} - e^{ i\pi n} = \cos(-\pi n) + i\sin(-\pi n) - \cos(\pi n) - i\sin(\pi n) = -2i\sin(\pi n) = 0
    \end{align*}
    Plugging back in:
        \begin{align*}
        \hat{f}(n) &= \frac{1}{2\pi}\left[\frac{-2 + (e^{-i\pi n} + e^{ i\pi n}) + i\pi n(e^{-i\pi n} - e^{i\pi n})}{n^2}\right]\\
        \ &= \frac{1}{2\pi}\left[\frac{-2 + 2\cos(\pi n)}{n^2}\right]\\
        &= \frac{-1 + \cos(\pi n)}{\pi n^2}
    \end{align*}
    Thus, 
\begin{align*}
    S_N(f) &= \hat{f}(0) + 2\sum_{n=1}^N \hat{f}(n)\cos(nx)\\
    &= \frac{\pi}{2} + 2\sum_{n=1}^N\frac{-1 + \cos(\pi n)}{\pi n^2} \cos(nx)
\end{align*}
\end{solution}
\item 
Deduce that $\sum_{n=1}^\infty \frac{1}{n^2} = \frac{\pi^2}{6}$
   \begin{solution}
       \begin{lemma}

    Suppose that for $t\in (-\delta, \delta),$ there exists a $C\in \bbR$ such that $|f(x-t) - f(x)|\leq C(t),$ (locally Lipshitz), then $S_n(x)\to f(x).$
\end{lemma}
\begin{proof}
    Define 
    \[g(t) = \frac{f(x-t) - f(x)}{\sin(\frac{t}{2})},\] and define
    \[D_N(x) = \sum_{-N}^N e^{inx} = \frac{\sin(N + \frac{1}{2})x}{\sin(\frac{x}{2})}.\]

    Then 
     \begin{align*}
     S_N(f,x) &= \sum_{-N}^N \frac{1}{2\pi}\int_{-\pi}^\pi f e^{-int} e^{inx}\,dt\\
     &= \frac{1}{2\pi0}f(t)\sum_{-N}^N e^{in(x-t)}dt\\
     &= \frac{1}{2\pi}\int_{-\pi}^\pi f(t)D_N(x-t)dt\\
     &= \frac{1}{2\pi}\int_{-\pi}^\pi f(x-t)D_N(t)dt\\
     &= \frac{1}{2\pi}(f \ast D_N)(t)
 \end{align*}
Thus,
    \begin{align*}
        |S_n(x) - f(x)| &= \frac{1}{2\pi}\int_{-\pi}^\pi (f(x-t) - f(x))D_N(t)dt\\
        &= \frac{1}{2\pi}\int_{-\pi}^\pi g(t)\sin(Nt + \frac{t}{2})dt\\
        &= \frac{1}{2\pi}\int_{-\pi}^\pi g(t)\cos(\frac{t}{2})\sin(Nt)dt + \frac{1}{2\pi}\int_{-\pi}^\pi g(t)\sin(\frac{t}{2})\cos(Nt)dt\\
        &\to 0.
    \end{align*}
    The last equality holds because $g(t)$ is bounded and because $|\hat{f}(n)|\to 0$ by Bessel's inequality and thus both the real and imaginary components of $\hat{f}(n)$ go to $0.$
\end{proof}
\rule{\linewidth}{0.4pt}
We first show that $|x|$ is locally Lipshitz around $x = 0.$ Let $t\in (-\delta, \delta),$ then 
\[|f(0 - t) - f(0)| = |t|  = C(t).\]Thus, by our lemma, $S_N(f,0)\to f(0) = 0,$ thus, 
\begin{align*}
0 &= \frac{\pi}{2} + 2\sum_{n=1}^\infty \frac{-1 + \cos(\pi n)}{\pi n^2}\cos(n (0))\\
&= \frac{\pi}{2} + \frac{2}{\pi}\sum_{n=1}^\infty \frac{-1 + (-1)^n}{n^2}\\
&= \frac{\pi}{2} - \frac{4}{\pi}\sum_{n \text{ odd}}^\infty \frac{1}{n^2}
\end{align*}
Thus, \[\sum_{n \text{ odd}}^\infty \frac{1}{n^2} = \frac{\pi^2}{8}.\] Thus, 
\[\sum_{n=1}^\infty \frac{1}{n^2}= \sum_{n=1}^\infty \frac{1}{(2n)^2} + \frac{\pi^2}{8} = \frac{1}{4}\sum_{n=1}^\infty \frac{1}{n^2} + \frac{\pi^2}{8},\] and thus 
\[\frac{3}{4}\sum_{n=1}^\infty \frac{1}{n^2} = \frac{\pi^2}{8} \implies \sum_{n=1}^\infty \frac{1}{n^2} = \frac{\pi^2}{6}.\]
   \end{solution}   
\begin{reflection}
    In this problem, we actually have stronger convergence. In fact, we have that $S_N(f)\uconv f.$ To see this, it suffices to note that $f$ is continous, $2\pi-$periodic on $[-\pi, \pi],$ and $S_N(f)$ converges absolutely since 
    \[\|S_N(f)\| \leq \frac{\pi}{2} + 2\sum_{n=1}^N \frac{2}{\pi n^2} < \infty\]
\end{reflection}
    \end{enumerate}
\end{problem}



\newpage
\section*{Problem 3}
\begin{problem}
    Show that in $\cal R,$ the space of $2\pi-$integrable functions, the Pythagorean theorem, the C-S inequality, and the triangle inequality all hold.
\end{problem}
\begin{solution}
    (Pythagorean Theorem) Let $f,g \in \cal R$ such that $f \perp g.$  Then 
    \begin{align*}
        \|f + g\|^2 &= (f + g, f + g)\\
        &= (f,f + g) + (g, f + g)\\
        &= (f,f) + (f,g) + (g,f) + (g,g)\\
        &= \|f\|^2 + 0 + 0 + \|g\|^2\\
        &= \|f\|^2 + \|g\|^2
    \end{align*}


    (H\"older's Inequality). Let $p,q \in \bbR$ such that $\frac{1}{p} + \frac{1}{q} = 1.$ Consider the degenerate case when 
    \[\int_{-\pi}^\pi |f|^p = \int_{-\pi}^\pi |g|^q = 1.\] Note that for any $x\in [-\pi, \pi],$ we have that by properties of the logarithm and by its convexity,
    \[\log(|f(x)| |g(x)|) = \frac{1}{p}\log(f(x)^p) + \frac{1}{q}\log(g(x)^q) \leq \log\left(\frac{1}{p}f(x)^p + \frac{1}{q}g(x)^q\right).\] Since the logarithm function is monotonically increasing, we have that since the integral is monotonic,
    \[|f(x)||g(x)| \leq \frac{1}{p}|f(x)|^p + \frac{1}{q}|g(x)|^q \implies \int_{-\pi}^\pi |f||g| \leq \frac{1}{p}\int_{-\pi}^\pi f^p + \frac{1}{q}\int_{-\pi}^\pi g^q = \frac{1}{p} + \frac{1}{q} = 1.\] Thus, in the degenerate case,
    \begin{align}
        \int_{-\pi}^\pi |f||g| \leq 1. 
    \end{align}
    Now for the general case. Define 
    \[f^*:= \frac{f}{\left(\int_{-\pi}^\pi |f|^p\right)^{\frac{1}{p}}} \implies \int_{-\pi}^\pi |f^*|^p = \int_{-\pi}^\pi \frac{|f|^p}{\bigg(\left(\int_{-\pi}^\pi |f|^p\right)^\frac{1}{p}\bigg)^p} = 1\]
    \[g^*:=\frac{g}{\int_{-\pi}^\pi |g|^q} \implies \int_{-\pi}^\pi |g^*|^q = 1.\] By our degenerate case, we know that 
    \begin{align*}
        1 &\geq \int_{-\pi}^\pi |f^*||g^*|\\
        &= \int_{-\pi}^\pi \frac{|f|\cdot |g|}{\left(\int_{-\pi}^\pi |f|^p\right)^\frac{1}{p} \cdot \left(\int_{-\pi}^\pi |g|^p\right)^\frac{1}{q}},
    \end{align*}
and so 
\[\int_{-\pi}^\pi|f|\cdot|g| \leq \left(\int_{-\pi}^\pi |f|^p\right)^\frac{1}{p}\left(\int_{-\pi}^\pi |g|^p\right)^\frac{1}{q}.\] Thus, letting $p = q = 2,$ we have that 
\begin{align*}
    |(f,g)| &= \left|\int_{-\pi}^\pi f \cdot \overline{g}\right| \\
    &\leq \int_{-\pi}^\pi |f|\cdot |\overline{g}|\\
    &=\int_{-\pi}^\pi |f|\cdot |g|\\
    &\leq \left(\int_{-\pi}^\pi |f|^2\right)^\frac{1}{2}\left(\int_{-\pi}^\pi |g|^2\right)^\frac{1}{2}\\
    &= \|f|| \cdot \|g\|
\end{align*}

(Triangle Inequality) 
We have that for any $f,g,h \in \cal R,$
\begin{align*}
    \|f - g\| &= \int_{-\pi}^\pi |f - g|\\
    &= \int_{-\pi}^\pi |(f - h) + (h -g)|\\
    &\leq \int_{-\pi}^\pi |f - h| + |h - g|\\
    &= \int_{-\pi}^\pi |f - h| + \int_{-\pi}^\pi |h-g|\\
    &= \|f - h\| + \|h - g\|
\end{align*}
A more general proof:
\begin{align*}
\|f + g\|^2 &= (f + g, f + g)\\ &= (f,f) + (f,g)+ (g,f) + (g,g)\\ &\leq \|f\|^2 + \|g\|^2 + 2|(f,g)|\\ &\leq \|f\|^2 + \|g\|^2  + 2\|f\|    \|g\|\\
&= (\|f\| + \|g\|)^2.
\end{align*}
Take square roots of both sides and conclude.

\end{solution}


\newpage
\section*{Problem 4}
\begin{problem}
    Find the values of 
    \[\sum_{n=1}^\infty \frac{1}{(2n + 1)^4}, \qquad \sum_{n=1}^\infty \frac{1}{n^4}\]
\end{problem}
\begin{solution}
Parseval's Theorem states that 
\[\lim_{n\to \infty}(S_n(f))^2 = \frac{1}{2\pi}\int_{-\pi}^\pi |f|^2.\] Using Problem (1), we see that since 
\[(S_n(f))^2 = \frac{\pi^2}{4} + \frac{8}{\pi^2}\sum_{n=1}^\infty \frac{1}{(2n + 1)^4} \] and 
\[\frac{1}{2\pi}\int_{-\pi}^\pi |x|^2 \,dx = \frac{1}{2\pi}\int_{-\pi}^\pi x^2  = \frac{\pi^2}{3}\] Thus, we have that 
\[\frac{\pi^2}{3} = \frac{\pi^2}{4} + \frac{8}{\pi^2}\sum_{n=1}^\infty \frac{1}{(2n + 1)^4} \iff \frac{\pi^2}{96} = \sum_{n=1}^\infty \frac{1}{(2n + 1)^4}.\]

Moreover, we know that 
\[\sum_{n=1}^\infty \frac{1}{n^4} = \sum_{n=1}^\infty \frac{1}{(2n)^4} + \sum_{n=1}^\infty \frac{1}{(2n + 1)^4} = \frac{1}{16}\sum_{n=1}^\infty + \frac{\pi^2}{96}\] and so 
\[\frac{15}{16}\sum_{n=1}^\infty \frac{1}{n^4} = \frac{\pi^2}{96} \implies \sum_{n=1}^\infty \frac{1}{n^4} = \frac{\pi^2}{90}\]
\end{solution}

\end{document}