\documentclass[11pt]{article}

% NOTE: Add in the relevant information to the commands below; or, if you'll be using the same information frequently, add these commands at the top of paolo-pset.tex file. 
\newcommand{\name}{Agustín Esteva}
\newcommand{\email}{aesteva@uchicago.edu}
\newcommand{\classnum}{20510}
\newcommand{\subject}{Accelerated Analysis in $\bbR^n$ III}
\newcommand{\instructors}{Don Stull}
\newcommand{\assignment}{Problem Set 7}
\newcommand{\semester}{Spring 2025}
\newcommand{\duedate}{05-14-2025}
\newcommand{\bA}{\mathbf{A}}
\newcommand{\bB}{\mathbf{B}}
\newcommand{\bC}{\mathbf{C}}
\newcommand{\bD}{\mathbf{D}}
\newcommand{\bE}{\mathbf{E}}
\newcommand{\bF}{\mathbf{F}}
\newcommand{\bG}{\mathbf{G}}
\newcommand{\bH}{\mathbf{H}}
\newcommand{\bI}{\mathbf{I}}
\newcommand{\bJ}{\mathbf{J}}
\newcommand{\bK}{\mathbf{K}}
\newcommand{\bL}{\mathbf{L}}
\newcommand{\bM}{\mathbf{M}}
\newcommand{\bN}{\mathbf{N}}
\newcommand{\bO}{\mathbf{O}}
\newcommand{\bP}{\mathbf{P}}
\newcommand{\bQ}{\mathbf{Q}}
\newcommand{\bR}{\mathbf{R}}
\newcommand{\bS}{\mathbf{S}}
\newcommand{\bT}{\mathbf{T}}
\newcommand{\bU}{\mathbf{U}}
\newcommand{\bV}{\mathbf{V}}
\newcommand{\bW}{\mathbf{W}}
\newcommand{\bX}{\mathbf{X}}
\newcommand{\bY}{\mathbf{Y}}
\newcommand{\bZ}{\mathbf{Z}}
\newcommand{\Vol}{\text{Vol}}

%% blackboard bold math capitals
\newcommand{\bbA}{\mathbb{A}}
\newcommand{\bbB}{\mathbb{B}}
\newcommand{\bbC}{\mathbb{C}}
\newcommand{\bbD}{\mathbb{D}}
\newcommand{\bbE}{\mathbb{E}}
\newcommand{\bbF}{\mathbb{F}}
\newcommand{\bbG}{\mathbb{G}}
\newcommand{\bbH}{\mathbb{H}}
\newcommand{\bbI}{\mathbb{I}}
\newcommand{\bbJ}{\mathbb{J}}
\newcommand{\bbK}{\mathbb{K}}
\newcommand{\bbL}{\mathbb{L}}
\newcommand{\bbM}{\mathbb{M}}
\newcommand{\bbN}{\mathbb{N}}
\newcommand{\bbO}{\mathbb{O}}
\newcommand{\bbP}{\mathbb{P}}
\newcommand{\bbQ}{\mathbb{Q}}
\newcommand{\bbR}{\mathbb{R}}
\newcommand{\bbS}{\mathbb{S}}
\newcommand{\bbT}{\mathbb{T}}
\newcommand{\bbU}{\mathbb{U}}
\newcommand{\bbV}{\mathbb{V}}
\newcommand{\bbW}{\mathbb{W}}
\newcommand{\bbX}{\mathbb{X}}
\newcommand{\bbY}{\mathbb{Y}}
\newcommand{\bbZ}{\mathbb{Z}}

%% script math capitals
\newcommand{\sA}{\mathscr{A}}
\newcommand{\sB}{\mathscr{B}}
\newcommand{\sC}{\mathscr{C}}
\newcommand{\sD}{\mathscr{D}}
\newcommand{\sE}{\mathscr{E}}
\newcommand{\sF}{\mathscr{F}}
\newcommand{\sG}{\mathscr{G}}
\newcommand{\sH}{\mathscr{H}}
\newcommand{\sI}{\mathscr{I}}
\newcommand{\sJ}{\mathscr{J}}
\newcommand{\sK}{\mathscr{K}}
\newcommand{\sL}{\mathscr{L}}
\newcommand{\sM}{\mathscr{M}}
\newcommand{\sN}{\mathscr{N}}
\newcommand{\sO}{\mathscr{O}}
\newcommand{\sP}{\mathscr{P}}
\newcommand{\sQ}{\mathscr{Q}}
\newcommand{\sR}{\mathscr{R}}
\newcommand{\sS}{\mathscr{S}}
\newcommand{\sT}{\mathscr{T}}
\newcommand{\sU}{\mathscr{U}}
\newcommand{\sV}{\mathscr{V}}
\newcommand{\sW}{\mathscr{W}}
\newcommand{\sX}{\mathscr{X}}
\newcommand{\sY}{\mathscr{Y}}
\newcommand{\sZ}{\mathscr{Z}}


\renewcommand{\emptyset}{\O}

\newcommand{\abs}[1]{\lvert #1 \rvert}
\newcommand{\norm}[1]{\lVert #1 \rVert}
\newcommand{\sm}{\setminus}


\newcommand{\sarr}{\rightarrow}
\newcommand{\arr}{\longrightarrow}

% NOTE: Defining collaborators is optional; to not list collaborators, comment out the line below.
%\newcommand{\collaborators}{Alyssa P. Hacker (\texttt{aphacker}), Ben Bitdiddle (\texttt{bitdiddle})}

\input{paolo-pset.tex}

% NOTE: To compile a version of this pset without problems, solutions, or reflections, uncomment the relevant line below.

%\excludeversion{problem}
%\excludeversion{solution}
%\excludeversion{reflection}

\begin{document}	
	
	% Use the \psetheader command at the beginning of a pset. 
	\psetheader
    
    
    \section*{Problem 1}
Let $E\subseteq\bbR^n$ and $F \subseteq \bbR^\ell.$ Suppose $T \in C^1(E,F).$ Let $\omega \in \Lambda^k(F)$ and $\lambda \in \Lambda^m(F).$ Prove that 
\[(\omega \wedge \lambda)_T = \omega_T \wedge \lambda_T\]
\begin{solution}
We remark that by definition, $(dx_{i})_T = dt_i$
    Suppose $r = 1,$ then by definition 
    \[(dx_1)_T = (dt_1).\]
    Suppose that we preserve order in the pullback when $r = n,$ i.e,
    \[(dx_{i_1} \wedge \cdots \wedge dx_{i_n})_T  = (dx_{i_1})_T \wedge \cdots \wedge (dx_{i_n})_T  = dt_{i_1}\wedge \cdots \wedge dt_{i_n}\]
    Now for $r = n+1,$ we have that (if we denote $dx_I$ to be the standard presentation of the form), then if $\alpha$ is the number of permutations necessary to make $i_1, \dots, i_{n+1}$ into the standard presentation $I'= i_{1}', \dots, i'_{n+1}$
    \begin{align*}
        (dx_{i_1}\wedge \dots \wedge dx_{i_n} \wedge dx_{i_{n+1}})_T &= (-1^\alpha dx_{I'})_T\\
        &= (-1)^\alpha (dx_{I'})_T\\
        &= (-1)^\alpha \left(dt_{i'_1} \wedge \cdots \wedge dt_{i'_{n+1}}\right)\\
        &= dt_{i_1} \wedge \cdots \wedge dt_{i_n} \wedge dt_{i_{n+1}}\\
        &= (dt_{i_1} \wedge \cdots \wedge dt_{i_n}) \wedge dt_{i_{n+1}}\\
        &= (dt_{i_1} \wedge \cdots \wedge dt_{i_n}) \wedge (dx_{i_{n+1}})_T\\
        &= (dx_{i_1} \wedge \cdots \wedge dx_{i_n})_T \wedge (dx_{i_{n+1}})_T\\
        &= (dx_{i_1})_T \wedge \cdots \wedge (dx_{i_n})_T\wedge (dx_{i_{n+1}})_T\\
    \end{align*}

Consider first the case when 
\[\omega = f\,dx_I = f\,dx_{i_1}\wedge \cdots \wedge dx_{i_k}, \qquad \lambda = g\,dx_J = g\,dx_{j_1}\wedge \cdots \wedge dx_{j_m}.\] Then by the lemma, 
\begin{align*}
(\omega\wedge\lambda)_T &=  (fg\,dx_{i_1}\wedge \cdots \wedge dx_{i_k}\wedge dx_{j_{1}}\wedge \cdots \wedge dx_{j_{m}})_T\\
&= fg(T(\textbf{x}))\,dt_{i_1}\wedge \cdots \wedge dt_{i_k}\wedge dt_{j_{1}}\wedge \cdots \wedge dt_{j_{m}}\\
&= (f(T(\textbf{x}))\,dt_{i_1}\wedge \cdots \wedge dt_{i_k}) \wedge (g(T(\textbf{x}))\,dt_{j_{1}}\wedge \cdots \wedge dt_{j_{m}})\\
&= (f\,dx_{i_1}\wedge\cdots\wedge dx_{i_k})_T\wedge (g\,dx_{j_{1}}\wedge \cdots \wedge dx_{j_{m}})_T\\
&= \omega_T \wedge \lambda_T
\end{align*}
Now consider general $\omega$ and $\lambda.$ We can express 
\[\omega = \sum_I f_I dx_I, \qquad \lambda = \sum_J g_J dx_J.\] We have that using part (a) of the Theorem,
\begin{align*}
    \omega_T \wedge \lambda_T &= (\sum_I f_I dx_I)_T \wedge (\sum_J g_J dx_J)_T\\
    &= \sum_I (f_I dx_I)_T  \wedge \sum_J (g_J dx_J)_T\\
    &= \sum_{I,J}(f_Idx_I)_T\wedge (g_J dx_J)_T \\
    &= \sum_{I,J} (f_I g_J dx_I \wedge dx_J)_T\\
    &= (\omega \wedge \lambda)_T
\end{align*}


\end{solution}

\newpage

\section*{Problem 2}

Let $\omega$ be a 1-form on $\mathbb{R}^n$ and let $\gamma : [a, b] \to \mathbb{R}^n$ be a $C^1$ curve. Let $\Delta : [a, b] \to [a, b]$ be the identity function (which is a curve in $\mathbb{R}^1$). Prove that

\[
\int_{\gamma} \omega = \int_{\Delta} \omega_{\gamma}.
\]

Do not just apply Theorem 10.24 or 10.25, please give a direct proof.
\begin{solution}
Suppose first $\omega = \sum_I f_I dx_I$ Then by definition of integrating $k-$forms,
    \begin{align*}
        \int_\gamma \omega &= \int_a^b  \sum_{I}f_I(\gamma(x))dy_I(\gamma_1'(x), \dots,\gamma_n'(x))\,dx\\
        &= \int_a^b \sum_{I} f_I(\gamma(x))\gamma_I'(x)\,dx\\
        &= \int_a^b \sum_If_I(\gamma(x))dt_I\,dx\\
        &=\int_a^b \omega_\gamma(x)\,dx\\
        &= \int_a^b \omega_\gamma(\Delta(x)) \Delta'(x)\,dx\\
        &= \int_\Delta \omega_\gamma.
    \end{align*}
    For general $\omega= \sum_I f_I dx_I,$ we use the linearity of the integral to conclude. 
\end{solution}


\newpage

\section*{Problem 3}

Define the forms

\begin{align*}
\omega_1 &= x \, dx - y \, dy \\
\omega_2 &= z \, dx \wedge dy + x \, dy \wedge dz \\
\omega_3 &= z \, dy.
\end{align*}

Compute $\omega_1 \wedge \omega_2$, $\omega_1 \wedge \omega_3$ and $\omega_2 \wedge \omega_3$. Write all forms in standard presentation.
\begin{solution}
    \begin{itemize}
        \item To compute $\omega_1 \wedge \omega_2,$ we see that 
\begin{align*}
    \omega_1 \wedge \omega_2 &= (x \, dx - y \, dy )\wedge ( z \, dx \wedge dy + x \, dy \wedge dz)\\
    &= xz\, dx \wedge dx \wedge dy+ x^2 \, dx \wedge dy\wedge dz - yz \,dy \wedge dx \wedge dy - yx \, dy \wedge dy\wedge dz\\
    &= \boxed{x^2 \,dx\wedge dy\wedge dz}
\end{align*}
\item For $\omega_1 \wedge \omega_3,$ we compute
\begin{align*}
    \omega_1 \wedge \omega_3 &= (x\, dx - y \, dy) \wedge z\, dy\\
    &= xz \, dx \wedge dy - yz\, dy \wedge dy\\
    &= \boxed{xz \, dx \wedge dy}
\end{align*}
\item For $\omega_2 \wedge \omega_3,$ we compute
\begin{align*}
    \omega_2 \wedge \omega_4 &= (z\, dx \wedge dy  + x\,dy \wedge dz) \wedge z\,dy\\
    &= z^2\, dx \wedge dy\wedge dy + xy\, dy \wedge dz \wedge dy\\
    &= \boxed{0}
\end{align*}
    \end{itemize}
\end{solution}



\newpage

\section*{Problem 4}

Let $\omega = xy \, dx \wedge dz + z \, dx \wedge dy$ be a 2-form in $\mathbb{R}^3$. Compute $d\omega$.
\begin{solution}
    Computing, 
    \begin{align*}
        d(\omega) &= d\left(xy \, dx \wedge dz + z \,dx \wedge dy\right)\\
        &= y\,dx \wedge dx\wedge dz + 0 + x\,dy\wedge dx\wedge dz + 0 + \,dz \wedge dx \wedge dy\\
        &= -x \,dx \wedge dy \wedge dz + dx \wedge dy \wedge dz\\
        &= (1-x)\,dx \wedge dy \wedge dz
    \end{align*}
\end{solution}
\newpage

\section*{Problem 5}

Let $T : \mathbb{R}^3 \to \mathbb{R}^3$ be the function defined by $T (x, y, z) = (xy, xz, yz)$. Find the following forms:

(a) $(dx)_T$, $(dy)_T$ and $(dz)_T$.

(b) $(dx \wedge dy)_T$

(c) $(dx \wedge dy \wedge dz)_T$.

Write all forms in standard presentation.
\begin{solution}
We consider 
\[J = \begin{pmatrix}
    y & x & 0\\
    z & 0 & x\\
    0 & z & y
\end{pmatrix}\] and we can immediately see
\[dt_1 = ydx + xdy\]
\[dt_2 = zdx + xdz\]
\[dt_3 = zdy + ydz\]
    \begin{enumerate}
        \item We have by definition
        \[(dx)_T = dt_1 = ydx + xdy\]
        \[(dy)_T = dt_2 = zdx + xdz\]
        \[(dz)_T = dt_3 = zdy + ydz\]
        \item The pullback is distributive, so
        \[(dx \wedge dy)_T = (dx)_T \wedge (dy)_T = (y\, dx + x\,dy) \wedge (z\,dx + x\,dz) = -xz \,dx \wedge dy + xy\, dx\wedge dz + x^2 \, dy\wedge dz\]
        \item Similarly to (b) but just more annoying
        \begin{align*}
            (dx \wedge dy \wedge dz)_T &= (dx \wedge dy )_T\wedge (dz)_T\\
            &= (-xz \,dx \wedge dy + xy\, dx\wedge dz + x^2 \, dy\wedge dz)\wedge z\,dy + y\,dz\\
            &= -xyz\,dx \wedge dy\wedge dz - xyz\, dx\wedge dy\wedge dz\\
            &= -2xyz\, dx\wedge dy\wedge dz
        \end{align*}
    \end{enumerate}
\end{solution}

\newpage

\section*{Problem 6}

Let $T (r, \theta, \phi) = (r \cos \theta \sin\phi, r \sin \theta \sin\phi, r \cos\phi)$ (this function gives the spherical coordinates of $\mathbb{R}^3$). Calculate $\omega_T$ for each of the following forms $\omega$:

$dx$, $dy$, $dz$, $dx \wedge dy$, $dx \wedge dz$, $dy \wedge dz$, $dx \wedge dy \wedge dz$.
\begin{solution}
    Holy moly. It is a lot easier if we just look at the Jacobian, which is given by 
    \[J_T = \begin{pmatrix}
    \cos \theta \sin\phi & -r\sin \theta \sin \phi & r\cos \theta \cos \phi\\\sin \theta \sin \phi & r\cos \theta \sin \phi & r\sin \theta \cos \phi\\\cos \phi & 0 & -r\sin \phi
    \end{pmatrix}\]
    \begin{align*}
        (dx)_T &= d(T_x) = \cos \theta \sin\phi \,dr -r\sin \theta \sin\phi \,d\theta + r\cos \theta \cos \phi \, d\phi\\
        (dy)_T &= d(T_y) = \sin \theta \sin \phi \,d\theta +r\cos \theta \sin \phi\,d\theta + r\sin \theta \cos \phi\,d\theta\\
        (dz)_T &= d(T_z) = \cos \phi \,dr - r\sin \phi \, d\phi\\
    \end{align*}
    We do not show our work for the following, but we make a lot of use of the fact that wedge products are zero when indices are shared.
    \begin{align*}
        (dx \wedge dy)_T &= (dx)_T \wedge (dy)_T =  \\
        &= r\sin ^2 \phi dr \wedge d\theta - r^2 \sin \phi \cos \phi d\theta \wedge d\phi\\
        (dx \wedge dz)_T &= (dx)_T \wedge (dz)_T \\
        &= (r\sin \theta \sin \phi \cos \phi)dr\wedge d\theta - r\cos \theta dr \wedge d\phi + r^2 \sin \theta \sin^2 \phi d\theta \wedge d\phi\\
        (dy \wedge dz)_T &= (dy)_T \wedge (dz)_T\\
        &= -r\cos \theta \sin \phi \cos \phi dr \wedge d\theta - r\sin \theta dr\wedge d\phi - r^2 \cos \theta\sin^2 \phi d\theta \wedge d\phi\\
        (dx \wedge dy \wedge dz)_T &= (dx\wedge dy)_T \wedge dz = -r^2 \sin \phi dr\wedge d\theta \wedge d\phi
    \end{align*}
\end{solution}
\newpage

\section*{Problem 7}

Consider the 2-form $dx \wedge dy$ in $\mathbb{R}^2$. Find all linear maps $T : \mathbb{R}^2 \to \mathbb{R}^2$ such that $\omega_T = \omega$.
\begin{solution}
    Since $T: \bbR^2 \to \bbR^2$ is linear, then 
    \[T= \begin{pmatrix}
        a & b\\ c & d
    \end{pmatrix} = \begin{pmatrix}
        ax + by\\
        cx + dy
    \end{pmatrix}\] We see the Jacobian is given by 
    \[J = \begin{pmatrix}
        a & b\\
        c & d
    \end{pmatrix}\] and thus
    \begin{align*}
    (dx\wedge dy)_T &= d(T_x) \wedge d(T_y)\\&= (a\,dx + b\,dy)\wedge (c\,dx + d\,dy)\\
    &= (ad)\,dx\wedge dy + (cb)\,dy \wedge dx\\
    &= (ad -bc)\,dx\wedge dy\\
    &= dx\wedge dy
    \end{align*}
    Thus, the only linear map $T$ is one such that $ad - bc  =1.$
\end{solution}

\newpage

\section*{Problem 8}

Let $f : \mathbb{R}^n \to \mathbb{R}$ be $C^1$, and let $df$ be the 1-form which is the derivative of the 0-form $f$. For any curve $\gamma : [a, b] \to \mathbb{R}^n$, prove that

\[
\int_{\gamma} df = f(\gamma(b)) - f(\gamma(a)).
\]
\begin{solution}
    By definition, we have that 
    \begin{align*}
        \int_\gamma df &= \int_a^b df(\gamma(u))J(u)\,du\\
        &= \int_a^b \sum_{j=1}^n (D_jf)(\gamma(u))dx_j J(u)\,du\\
        &= \int_a^b \sum_{j=1}^n (D_jf(\gamma(u)))(\gamma_j'(u))\,du\\
        &= \int_a^b \langle\nabla f(\gamma(u)), \gamma'(u)\rangle\,du\\
        &= \int_a^b (f(\gamma(u)))'\,du\\
        &= f(\gamma(b)) - f(\gamma(a))
    \end{align*}
    Where the normal FTC was used in the last step.
\end{solution}


\newpage

\section*{Problem 9}

Let $\omega$ be the 1-form on $\mathbb{R}^2 \setminus \{0\}$ given by

\[
\omega = \frac{y \, dx - x \, dy}{x^2 + 4y^2}
\]

Let $\gamma : [0, 1] \to \mathbb{R}^2$ be the curve defined by $\gamma(t) = (2 \cos(2\pi t), \sin(2\pi t))$.

(a) Compute $d\omega$.
\begin{solution}
    \begin{align*}
        d\omega&= d\left(\frac{y}{x^2 + 4y^2}\,dx - \frac{x}{x^2 + 4y^2}\,dy\right)\\
        &= d\left(\frac{y}{x^2 + 4y^2}\, dx\right) - d\left(\frac{x}{x^2 + 4y^2}\,dy\right)\\
        &= \frac{(x^2 + 4y^2) - (y(8y))}{(x^2 + 4y^2)^2}dy \wedge dx - \frac{(x^2 + 4y^2) - (x(2x))}{(x^2 + 4y^2)^2}\,dx\wedge dy\\
        &= \frac{-x^2 + 4y^2}{(x^2 + 4y^2)^2}dx \wedge dy - \frac{-x^2 + 4y^2}{(x^2 + 4y^2)^2}\,dx\wedge dy\\
        &= 0
    \end{align*}
\end{solution}

(b) Compute $\int_{\gamma} \omega$.
\begin{solution}
Notice that 
\[\gamma'(x) = \begin{pmatrix}
    -4\pi\sin (2\pi t) & 2\pi \cos (2\pi t) 
\end{pmatrix}\]
    By definition,
\begin{align*}
    \int_\gamma \omega &= \int_0^1\omega(\gamma(t))\gamma'(t)\,dt\\
    &= \int_0^1 \frac{\sin(2\pi t))}{4 \cos^2(2\pi t)^2 + 4\sin^2(2\pi t)}(-4\pi\sin (2\pi t))- \int_0^1\frac{2 \cos(2\pi t)}{4 \cos^2(2\pi t)^2 + 4\sin^2(2\pi t)}(2\pi \cos (2\pi t))\\
    &= \int_0^1-\pi \sin^2(2\pi t)\,dt  - \int_0^1 \pi \cos^2(2\pi t)\,dt\\
    &= -\pi
\end{align*}
\end{solution}

(c) Is $\omega$ closed? Is it exact? (Hint: For exactness, use the previous Problem.)
\begin{solution}
    Part (a) shows that $\omega$ is closed. 

    Suppose $\omega$ is exact. Since $\omega$ is a one form, then there exists some $f\in C^1(\bbR^n, \bbR)$ such that $df = \omega.$ Since $(2, 0)=\gamma(1) = \gamma(0) = (2,0),$ we have by Problem 8 that 
    \[\int_\gamma df = f(\gamma(b)) - f(\gamma(a)) = 0.\] But by (b) we have that
    \[\int_\gamma df = \int_\gamma \omega \neq \pi.\] Thus, $\omega$ cannot be exact.
\end{solution}


\end{document}