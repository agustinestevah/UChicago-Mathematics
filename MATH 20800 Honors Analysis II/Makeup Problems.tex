\documentclass[11pt]{article}
\usepackage{float}
% NOTE: Add in the relevant information to the commands below; or, if you'll be using the same information frequently, add these commands at the top of paolo-pset.tex file. 
\newcommand{\name}{Agustín Esteva}
\newcommand{\email}{aesteva@uchicago.edu}
\newcommand{\classnum}{Marrs}
\newcommand{\subject}{Honors Marrs in $\bbR^n$}
\newcommand{\instructors}{Marrs}
\newcommand{\assignment}{Problem Set Marrs}
\newcommand{\semester}{Marrs 2024}
\newcommand{\duedate}{2024-13-Marrs}
\newcommand{\bA}{\mathbf{A}}
\newcommand{\bB}{\mathbf{B}}
\newcommand{\bC}{\mathbf{C}}
\newcommand{\bD}{\mathbf{D}}
\newcommand{\bE}{\mathbf{E}}
\newcommand{\bF}{\mathbf{F}}
\newcommand{\bG}{\mathbf{G}}
\newcommand{\bH}{\mathbf{H}}
\newcommand{\bI}{\mathbf{I}}
\newcommand{\bJ}{\mathbf{J}}
\newcommand{\bK}{\mathbf{K}}
\newcommand{\bL}{\mathbf{L}}
\newcommand{\bM}{\mathbf{M}}
\newcommand{\bN}{\mathbf{N}}
\newcommand{\bO}{\mathbf{O}}
\newcommand{\bP}{\mathbf{P}}
\newcommand{\bQ}{\mathbf{Q}}
\newcommand{\bR}{\mathbf{R}}
\newcommand{\bS}{\mathbf{S}}
\newcommand{\bT}{\mathbf{T}}
\newcommand{\bU}{\mathbf{U}}
\newcommand{\bV}{\mathbf{V}}
\newcommand{\bW}{\mathbf{W}}
\newcommand{\bX}{\mathbf{X}}
\newcommand{\bY}{\mathbf{Y}}
\newcommand{\bZ}{\mathbf{Z}}

%% blackboard bold math capitals
\newcommand{\bbA}{\mathbb{A}}
\newcommand{\bbB}{\mathbb{B}}
\newcommand{\bbC}{\mathbb{C}}
\newcommand{\bbD}{\mathbb{D}}
\newcommand{\bbE}{\mathbb{E}}
\newcommand{\bbF}{\mathbb{F}}
\newcommand{\bbG}{\mathbb{G}}
\newcommand{\bbH}{\mathbb{H}}
\newcommand{\bbI}{\mathbb{I}}
\newcommand{\bbJ}{\mathbb{J}}
\newcommand{\bbK}{\mathbb{K}}
\newcommand{\bbL}{\mathbb{L}}
\newcommand{\bbM}{\mathbb{M}}
\newcommand{\bbN}{\mathbb{N}}
\newcommand{\bbO}{\mathbb{O}}
\newcommand{\bbP}{\mathbb{P}}
\newcommand{\bbQ}{\mathbb{Q}}
\newcommand{\bbR}{\mathbb{R}}
\newcommand{\bbS}{\mathbb{S}}
\newcommand{\bbT}{\mathbb{T}}
\newcommand{\bbU}{\mathbb{U}}
\newcommand{\bbV}{\mathbb{V}}
\newcommand{\bbW}{\mathbb{W}}
\newcommand{\bbX}{\mathbb{X}}
\newcommand{\bbY}{\mathbb{Y}}
\newcommand{\bbZ}{\mathbb{Z}}

%% script math capitals
\newcommand{\sA}{\mathscr{A}}
\newcommand{\sB}{\mathscr{B}}
\newcommand{\sC}{\mathscr{C}}
\newcommand{\sD}{\mathscr{D}}
\newcommand{\sE}{\mathscr{E}}
\newcommand{\sF}{\mathscr{F}}
\newcommand{\sG}{\mathscr{G}}
\newcommand{\sH}{\mathscr{H}}
\newcommand{\sI}{\mathscr{I}}
\newcommand{\sJ}{\mathscr{J}}
\newcommand{\sK}{\mathscr{K}}
\newcommand{\sL}{\mathscr{L}}
\newcommand{\sM}{\mathscr{M}}
\newcommand{\sN}{\mathscr{N}}
\newcommand{\sO}{\mathscr{O}}
\newcommand{\sP}{\mathscr{P}}
\newcommand{\sQ}{\mathscr{Q}}
\newcommand{\sR}{\mathscr{R}}
\newcommand{\sS}{\mathscr{S}}
\newcommand{\sT}{\mathscr{T}}
\newcommand{\sU}{\mathscr{U}}
\newcommand{\sV}{\mathscr{V}}
\newcommand{\sW}{\mathscr{W}}
\newcommand{\sX}{\mathscr{X}}
\newcommand{\sY}{\mathscr{Y}}
\newcommand{\sZ}{\mathscr{Z}}


\renewcommand{\emptyset}{\O}

\newcommand{\abs}[1]{\lvert #1 \rvert}
\newcommand{\norm}[1]{\lVert #1 \rVert}
\newcommand{\sm}{\setminus}


\newcommand{\sarr}{\rightarrow}
\newcommand{\arr}{\longrightarrow}

% NOTE: Defining collaborators is optional; to not list collaborators, comment out the line below.
%\newcommand{\collaborators}{Alyssa P. Hacker (\texttt{aphacker}), Ben Bitdiddle (\texttt{bitdiddle})}

% Copyright 2021 Paolo Adajar (padajar.com, paoloadajar@mit.edu)
% 
% Permission is hereby granted, free of charge, to any person obtaining a copy of this software and associated documentation files (the "Software"), to deal in the Software without restriction, including without limitation the rights to use, copy, modify, merge, publish, distribute, sublicense, and/or sell copies of the Software, and to permit persons to whom the Software is furnished to do so, subject to the following conditions:
%
% The above copyright notice and this permission notice shall be included in all copies or substantial portions of the Software.
% 
% THE SOFTWARE IS PROVIDED "AS IS", WITHOUT WARRANTY OF ANY KIND, EXPRESS OR IMPLIED, INCLUDING BUT NOT LIMITED TO THE WARRANTIES OF MERCHANTABILITY, FITNESS FOR A PARTICULAR PURPOSE AND NONINFRINGEMENT. IN NO EVENT SHALL THE AUTHORS OR COPYRIGHT HOLDERS BE LIABLE FOR ANY CLAIM, DAMAGES OR OTHER LIABILITY, WHETHER IN AN ACTION OF CONTRACT, TORT OR OTHERWISE, ARISING FROM, OUT OF OR IN CONNECTION WITH THE SOFTWARE OR THE USE OR OTHER DEALINGS IN THE SOFTWARE.

\usepackage{fullpage}
\usepackage{enumitem}
\usepackage{amsfonts, amssymb, amsmath,amsthm}
\usepackage{mathtools}
\usepackage[pdftex, pdfauthor={\name}, pdftitle={\classnum~\assignment}]{hyperref}
\usepackage[dvipsnames]{xcolor}
\usepackage{bbm}
\usepackage{graphicx}
\usepackage{mathrsfs}
\usepackage{pdfpages}
\usepackage{tabularx}
\usepackage{pdflscape}
\usepackage{makecell}
\usepackage{booktabs}
\usepackage{natbib}
\usepackage{caption}
\usepackage{subcaption}
\usepackage{physics}
\usepackage[many]{tcolorbox}
\usepackage{version}
\usepackage{ifthen}
\usepackage{cancel}
\usepackage{listings}
\usepackage{courier}

\usepackage{tikz}
\usepackage{istgame}

\hypersetup{
	colorlinks=true,
	linkcolor=blue,
	filecolor=magenta,
	urlcolor=blue,
}

\setlength{\parindent}{0mm}
\setlength{\parskip}{2mm}

\setlist[enumerate]{label=({\alph*})}
\setlist[enumerate, 2]{label=({\roman*})}

\allowdisplaybreaks[1]

\newcommand{\psetheader}{
	\ifthenelse{\isundefined{\collaborators}}{
		\begin{center}
			{\setlength{\parindent}{0cm} \setlength{\parskip}{0mm}
				
				{\textbf{\classnum~\semester:~\assignment} \hfill \name}
				
				\subject \hfill \href{mailto:\email}{\tt \email}
				
				Instructor(s):~\instructors \hfill Due Date:~\duedate	
				
				\hrulefill}
		\end{center}
	}{
		\begin{center}
			{\setlength{\parindent}{0cm} \setlength{\parskip}{0mm}
				
				{\textbf{\classnum~\semester:~\assignment} \hfill \name\footnote{Collaborator(s): \collaborators}}
				
				\subject \hfill \href{mailto:\email}{\tt \email}
				
				Instructor(s):~\instructors \hfill Due Date:~\duedate	
				
				\hrulefill}
		\end{center}
	}
}

\renewcommand{\thepage}{\classnum~\assignment \hfill \arabic{page}}

\makeatletter
\def\points{\@ifnextchar[{\@with}{\@without}}
\def\@with[#1]#2{{\ifthenelse{\equal{#2}{1}}{{[1 point, #1]}}{{[#2 points, #1]}}}}
\def\@without#1{\ifthenelse{\equal{#1}{1}}{{[1 point]}}{{[#1 points]}}}
\makeatother

\newtheoremstyle{theorem-custom}%
{}{}%
{}{}%
{\itshape}{.}%
{ }%
{\thmname{#1}\thmnumber{ #2}\thmnote{ (#3)}}

\theoremstyle{theorem-custom}

\newtheorem{theorem}{Theorem}
\newtheorem{lemma}[theorem]{Lemma}
\newtheorem{example}[theorem]{Example}

\newenvironment{problem}[1]{\color{black} #1}{}

\newenvironment{solution}{%
	\leavevmode\begin{tcolorbox}[breakable, colback=green!5!white,colframe=green!75!black, enhanced jigsaw] \proof[\scshape Solution:] \setlength{\parskip}{2mm}%
	}{\renewcommand{\qedsymbol}{$\blacksquare$} \endproof \end{tcolorbox}}

\newenvironment{reflection}{\begin{tcolorbox}[breakable, colback=black!8!white,colframe=black!60!white, enhanced jigsaw, parbox = false]\textsc{Reflections:}}{\end{tcolorbox}}

\newcommand{\qedh}{\renewcommand{\qedsymbol}{$\blacksquare$}\qedhere}

\definecolor{mygreen}{rgb}{0,0.6,0}
\definecolor{mygray}{rgb}{0.5,0.5,0.5}
\definecolor{mymauve}{rgb}{0.58,0,0.82}

% from https://github.com/satejsoman/stata-lstlisting
% language definition
\lstdefinelanguage{Stata}{
	% System commands
	morekeywords=[1]{regress, reg, summarize, sum, display, di, generate, gen, bysort, use, import, delimited, predict, quietly, probit, margins, test},
	% Reserved words
	morekeywords=[2]{aggregate, array, boolean, break, byte, case, catch, class, colvector, complex, const, continue, default, delegate, delete, do, double, else, eltypedef, end, enum, explicit, export, external, float, for, friend, function, global, goto, if, inline, int, local, long, mata, matrix, namespace, new, numeric, NULL, operator, orgtypedef, pointer, polymorphic, pragma, private, protected, public, quad, real, return, rowvector, scalar, short, signed, static, strL, string, struct, super, switch, template, this, throw, transmorphic, try, typedef, typename, union, unsigned, using, vector, version, virtual, void, volatile, while,},
	% Keywords
	morekeywords=[3]{forvalues, foreach, set},
	% Date and time functions
	morekeywords=[4]{bofd, Cdhms, Chms, Clock, clock, Cmdyhms, Cofc, cofC, Cofd, cofd, daily, date, day, dhms, dofb, dofC, dofc, dofh, dofm, dofq, dofw, dofy, dow, doy, halfyear, halfyearly, hh, hhC, hms, hofd, hours, mdy, mdyhms, minutes, mm, mmC, mofd, month, monthly, msofhours, msofminutes, msofseconds, qofd, quarter, quarterly, seconds, ss, ssC, tC, tc, td, th, tm, tq, tw, week, weekly, wofd, year, yearly, yh, ym, yofd, yq, yw,},
	% Mathematical functions
	morekeywords=[5]{abs, ceil, cloglog, comb, digamma, exp, expm1, floor, int, invcloglog, invlogit, ln, ln1m, ln, ln1p, ln, lnfactorial, lngamma, log, log10, log1m, log1p, logit, max, min, mod, reldif, round, sign, sqrt, sum, trigamma, trunc,},
	% Matrix functions
	morekeywords=[6]{cholesky, coleqnumb, colnfreeparms, colnumb, colsof, corr, det, diag, diag0cnt, el, get, hadamard, I, inv, invsym, issymmetric, J, matmissing, matuniform, mreldif, nullmat, roweqnumb, rownfreeparms, rownumb, rowsof, sweep, trace, vec, vecdiag, },
	% Programming functions
	morekeywords=[7]{autocode, byteorder, c, _caller, chop, abs, clip, cond, e, fileexists, fileread, filereaderror, filewrite, float, fmtwidth, has_eprop, inlist, inrange, irecode, matrix, maxbyte, maxdouble, maxfloat, maxint, maxlong, mi, minbyte, mindouble, minfloat, minint, minlong, missing, r, recode, replay, return, s, scalar, smallestdouble,},
	% Random-number functions
	morekeywords=[8]{rbeta, rbinomial, rcauchy, rchi2, rexponential, rgamma, rhypergeometric, rigaussian, rlaplace, rlogistic, rnbinomial, rnormal, rpoisson, rt, runiform, runiformint, rweibull, rweibullph,},
	% Selecting time-span functions
	morekeywords=[9]{tin, twithin,},
	% Statistical functions
	morekeywords=[10]{betaden, binomial, binomialp, binomialtail, binormal, cauchy, cauchyden, cauchytail, chi2, chi2den, chi2tail, dgammapda, dgammapdada, dgammapdadx, dgammapdx, dgammapdxdx, dunnettprob, exponential, exponentialden, exponentialtail, F, Fden, Ftail, gammaden, gammap, gammaptail, hypergeometric, hypergeometricp, ibeta, ibetatail, igaussian, igaussianden, igaussiantail, invbinomial, invbinomialtail, invcauchy, invcauchytail, invchi2, invchi2tail, invdunnettprob, invexponential, invexponentialtail, invF, invFtail, invgammap, invgammaptail, invibeta, invibetatail, invigaussian, invigaussiantail, invlaplace, invlaplacetail, invlogistic, invlogistictail, invnbinomial, invnbinomialtail, invnchi2, invnF, invnFtail, invnibeta, invnormal, invnt, invnttail, invpoisson, invpoissontail, invt, invttail, invtukeyprob, invweibull, invweibullph, invweibullphtail, invweibulltail, laplace, laplaceden, laplacetail, lncauchyden, lnigammaden, lnigaussianden, lniwishartden, lnlaplaceden, lnmvnormalden, lnnormal, lnnormalden, lnwishartden, logistic, logisticden, logistictail, nbetaden, nbinomial, nbinomialp, nbinomialtail, nchi2, nchi2den, nchi2tail, nF, nFden, nFtail, nibeta, normal, normalden, npnchi2, npnF, npnt, nt, ntden, nttail, poisson, poissonp, poissontail, t, tden, ttail, tukeyprob, weibull, weibullden, weibullph, weibullphden, weibullphtail, weibulltail,},
	% String functions 
	morekeywords=[11]{abbrev, char, collatorlocale, collatorversion, indexnot, plural, plural, real, regexm, regexr, regexs, soundex, soundex_nara, strcat, strdup, string, strofreal, string, strofreal, stritrim, strlen, strlower, strltrim, strmatch, strofreal, strofreal, strpos, strproper, strreverse, strrpos, strrtrim, strtoname, strtrim, strupper, subinstr, subinword, substr, tobytes, uchar, udstrlen, udsubstr, uisdigit, uisletter, ustrcompare, ustrcompareex, ustrfix, ustrfrom, ustrinvalidcnt, ustrleft, ustrlen, ustrlower, ustrltrim, ustrnormalize, ustrpos, ustrregexm, ustrregexra, ustrregexrf, ustrregexs, ustrreverse, ustrright, ustrrpos, ustrrtrim, ustrsortkey, ustrsortkeyex, ustrtitle, ustrto, ustrtohex, ustrtoname, ustrtrim, ustrunescape, ustrupper, ustrword, ustrwordcount, usubinstr, usubstr, word, wordbreaklocale, worcount,},
	% Trig functions
	morekeywords=[12]{acos, acosh, asin, asinh, atan, atanh, cos, cosh, sin, sinh, tan, tanh,},
	morecomment=[l]{//},
	% morecomment=[l]{*},  // `*` maybe used as multiply operator. So use `//` as line comment.
	morecomment=[s]{/*}{*/},
	% The following is used by macros, like `lags'.
	morestring=[b]{`}{'},
	% morestring=[d]{'},
	morestring=[b]",
	morestring=[d]",
	% morestring=[d]{\\`},
	% morestring=[b]{'},
	sensitive=true,
}

\lstset{ 
	backgroundcolor=\color{white},   % choose the background color; you must add \usepackage{color} or \usepackage{xcolor}; should come as last argument
	basicstyle=\footnotesize\ttfamily,        % the size of the fonts that are used for the code
	breakatwhitespace=false,         % sets if automatic breaks should only happen at whitespace
	breaklines=true,                 % sets automatic line breaking
	captionpos=b,                    % sets the caption-position to bottom
	commentstyle=\color{mygreen},    % comment style
	deletekeywords={...},            % if you want to delete keywords from the given language
	escapeinside={\%*}{*)},          % if you want to add LaTeX within your code
	extendedchars=true,              % lets you use non-ASCII characters; for 8-bits encodings only, does not work with UTF-8
	firstnumber=0,                % start line enumeration with line 1000
	frame=single,	                   % adds a frame around the code
	keepspaces=true,                 % keeps spaces in text, useful for keeping indentation of code (possibly needs columns=flexible)
	keywordstyle=\color{blue},       % keyword style
	language=Octave,                 % the language of the code
	morekeywords={*,...},            % if you want to add more keywords to the set
	numbers=left,                    % where to put the line-numbers; possible values are (none, left, right)
	numbersep=5pt,                   % how far the line-numbers are from the code
	numberstyle=\tiny\color{mygray}, % the style that is used for the line-numbers
	rulecolor=\color{black},         % if not set, the frame-color may be changed on line-breaks within not-black text (e.g. comments (green here))
	showspaces=false,                % show spaces everywhere adding particular underscores; it overrides 'showstringspaces'
	showstringspaces=false,          % underline spaces within strings only
	showtabs=false,                  % show tabs within strings adding particular underscores
	stepnumber=2,                    % the step between two line-numbers. If it's 1, each line will be numbered
	stringstyle=\color{mymauve},     % string literal style
	tabsize=2,	                   % sets default tabsize to 2 spaces
%	title=\lstname,                   % show the filename of files included with \lstinputlisting; also try caption instead of title
	xleftmargin=0.25cm
}

% NOTE: To compile a version of this pset without problems, solutions, or reflections, uncomment the relevant line below.

%\excludeversion{problem}
%\excludeversion{solution}
%\excludeversion{reflection}

\begin{document}	
	
	% Use the \psetheader command at the beginning of a pset. 
	\psetheader

\section*{Problem 1}

\begin{problem}
    Let $E$ be a Banach space and $T: E\to E^*$ such that for all $x\in E,$
    \[\langle Tx, x\rangle \geq 0.\] Show that $T$ is a bounded operator.
\end{problem}
\begin{solution}
By the closed graph theorem, it suffices to see that $G(T)\subset E \times E^*$ is closed. Let $Tx_n \to y$ with $x_n \to x.$ Evidently $x\in E.$ It suffices to show that $Tx = y,$ that is, for any $u\in E,$ 
we have that 
\[\langle T x, u \rangle = \langle y, u\rangle.\] Let $u \in E.$ There exists some $\lambda \in \bbR$ such that  $x\lambda = u,$ where $\lambda \in \bbR.$ By assumption,
\[\langle Tx_n - Tu, x_n - u\rangle \geq 0 \to \langle y - Tu, x-u\rangle \geq 0.\] Thus,
\[\langle y, x - u \rangle \geq \langle Tu, x-u\rangle,\] and so plugging in our definition of $u,$ we see that 
\[\langle y, \frac{u}{\lambda} - u\rangle \geq \langle T(\lambda x), \frac{u}{\lambda} - u\rangle,\] and so 
\[(\frac{1}{\lambda} -1)\langle y, u\rangle \geq (\frac{1}{\lambda} -1)\langle T(\lambda), u\rangle \implies \langle y, u\rangle \geq \lambda \langle Tx, u\rangle\] Letting $\lambda \to 1$ we see that $\langle y, u\rangle \geq \langle Tx, u\rangle$ and letting $\lambda \to -1,$ see the opposite inequality.
\end{solution}

\newpage
\section*{Problem 2}
\begin{problem}
    Let $E$ be Banach and $A: D(A) \subset E \to E^*$ be a densely defined unbounded operator. 
    \begin{enumerate}
        \item Suppose that there exists some $C$ such that for all $u\in D(A),$ we have that 
        \begin{align}
        \langle Au, u\rangle \geq -C\|Au\|^2    
        \end{align}
         Show that $N(A)\subset N(A^*)$
        \begin{solution}
    Since \[A: D(A)\subset E \to E^*\implies A^*:D(A^*)\subset E^{**} \to E^*\]
    By that lecture Marr's did in class, it suffices to show that since $N(A^*) = R(T)^\perp,$  we have that $N(A)\subset R(T)^\perp.$ That is, 
    we want to show that 
    \[\{u \in E \: |\: Au = 0\} \subset \{u\in E^{**} \; : \; \langle u, Av\rangle = 0 \; \forall v \in D(A)\}.\] In non-reflexive spaces, we have that $E\subset E^{**},$ so this might be an equality in reflexive spaces. Anyways, let $u\in N(A),$ then $Au = 0.$ Let $v\in D(A).$ Then for any $t \in \bbR,$ we have that by the given inequality, 
    \[\langle A(u + tv), u + tv \rangle \geq -C \|A(u + tv)\|^2\] Expanding the left side, we see that 
    \begin{align*}
      \langle A(u + tv), u + tv \rangle &= \langle Au, u + tv\rangle + \langle tAv, u + tv \rangle\\ 
      &=0 + t\langle Av, u\rangle +t \langle Av, u\rangle + t^2\langle Av, v\rangle\\
      &\geq -C t^2\|Av\|^2  
    \end{align*}
    Thus, 
    \[2t\langle Av, u\rangle + t^2(\langle Av,v\rangle +C\|Av\|^2) \geq 0.\] This is a quadratic equation in terms of $t,$ and thus 
    \[at^2 + bt \geq 0 \implies b^2 - 4ac \leq 0 \implies b^2 \leq 0,\] but $b^2 = (2\langle Av, u\rangle)^2 \leq 0 \implies b = 0$ and so $\langle Av, u\rangle = 0,$ and so $u \in R(A)^\perp.$ 
        \end{solution}
    \item Prove that the converse holds if $A$ is closed and $R(A)$ is closed. 
    \begin{solution}
        Since $N(A)\subset N(A^*)$ by assumption, then 
        $N(A)\subset R(A)^\perp,$ and so for any $u\in N(A),$ we have that $u\in R(A)^\perp.$ We claim that it suffices to find some $v\in D(A)$ for any $u\in D(A)$ such that $Au = Av.$ That is, 
        \[\langle Au, x\rangle = \langle Av, x\rangle, \quad \forall x\in D(A) \implies Au- Av = A(u-v) = 0 \implies u-v \in \ker A.\] Thus, we see that by assumption, $u-v \in R(A)^\perp,$ and so for any $x\in D(A)$
        \[\langle u-v, Ax \rangle = 0.\]
        In particular, letting $x = u$ we see that 
        \[\langle Au, u\rangle = \langle Au, v\rangle\]
        Letting $x = v,$ we see that 
        \[\langle Av, u\rangle = \langle Av, v\rangle.\]
        Thus, 
        \begin{align}
        \langle Au, u \rangle= \langle Av,v\rangle    
        \end{align}
        

        To find such a $v,$ consider that since $R(A)\subset E^*$ is a closed subset of a Banach space, then $R(A)$ is a Banach space. Thus, $A$ is an open mapping, i.e, there exists some $c >0$ such that
        \[B_c^{R(A)}(0)\subset A(B_1^{(D(A))}(0)).\] Thus, for any $f \in R(A),$ with $\|f\|\leq c,$ there exists some $u \in D(A)$ such that $Au = cf,$ and so $Au' = f,$ where $u' = \frac{u}{c}\in D(A)$ since $D(A)$ is densely defined linear subspace.  Thus, $\|u'\| = \|\frac{u}{c}\| \leq \|f\|\implies \|u\|\leq c\|f\|$ Because this holds for any $f\in R(A)$ (you can just scale any $f$ not in the $c-$ball) and $A$ is surjective, then let $u\in D(A)$ with $Au = f.$ But since $f\in R(A),$ then we know by the open mapping theorem there exists some $v\in D(A)$ with $Av = f$ and $-c \|Au\|\leq \|v\| \leq \|c\|\|Au\|.$ Thus, we have found our $u-v \in N(A).$ Hence by (2), we have that since $|\langle f, u\rangle|\leq \|f\|\|u\|,$ then 
        \[\langle Au, u \rangle \geq -c\|Au\|\|v\| \geq -c\|Au\|^2\]
    \end{solution}
    \end{enumerate}
\end{problem}

\newpage
\section*{Problem 3}
\begin{problem}
    Suppose $X$ is a separable Banach space and $M\subset X$ is a closed subspace. Then $X/M$ is separable.
\end{problem}
\begin{solution}
    Let $\pi: X\to X/M$ be the canonical linear surjection and let $(v_n)$ be a countably dense subset of $X.$ We claim that $\pi(v_n)$ is a countably dense subset of $X/M.$ Recall that for any since $X$ is Banach, then $X/M$ is Banach with respect to the norm
    \[\|u\|_{X/M} = \inf_{m\in M}\|u - m\|_X.\] Let $u \in X/M,$ and $\epsilon>0.$ Since $\pi$ is surjective, there exists some $v\in X$ such that $\pi(v) = x,$ and thus there exists some $v_k \in (v_n)$ such that 
    \[\|v_k - v\|< \epsilon\] 
    Thus, we have that 
    \begin{align*}
        \|\pi(v_n) - u\|_{X/M} &= \|\pi(v_n) - \pi(v)\|_{X/M}\\ 
        &=\|\pi(v_n  - v)\|_{X/M}\\
        &= \inf_{m\in M} \|v_n - v - m\|_X\\
        &\leq \|v_n - v\|_X + \inf_{m\in M}\|m\|_X\\
        &= \|v_n - v\|\\
        &< \epsilon.
    \end{align*}
Then we have that $(\pi(v_n))$ is countably dense in $X/M,$ as required.
\end{solution}

\newpage
\section*{Problem 4}
\begin{problem}
    Suppose that $X$ is a Banach space, $M\subset X$ is closed and separable. If $X/M$ is separable, then $X$ is separable.
\end{problem}
\begin{solution}
    Let $(u_n)\subset M$ be a countably dense subset of $M$ and let $([w_n])\subset X/M$ be a countably dense subset of $X/M$ and choose a representative $w_n \in X$ such that $[w_n] = \pi(w_n).$ Thus, let $u \in M$ and $[w] \in X/M,$ then there exist $u_n\in (u_n),$ $[w_k]\in ([w_n])$ such that 
    \begin{align}
        \|u_n - u\|_X< \frac{\epsilon}{2}
    \end{align}
    \begin{align}
        \|[w_k] - [w]\|_{X/M} = \|\pi(w_k - w)\|_{X/M} = \inf_{m\in M}\|w_k - w - m\|< \frac{\epsilon}{2}
    \end{align}

    Consider $F = \{u_n + w_n \; : \: u_n \in (u_n), w_n \in (w_n)\},$ we claim that $F$ is countably dense in $X.$ The countability comes from the fact that 
    \[F = \bigcup_{k=1}^\infty\bigcup_{n=k}^\infty (u_n + w_k).\] Let $x\in X.$ Then for any $\epsilon>0,$ we have by the above that if $m\in M$ and $u_n$ and $w_k$ are chosen such that (3) and (4) are satisfied, then
    \begin{align*}
        \|x - (u_n + w_n)\|_X &\leq \|x - w_n - m\| + \|m - u_n\|\\
        &< \epsilon
    \end{align*}
    Thus, we are done.
\end{solution}
\begin{reflection}
    This might be wrong, I didn't use the fact that $X$ was Banach, but in the book the quotient space is only defined for Banach spaces, so maybe I did?
\end{reflection}


\end{document}