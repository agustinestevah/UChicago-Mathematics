\documentclass[11pt]{article}

% NOTE: Add in the relevant information to the commands below; or, if you'll be using the same information frequently, add these commands at the top of paolo-pset.tex file. 
\newcommand{\name}{Agustín Esteva}
\newcommand{\email}{aesteva@uchicago.edu}
\newcommand{\classnum}{207}
\newcommand{\subject}{Honors Analysis in $\bbR^n$}
\newcommand{\instructors}{Panagiotis E. Souganidis}
\newcommand{\assignment}{Problem Set 1}
\newcommand{\semester}{Fall 2024}
\newcommand{\duedate}{2024-13-01}
\newcommand{\bA}{\mathbf{A}}
\newcommand{\bB}{\mathbf{B}}
\newcommand{\bC}{\mathbf{C}}
\newcommand{\bD}{\mathbf{D}}
\newcommand{\bE}{\mathbf{E}}
\newcommand{\bF}{\mathbf{F}}
\newcommand{\bG}{\mathbf{G}}
\newcommand{\bH}{\mathbf{H}}
\newcommand{\bI}{\mathbf{I}}
\newcommand{\bJ}{\mathbf{J}}
\newcommand{\bK}{\mathbf{K}}
\newcommand{\bL}{\mathbf{L}}
\newcommand{\bM}{\mathbf{M}}
\newcommand{\bN}{\mathbf{N}}
\newcommand{\bO}{\mathbf{O}}
\newcommand{\bP}{\mathbf{P}}
\newcommand{\bQ}{\mathbf{Q}}
\newcommand{\bR}{\mathbf{R}}
\newcommand{\bS}{\mathbf{S}}
\newcommand{\bT}{\mathbf{T}}
\newcommand{\bU}{\mathbf{U}}
\newcommand{\bV}{\mathbf{V}}
\newcommand{\bW}{\mathbf{W}}
\newcommand{\bX}{\mathbf{X}}
\newcommand{\bY}{\mathbf{Y}}
\newcommand{\bZ}{\mathbf{Z}}

%% blackboard bold math capitals
\newcommand{\bbA}{\mathbb{A}}
\newcommand{\bbB}{\mathbb{B}}
\newcommand{\bbC}{\mathbb{C}}
\newcommand{\bbD}{\mathbb{D}}
\newcommand{\bbE}{\mathbb{E}}
\newcommand{\bbF}{\mathbb{F}}
\newcommand{\bbG}{\mathbb{G}}
\newcommand{\bbH}{\mathbb{H}}
\newcommand{\bbI}{\mathbb{I}}
\newcommand{\bbJ}{\mathbb{J}}
\newcommand{\bbK}{\mathbb{K}}
\newcommand{\bbL}{\mathbb{L}}
\newcommand{\bbM}{\mathbb{M}}
\newcommand{\bbN}{\mathbb{N}}
\newcommand{\bbO}{\mathbb{O}}
\newcommand{\bbP}{\mathbb{P}}
\newcommand{\bbQ}{\mathbb{Q}}
\newcommand{\bbR}{\mathbb{R}}
\newcommand{\bbS}{\mathbb{S}}
\newcommand{\bbT}{\mathbb{T}}
\newcommand{\bbU}{\mathbb{U}}
\newcommand{\bbV}{\mathbb{V}}
\newcommand{\bbW}{\mathbb{W}}
\newcommand{\bbX}{\mathbb{X}}
\newcommand{\bbY}{\mathbb{Y}}
\newcommand{\bbZ}{\mathbb{Z}}

%% script math capitals
\newcommand{\sA}{\mathscr{A}}
\newcommand{\sB}{\mathscr{B}}
\newcommand{\sC}{\mathscr{C}}
\newcommand{\sD}{\mathscr{D}}
\newcommand{\sE}{\mathscr{E}}
\newcommand{\sF}{\mathscr{F}}
\newcommand{\sG}{\mathscr{G}}
\newcommand{\sH}{\mathscr{H}}
\newcommand{\sI}{\mathscr{I}}
\newcommand{\sJ}{\mathscr{J}}
\newcommand{\sK}{\mathscr{K}}
\newcommand{\sL}{\mathscr{L}}
\newcommand{\sM}{\mathscr{M}}
\newcommand{\sN}{\mathscr{N}}
\newcommand{\sO}{\mathscr{O}}
\newcommand{\sP}{\mathscr{P}}
\newcommand{\sQ}{\mathscr{Q}}
\newcommand{\sR}{\mathscr{R}}
\newcommand{\sS}{\mathscr{S}}
\newcommand{\sT}{\mathscr{T}}
\newcommand{\sU}{\mathscr{U}}
\newcommand{\sV}{\mathscr{V}}
\newcommand{\sW}{\mathscr{W}}
\newcommand{\sX}{\mathscr{X}}
\newcommand{\sY}{\mathscr{Y}}
\newcommand{\sZ}{\mathscr{Z}}


\renewcommand{\emptyset}{\O}

\newcommand{\abs}[1]{\lvert #1 \rvert}
\newcommand{\norm}[1]{\lVert #1 \rVert}
\newcommand{\sm}{\setminus}


\newcommand{\sarr}{\rightarrow}
\newcommand{\arr}{\longrightarrow}

% NOTE: Defining collaborators is optional; to not list collaborators, comment out the line below.
%\newcommand{\collaborators}{Alyssa P. Hacker (\texttt{aphacker}), Ben Bitdiddle (\texttt{bitdiddle})}

% Copyright 2021 Paolo Adajar (padajar.com, paoloadajar@mit.edu)
% 
% Permission is hereby granted, free of charge, to any person obtaining a copy of this software and associated documentation files (the "Software"), to deal in the Software without restriction, including without limitation the rights to use, copy, modify, merge, publish, distribute, sublicense, and/or sell copies of the Software, and to permit persons to whom the Software is furnished to do so, subject to the following conditions:
%
% The above copyright notice and this permission notice shall be included in all copies or substantial portions of the Software.
% 
% THE SOFTWARE IS PROVIDED "AS IS", WITHOUT WARRANTY OF ANY KIND, EXPRESS OR IMPLIED, INCLUDING BUT NOT LIMITED TO THE WARRANTIES OF MERCHANTABILITY, FITNESS FOR A PARTICULAR PURPOSE AND NONINFRINGEMENT. IN NO EVENT SHALL THE AUTHORS OR COPYRIGHT HOLDERS BE LIABLE FOR ANY CLAIM, DAMAGES OR OTHER LIABILITY, WHETHER IN AN ACTION OF CONTRACT, TORT OR OTHERWISE, ARISING FROM, OUT OF OR IN CONNECTION WITH THE SOFTWARE OR THE USE OR OTHER DEALINGS IN THE SOFTWARE.

\usepackage{fullpage}
\usepackage{enumitem}
\usepackage{amsfonts, amssymb, amsmath,amsthm}
\usepackage{mathtools}
\usepackage[pdftex, pdfauthor={\name}, pdftitle={\classnum~\assignment}]{hyperref}
\usepackage[dvipsnames]{xcolor}
\usepackage{bbm}
\usepackage{graphicx}
\usepackage{mathrsfs}
\usepackage{pdfpages}
\usepackage{tabularx}
\usepackage{pdflscape}
\usepackage{makecell}
\usepackage{booktabs}
\usepackage{natbib}
\usepackage{caption}
\usepackage{subcaption}
\usepackage{physics}
\usepackage[many]{tcolorbox}
\usepackage{version}
\usepackage{ifthen}
\usepackage{cancel}
\usepackage{listings}
\usepackage{courier}

\usepackage{tikz}
\usepackage{istgame}

\hypersetup{
	colorlinks=true,
	linkcolor=blue,
	filecolor=magenta,
	urlcolor=blue,
}

\setlength{\parindent}{0mm}
\setlength{\parskip}{2mm}

\setlist[enumerate]{label=({\alph*})}
\setlist[enumerate, 2]{label=({\roman*})}

\allowdisplaybreaks[1]

\newcommand{\psetheader}{
	\ifthenelse{\isundefined{\collaborators}}{
		\begin{center}
			{\setlength{\parindent}{0cm} \setlength{\parskip}{0mm}
				
				{\textbf{\classnum~\semester:~\assignment} \hfill \name}
				
				\subject \hfill \href{mailto:\email}{\tt \email}
				
				Instructor(s):~\instructors \hfill Due Date:~\duedate	
				
				\hrulefill}
		\end{center}
	}{
		\begin{center}
			{\setlength{\parindent}{0cm} \setlength{\parskip}{0mm}
				
				{\textbf{\classnum~\semester:~\assignment} \hfill \name\footnote{Collaborator(s): \collaborators}}
				
				\subject \hfill \href{mailto:\email}{\tt \email}
				
				Instructor(s):~\instructors \hfill Due Date:~\duedate	
				
				\hrulefill}
		\end{center}
	}
}

\renewcommand{\thepage}{\classnum~\assignment \hfill \arabic{page}}

\makeatletter
\def\points{\@ifnextchar[{\@with}{\@without}}
\def\@with[#1]#2{{\ifthenelse{\equal{#2}{1}}{{[1 point, #1]}}{{[#2 points, #1]}}}}
\def\@without#1{\ifthenelse{\equal{#1}{1}}{{[1 point]}}{{[#1 points]}}}
\makeatother

\newtheoremstyle{theorem-custom}%
{}{}%
{}{}%
{\itshape}{.}%
{ }%
{\thmname{#1}\thmnumber{ #2}\thmnote{ (#3)}}

\theoremstyle{theorem-custom}

\newtheorem{theorem}{Theorem}
\newtheorem{lemma}[theorem]{Lemma}
\newtheorem{example}[theorem]{Example}

\newenvironment{problem}[1]{\color{black} #1}{}

\newenvironment{solution}{%
	\leavevmode\begin{tcolorbox}[breakable, colback=green!5!white,colframe=green!75!black, enhanced jigsaw] \proof[\scshape Solution:] \setlength{\parskip}{2mm}%
	}{\renewcommand{\qedsymbol}{$\blacksquare$} \endproof \end{tcolorbox}}

\newenvironment{reflection}{\begin{tcolorbox}[breakable, colback=black!8!white,colframe=black!60!white, enhanced jigsaw, parbox = false]\textsc{Reflections:}}{\end{tcolorbox}}

\newcommand{\qedh}{\renewcommand{\qedsymbol}{$\blacksquare$}\qedhere}

\definecolor{mygreen}{rgb}{0,0.6,0}
\definecolor{mygray}{rgb}{0.5,0.5,0.5}
\definecolor{mymauve}{rgb}{0.58,0,0.82}

% from https://github.com/satejsoman/stata-lstlisting
% language definition
\lstdefinelanguage{Stata}{
	% System commands
	morekeywords=[1]{regress, reg, summarize, sum, display, di, generate, gen, bysort, use, import, delimited, predict, quietly, probit, margins, test},
	% Reserved words
	morekeywords=[2]{aggregate, array, boolean, break, byte, case, catch, class, colvector, complex, const, continue, default, delegate, delete, do, double, else, eltypedef, end, enum, explicit, export, external, float, for, friend, function, global, goto, if, inline, int, local, long, mata, matrix, namespace, new, numeric, NULL, operator, orgtypedef, pointer, polymorphic, pragma, private, protected, public, quad, real, return, rowvector, scalar, short, signed, static, strL, string, struct, super, switch, template, this, throw, transmorphic, try, typedef, typename, union, unsigned, using, vector, version, virtual, void, volatile, while,},
	% Keywords
	morekeywords=[3]{forvalues, foreach, set},
	% Date and time functions
	morekeywords=[4]{bofd, Cdhms, Chms, Clock, clock, Cmdyhms, Cofc, cofC, Cofd, cofd, daily, date, day, dhms, dofb, dofC, dofc, dofh, dofm, dofq, dofw, dofy, dow, doy, halfyear, halfyearly, hh, hhC, hms, hofd, hours, mdy, mdyhms, minutes, mm, mmC, mofd, month, monthly, msofhours, msofminutes, msofseconds, qofd, quarter, quarterly, seconds, ss, ssC, tC, tc, td, th, tm, tq, tw, week, weekly, wofd, year, yearly, yh, ym, yofd, yq, yw,},
	% Mathematical functions
	morekeywords=[5]{abs, ceil, cloglog, comb, digamma, exp, expm1, floor, int, invcloglog, invlogit, ln, ln1m, ln, ln1p, ln, lnfactorial, lngamma, log, log10, log1m, log1p, logit, max, min, mod, reldif, round, sign, sqrt, sum, trigamma, trunc,},
	% Matrix functions
	morekeywords=[6]{cholesky, coleqnumb, colnfreeparms, colnumb, colsof, corr, det, diag, diag0cnt, el, get, hadamard, I, inv, invsym, issymmetric, J, matmissing, matuniform, mreldif, nullmat, roweqnumb, rownfreeparms, rownumb, rowsof, sweep, trace, vec, vecdiag, },
	% Programming functions
	morekeywords=[7]{autocode, byteorder, c, _caller, chop, abs, clip, cond, e, fileexists, fileread, filereaderror, filewrite, float, fmtwidth, has_eprop, inlist, inrange, irecode, matrix, maxbyte, maxdouble, maxfloat, maxint, maxlong, mi, minbyte, mindouble, minfloat, minint, minlong, missing, r, recode, replay, return, s, scalar, smallestdouble,},
	% Random-number functions
	morekeywords=[8]{rbeta, rbinomial, rcauchy, rchi2, rexponential, rgamma, rhypergeometric, rigaussian, rlaplace, rlogistic, rnbinomial, rnormal, rpoisson, rt, runiform, runiformint, rweibull, rweibullph,},
	% Selecting time-span functions
	morekeywords=[9]{tin, twithin,},
	% Statistical functions
	morekeywords=[10]{betaden, binomial, binomialp, binomialtail, binormal, cauchy, cauchyden, cauchytail, chi2, chi2den, chi2tail, dgammapda, dgammapdada, dgammapdadx, dgammapdx, dgammapdxdx, dunnettprob, exponential, exponentialden, exponentialtail, F, Fden, Ftail, gammaden, gammap, gammaptail, hypergeometric, hypergeometricp, ibeta, ibetatail, igaussian, igaussianden, igaussiantail, invbinomial, invbinomialtail, invcauchy, invcauchytail, invchi2, invchi2tail, invdunnettprob, invexponential, invexponentialtail, invF, invFtail, invgammap, invgammaptail, invibeta, invibetatail, invigaussian, invigaussiantail, invlaplace, invlaplacetail, invlogistic, invlogistictail, invnbinomial, invnbinomialtail, invnchi2, invnF, invnFtail, invnibeta, invnormal, invnt, invnttail, invpoisson, invpoissontail, invt, invttail, invtukeyprob, invweibull, invweibullph, invweibullphtail, invweibulltail, laplace, laplaceden, laplacetail, lncauchyden, lnigammaden, lnigaussianden, lniwishartden, lnlaplaceden, lnmvnormalden, lnnormal, lnnormalden, lnwishartden, logistic, logisticden, logistictail, nbetaden, nbinomial, nbinomialp, nbinomialtail, nchi2, nchi2den, nchi2tail, nF, nFden, nFtail, nibeta, normal, normalden, npnchi2, npnF, npnt, nt, ntden, nttail, poisson, poissonp, poissontail, t, tden, ttail, tukeyprob, weibull, weibullden, weibullph, weibullphden, weibullphtail, weibulltail,},
	% String functions 
	morekeywords=[11]{abbrev, char, collatorlocale, collatorversion, indexnot, plural, plural, real, regexm, regexr, regexs, soundex, soundex_nara, strcat, strdup, string, strofreal, string, strofreal, stritrim, strlen, strlower, strltrim, strmatch, strofreal, strofreal, strpos, strproper, strreverse, strrpos, strrtrim, strtoname, strtrim, strupper, subinstr, subinword, substr, tobytes, uchar, udstrlen, udsubstr, uisdigit, uisletter, ustrcompare, ustrcompareex, ustrfix, ustrfrom, ustrinvalidcnt, ustrleft, ustrlen, ustrlower, ustrltrim, ustrnormalize, ustrpos, ustrregexm, ustrregexra, ustrregexrf, ustrregexs, ustrreverse, ustrright, ustrrpos, ustrrtrim, ustrsortkey, ustrsortkeyex, ustrtitle, ustrto, ustrtohex, ustrtoname, ustrtrim, ustrunescape, ustrupper, ustrword, ustrwordcount, usubinstr, usubstr, word, wordbreaklocale, worcount,},
	% Trig functions
	morekeywords=[12]{acos, acosh, asin, asinh, atan, atanh, cos, cosh, sin, sinh, tan, tanh,},
	morecomment=[l]{//},
	% morecomment=[l]{*},  // `*` maybe used as multiply operator. So use `//` as line comment.
	morecomment=[s]{/*}{*/},
	% The following is used by macros, like `lags'.
	morestring=[b]{`}{'},
	% morestring=[d]{'},
	morestring=[b]",
	morestring=[d]",
	% morestring=[d]{\\`},
	% morestring=[b]{'},
	sensitive=true,
}

\lstset{ 
	backgroundcolor=\color{white},   % choose the background color; you must add \usepackage{color} or \usepackage{xcolor}; should come as last argument
	basicstyle=\footnotesize\ttfamily,        % the size of the fonts that are used for the code
	breakatwhitespace=false,         % sets if automatic breaks should only happen at whitespace
	breaklines=true,                 % sets automatic line breaking
	captionpos=b,                    % sets the caption-position to bottom
	commentstyle=\color{mygreen},    % comment style
	deletekeywords={...},            % if you want to delete keywords from the given language
	escapeinside={\%*}{*)},          % if you want to add LaTeX within your code
	extendedchars=true,              % lets you use non-ASCII characters; for 8-bits encodings only, does not work with UTF-8
	firstnumber=0,                % start line enumeration with line 1000
	frame=single,	                   % adds a frame around the code
	keepspaces=true,                 % keeps spaces in text, useful for keeping indentation of code (possibly needs columns=flexible)
	keywordstyle=\color{blue},       % keyword style
	language=Octave,                 % the language of the code
	morekeywords={*,...},            % if you want to add more keywords to the set
	numbers=left,                    % where to put the line-numbers; possible values are (none, left, right)
	numbersep=5pt,                   % how far the line-numbers are from the code
	numberstyle=\tiny\color{mygray}, % the style that is used for the line-numbers
	rulecolor=\color{black},         % if not set, the frame-color may be changed on line-breaks within not-black text (e.g. comments (green here))
	showspaces=false,                % show spaces everywhere adding particular underscores; it overrides 'showstringspaces'
	showstringspaces=false,          % underline spaces within strings only
	showtabs=false,                  % show tabs within strings adding particular underscores
	stepnumber=2,                    % the step between two line-numbers. If it's 1, each line will be numbered
	stringstyle=\color{mymauve},     % string literal style
	tabsize=2,	                   % sets default tabsize to 2 spaces
%	title=\lstname,                   % show the filename of files included with \lstinputlisting; also try caption instead of title
	xleftmargin=0.25cm
}

% NOTE: To compile a version of this pset without problems, solutions, or reflections, uncomment the relevant line below.

%\excludeversion{problem}
%\excludeversion{solution}
%\excludeversion{reflection}

\begin{document}	
	
	% Use the \psetheader command at the beginning of a pset. 
	\psetheader

\section*{Problem 1}
\begin{problem}
Suppose $\alpha$ increases on $[a,b]$ $a \leq x_0 \leq b,$ $\alpha$ is continuous at $x_0,$ $f(x_0) = 1$ and $f(x) = 0$ for all $x\neq x_0.$ Prove that $f\in \mathcal{R}(\alpha)$ and that $\int f d\alpha =0.$
\end{problem}
\begin{solution}
Let $\epsilon>0.$ Since $\alpha$ is continuous at $x_0,$ we have that that there exists a $\delta>0$ such that if $|x-t|< \delta,$ then $|\alpha(x) - \alpha(t)|< \frac{\epsilon}{2}.$ Thus, if we partition $P = \{x_0, x_1, \dots, x_n \}$ such that $||P||< \frac{\delta}{2},$ then we can say that $x_0 \in [x_{j-1}, x_j]$ for some $j \in [n]$ and thus $d(x_0), x_{j-1}< \delta$ and $d(x_0, x_j)< \delta.$ Moreover, we can say that by the triangle inequality, 
\[|\alpha(x_j) - \alpha(x_{j-1})| \leq |\alpha(x_j) - \alpha(x_0)| + |\alpha(x_0) - \alpha(x_{j-1})|< \frac{\epsilon}{2} + \frac{\epsilon}{2}.\]
Thus,
\[U(P,f, \alpha) -L(P, f, \alpha) = \sum (M_i - m_i)\Delta \alpha_i = 0 + f(x_0)(\alpha(x_j) - \alpha(x_{j-1})) < \epsilon.\] This shows $f \in \mathcal{R}(\alpha).$
We will show that $\inf_P U(P,f,\alpha) = 0 = \sup_P L(P,f,\alpha).$ Since for any $P,$ we have that $L(P,f,\alpha) \leq U(P,f\alpha),$ it will suffice to show that there exist partitions such that $U(P,f,\alpha)\leq 0$ and $L(P,f, \alpha)\geq 0.$ The lower sum is obvious, since for any partition, $\inf_{x\in [x_{i-1}, x_i]}f(x) = 0.$ Let $\epsilon>0,$ then we can take the same partition we took in the first part, which shows that $U(P,f,\alpha) < \epsilon,$ which implies that $U(P,f,\alpha)\leq 0.$
\end{solution}

\newpage
\section*{Problem 2}
\begin{problem}
Suppose $f\geq 0,$ $f$ is continuous on $[a,b]$, and $\int_a^b f d\alpha = 0.$ Prove that $f(x) = 0$ for all $x\in [a,b].$
\end{problem}
\begin{solution}
    Suppose not. Let $x_0 \in [a,b]$ with $f(x_0) > 0 $. By continuity of $f,$ there exists some $\delta>0$ such that if $x\in (x_0 - \frac{\delta}{2}, x_0 + \frac{\delta}{2}),$ then $f(x) > \frac{f(x_0)}{2}.$ Thus we have that 
    \[\int_a^b fdx = \int_{[a,b]\setminus(x_0 - \delta, x_0 + \delta)}f dx + \int_{x_0-\delta}^ {x_0 + \delta}f dx\geq \int_{x_0-\delta}^ {x_0 + \delta}f dx \geq \frac{f(x_0)}{2}(2\delta) > 0,\] a contradiction! It might be worth noting that we know the second equality holds since $0\leq f$ on $[a,b]\setminus(x_0 - \delta, x_0 + \delta)$ and thus $0 \leq \int_{[a,b]\setminus(x_0 - \delta, x_0 + \delta)}f dx.$ Comparing this with problem 1, we see that 
\end{solution}

\newpage
\section*{Problem 3}
\begin{problem}
    Define three functions, $\beta_1, \beta_2, \beta_3$ as follows: $\beta_j(x) = 0$ if $x<0,$ $\beta_j = 1$ if $x>0$ for $j = 1,2,3;$ and $\beta_1(0) = 0,$ $\beta_2(0) = 1,$ and $\beta_3(0) = \frac{1}{2}.$ Let $f$ be a bounded function on $[-1,1].$ 
\end{problem}
\begin{enumerate}
    \item 
    \begin{problem}
        Prove that $f\in \mathcal{R}(\beta_1)$ if and only if $f(0 + ) = f(0)$ and that then $\int f d\beta_1 = f(0).$
    \end{problem}
    \begin{solution}
        \begin{itemize}
            \item ($\implies$) Suppose $f\in \mathcal{R}(\beta_1)$ and assume that $f(0+)\neq f(0).$ Thus, for any $x>0,$ we can find some $x_0 \in (0,x)$ such that $f(x_0)\neq f(0).$ For any partition, can split the sum over the partitions into the terms where the intervals are less than $0,$ in which case $\Delta \beta_1 = 0$ since $\beta_1$ is identically $0$ here, the case when the interval contains $0$ and some points on the right, and the case when the interval are strictly greater than $0,$ in which the same thing as in the first case applies. Thus, we only have to worry about the interval $[x_{j-1}, x_j],$ where $0 \in [x_{j-1}, x_j]$ and $x_j > 0.$ Thus:
            \[U(P, f, \beta_1) - L(P, f, \beta_1) =  (M_j  - m_j)(\beta_1(x_j) - \beta_1(x_{j-1})) = M_j - m_j \geq |f(x_0)- f(0)|,\] where $x_0$ is the point discussed at the beginning. Thus, we we have $f\notin \mathcal{R}(\beta_1),$ a contradiction.
            \item ($\impliedby$) Suppose $\lim_{x\to 0^+}f(x) = 0.$ Thus, for any $\epsilon>0,$ there exists a $\delta>0$ such that if $x<\delta,$ then $|f(x) - f(0)|< \frac{\epsilon}{2}.$ Thus, we have that we can partition $[a,b]$ by $P = \{x_0,x_1, x_2, x_3\}$ where $x_2 - x_1 < \delta$ and $0 \in (x_1,x_2)$ Thus, we have that
            \[U(P, f, \beta_1) - L(P, f, \beta_1) = (M_1 - m_1)\Delta_1 \beta_1 + (M_2- m_2)\Delta_2 \beta_1 + (M_3 - m_3)\Delta_3\beta_1.\] The first term is $0$ since for all $x\in [x_0, x_1],$ $x<0$ and so $\beta_1(x) = 0.$ Similarly for the third term. For the middle term, we have that $\beta(x_2) - \beta(x_1) = 1.$ For any $x\in [x_1, x_2],$ we have that $|x|< \delta,$ and thus $|f(x) - f(0)| < \frac{\epsilon}{2},$ and thus if $x,y \in [x_1, x_2],$ we have that 
            \[|f(x) - f(y)|\leq |f(x) - f(0)| + |f(0) - f(y)|< \epsilon \implies M_2 - m_2 \leq \epsilon.\]
        \end{itemize}
    \end{solution}
    \item 
    \begin{problem}
        State and prove a similar result for $\beta_2.$
    \end{problem}
    \begin{solution}
        $f \in \mathcal{R}(\beta_2)$ if and only if $f(0-) = f(0)$ and that then 
        \[\int f d\beta_2 = f(0).\] The proof is identical to the above, don't make me do this again (please).
    \end{solution}
    \item 
    \begin{problem}
            Prove that $f\in \mathcal{R}(\beta_3)$ if and only if $f$ is continuous at $0.$
    \end{problem}
    \begin{solution}
        \begin{itemize}
            \item ($\implies$) Suppose $f\in \mathcal{R}(\beta_3)$ but assume $f$ is not continuous at $0.$ Thus, for there exists some $\epsilon>0$ such that for any $\delta>0,$ if $|x|< \delta,$ then $|f(x) - f(0)| \geq \epsilon.$ Thus, for any partition $P,$ we have that there exists some $x \in [x_{j-1}, x_j]$ where $0 \in [x_{j-1}, x_j]$ such that $|f(x) - f(0)|\geq \epsilon.$ Thus we can say that 
            \[U(P, f, \beta_1) - L(P, f, \beta_1) = \sum_{i=1}^n (M_i - m_i)\Delta_i \beta_3 \geq (M_j - m_j)(\beta(x_j) - \beta(x_{j-1}))\geq \frac{1}{2}\epsilon.\] Thus, $f\notin \mathcal{R}(\beta_3).$
            \item ($\impliedby$) Suppose $f$ is continuous at $0.$ Let $\epsilon>0.$ By continuity of $f,$ we have a $\delta>0$ such that if $|x|< \delta,$ then $|f(x) - f(0)|< \frac{\epsilon}{2}.$ Thus let $P = \{x_0, x_1, x_2, x_3\}$ be a partition of $[a,b]$ such that $0 \in (x_1, x_2)$ and $x_2 - x_1< \delta.$ Thus, for any $x,y \in [x_1, x_2],$ \[|f(x) - f(y)| \leq |f(x) - f(0)| + |f(0) - f(y)| < \frac{\epsilon}{2} + \frac{\epsilon}{2}  = \epsilon \implies M_2 - m_2 \leq \epsilon.\] Thus, 
            \[U(P,f,\beta_3) - L(P, f, \beta_3) = (M_1 - m_1)\Delta_1\beta_3 + (M_2 - m_2)\Delta_2 \beta_3 + (M_3 - m_3)\Delta_3 \beta_3.\] The first term goes away since $\beta(x_1) = \beta(x_0) = 0$ and the third dies off since $\beta(x_3) = \beta(x_2) = 1.$ Thus, we have that 
            \[U(P,f, \beta_3) - L(P, f, \beta_3) < \epsilon \Delta_2 \beta_3 = \epsilon.\]
        \end{itemize}
    \end{solution}
    \item 
    \begin{problem}
        If $f$ is continuous at $0,$ prove that 
        \[\int f d\beta_1 = \int f d\beta_2 = \int f d\beta_3 = f(0).\] 
    \end{problem}
    \begin{solution}
        Since $f$ is continuous at $0,$ then we have that $f(0+) = f(0-) = f(0)$ and so by the previous parts we have that $f\in \mathcal{R}(\beta_j)$ and that 
        \[\int f d\beta_1 = \int f d\beta_2 = f(0).\] It suffices to show the last equality. We can use the same partition $P$ as we did in part (c) to show that 
        \[U(P, f, \beta_3) = M_2.\] Thus, it suffices to notice that $M_2 \to f(0)$ as $|x_2 - x_1| \to 0,$ which is true continuity of $f$, however we can also note that $M_2 \geq f(0)$ and so since any partition will include at least one interval (and at most two) which contains $0,$ then for all partitions $P,$ $U(P,f,\beta_3) \geq f(0).$ Similarly, we have that for all partitions, $L(P,f,\beta_3)\leq f(0).$ Thus, since we know that $f$ is Stieltjes integrable, then these two lower and upper integrals must equal and so 
        \[\int f d\beta_3 = \underline{\int} f d\beta_3 = \overline{\int} f d\beta_3 = f(0).\]
    \end{solution}
\end{enumerate}

\newpage
\section*{Problem 4}
\begin{problem}
    If $f(x) = 0$ for all irrational $x,$ $f(x) = 1$ for all rational $x,$ prove that $f\notin \mathcal{R}$ on $[a,b]$ for any $a<b.$
\end{problem}
    \begin{solution}
        Let $P$ be a partition on $[a,b].$ For any interval in such partition, we have by the density of the rationals and irrationals that a rational and an irrational is in that interval, and thus for all $i,$ $M_i = 1$ and $m_i  = 0.$ Thus, 
        \[U(P, f)  - L(P,f)= \sum_{i =1}^n M_i - m_i \Delta x_i = \sum_{i=1}^n \Delta x_i = b-a.\] Thus, $f$ is not Riemann-integrable.
    \end{solution}
    
\newpage
\section*{Problem 5}
\begin{problem}
    Suppose $f$ is a bounded real function on $[a,b],$ and $f^2 \in \mathcal{R}$ on $[a,b],$ does it follow that $f \in \mathcal{R}?$ Does the answer change if we assume that $f^3 \in \mathcal{R}.$
\end{problem}
\begin{solution}
    Define $f:[a,b]\to \bbR$ such that 
    \[f(x) = 
    \begin{cases}
        1, \qquad x \in \bbQ\\
        -1, \qquad x \in \bbR \setminus \bbQ
    \end{cases}.\]
    We have that $f^2 \equiv 1$ which is obviously Riemann integrable, but by the previous part we showed that $f \notin \mathcal{R}.$ However if $f^3 \in \mathcal{R},$ then we claim that $f\in \mathcal{R}.$ This is due to the following lemma: if $\varphi$ is continuous and $f$ is bounded and integrable, then $\varphi \circ f$ is Riemann integrable (which was proved last quarter) in the book using the Riemann-Lebesgue criterion. Thus, since $\varphi = \sqrt[3]{x}$ is continuous and bounded on $[a,b]$ and $f^3\in \mathcal{R},$ then $\varphi \circ f^3 = \sqrt[3]{f^3} = f\in n\mathcal{R}.$   
\end{solution}
\newpage
\section*{Problem 6}
\begin{problem}
    Let $C$ be the Cantor set. Let $f$ be a bounded real function on $[0,1]$ which is continuous at every point outside of $C.$ Prove that $f\in \mathcal{R}$ on $[0,1].$
\end{problem}
\begin{solution}
Let $\epsilon>0,$ take $M = \sup_{x\in [0,1]}|f(x)|.$ Since $C$ has zero measure, there exists a countable covering $\{(a_k, b_k)\}$ such that 
\[\sum_{k=1}^\infty b_k - a_k< \frac{\epsilon}{4M}.\] Since $C$ is compact, there exists a finite subcover and thus
\[\sum_{k=1}^n b_k - a_k < \frac{\epsilon}{4M}.\] Split $[0,1]$ into $[0,1] = \{[a_k, b_k]\}_{k\in [n]} \cup [0,1]\setminus \bigcup\{(a_k, b_k)\}_{k\in [n]},$ which gives a natural partition of $[0,1]$ that contains the finite subcover. Consider that $[0,1]\setminus \bigcup_{k=1}^n (a_k, b_k)$ is compact in $\bbR,$ and thus $f$ is uniformly continuous on this set. Let $P'$ be a refinement of $P$ such that $||P'||< \delta,$ where $\delta$ is the one from uniform continuity of $f$ in the set outside the cover of $C$ that gives $|f(x) - f(y)| < \frac{\epsilon}{2(b-a)}$ 
Then 
\[U(P',f) - L(P',f) = \sum_{i=1}^n (M_i - m_i)\Delta x_i = \sum_{x_i \in \{[a_k, b_k]\}_{k\in [n]}} (M_i - m_i)\Delta x_i + \sum_{x_i\notin \{[a_k, b_k]\}_{k\in [n]}}(M_i - m_i)\Delta x_i.\] We need to bound both of these terms. The second one is easy, and we already did the work by setting $||P'||< \delta,$ since
\[\sum_{x_i\notin \{[a_k, b_k]\}_{k\in [n]}}(M_i - m_i)\Delta x_i \leq \frac{\epsilon}{2(b-a)}\sum_{x_i\notin \{[a_k, b_k]\}_{k\in [n]}} \Delta x_i \leq \frac{\epsilon}{2}.\] For the first term, 
\[\sum_{i \in \{[a_k, b_k]\}_{k\in [n]}} (M_i - m_i)\Delta x_i \leq 2M \sum_{k=1}^n b_k - a_k < \frac{\epsilon}{2}.\] We are done, but it is worth mentioning that if $M=0,$ our proof fails, but we could just replace $M$ everywhere with $K =  M+1$ and then our proof would hold so I don't really care.
\end{solution}

\newpage
\section*{Problem 7}
\begin{problem}
    Suppose $f$ is a real function on $(0,1]$ and $f\in \mathcal{R}$ on $[c,1]$ for every $c>0.$ Define 
    \[\int_0^1 f(x)dx = \lim_{c\to 0}\int_c^1 f(x)dx\] if this limit exists (and it finite).
\end{problem}
\begin{enumerate}
    \item 
    \begin{problem}
        If $f\in \mathcal{R}$ on $[0,1],$ show that this definition of the integral agrees with the old one.
    \end{problem}
    \begin{solution}
        Let $\epsilon>0.$ Let $M= \sup_{x\in [0,1]} |f(x)|.$ Then as $c\to 0,$ we claim that
        \[\left|\int_c^1 f(x)dx - \int_0^1 f(x)dx\right| \leq \epsilon.\] Since $f\in \mathcal{R},$ we have that there exists some $P$ partition of $[0,1]$ such that 
        \[U(P,f) - L(P,f) = \sum_{i=1}^n (M_i - m_i)\Delta x_i < \frac{\epsilon}{4}.\] This in turn implies that 
        \[|\int_0^1 f - U(P,f)| < \frac{\epsilon}{4}, \qquad |\int_0^1 f - L(P,f)|< \frac{\epsilon}{4}.\]
        
        Thus, because $c\in [x_{k-1}, x_k]$ for some $k\in [n],$ then since $c\to 0,$ we can make $c< \frac{\epsilon}{4M},$ and thus for the interval $[c,1],$ we obviously still have that 
        \[\sum_{i=k}^n (M_i - m_i)\Delta x_i < \frac{\epsilon}{4},\] 
        \[|\int_c^1 f - U(P_c,f)| < \frac{\epsilon}{4}, \qquad |\int_c^1 f - L(P_c,f)|< \frac{\epsilon}{4}.\]
        but we also know that 
        \[|\sum_{i=1}^n M_i \Delta x_i - \sum_{i=k}^n M_i \Delta x_i| = |\sum_{i=1}^k M_i \Delta x_i|\leq M\sum_{i=1}^k \Delta x_i| < M\frac{\epsilon}{4M} = \frac{\epsilon}{4}\] Similarly, 
        \[|L(P,f) - L(P_c,f)| < \frac{\epsilon}{4}.\] We are pretty much done now, since we have shown that
        \begin{align*}
        \left|\int_c^1 f(x)dx - \int_0^1 f(x)dx\right| &\leq \left| \int_0^1 f - U(P,f)\right| + \left|U(P,f) - L(P,f)\right|\\
        &\quad + \left|L(P,f) - L(P_c,f)\right| + \left|L(P_c, f) - \int_c^1 f\right|\\
        &< \epsilon.
        \end{align*}
    
    \end{solution}
    \item 
    \begin{problem}
        Construct a function $f$ such that the above limit exists, although it fails to exist with $|f|$ in place of $f.$
    \end{problem}
    \begin{solution}
        Consider $f:(0,1]\to \bbR$ such that 
        \[f(x) = 
        \begin{cases}
            0, \qquad x = 0\\
            \frac{2^{2n}}{2n-1}, \qquad x\in [\frac{1}{2^{2n}}, \frac{1}{2^{2n-1}})\\
            \frac{-2^{2n-1}}{2n}, \qquad x\in (\frac{1}{2^{2n-1}}, \frac{1}{2^{2n-2}}]
        \end{cases}
        \]
        then we sum over all the intervals of length 
        \[\int_{\frac{1}{2^{2n}}}^\frac{1}{2^{2n-1}} f(x) = \frac{1}{2n-1}, \qquad \int_\frac{1}{2^{2n-1}}^ \frac{1}{2^{2n-2}}f(x) = \frac{-1}{n}\]
        \[\int_c^1 f(x) = \sum_{n=1}^N (\frac{1}{2n-1} - \frac{1}{n}) = \sum_{n=1}^N \frac{(-1)^{n}}{n}.\] Thus, as $c\to 0,$ we take $N\to \infty$ and so the integral converges because its the alternating harmonic test. However, $|f(x)|$ results in the harmonic series which is divergent.
    \end{solution}
\end{enumerate}

\newpage
\section*{Problem 8}
\begin{problem}
    Suppose $f\in \mathcal{R}$ on $[a,b]$ for every $b>a$ where $a$ is fixed. Define 
    \[\int_a^\infty f(x)dx =\lim_{b\to \infty}\int_a^b f(x)dx\] if this limit exists (and is finite). In that case, we say that the integral on the left converges. If it also converges after $f$ has been replaced by $|f|,$ it is said to converge absolutely.\\
    Assume that $f(x)\geq 0$ and that $f$ decreases monotonically on $[1, \infty).$ Prove that 
    \[\int_a^\infty f(x)dx\] converges if and only if 
    \[\sum_{n=1}^\infty f(n)\] converges.
\end{problem}
\begin{solution}
Without loss of generality since we only care about the behavior of the tail section, we let $a = 1.$ Thus, consider that 
\[\int_0^1f(x)dx \geq f(1), \quad \int_1^2 f(x)dx \geq f(2), \dots \implies \sum_{k=1}^n f(k) \leq \sum_{k=0}^n \int_{k-1}^{k}f(x)dx= f(0) + \int_1^n f(x)dx.\] By the comparison test, $F_n = \sum_{n=1}^\infty f(n)$ converges. Conversely, 
\[f(1)\geq \int_1^2 f(x)dx, \dots \implies \sum_{k=1}^n f(k) \geq \sum_{k=1}^n \int_{k}^{k+1}f(x)dx\] which again converges by the comparison test. Moreover, we have that from both equations above:
\[\sum_{k=1}^n [f(k)] - f(0) \leq \int_1^n f \leq \sum_{k=1}^{n-1} f(n),\] and so the integral and the sum converge and diverge together.
\end{solution}

\newpage
\section*{Problem 9}
\begin{problem}
    Show that integration by parts can sometimes be applied to the ``improper" integrals defined in Problems 7 and 8 (state appropriate hypothesis, formulate a theorem, and prove it.) For instance show that 
\[\int_0^\infty \frac{\cos x}{1 + x}dx = \int_0^\infty \frac{\sin x}{(1 + x)^2}dx.\] Show that one of these integrals converges \textit{absolutely,} but that the other does not.
\end{problem}
\begin{solution}
    Let $f, g$ be functions defined on some open interval containing $[a, b]$ such that $f'$ and $g'$ exist and are continuous on $(0, b]$ (Problem 7) or $[a, \infty)$ (Problem 7). Then if (7) $\lim_{a\to \infty} f(a)g(a)$ exists or $\lim_{b\to \infty}f(g)g(b)$ exists and $\lim_{a\to \infty} \int_a^b f'g < \infty $ or (8) $\lim_{b\to \infty} \int_a^b f'g < \infty,$ then 
	\[
		\int_0^b fg' = \lim_{a\to 0} [f(b)g(b) - f(a)g(a)] - \lim_{a\to 0}\int_a^b f' g.
	\]
    \[\int_a^\infty fg' = \lim_{b\to \infty} [f(b)g(b) - f(a)g(a)] - \lim_{b\to \infty}\int_a^b f' g.\] To prove this theorem, simply look at positive $a$ or finite $b$ and run integration by parts on them, then, for example
    \[\int_a^b fg' = [f(b)g(b) - f(a)g(a)] - \int_a^b f'g,\] where the RHS exists by assumption, so as $a\to 0$ or $b\to \infty,$ the LHS will converge. 
    Letting $f = \sin(x)$ and $g = (1 + x)^{-1},$ we get that using our theorem above, since 
    \[\lim_{b\to \infty} f(b)g(b) = 0, \qquad \lim_{b\to \infty}\int_0^b f'g = \int_0^\infty \frac{\sin(x)}{(1 + x)^2}dx < \int_0^\infty \frac{1}{(1 + x)^2}dx = \frac{1}{2},\] then we can use our integration by parts theorem and show that 
    \[-\int_0^\infty \frac{\sin(x)}{(1 + x)^2} = 0 - \int_0^\infty \frac{\cos(x)}{1 + x} dx.\] We showed above that 
    \[\int_0^\infty \frac{\sin(x)}{(1 + x)^2}\] converges absolutely, and we claim the other doesn't converge absolutely:
    \begin{align*}
        \int_0^\infty \frac{|\cos(x)|}{|1 + x|}dx  &= \int_0^\infty \frac{|\cos x|}{1 + x}dx\\
        &\geq \sum_{k=0}^\infty \int_{\pi k}^{\pi(k+1)}\frac{|\cos x|}{1 + x} dx
    \end{align*}
    Since $1+x$ is increasing on each $[\pi k, \pi (k+1)]$ interval, then $\frac{1}{1 + x} \geq \frac{1}{1 + \pi(k+1)}$ on each interval. Thus, since we know that $\int_0^\pi |\cos(x)|dx = \int_\pi^{2\pi}\sin(x) = 2,$ then
    \begin{align*}
        \sum_{k=0}^\infty \int_{\pi k}^{\pi(k+1)}\frac{|\cos x|}{1 + x} dx &\geq \sum_{k=0}^{\infty}\frac{1}{1 + \pi(k+1)}\int_{\pi k}^{\pi(k+1)}|\cos(x)|dx\\
        &= \sum_{k=0}^\infty \frac{2}{1 + \pi k + \pi)}\\
        &\geq \sum_{k=0}^\infty \frac{2}{\pi (k+2)}\\
        &= \frac{2}{\pi}\sum_{k=0}^\infty \frac{1}{k+2}\\
        &> \infty
    \end{align*}
\end{solution}


\newpage
\section*{Problem 10}
\begin{problem}
    Suppose $f$ is a real, continuously differentiable function on $[a,b],$ $f(a) = f(b) = 0,$ and 
    \[\int_a^bf^2(x)dx = 1.\] Prove that 
    \[\int_a^b xf(x)f'(x)dx = -\frac{1}{2}\]
    and that 
    \[\int_a^b[f'(x)]^2dx \cdot \int_a^b x^2 f^2(x)dx \geq \frac{1}{4}.\]
\end{problem}
\begin{solution}
    Using the integration by parts formula:
    \[\int uv' = uv - \int u'v,\] if we let
    \[u = f^2(x), \quad u' = 2f(x)f'(x), \quad v = x, \quad v' = 1\]
    then
    \[1 = \int_a^b f^2(x)dx = 2xf^2(x)|_a^b - \int_a^b 2f(x)f'(x)dx.\] The first term on the RHS is $0$ due to the fact that $f(a) = f(b) = 0$ by assumption and thus we get the desired result.\\
    Apply the Cauchy-Shwartz Inequality, which states that 
    \[\langle \varphi, g\rangle\leq |\varphi| \cdot |g|,\] we let $\varphi = f'(x)$ and $g = xf(x)$ to find that in the usual inner product of function spaces,
    \[\frac{-1}{2} = \int_a^b xf(x)f'(x)\leq \sqrt{\int_a^bx^2f^2(x)dx} \sqrt{\int_a^b[f'(x)]^2dx}.\] Square both sides and we are done.
\end{solution}

\newpage
\section*{Problem 11}
\begin{problem}
    For $1< s< \infty,$ define 
    \[\zeta(s) = \sum_{n=1}^\infty \frac{1}{n^s}.\] Prove that 
    \end{problem}

    \begin{enumerate}
        \item 
        \begin{problem}
            \[\zeta(s) = s\int_a^\infty \frac{\lfloor x\rfloor}{x^{s+1}}dx\]
        \end{problem}
        \begin{solution}
        Since the floor function remains constant in intervals $[n, n+1],$ we get that
        \begin{align*}
            s\int_1^\infty \frac{\lfloor c \rfloor}{x^{s+1}}dx &= \sum_{n=1}^\infty \int_n^{n+1}\frac{sn}{x^{s+1}}dx\\
            &= \sum_{n=1}^\infty sn\left[\frac{1}{sn^s} - \frac{1}{s(n+1)^s}| \right]\\
            &= \sum_{n=1}^\infty \frac{n}{n^s} - \sum_{n=1}^\infty \frac{n}{(n+1)^s}
        \end{align*}
            Using a change of variable in the second summation:
            \[\sum_{n=1}^\infty \frac{n}{n^s} - \sum_{n=1}^\infty \frac{n}{(n+1)^s} = \sum_{n=1}^\infty \frac{n}{n^s} - \sum_{n=1}^\infty \frac{n-1}{(n)^s} = \sum_{n=1}^\infty \frac{1}{n^s} = \zeta(s)\]
        \end{solution}
        \item 
        \begin{problem}
            \[\zeta(s) = \frac{s}{s-1} - s\int_1^\infty \frac{x - \lfloor x \rfloor }{x^{s+1}}.\]
        \end{problem}
        \begin{solution}
            By part $a,$ we have that 
            \[\zeta(s) = s\int_1^\infty \frac{\lfloor x\rfloor}{x^{s+1}} \implies \zeta(s) - s\int_1^\infty \frac{x}{x^{s+1}} = s\int_1^\infty \frac{x - \lfloor x \rfloor }{x^{s+1}}.\] Thus, it suffices to show that 
            \[s\int_1^\infty \frac{x}{x^{s+1}} = \frac{s}{s-1}.\] Simplify
            \[s\int_1^\infty \frac{x}{x^{s+1}} = s\int_1^\infty \frac{1}{x^s} = s\frac{x^{s-1}}{s-1}\bigg|_1^\infty = \frac{s}{s-1}\]
        \end{solution}
    \end{enumerate}
\newpage
\section*{Problem 12}
\begin{problem}
    Suppose $\alpha$ increases monotonically on $[a,b],$ $g$ is continuous, and $g(x) = G'(x)$ for $a\leq x \leq b.$ Prove that
    \[\int_a^b \alpha(x)g(x)dx = G(b)\alpha(b) - G(a)\alpha(a) - \int_a^b G d\alpha.\]
\end{problem}
\begin{solution}
    Let $P$ be a partition on $[a,b].$ By the mean value theorem, we know that there exists some $t_i$ for every interval $t_i\in[x_{i-1}, x_i]$ such that 
    \[G(x_i) - G(x_{i-1}) = G'(t_i)\Delta x_i = g(t_i)\Delta x_i.\] Thus, we see that by opening parenthesis:
    \begin{align*}
    \sum_{i=1}^n \alpha(x_i)g(t_i)\Delta x_i &= \sum_{i=1}^n \alpha(x_i)(G(x_i) - G(x_{i-1}))\\ &= \alpha(x_1)(G(x_1) - G(x_0)) + \alpha(x_2)(G(x_2) - G(x_1) + \dots + \alpha(x_n)(G(x_n) - G(x_{n-1}))\\
    &= \alpha(x_n)G(x_n) - \alpha(x_1)G(x_0) + \alpha(x_1)G(x_1) - \alpha(x_2)G(x_1) + \dots\\
    &= \alpha(x_n)G(x_n) - \alpha(x_0)G(x_0) + G(x_1)(\alpha(x_1) - \alpha(x_2)) + G(x_2)(\alpha(x_2) - \alpha(x_3)) + \dots\\
    &= \alpha(x_n)G(x_n) - \alpha(x_0)G(x_0) - \sum_{i=1}^n G(x_{i-1})\Delta \alpha_i.
    \end{align*}
    By continuity of $G,$ $G\in \mathcal{R}(\alpha)$ and moreover using the same process as we have in class and in previous exercises, it is obvious that by taking $n\to \infty,$ we have by the continuity of $G$ that 
    \[\overline{\int_a^b} Gd\alpha - \sum_{i=1}^n G\Delta \alpha_i < \epsilon, \qquad \underline{\int_a^b} Gd\alpha - \sum_{i=1}^n G\Delta \alpha_i < \epsilon.\] Thus, by taking limits in the equation above, we reach our desired result (noting that $\alpha g\in \mathcal{R}$ by the Riemann-Lebesgue Theorem discussed last quarter).\footnote{The left hand side of the equation similarly gets $\epsilon$ close to $\int_a^b \alpha g dx$ by the same argument as the right hand side}
\end{solution}
\end{document}