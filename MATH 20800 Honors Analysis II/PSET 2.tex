\documentclass[11pt]{article}

% NOTE: Add in the relevant information to the commands below; or, if you'll be using the same information frequently, add these commands at the top of paolo-pset.tex file. 
\newcommand{\name}{Agustín Esteva}
\newcommand{\email}{aesteva@uchicago.edu}
\newcommand{\classnum}{208}
\newcommand{\subject}{Honors Analysis in $\bbR^n$ II}
\newcommand{\instructors}{Panagiotis E. Souganidis}
\newcommand{\assignment}{Problem Set 2}
\newcommand{\semester}{Winter 2025}
\newcommand{\duedate}{2024-20-01}
\newcommand{\bA}{\mathbf{A}}
\newcommand{\bB}{\mathbf{B}}
\newcommand{\bC}{\mathbf{C}}
\newcommand{\bD}{\mathbf{D}}
\newcommand{\bE}{\mathbf{E}}
\newcommand{\bF}{\mathbf{F}}
\newcommand{\bG}{\mathbf{G}}
\newcommand{\bH}{\mathbf{H}}
\newcommand{\bI}{\mathbf{I}}
\newcommand{\bJ}{\mathbf{J}}
\newcommand{\bK}{\mathbf{K}}
\newcommand{\bL}{\mathbf{L}}
\newcommand{\bM}{\mathbf{M}}
\newcommand{\bN}{\mathbf{N}}
\newcommand{\bO}{\mathbf{O}}
\newcommand{\bP}{\mathbf{P}}
\newcommand{\bQ}{\mathbf{Q}}
\newcommand{\bR}{\mathbf{R}}
\newcommand{\bS}{\mathbf{S}}
\newcommand{\bT}{\mathbf{T}}
\newcommand{\bU}{\mathbf{U}}
\newcommand{\bV}{\mathbf{V}}
\newcommand{\bW}{\mathbf{W}}
\newcommand{\bX}{\mathbf{X}}
\newcommand{\bY}{\mathbf{Y}}
\newcommand{\bZ}{\mathbf{Z}}

%% blackboard bold math capitals
\newcommand{\bbA}{\mathbb{A}}
\newcommand{\bbB}{\mathbb{B}}
\newcommand{\bbC}{\mathbb{C}}
\newcommand{\bbD}{\mathbb{D}}
\newcommand{\bbE}{\mathbb{E}}
\newcommand{\bbF}{\mathbb{F}}
\newcommand{\bbG}{\mathbb{G}}
\newcommand{\bbH}{\mathbb{H}}
\newcommand{\bbI}{\mathbb{I}}
\newcommand{\bbJ}{\mathbb{J}}
\newcommand{\bbK}{\mathbb{K}}
\newcommand{\bbL}{\mathbb{L}}
\newcommand{\bbM}{\mathbb{M}}
\newcommand{\bbN}{\mathbb{N}}
\newcommand{\bbO}{\mathbb{O}}
\newcommand{\bbP}{\mathbb{P}}
\newcommand{\bbQ}{\mathbb{Q}}
\newcommand{\bbR}{\mathbb{R}}
\newcommand{\bbS}{\mathbb{S}}
\newcommand{\bbT}{\mathbb{T}}
\newcommand{\bbU}{\mathbb{U}}
\newcommand{\bbV}{\mathbb{V}}
\newcommand{\bbW}{\mathbb{W}}
\newcommand{\bbX}{\mathbb{X}}
\newcommand{\bbY}{\mathbb{Y}}
\newcommand{\bbZ}{\mathbb{Z}}

%% script math capitals
\newcommand{\sA}{\mathscr{A}}
\newcommand{\sB}{\mathscr{B}}
\newcommand{\sC}{\mathscr{C}}
\newcommand{\sD}{\mathscr{D}}
\newcommand{\sE}{\mathscr{E}}
\newcommand{\sF}{\mathscr{F}}
\newcommand{\sG}{\mathscr{G}}
\newcommand{\sH}{\mathscr{H}}
\newcommand{\sI}{\mathscr{I}}
\newcommand{\sJ}{\mathscr{J}}
\newcommand{\sK}{\mathscr{K}}
\newcommand{\sL}{\mathscr{L}}
\newcommand{\sM}{\mathscr{M}}
\newcommand{\sN}{\mathscr{N}}
\newcommand{\sO}{\mathscr{O}}
\newcommand{\sP}{\mathscr{P}}
\newcommand{\sQ}{\mathscr{Q}}
\newcommand{\sR}{\mathscr{R}}
\newcommand{\sS}{\mathscr{S}}
\newcommand{\sT}{\mathscr{T}}
\newcommand{\sU}{\mathscr{U}}
\newcommand{\sV}{\mathscr{V}}
\newcommand{\sW}{\mathscr{W}}
\newcommand{\sX}{\mathscr{X}}
\newcommand{\sY}{\mathscr{Y}}
\newcommand{\sZ}{\mathscr{Z}}


\renewcommand{\emptyset}{\O}

\newcommand{\abs}[1]{\lvert #1 \rvert}
\newcommand{\norm}[1]{\lVert #1 \rVert}
\newcommand{\sm}{\setminus}


\newcommand{\sarr}{\rightarrow}
\newcommand{\arr}{\longrightarrow}

% NOTE: Defining collaborators is optional; to not list collaborators, comment out the line below.
%\newcommand{\collaborators}{Alyssa P. Hacker (\texttt{aphacker}), Ben Bitdiddle (\texttt{bitdiddle})}

% Copyright 2021 Paolo Adajar (padajar.com, paoloadajar@mit.edu)
% 
% Permission is hereby granted, free of charge, to any person obtaining a copy of this software and associated documentation files (the "Software"), to deal in the Software without restriction, including without limitation the rights to use, copy, modify, merge, publish, distribute, sublicense, and/or sell copies of the Software, and to permit persons to whom the Software is furnished to do so, subject to the following conditions:
%
% The above copyright notice and this permission notice shall be included in all copies or substantial portions of the Software.
% 
% THE SOFTWARE IS PROVIDED "AS IS", WITHOUT WARRANTY OF ANY KIND, EXPRESS OR IMPLIED, INCLUDING BUT NOT LIMITED TO THE WARRANTIES OF MERCHANTABILITY, FITNESS FOR A PARTICULAR PURPOSE AND NONINFRINGEMENT. IN NO EVENT SHALL THE AUTHORS OR COPYRIGHT HOLDERS BE LIABLE FOR ANY CLAIM, DAMAGES OR OTHER LIABILITY, WHETHER IN AN ACTION OF CONTRACT, TORT OR OTHERWISE, ARISING FROM, OUT OF OR IN CONNECTION WITH THE SOFTWARE OR THE USE OR OTHER DEALINGS IN THE SOFTWARE.

\usepackage{fullpage}
\usepackage{enumitem}
\usepackage{amsfonts, amssymb, amsmath,amsthm}
\usepackage{mathtools}
\usepackage[pdftex, pdfauthor={\name}, pdftitle={\classnum~\assignment}]{hyperref}
\usepackage[dvipsnames]{xcolor}
\usepackage{bbm}
\usepackage{graphicx}
\usepackage{mathrsfs}
\usepackage{pdfpages}
\usepackage{tabularx}
\usepackage{pdflscape}
\usepackage{makecell}
\usepackage{booktabs}
\usepackage{natbib}
\usepackage{caption}
\usepackage{subcaption}
\usepackage{physics}
\usepackage[many]{tcolorbox}
\usepackage{version}
\usepackage{ifthen}
\usepackage{cancel}
\usepackage{listings}
\usepackage{courier}

\usepackage{tikz}
\usepackage{istgame}

\hypersetup{
	colorlinks=true,
	linkcolor=blue,
	filecolor=magenta,
	urlcolor=blue,
}

\setlength{\parindent}{0mm}
\setlength{\parskip}{2mm}

\setlist[enumerate]{label=({\alph*})}
\setlist[enumerate, 2]{label=({\roman*})}

\allowdisplaybreaks[1]

\newcommand{\psetheader}{
	\ifthenelse{\isundefined{\collaborators}}{
		\begin{center}
			{\setlength{\parindent}{0cm} \setlength{\parskip}{0mm}
				
				{\textbf{\classnum~\semester:~\assignment} \hfill \name}
				
				\subject \hfill \href{mailto:\email}{\tt \email}
				
				Instructor(s):~\instructors \hfill Due Date:~\duedate	
				
				\hrulefill}
		\end{center}
	}{
		\begin{center}
			{\setlength{\parindent}{0cm} \setlength{\parskip}{0mm}
				
				{\textbf{\classnum~\semester:~\assignment} \hfill \name\footnote{Collaborator(s): \collaborators}}
				
				\subject \hfill \href{mailto:\email}{\tt \email}
				
				Instructor(s):~\instructors \hfill Due Date:~\duedate	
				
				\hrulefill}
		\end{center}
	}
}

\renewcommand{\thepage}{\classnum~\assignment \hfill \arabic{page}}

\makeatletter
\def\points{\@ifnextchar[{\@with}{\@without}}
\def\@with[#1]#2{{\ifthenelse{\equal{#2}{1}}{{[1 point, #1]}}{{[#2 points, #1]}}}}
\def\@without#1{\ifthenelse{\equal{#1}{1}}{{[1 point]}}{{[#1 points]}}}
\makeatother

\newtheoremstyle{theorem-custom}%
{}{}%
{}{}%
{\itshape}{.}%
{ }%
{\thmname{#1}\thmnumber{ #2}\thmnote{ (#3)}}

\theoremstyle{theorem-custom}

\newtheorem{theorem}{Theorem}
\newtheorem{lemma}[theorem]{Lemma}
\newtheorem{example}[theorem]{Example}

\newenvironment{problem}[1]{\color{black} #1}{}

\newenvironment{solution}{%
	\leavevmode\begin{tcolorbox}[breakable, colback=green!5!white,colframe=green!75!black, enhanced jigsaw] \proof[\scshape Solution:] \setlength{\parskip}{2mm}%
	}{\renewcommand{\qedsymbol}{$\blacksquare$} \endproof \end{tcolorbox}}

\newenvironment{reflection}{\begin{tcolorbox}[breakable, colback=black!8!white,colframe=black!60!white, enhanced jigsaw, parbox = false]\textsc{Reflections:}}{\end{tcolorbox}}

\newcommand{\qedh}{\renewcommand{\qedsymbol}{$\blacksquare$}\qedhere}

\definecolor{mygreen}{rgb}{0,0.6,0}
\definecolor{mygray}{rgb}{0.5,0.5,0.5}
\definecolor{mymauve}{rgb}{0.58,0,0.82}

% from https://github.com/satejsoman/stata-lstlisting
% language definition
\lstdefinelanguage{Stata}{
	% System commands
	morekeywords=[1]{regress, reg, summarize, sum, display, di, generate, gen, bysort, use, import, delimited, predict, quietly, probit, margins, test},
	% Reserved words
	morekeywords=[2]{aggregate, array, boolean, break, byte, case, catch, class, colvector, complex, const, continue, default, delegate, delete, do, double, else, eltypedef, end, enum, explicit, export, external, float, for, friend, function, global, goto, if, inline, int, local, long, mata, matrix, namespace, new, numeric, NULL, operator, orgtypedef, pointer, polymorphic, pragma, private, protected, public, quad, real, return, rowvector, scalar, short, signed, static, strL, string, struct, super, switch, template, this, throw, transmorphic, try, typedef, typename, union, unsigned, using, vector, version, virtual, void, volatile, while,},
	% Keywords
	morekeywords=[3]{forvalues, foreach, set},
	% Date and time functions
	morekeywords=[4]{bofd, Cdhms, Chms, Clock, clock, Cmdyhms, Cofc, cofC, Cofd, cofd, daily, date, day, dhms, dofb, dofC, dofc, dofh, dofm, dofq, dofw, dofy, dow, doy, halfyear, halfyearly, hh, hhC, hms, hofd, hours, mdy, mdyhms, minutes, mm, mmC, mofd, month, monthly, msofhours, msofminutes, msofseconds, qofd, quarter, quarterly, seconds, ss, ssC, tC, tc, td, th, tm, tq, tw, week, weekly, wofd, year, yearly, yh, ym, yofd, yq, yw,},
	% Mathematical functions
	morekeywords=[5]{abs, ceil, cloglog, comb, digamma, exp, expm1, floor, int, invcloglog, invlogit, ln, ln1m, ln, ln1p, ln, lnfactorial, lngamma, log, log10, log1m, log1p, logit, max, min, mod, reldif, round, sign, sqrt, sum, trigamma, trunc,},
	% Matrix functions
	morekeywords=[6]{cholesky, coleqnumb, colnfreeparms, colnumb, colsof, corr, det, diag, diag0cnt, el, get, hadamard, I, inv, invsym, issymmetric, J, matmissing, matuniform, mreldif, nullmat, roweqnumb, rownfreeparms, rownumb, rowsof, sweep, trace, vec, vecdiag, },
	% Programming functions
	morekeywords=[7]{autocode, byteorder, c, _caller, chop, abs, clip, cond, e, fileexists, fileread, filereaderror, filewrite, float, fmtwidth, has_eprop, inlist, inrange, irecode, matrix, maxbyte, maxdouble, maxfloat, maxint, maxlong, mi, minbyte, mindouble, minfloat, minint, minlong, missing, r, recode, replay, return, s, scalar, smallestdouble,},
	% Random-number functions
	morekeywords=[8]{rbeta, rbinomial, rcauchy, rchi2, rexponential, rgamma, rhypergeometric, rigaussian, rlaplace, rlogistic, rnbinomial, rnormal, rpoisson, rt, runiform, runiformint, rweibull, rweibullph,},
	% Selecting time-span functions
	morekeywords=[9]{tin, twithin,},
	% Statistical functions
	morekeywords=[10]{betaden, binomial, binomialp, binomialtail, binormal, cauchy, cauchyden, cauchytail, chi2, chi2den, chi2tail, dgammapda, dgammapdada, dgammapdadx, dgammapdx, dgammapdxdx, dunnettprob, exponential, exponentialden, exponentialtail, F, Fden, Ftail, gammaden, gammap, gammaptail, hypergeometric, hypergeometricp, ibeta, ibetatail, igaussian, igaussianden, igaussiantail, invbinomial, invbinomialtail, invcauchy, invcauchytail, invchi2, invchi2tail, invdunnettprob, invexponential, invexponentialtail, invF, invFtail, invgammap, invgammaptail, invibeta, invibetatail, invigaussian, invigaussiantail, invlaplace, invlaplacetail, invlogistic, invlogistictail, invnbinomial, invnbinomialtail, invnchi2, invnF, invnFtail, invnibeta, invnormal, invnt, invnttail, invpoisson, invpoissontail, invt, invttail, invtukeyprob, invweibull, invweibullph, invweibullphtail, invweibulltail, laplace, laplaceden, laplacetail, lncauchyden, lnigammaden, lnigaussianden, lniwishartden, lnlaplaceden, lnmvnormalden, lnnormal, lnnormalden, lnwishartden, logistic, logisticden, logistictail, nbetaden, nbinomial, nbinomialp, nbinomialtail, nchi2, nchi2den, nchi2tail, nF, nFden, nFtail, nibeta, normal, normalden, npnchi2, npnF, npnt, nt, ntden, nttail, poisson, poissonp, poissontail, t, tden, ttail, tukeyprob, weibull, weibullden, weibullph, weibullphden, weibullphtail, weibulltail,},
	% String functions 
	morekeywords=[11]{abbrev, char, collatorlocale, collatorversion, indexnot, plural, plural, real, regexm, regexr, regexs, soundex, soundex_nara, strcat, strdup, string, strofreal, string, strofreal, stritrim, strlen, strlower, strltrim, strmatch, strofreal, strofreal, strpos, strproper, strreverse, strrpos, strrtrim, strtoname, strtrim, strupper, subinstr, subinword, substr, tobytes, uchar, udstrlen, udsubstr, uisdigit, uisletter, ustrcompare, ustrcompareex, ustrfix, ustrfrom, ustrinvalidcnt, ustrleft, ustrlen, ustrlower, ustrltrim, ustrnormalize, ustrpos, ustrregexm, ustrregexra, ustrregexrf, ustrregexs, ustrreverse, ustrright, ustrrpos, ustrrtrim, ustrsortkey, ustrsortkeyex, ustrtitle, ustrto, ustrtohex, ustrtoname, ustrtrim, ustrunescape, ustrupper, ustrword, ustrwordcount, usubinstr, usubstr, word, wordbreaklocale, worcount,},
	% Trig functions
	morekeywords=[12]{acos, acosh, asin, asinh, atan, atanh, cos, cosh, sin, sinh, tan, tanh,},
	morecomment=[l]{//},
	% morecomment=[l]{*},  // `*` maybe used as multiply operator. So use `//` as line comment.
	morecomment=[s]{/*}{*/},
	% The following is used by macros, like `lags'.
	morestring=[b]{`}{'},
	% morestring=[d]{'},
	morestring=[b]",
	morestring=[d]",
	% morestring=[d]{\\`},
	% morestring=[b]{'},
	sensitive=true,
}

\lstset{ 
	backgroundcolor=\color{white},   % choose the background color; you must add \usepackage{color} or \usepackage{xcolor}; should come as last argument
	basicstyle=\footnotesize\ttfamily,        % the size of the fonts that are used for the code
	breakatwhitespace=false,         % sets if automatic breaks should only happen at whitespace
	breaklines=true,                 % sets automatic line breaking
	captionpos=b,                    % sets the caption-position to bottom
	commentstyle=\color{mygreen},    % comment style
	deletekeywords={...},            % if you want to delete keywords from the given language
	escapeinside={\%*}{*)},          % if you want to add LaTeX within your code
	extendedchars=true,              % lets you use non-ASCII characters; for 8-bits encodings only, does not work with UTF-8
	firstnumber=0,                % start line enumeration with line 1000
	frame=single,	                   % adds a frame around the code
	keepspaces=true,                 % keeps spaces in text, useful for keeping indentation of code (possibly needs columns=flexible)
	keywordstyle=\color{blue},       % keyword style
	language=Octave,                 % the language of the code
	morekeywords={*,...},            % if you want to add more keywords to the set
	numbers=left,                    % where to put the line-numbers; possible values are (none, left, right)
	numbersep=5pt,                   % how far the line-numbers are from the code
	numberstyle=\tiny\color{mygray}, % the style that is used for the line-numbers
	rulecolor=\color{black},         % if not set, the frame-color may be changed on line-breaks within not-black text (e.g. comments (green here))
	showspaces=false,                % show spaces everywhere adding particular underscores; it overrides 'showstringspaces'
	showstringspaces=false,          % underline spaces within strings only
	showtabs=false,                  % show tabs within strings adding particular underscores
	stepnumber=2,                    % the step between two line-numbers. If it's 1, each line will be numbered
	stringstyle=\color{mymauve},     % string literal style
	tabsize=2,	                   % sets default tabsize to 2 spaces
%	title=\lstname,                   % show the filename of files included with \lstinputlisting; also try caption instead of title
	xleftmargin=0.25cm
}

% NOTE: To compile a version of this pset without problems, solutions, or reflections, uncomment the relevant line below.

%\excludeversion{problem}
%\excludeversion{solution}
%\excludeversion{reflection}

\begin{document}	
	
	% Use the \psetheader command at the beginning of a pset. 
	\psetheader

\section*{Problem 1}
\begin{problem}
Suppose \(0 < \delta < \pi\), \( f(x) = 1 \) if \( |x| \leq \delta \), \( f(x) = 0 \) if \( \delta < |x| \leq \pi \), and \( f(x + 2\pi) = f(x) \) for all \( x \).
\end{problem}
\begin{itemize}
\begin{problem}
        \item[(a)] Compute the Fourier coefficients of \( f \).
  \end{problem}
\begin{solution}
\[f(x) = \begin{cases}
    1, \qquad |x|\leq \delta\\
    0, \qquad \delta < |x|\leq \pi
\end{cases}\]
    We split the integral into the natural segments and use the definition of $f$ to compute $a_n$ and $b_n$ and $a_0$  (the complex coefficients of $f,$ wouldn't make sense since it is a real valued function).\footnote{Or at least I could not figure out how to make them work lol} We use the fact that $\sin(-kx) = -\sin(kx)$
    \begin{align*}
        a_k &= \langle f, \cos(kx)\rangle\\
        &= \frac{1}{\pi}\int_{-\pi}^\pi f(x)\cos(kx)dx\\
        &= \frac{1}{\pi}\int_{-\delta}^\delta f(x) \cos(kx)dx + 2\int_{\delta}^{\pi}f(x)\cos(kx)dx\\
        &= \frac{1}{\pi}\int_{-\delta}^\delta \frac{\cos(kx)}{k}dx\\
        &= \frac{2\sin(\delta k)}{\pi k}
    \end{align*}
    and for $b_k:$
    \begin{align*}
        b_m &= \langle f, \sin(mx)\rangle\\
        &= \frac{1}{\pi}\int_{-\delta}^\delta f(x) \sin(mx)dx + 2\int_{\delta}^{\pi}f(x)\sin(mx)dx\\
        &= \frac{1}{\pi}\int_{-\delta}^\delta \sin(mx)dx\\
        &= 0
    \end{align*}
    and finally:
    \[a_0 = \frac{1}{\pi}\int_{-\delta}^\delta f(x)dx = \frac{2\delta}{\pi}\]
\end{solution}  
    \item[(b)] Conclude that
    \[
    \sum_{n=1}^{\infty} \frac{\sin (n\delta)}{n} = \frac{\pi - \delta}{2}, \quad (0 < \delta < \pi).
    \]
    \begin{solution}
    We apply the localization theorem for $x = 0,$ since we know that \[|f(x + t) - f(x)|= 1\] for $0<|t|< \delta,$ then $f$ is locally Lipshitz. Thus, we have that $s_n(f,0) = f(0)$ in the limit, and so by the previous part:
    \[s_n(f,0) =a_0  + \sum_{n=1}^N a_n \cos(n(0)) = \frac{2\delta}{\pi} + \sum_{n=1}^N\frac{2\sin(\delta n)}{\pi n} \to f(x)\]
        Thus, 
        \[ \frac{2\delta}{\pi} + \sum_{n=1}^\infty\frac{2\sin(\delta n)}{\pi n}= 1 \iff \sum_{n=1}^{\infty} \frac{\sin (n\delta)}{n} = \frac{\pi - \delta}{2}\]
    \end{solution}
    \begin{problem}
            \item[(c)] Deduce from Parseval's theorem that
    \[
    \sum_{n=1}^{\infty} \frac{\sin^2 (n\delta)}{n^2 \delta} = \frac{\pi - \delta}{2}.
    \]
        \end{problem}    
    \begin{solution}
        From Parseval's theorem, we get that 
        \[
\frac{1}{\pi} \int_{-\pi}^{\pi} f^2(x) \, dx = \frac{1}{2} a_0^2 + \sum_{n=1}^{\infty} \left( a_n^2 + b_n^2 \right).
\]
Thus, plugging in all the stuff from before, we get that after some algebraic manipulation (multiplying by $\frac{\pi^2}{4\delta}$ and moving stuff around):
\[\frac{2\delta}{\pi} = \frac{1}{2}(\frac{2\delta}{\pi})^2 + \sum_{n=1}^\infty \frac{4\sin^2(\delta n)}{\pi^2 n^2} \iff \frac{\pi - \delta}{2} = \sum_{n=1}^\infty \frac{\sin^2(\delta n)}{n^2 \delta}\]
    \end{solution}
    
    \item[(d)] Let \( \delta \to 0 \) and prove that
    \[
    \int_0^{\infty} \left( \frac{\sin x}{x} \right)^2 dx = \frac{\pi}{2}.
    \]
    \begin{solution}
    The right hand side is fine by letting $\delta \to 0$ in the above equation, so we need to show that for all $\epsilon>0,$ as $\delta \to 0$
        \[\left|\int_0^\infty \left(\frac{\sin x}{x}\right)^2dx - \sum_{n=1}^\infty \frac{\sin^2(n\delta)}{n^2\delta}\right|< \epsilon.\]
        To provide intuition, we need to show that 
        \[\lim_{\delta \to 0}\left(\lim_{N\to \infty}\sum_{i=1}^N \frac{\sin^2(n\delta)}{n^2\delta}\right) = \int_0^\infty \left(\frac{\sin x}{x}\right)^2.\]
Let $b$ be large, and partition $[0,b]$ by $P = \{0, \delta, 2\delta, \dots, N\delta = b\}.$ By letting $x_n = n\delta$ and $\Delta x_n = \delta,$ then 
\begin{align}
\lim_{\delta \to 0}\sum_{n=1}^N \frac{\sin^2(n\delta)}{n^2 \delta} = \lim_{\Delta x_n \to 0} \sum_{n =1}^{N}\frac{\sin^2(x_n)}{x_n^2}\delta = \int_0^b \frac{\sin^2 (x)}{x^2}dx    
\end{align}
 Let $\epsilon>0.$ Let $b$ large such that 
\[\left|\int_0^\infty \frac{\sin^2 (x)}{x^2}dx - \int_0^b \frac{\sin^2 (x)}{x^2}dx\right|< \frac{\epsilon}{2}.\] Since $\delta \to 0,$ we find $\delta$ small enough such that by (1):
\[\left|\int_0^b\frac{\sin^2 (x)}{x^2}dx - \sum_{n=1}^N \frac{\sin^2(n\delta)}{n^2 \delta} \right| < \frac{\epsilon}{2}\] Combining the inequalities:
\begin{align*}
    \left|\int_0^\infty \left(\frac{\sin x}{x}\right)^2dx - \sum_{n=1}^\infty \frac{\sin^2(n\delta)}{n^2\delta}\right| &\leq \left| \int_0^\infty \frac{\sin^2 (x)}{x^2}dx - \int_0^b \frac{\sin^2 (x)}{x^2}dx\right| \\
    &\qquad + \left|\int_0^b\frac{\sin^2 (x)}{x^2}dx - \sum_{n=1}^N \frac{\sin^2(n\delta)}{n^2 \delta} \right|\\ &< \epsilon.
\end{align*}
Thus, by part (c), we are done (one could have done an $\frac{\epsilon}{3}$ argument to not be so wishy washy about connecting the integral to part (c), but the author thinks that is obvious enough).
        
    \end{solution}
    \begin{problem}
        
    \item[(e)] Put \( \delta = \pi/2 \) in (c). What do you get?
        \end{problem}
\begin{solution}
Define the Esteva function as 
\[\chi_{\mod 4}(n) = \begin{cases}
    1, \qquad n\mod 4 = 1\\
    0, \qquad n\mod 4 = 0\\
    -1, \qquad n\mod 4 = 3
\end{cases}\]
Then we can make the result prettier:
\begin{align*}
    \sum_{n=1}^\infty \frac{2\sin^2(n\frac{\pi}{2})}{n^2 \pi} = \frac{\pi}{4} &\iff \sum_{n=1}^\infty \frac{\sin^2(\frac{n\pi}{2})}{n^2} = \frac{\pi^2}{8} \iff \sum_{n=1}^\infty \frac{\chi_{\mod 4}(n)}{n^2}\\ &= 1 -\frac{1}{9} + \frac{1}{25} - \frac{1}{49} + \dots = \sum_{n=1}^\infty \frac{1}{(2n-1)^2}\\ &= 
    \frac{\pi^2}{8}
\end{align*}
    
\end{solution}
\end{itemize}
\newpage
\section*{Problem 2}
\begin{problem}
    Put $f(x) = x$ for $x\in [0,2\pi),$ and apply Parseval's Theorem to conclude that 
    \[\sum_{n=1}^\infty = \frac{\pi^2}{6}\]
\end{problem}
\begin{solution}
    We get that since $x$ is odd, we do not worry about the even cosine, and thus:
    \[a_0 = \frac{1}{\pi}\int_{0}^{2\pi} x dx = 2\pi,\]
    and using integration by parts:
    \[a_n = \frac{1}{\pi}\int_{-\pi}^\pi x\sin(nx)dx = \frac{-2}{n}.\] Thus:
    \[f(x) = \pi + \sum_{n=1}^\infty \frac{-2}{n}\sin(nx).\] Parseval gives us then that 
    \[\frac{1}{\pi}\int_{0}^{2\pi} x^2dx = \frac{8\pi^2}{3}= 2\pi^2 + \sum_{n=1}^\infty \frac{4}{n^2} \iff \frac{\pi^2}{6} = \sum_{n=1}^\infty \frac{1}{n^2}\]
\end{solution}
\newpage
\section*{Problem 3}
\begin{problem}
    If \( f(x) = (\pi - |x|)^2 \) on \( [-\pi, \pi] \), prove that

\[
f(x) = \frac{\pi^2}{3} + \sum_{n=1}^{\infty} \frac{4}{n^2} \cos nx
\]

and deduce that

\[
\sum_{n=1}^{\infty} \frac{1}{n^2} = \frac{\pi^2}{6}, \quad
\sum_{n=1}^{\infty} \frac{1}{n^4} = \frac{\pi^4}{90}.
\]
\end{problem}
\begin{solution}
    We calculate the Fourier series, by first noting that $f(x)$ is even and thus 
    \[\int_{-\pi}^\pi (\pi - |x|)^2dx = 2\int_{0}^\pi (\pi - x)^2 dx.\] Since $f$ is even, we need not even both calculating the coefficients of the odd $\sin(x)$ function, since we know them to be $0.$ Thus
    \[a_0 = \frac{1}{\pi}\int_{-\pi}^\pi (\pi - |x|)^2 = \frac{2}{\pi}\int_0^\pi (\pi - x)^2dx = -\frac{1}{3\pi}\left[(\pi - x)^3\right]_{0}^\pi =\frac{2\pi^2}{3}\] 
    \[a_n = \frac{2}{\pi}\int_0^\pi(\pi - x)^2 \cos(nx)dx = \frac{4}{n^2}\]
    To deduce the two relations, we first let $x = 0$ in the equation derived above and notice that 
    \[f(0) = \pi^2 = \frac{\pi^2}{3} + \sum_{n=1}^\infty \frac{4}{n^2} \iff \frac{2\pi^2}{3} = \sum_{n=1}^\infty \frac{4}{n^2} \iff \frac{\pi^2}{6} = \sum_{n=1}^\infty \frac{1}{n^2}.\] For the second relation, we use Parseval's identity:
    \begin{align*}
        \frac{1}{\pi}\int_{-\pi}^\pi |f^2|dx &= \frac{1}{\pi}\int_{-\pi}^\pi (\pi - |x|)^4dx\\
        &= \frac{2}{\pi}\int_{0}^\pi (\pi - x)^4dx\\
        &= \frac{2\pi^4}{5}\\
        &= \frac{1}{2}(\frac{2\pi^2}{3})^2 + \sum_{n=1}^\infty(\frac{4}{n^2})^2\\
        &= \frac{2\pi^4}{9} + \sum_{n=1}^\infty\frac{16}{n^4}
    \end{align*}
    Thus, we get that 
    \[\frac{36\pi^4}{90} = \frac{20\pi^4}{90} + \sum_{n=1}^\infty \frac{16}{n^4} \iff \frac{\pi^4}{90} = \sum_{n=1}^\infty \frac{16}{n^4}.\]
    
\end{solution}

\newpage
\section*{Problem 4}
\begin{problem}
    
With \( D_n \) as defined in (77), put


\[ K_N(x) = \frac{1}{N+1} \sum_{n=0}^{N} D_n(x). \]



Prove that


\[ K_N(x) = \frac{1}{N+1} \cdot \frac{1 - \cos (N+1)x}{1 - \cos x} \]
\begin{solution}
    By work done in class, we know that 
    \[D_n(x) = \frac{\sin(nx+\frac{x}{2})}{\sin(\frac{x}{2})}.\] Thus, it suffices to show that 
    \[\sum_{n=0}^N\frac{\sin(nx+\frac{x}{2})}{\sin(\frac{x}{2})} = \frac{1-\cos(Nx + x)}{1 - \cos(x)}\] Thus, consider that 
    \[(1-\cos(x))K_N(x) = \frac{1}{N+1}\sum_{n=0}^N\frac{\sin(nx+\frac{x}{2})}{\sin(\frac{x}{2})}(1-\cos(x)) = \frac{1}{N+1}\sum_{n=0}^N2\sin(\frac{x}{2})\sin(nx+\frac{x}{2})\]
    We use the multiplication of sin identity
    \[(1-\cos(x))K_N(x) = \frac{1}{N+1}\sum_{n=0}^N\cos(-nx) - \cos(x(n+1)) = \frac{1}{N+1}\sum_{n=0}^N\cos(nx) - \cos(nx+x)\] we then telescope the above sum:
    \[1 - \cos(x)K_N(x) = 1-\cos((N+1)x)\iff K_N(x) = \frac{1}{N+1} \cdot \frac{1 - \cos (N+1)x}{1 - \cos x}\]
\end{solution}



and that\end{problem}

\begin{itemize}
\begin{problem}
    
    \item[(a)] \( K_N \geq 0 \),
    \end{problem}
    \begin{solution}
        This is fairly obvious, as $\cos(x) \leq 1,$ then the denominator $1-\cos(x)\geq 0.$ Similarly, $\cos((N+1)x)\leq 1$ for all $N.$ 
    \end{solution}
\begin{problem}
    
    \item[(b)] \( \frac{1}{2\pi} \int_{-\pi}^{\pi} K_N(x) \, dx = 1 \),
    \end{problem}
\begin{solution}
    Recall that 
    \[\int_{-\pi}^\pi D_N(x)dx = 2\pi,\] thus we have that
    \[\frac{1}{2\pi}\int_{-\pi}^\pi \frac{1}{N+1}\sum_{n=0}^ND_n(x) = \frac{1}{2\pi}\frac{1}{N+1}\sum_{n=0}^N\int_{-\pi}^\pi D_n(x) = \frac{1}{2\pi}\frac{1}{N+1}(2\pi(N+1)) = 1\]
\end{solution}
\begin{problem}
        \item[(c)] \( K_N(x) \leq \frac{1}{N+1} \cdot \frac{2}{1 - \cos \delta} \) if \( 0 < \delta \leq |x| \leq \pi \).
        \end{problem}
        \begin{solution}
            
Since $\cos(x)$ is decreasing on $[0,\pi],$ then $1-\cos(x)$ is increasing on the interval, and thus since $\delta \leq |x|,$ then $1-\cos(\delta)\leq 1-\cos(|x|) = 1-\cos(x),$ which implies that $\frac{1}{1-\cos(\delta)}\geq \frac{1}{1-\cos(x)}.$ We also know that $\cos((N+1)x)\leq 2$ for all $N.$ Thus, 
\[K_N(x) = \frac{1}{N+1} \cdot \frac{1 - \cos (N+1)x}{1 - \cos x} \leq \frac{2}{1-\cos(\delta)}.\]

        \end{solution}
\end{itemize}
If \( s_N = s_N(f; x) \) is the \( N \)th partial sum of the Fourier series of \( f \), consider the arithmetic means


\[ \sigma_N = \frac{s_0 + s_1 + \cdots + s_N}{N+1}. \]



Prove that


\[ \sigma_N(f; x) = \frac{1}{2\pi} \int_{-\pi}^{\pi} f(x - t) K_N(t) \, dt, \]

\begin{solution}
    This follows directly from the definition of $K_N(t):$
    \begin{align*}
    \sigma_N &= \frac{\sum_{n=0}^N s_n(f,x)}{N+1}\\ &= \frac{\sum_{n=0}^N \frac{1}{2\pi} \int_{-\pi}^\pi f(x-t) D_n(t)}{N+1}dt\\ &= \frac{1}{2\pi}\int_{-\pi}^\pi f(x-t)\frac{1}{N+1}\sum_{n=0}^N D_n(t)dt\\ &= \frac{1}{2\pi}\int_{-\pi}^\pi f(x-t)K_N(t)dt.    
    \end{align*}
    
\end{solution}

and hence prove Fejér's theorem:
\textit{If \( f \) is continuous, with period \( 2\pi \), then \( \sigma_N(f; x) \to f(x) \) uniformly on \([- \pi, \pi]\).}
\begin{solution}
    Let $\epsilon>0$ and $x\in [-\pi, \pi].$ Since $f$ is continuous, then $f$ is uniformly continuous on $[-\pi, \pi].$ Thus, there exists some $\delta>0$ such that if $|t|< \delta,$ then $|f(x-t) - f(x)|< \epsilon$ for any $x\in [-\pi, \pi].$
    \begin{align*}
        \left| \sigma_N(f,x) - f(x) \right| &= \left|\frac{1}{2\pi}\int_{-\pi}^\pi f(x-t)K_N(t)dt - f(x)\right|\\
        &= \left|\frac{1}{2\pi}\int_{-\pi}^\pi f(x-t)K_N(t)dt - f(x)\frac{1}{2\pi} \int_{-\pi}^\pi K_N(t)dt\right|\\
        &= \left|\frac{1}{2\pi} \int_{-\pi}^\pi \left(f(x-t) - f(x)\right) K_N(t)dt\right|\\
        &\leq \frac{1}{2\pi} \int_{-\pi}^\pi \left|f(x-t) - f(x)\right| K_N(t)dt\\
        &= \frac{1}{2\pi}\int_{-\pi}^{-\delta}|f(x-t)- f(x)|K_N(t)dt\\ &\qquad + \frac{1}{2\pi}\int_{-\delta}^\delta |f(x -t) - f(x)|K_N(t)dt\\ &\qquad + \frac{1}{2\pi}\int_{\delta}^\pi |f(x-t) - f(x)|K_N(t)dt\\
    \end{align*}
    We know $f$ to be uniformly continuous, and thus $f$ is bounded. Let $M = \sup_{x\in [-\pi, \pi]}|f(x)|.$ Thus, for $x\in [\pm\pi, \pm\delta],$ we get that $|f(x-t) - f(x)|\leq 2M.$ Thus, we can pick some large $N,$ not dependent on $x,$ and then
    \begin{align*}
        |\sigma_N(f,x) - f(x)|&\leq \frac{2}{\pi}\int_{\delta}^\pi K_N(t)dt + \frac{1}{2\pi}\int_{-\delta}^\delta |f(x -t) - f(x)|K_N(t)dt\\
        &< \frac{4}{\pi(N+1)}\int_\delta^\pi\frac{1}{1-\cos\delta}dt + \epsilon \frac{1}{2\pi}\int_{-\delta}^\delta K_N(t)dt\\
        \to 0
    \end{align*}
    as $N\to \infty$ and $\epsilon\to 0.$ 
\end{solution}
\begin{reflection}
    I've realized that I have stopped caring so much about making my $\epsilon's$ prettier as time has gone along. Nowadays, I see an $\epsilon$ and I let it go to $0$ without arguing very hard.
\end{reflection}
\newpage
\section*{Problem 5}
\begin{problem}
If $f\in \mathcal{R}$ and $f(x+)$ and $f(x-),$ exist for some $x,$ then 
\[\lim_{n\to \infty}\sigma_n(f,x) = \frac{1}{2}[f(x+) + f(x-)].\]
\end{problem}
\begin{solution}
    Let $\epsilon>0.$ By the existence of the left hand limit, we know that there exists some $\delta_L$ such that if $t\in (\delta_L, 0),$ then $|f(x-t) - f(x-)|< \epsilon.$ 
    
    Similarly for the existence of such a $\delta_R.$ Thus, we get that
    \begin{align*}
        \left| \sigma_n(f,x) - \frac{1}{2}[f(x +) + f(x-)]\right| &= \left| \frac{1}{2\pi}\int_{x-\pi}^{x + \pi} f(x-t)K_n(t)dt - \frac{1}{2}[f(x +) + f(x-)] \right|\\
        &= \left|\frac{1}{2\pi}\int_{x-\pi}^{x + \pi} f(x-t)K_n(t)dt - \frac{1}{2\pi}\int_{[x-\pi, x+\pi]\sm \{x\}} \frac{1}{2}[f(x +) + f(x-)]K_ndt \right|\\
        &= \left|\frac{1}{2\pi}\int_{x-\pi}^{x + \pi} f(x-t)K_n(t)dt - \frac{1}{2\pi}\int_{x-\pi}^{x}f(x -)K_n(t)dt - \frac{1}{2\pi}\int_{x}^{x+\pi}f(x+)K_ndt \right|
        &= \frac{1}{2\pi}\left|\int_{x-\pi}^x [f(x-t) - f(x-)]K_n(t)dt + \int_{x}^{x+\pi} [f(x-t) - f(x+)]K_n(t)dt\right|\\
        &\leq \frac{1}{2\pi}\int_{x-\pi}^{x} |f(x-t) - f(x-)|K_n(t)dt + \int_{x}^{x+\pi} |f(x-t) - f(x+)|K_n(t)dt\\
    \end{align*}
    First, in order to make my life simpler, we know by Rudin that the interval over which we integrate does not matter as long as its total length is $2\pi.$ Thus, we translate everything by $-x,$ and we now split up the terms and argue symmetrically,
    \begin{align*}
        \frac{1}{2\pi}\int_{-\pi}^0 |f(-t) - f(0-)|K_n(t)dt &= \frac{1}{2\pi}\int_{-\pi}^{-\delta_L} |f(-t) - f(0-)|K_n(t)dt + \frac{1}{2\pi}\int_{-\delta_L}^{0}|f(-t) - f(0-)|K_n(t)dt
    \end{align*}
    Since $f\in \mathcal{R},$ then $f$ is bounded, so we can bound $|f(x-t) - \frac{1}{2}f(x)|\leq 2M$ for $x\in [-\pi, \delta_L].$ Then we do an identical argument to the previous problem that the difference goes to $0.$ Thus, 
    \[\left| \sigma_n(f,x) - \frac{1}{2}[f(x +) + f(x-)]\right| \to 0\]
\end{solution}

\newpage
\begin{problem}
    Assume $f$ is bounded and monotonic on $[-\pi, \pi)$, with Fourier coefficients $c_n$, as given by (62).

\begin{itemize}
    \item[(a)] 
    \begin{problem}
        
   Use Exercise 17 of Chap. 6 to prove that $\{n c_n\}$ is a bounded sequence.
    \end{problem}
    \begin{solution}
        Since $f$ is monotonic on $[-\pi, \pi),$ $e^{-inx}$ is continuous, $g(x) = \frac{1}{-in}e^{-inx} = \frac{i}{n}e^{-inx}$ for $-\pi\leq x\leq \pi,$ then exercise 17 of chapter 6 tells us that
        \[\frac{1}{2\pi}\int_{-\pi}^\pi f(x)e^{-inx}dx = \frac{i}{n}e^{-in\pi}f(\pi) - \frac{i}{n}e^{in\pi}f(-\pi) - \frac{i}{2\pi n}\int_{-\pi}^\pi e^{-inx}df(x)\] By definition, the left hand side is equal to $c_n.$ The first two terms in the right hand side go away since $f$ is $2-\pi$ periodic and thus $f(\pi) = f(-\pi)$ and thus using trig properties:
        \begin{align*}
            \frac{i}{n}e^{-in\pi}f(\pi) - \frac{i}{n}e^{in\pi}f(-\pi) &= \frac{i}{n}(e^{-in\pi}f(\pi) - e^{in\pi}f(-\pi))\\
            &= \frac{i}{n}(\cos(-n\pi)f(\pi) + i\sin(-n\pi)f(\pi) - \cos(n\pi)f(-\pi) - i\sin(n\pi)f(-\pi))\\
            &= \frac{i}{n}(\cos(-n\pi)f(\pi) - \cos(n\pi)f(-\pi))\\
            &= \frac{i\cos(n\pi)}{n}(f(\pi) - f(-\pi))\\
            &=\frac{i}{n}(f(\pi) - f(-\pi))
        \end{align*}
        Thus, 
        \begin{align*}
        c_n &= \frac{i}{n}(f(\pi) - f(-\pi))-\frac{i}{2\pi n}\int_{-\pi}^\pi e^{-inx} df(x)\\ \implies |c_n| &= \frac{1}{n}f(\pi) - f(-\pi)) + \frac{1}{2\pi n}\left|\int_{-\pi}^\pi e^{-inx}df(x)\right|\\
        &\leq \frac{1}{n}f(\pi) - f(-\pi)) + \frac{1}{2\pi n}\int_{-\pi}^\pi |e^{-inx}|df(x)
        \end{align*}
        Consider that 
        \[|e^{inx}| = 1\] Thus, since $|e^{-inx}|$ is bounded, then we have that if $M= \sup_{x\in [-\pi, \pi]} e^{-inx} = 1,$ then by Theorem 6.12 on Rudin, we get that
        \[|c_n| \leq \frac{1}{n}f(\pi) - f(-\pi)) + \frac{1}{2\pi n}\int_{-\pi}^\pi df(x) \leq \frac{1}{n}f(\pi) - f(-\pi)) + \frac{1}{2\pi n} [f(\pi) - f(-\pi)]\] and by multiplying by $n$ on both sides we find that 
        \[|nc_n|\leq \frac{1}{2\pi n}[f(\pi) - f(-\pi)]\] and is thus bounded.
        
    \end{solution}
    \item[(b)] 
    \begin{problem}
            Combine (a) with Exercise 16 and with Exercise 14(e) of Chap. 3, to conclude that
    \[
    \lim_{N \to \infty} s_N(f; x) = \frac{1}{2} [f(x+) + f(x-)]
    \]
    for every $x$.
        \end{problem}
\begin{solution}
    By exercise 14 in chapter 3, we know that since $|nc_n|\leq M$ for some $M,$ and since $f\in \mathcal{R}[(-\pi, \pi)]$\footnote{We know this because $f$ is monotonic and bounded, and thus has at most countably many discontinuities, and is thus integrable by the Riemann-Lebesgue Theorem}, then by the previous problem, 
    \[\lim_{n\to \infty} \sigma_n(f,x) = \frac{1}{2}[f(x-) + f(x+)],\] and thus
    \[\lim_{n\to \infty}s_n(f,x) = \frac{1}{2}[f(x-) + f(x+)].\] The only assumption left to check\footnote{We know $f(x-), f(x+)$ exist on every point because $f$ is mononic} is that 
    \[s_n - \sigma_n = \frac{1}{n+1}\sum_{k=1}^n k a_k,\] where $|k a_k| \leq M$ This is just definitional however, since 
    \begin{align*}
        s_n - \sigma_n &= s_n - \frac{s_0 + s_1 + \cdots + s_n}{n+1}\\
        &= \frac{(n+1)s_n}{n+1} - \frac{s_0 + s_1 + \cdots + s_n}{n+1}\\
        &= \frac{1}{n+1}\sum_{k=1}^n k c_k e^{inx},
    \end{align*}
    and thus $a_k = c_k e^{inx}$ which has a norm of just $c_k.$
    
\end{solution}
    \item[(c)] 
    \begin{problem}
            Assume only that $f \in \mathcal{R}$ on $[-\pi, \pi]$ and that $f$ is monotonic in some segment $(\alpha, \beta) \subset [-\pi, \pi]$. Prove that the conclusion of (b) holds for every $x \in (\alpha, \beta)$.
    (This is an application of the localization theorem.)
        \end{problem}
    \begin{solution}
        Let 
        \[\varphi(x) = \begin{cases}
            f(\alpha), \quad x\in (-\pi, \alpha]\\
            f(x), \quad x\in (\alpha, \beta)\\
            f(\beta), \quad x\in (\beta, \pi)
        \end{cases}.\]
        Since $\varphi$ is bounded and monotonic on $[\alpha, \beta),$ and we know that $\varphi(x-), \varphi(x+)$ exist on the interval, then we must have that 
        \[\lim_{N\to \infty} s_N(\varphi, x) = \frac{1}{2}[\varphi(x+) + \varphi(x-)] = .\] By the localization theorem, we know moreover that since $\varphi(x) = f(x)$ for $x\in (\alpha, \beta),$ then:
        \[\lim_{N\to \infty} s_N(\varphi - f,x) = \lim_{N\to \infty} s_N(\varphi,x) - s_N(f,x) = 0,\] and so
        \[\lim_{N\to \infty} s_N(f,x) = \frac{1}{2}[\varphi(x+) + \varphi(x-)] = \frac{1}{2}[f(x+) - f(x-)].\]
    \end{solution}
\end{itemize}
\end{problem}

\end{document}


