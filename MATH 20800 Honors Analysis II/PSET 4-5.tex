\documentclass[11pt]{article}

% NOTE: Add in the relevant information to the commands below; or, if you'll be using the same information frequently, add these commands at the top of paolo-pset.tex file. 
\newcommand{\name}{Agustín Esteva}
\newcommand{\email}{aesteva@uchicago.edu}
\newcommand{\classnum}{208}
\newcommand{\subject}{Honors Analysis in $\bbR^n$ II}
\newcommand{\instructors}{Panagiotis E. Souganidis}
\newcommand{\assignment}{Problem Set 4-5}
\newcommand{\semester}{Fall 2024}
\newcommand{\duedate}{2024-10-02}
\newcommand{\bA}{\mathbf{A}}
\newcommand{\bB}{\mathbf{B}}
\newcommand{\bC}{\mathbf{C}}
\newcommand{\bD}{\mathbf{D}}
\newcommand{\bE}{\mathbf{E}}
\newcommand{\bF}{\mathbf{F}}
\newcommand{\bG}{\mathbf{G}}
\newcommand{\bH}{\mathbf{H}}
\newcommand{\bI}{\mathbf{I}}
\newcommand{\bJ}{\mathbf{J}}
\newcommand{\bK}{\mathbf{K}}
\newcommand{\bL}{\mathbf{L}}
\newcommand{\bM}{\mathbf{M}}
\newcommand{\bN}{\mathbf{N}}
\newcommand{\bO}{\mathbf{O}}
\newcommand{\bP}{\mathbf{P}}
\newcommand{\bQ}{\mathbf{Q}}
\newcommand{\bR}{\mathbf{R}}
\newcommand{\bS}{\mathbf{S}}
\newcommand{\bT}{\mathbf{T}}
\newcommand{\bU}{\mathbf{U}}
\newcommand{\bV}{\mathbf{V}}
\newcommand{\bW}{\mathbf{W}}
\newcommand{\bX}{\mathbf{X}}
\newcommand{\bY}{\mathbf{Y}}
\newcommand{\bZ}{\mathbf{Z}}

%% blackboard bold math capitals
\newcommand{\bbA}{\mathbb{A}}
\newcommand{\bbB}{\mathbb{B}}
\newcommand{\bbC}{\mathbb{C}}
\newcommand{\bbD}{\mathbb{D}}
\newcommand{\bbE}{\mathbb{E}}
\newcommand{\bbF}{\mathbb{F}}
\newcommand{\bbG}{\mathbb{G}}
\newcommand{\bbH}{\mathbb{H}}
\newcommand{\bbI}{\mathbb{I}}
\newcommand{\bbJ}{\mathbb{J}}
\newcommand{\bbK}{\mathbb{K}}
\newcommand{\bbL}{\mathbb{L}}
\newcommand{\bbM}{\mathbb{M}}
\newcommand{\bbN}{\mathbb{N}}
\newcommand{\bbO}{\mathbb{O}}
\newcommand{\bbP}{\mathbb{P}}
\newcommand{\bbQ}{\mathbb{Q}}
\newcommand{\bbR}{\mathbb{R}}
\newcommand{\bbS}{\mathbb{S}}
\newcommand{\bbT}{\mathbb{T}}
\newcommand{\bbU}{\mathbb{U}}
\newcommand{\bbV}{\mathbb{V}}
\newcommand{\bbW}{\mathbb{W}}
\newcommand{\bbX}{\mathbb{X}}
\newcommand{\bbY}{\mathbb{Y}}
\newcommand{\bbZ}{\mathbb{Z}}

%% script math capitals
\newcommand{\sA}{\mathscr{A}}
\newcommand{\sB}{\mathscr{B}}
\newcommand{\sC}{\mathscr{C}}
\newcommand{\sD}{\mathscr{D}}
\newcommand{\sE}{\mathscr{E}}
\newcommand{\sF}{\mathscr{F}}
\newcommand{\sG}{\mathscr{G}}
\newcommand{\sH}{\mathscr{H}}
\newcommand{\sI}{\mathscr{I}}
\newcommand{\sJ}{\mathscr{J}}
\newcommand{\sK}{\mathscr{K}}
\newcommand{\sL}{\mathscr{L}}
\newcommand{\sM}{\mathscr{M}}
\newcommand{\sN}{\mathscr{N}}
\newcommand{\sO}{\mathscr{O}}
\newcommand{\sP}{\mathscr{P}}
\newcommand{\sQ}{\mathscr{Q}}
\newcommand{\sR}{\mathscr{R}}
\newcommand{\sS}{\mathscr{S}}
\newcommand{\sT}{\mathscr{T}}
\newcommand{\sU}{\mathscr{U}}
\newcommand{\sV}{\mathscr{V}}
\newcommand{\sW}{\mathscr{W}}
\newcommand{\sX}{\mathscr{X}}
\newcommand{\sY}{\mathscr{Y}}
\newcommand{\sZ}{\mathscr{Z}}


\renewcommand{\emptyset}{\O}

\newcommand{\abs}[1]{\lvert #1 \rvert}
\newcommand{\norm}[1]{\lVert #1 \rVert}
\newcommand{\sm}{\setminus}


\newcommand{\sarr}{\rightarrow}
\newcommand{\arr}{\longrightarrow}

% NOTE: Defining collaborators is optional; to not list collaborators, comment out the line below.
%\newcommand{\collaborators}{Alyssa P. Hacker (\texttt{aphacker}), Ben Bitdiddle (\texttt{bitdiddle})}

% Copyright 2021 Paolo Adajar (padajar.com, paoloadajar@mit.edu)
% 
% Permission is hereby granted, free of charge, to any person obtaining a copy of this software and associated documentation files (the "Software"), to deal in the Software without restriction, including without limitation the rights to use, copy, modify, merge, publish, distribute, sublicense, and/or sell copies of the Software, and to permit persons to whom the Software is furnished to do so, subject to the following conditions:
%
% The above copyright notice and this permission notice shall be included in all copies or substantial portions of the Software.
% 
% THE SOFTWARE IS PROVIDED "AS IS", WITHOUT WARRANTY OF ANY KIND, EXPRESS OR IMPLIED, INCLUDING BUT NOT LIMITED TO THE WARRANTIES OF MERCHANTABILITY, FITNESS FOR A PARTICULAR PURPOSE AND NONINFRINGEMENT. IN NO EVENT SHALL THE AUTHORS OR COPYRIGHT HOLDERS BE LIABLE FOR ANY CLAIM, DAMAGES OR OTHER LIABILITY, WHETHER IN AN ACTION OF CONTRACT, TORT OR OTHERWISE, ARISING FROM, OUT OF OR IN CONNECTION WITH THE SOFTWARE OR THE USE OR OTHER DEALINGS IN THE SOFTWARE.

\usepackage{fullpage}
\usepackage{enumitem}
\usepackage{amsfonts, amssymb, amsmath,amsthm}
\usepackage{mathtools}
\usepackage[pdftex, pdfauthor={\name}, pdftitle={\classnum~\assignment}]{hyperref}
\usepackage[dvipsnames]{xcolor}
\usepackage{bbm}
\usepackage{graphicx}
\usepackage{mathrsfs}
\usepackage{pdfpages}
\usepackage{tabularx}
\usepackage{pdflscape}
\usepackage{makecell}
\usepackage{booktabs}
\usepackage{natbib}
\usepackage{caption}
\usepackage{subcaption}
\usepackage{physics}
\usepackage[many]{tcolorbox}
\usepackage{version}
\usepackage{ifthen}
\usepackage{cancel}
\usepackage{listings}
\usepackage{courier}

\usepackage{tikz}
\usepackage{istgame}

\hypersetup{
	colorlinks=true,
	linkcolor=blue,
	filecolor=magenta,
	urlcolor=blue,
}

\setlength{\parindent}{0mm}
\setlength{\parskip}{2mm}

\setlist[enumerate]{label=({\alph*})}
\setlist[enumerate, 2]{label=({\roman*})}

\allowdisplaybreaks[1]

\newcommand{\psetheader}{
	\ifthenelse{\isundefined{\collaborators}}{
		\begin{center}
			{\setlength{\parindent}{0cm} \setlength{\parskip}{0mm}
				
				{\textbf{\classnum~\semester:~\assignment} \hfill \name}
				
				\subject \hfill \href{mailto:\email}{\tt \email}
				
				Instructor(s):~\instructors \hfill Due Date:~\duedate	
				
				\hrulefill}
		\end{center}
	}{
		\begin{center}
			{\setlength{\parindent}{0cm} \setlength{\parskip}{0mm}
				
				{\textbf{\classnum~\semester:~\assignment} \hfill \name\footnote{Collaborator(s): \collaborators}}
				
				\subject \hfill \href{mailto:\email}{\tt \email}
				
				Instructor(s):~\instructors \hfill Due Date:~\duedate	
				
				\hrulefill}
		\end{center}
	}
}

\renewcommand{\thepage}{\classnum~\assignment \hfill \arabic{page}}

\makeatletter
\def\points{\@ifnextchar[{\@with}{\@without}}
\def\@with[#1]#2{{\ifthenelse{\equal{#2}{1}}{{[1 point, #1]}}{{[#2 points, #1]}}}}
\def\@without#1{\ifthenelse{\equal{#1}{1}}{{[1 point]}}{{[#1 points]}}}
\makeatother

\newtheoremstyle{theorem-custom}%
{}{}%
{}{}%
{\itshape}{.}%
{ }%
{\thmname{#1}\thmnumber{ #2}\thmnote{ (#3)}}

\theoremstyle{theorem-custom}

\newtheorem{theorem}{Theorem}
\newtheorem{lemma}[theorem]{Lemma}
\newtheorem{example}[theorem]{Example}

\newenvironment{problem}[1]{\color{black} #1}{}

\newenvironment{solution}{%
	\leavevmode\begin{tcolorbox}[breakable, colback=green!5!white,colframe=green!75!black, enhanced jigsaw] \proof[\scshape Solution:] \setlength{\parskip}{2mm}%
	}{\renewcommand{\qedsymbol}{$\blacksquare$} \endproof \end{tcolorbox}}

\newenvironment{reflection}{\begin{tcolorbox}[breakable, colback=black!8!white,colframe=black!60!white, enhanced jigsaw, parbox = false]\textsc{Reflections:}}{\end{tcolorbox}}

\newcommand{\qedh}{\renewcommand{\qedsymbol}{$\blacksquare$}\qedhere}

\definecolor{mygreen}{rgb}{0,0.6,0}
\definecolor{mygray}{rgb}{0.5,0.5,0.5}
\definecolor{mymauve}{rgb}{0.58,0,0.82}

% from https://github.com/satejsoman/stata-lstlisting
% language definition
\lstdefinelanguage{Stata}{
	% System commands
	morekeywords=[1]{regress, reg, summarize, sum, display, di, generate, gen, bysort, use, import, delimited, predict, quietly, probit, margins, test},
	% Reserved words
	morekeywords=[2]{aggregate, array, boolean, break, byte, case, catch, class, colvector, complex, const, continue, default, delegate, delete, do, double, else, eltypedef, end, enum, explicit, export, external, float, for, friend, function, global, goto, if, inline, int, local, long, mata, matrix, namespace, new, numeric, NULL, operator, orgtypedef, pointer, polymorphic, pragma, private, protected, public, quad, real, return, rowvector, scalar, short, signed, static, strL, string, struct, super, switch, template, this, throw, transmorphic, try, typedef, typename, union, unsigned, using, vector, version, virtual, void, volatile, while,},
	% Keywords
	morekeywords=[3]{forvalues, foreach, set},
	% Date and time functions
	morekeywords=[4]{bofd, Cdhms, Chms, Clock, clock, Cmdyhms, Cofc, cofC, Cofd, cofd, daily, date, day, dhms, dofb, dofC, dofc, dofh, dofm, dofq, dofw, dofy, dow, doy, halfyear, halfyearly, hh, hhC, hms, hofd, hours, mdy, mdyhms, minutes, mm, mmC, mofd, month, monthly, msofhours, msofminutes, msofseconds, qofd, quarter, quarterly, seconds, ss, ssC, tC, tc, td, th, tm, tq, tw, week, weekly, wofd, year, yearly, yh, ym, yofd, yq, yw,},
	% Mathematical functions
	morekeywords=[5]{abs, ceil, cloglog, comb, digamma, exp, expm1, floor, int, invcloglog, invlogit, ln, ln1m, ln, ln1p, ln, lnfactorial, lngamma, log, log10, log1m, log1p, logit, max, min, mod, reldif, round, sign, sqrt, sum, trigamma, trunc,},
	% Matrix functions
	morekeywords=[6]{cholesky, coleqnumb, colnfreeparms, colnumb, colsof, corr, det, diag, diag0cnt, el, get, hadamard, I, inv, invsym, issymmetric, J, matmissing, matuniform, mreldif, nullmat, roweqnumb, rownfreeparms, rownumb, rowsof, sweep, trace, vec, vecdiag, },
	% Programming functions
	morekeywords=[7]{autocode, byteorder, c, _caller, chop, abs, clip, cond, e, fileexists, fileread, filereaderror, filewrite, float, fmtwidth, has_eprop, inlist, inrange, irecode, matrix, maxbyte, maxdouble, maxfloat, maxint, maxlong, mi, minbyte, mindouble, minfloat, minint, minlong, missing, r, recode, replay, return, s, scalar, smallestdouble,},
	% Random-number functions
	morekeywords=[8]{rbeta, rbinomial, rcauchy, rchi2, rexponential, rgamma, rhypergeometric, rigaussian, rlaplace, rlogistic, rnbinomial, rnormal, rpoisson, rt, runiform, runiformint, rweibull, rweibullph,},
	% Selecting time-span functions
	morekeywords=[9]{tin, twithin,},
	% Statistical functions
	morekeywords=[10]{betaden, binomial, binomialp, binomialtail, binormal, cauchy, cauchyden, cauchytail, chi2, chi2den, chi2tail, dgammapda, dgammapdada, dgammapdadx, dgammapdx, dgammapdxdx, dunnettprob, exponential, exponentialden, exponentialtail, F, Fden, Ftail, gammaden, gammap, gammaptail, hypergeometric, hypergeometricp, ibeta, ibetatail, igaussian, igaussianden, igaussiantail, invbinomial, invbinomialtail, invcauchy, invcauchytail, invchi2, invchi2tail, invdunnettprob, invexponential, invexponentialtail, invF, invFtail, invgammap, invgammaptail, invibeta, invibetatail, invigaussian, invigaussiantail, invlaplace, invlaplacetail, invlogistic, invlogistictail, invnbinomial, invnbinomialtail, invnchi2, invnF, invnFtail, invnibeta, invnormal, invnt, invnttail, invpoisson, invpoissontail, invt, invttail, invtukeyprob, invweibull, invweibullph, invweibullphtail, invweibulltail, laplace, laplaceden, laplacetail, lncauchyden, lnigammaden, lnigaussianden, lniwishartden, lnlaplaceden, lnmvnormalden, lnnormal, lnnormalden, lnwishartden, logistic, logisticden, logistictail, nbetaden, nbinomial, nbinomialp, nbinomialtail, nchi2, nchi2den, nchi2tail, nF, nFden, nFtail, nibeta, normal, normalden, npnchi2, npnF, npnt, nt, ntden, nttail, poisson, poissonp, poissontail, t, tden, ttail, tukeyprob, weibull, weibullden, weibullph, weibullphden, weibullphtail, weibulltail,},
	% String functions 
	morekeywords=[11]{abbrev, char, collatorlocale, collatorversion, indexnot, plural, plural, real, regexm, regexr, regexs, soundex, soundex_nara, strcat, strdup, string, strofreal, string, strofreal, stritrim, strlen, strlower, strltrim, strmatch, strofreal, strofreal, strpos, strproper, strreverse, strrpos, strrtrim, strtoname, strtrim, strupper, subinstr, subinword, substr, tobytes, uchar, udstrlen, udsubstr, uisdigit, uisletter, ustrcompare, ustrcompareex, ustrfix, ustrfrom, ustrinvalidcnt, ustrleft, ustrlen, ustrlower, ustrltrim, ustrnormalize, ustrpos, ustrregexm, ustrregexra, ustrregexrf, ustrregexs, ustrreverse, ustrright, ustrrpos, ustrrtrim, ustrsortkey, ustrsortkeyex, ustrtitle, ustrto, ustrtohex, ustrtoname, ustrtrim, ustrunescape, ustrupper, ustrword, ustrwordcount, usubinstr, usubstr, word, wordbreaklocale, worcount,},
	% Trig functions
	morekeywords=[12]{acos, acosh, asin, asinh, atan, atanh, cos, cosh, sin, sinh, tan, tanh,},
	morecomment=[l]{//},
	% morecomment=[l]{*},  // `*` maybe used as multiply operator. So use `//` as line comment.
	morecomment=[s]{/*}{*/},
	% The following is used by macros, like `lags'.
	morestring=[b]{`}{'},
	% morestring=[d]{'},
	morestring=[b]",
	morestring=[d]",
	% morestring=[d]{\\`},
	% morestring=[b]{'},
	sensitive=true,
}

\lstset{ 
	backgroundcolor=\color{white},   % choose the background color; you must add \usepackage{color} or \usepackage{xcolor}; should come as last argument
	basicstyle=\footnotesize\ttfamily,        % the size of the fonts that are used for the code
	breakatwhitespace=false,         % sets if automatic breaks should only happen at whitespace
	breaklines=true,                 % sets automatic line breaking
	captionpos=b,                    % sets the caption-position to bottom
	commentstyle=\color{mygreen},    % comment style
	deletekeywords={...},            % if you want to delete keywords from the given language
	escapeinside={\%*}{*)},          % if you want to add LaTeX within your code
	extendedchars=true,              % lets you use non-ASCII characters; for 8-bits encodings only, does not work with UTF-8
	firstnumber=0,                % start line enumeration with line 1000
	frame=single,	                   % adds a frame around the code
	keepspaces=true,                 % keeps spaces in text, useful for keeping indentation of code (possibly needs columns=flexible)
	keywordstyle=\color{blue},       % keyword style
	language=Octave,                 % the language of the code
	morekeywords={*,...},            % if you want to add more keywords to the set
	numbers=left,                    % where to put the line-numbers; possible values are (none, left, right)
	numbersep=5pt,                   % how far the line-numbers are from the code
	numberstyle=\tiny\color{mygray}, % the style that is used for the line-numbers
	rulecolor=\color{black},         % if not set, the frame-color may be changed on line-breaks within not-black text (e.g. comments (green here))
	showspaces=false,                % show spaces everywhere adding particular underscores; it overrides 'showstringspaces'
	showstringspaces=false,          % underline spaces within strings only
	showtabs=false,                  % show tabs within strings adding particular underscores
	stepnumber=2,                    % the step between two line-numbers. If it's 1, each line will be numbered
	stringstyle=\color{mymauve},     % string literal style
	tabsize=2,	                   % sets default tabsize to 2 spaces
%	title=\lstname,                   % show the filename of files included with \lstinputlisting; also try caption instead of title
	xleftmargin=0.25cm
}

% NOTE: To compile a version of this pset without problems, solutions, or reflections, uncomment the relevant line below.

%\excludeversion{problem}
%\excludeversion{solution}
%\excludeversion{reflection}

\begin{document}	
	
	% Use the \psetheader command at the beginning of a pset. 
	\psetheader

\section*{Problem 1}
\begin{problem}
Let $E$ be a vector space of dimension $n$ and let $(e_i)_{1\leq i\leq n}$ be a basis of $E.$ That is, given $x\in E,$ we can write 
\[x = \sum_{i=1}^n x_i e_i,\] with $x_i \in \bbR.$ Given $f\in E^\ast,$ we can set $f_i = f(e_i).$ Recall that we define the duality map $F: E \to E^\ast$ as
\[F(x) = \{f\in E^* \; ; \; \|f\| = \|x\| \;\; \text{and} \;\; \langle f, x\rangle = \|x\|^2\}\]
    \begin{enumerate}
    \item Consider on \( E \) the norm
    

\[
    \|x\|_1 = \sum_{i=1}^{n} |x_i|.
    \]


    \begin{enumerate}
        \item Compute explicitly, in terms of the \( f_i \)'s, the dual norm \( \|f\|_{E^*} \) of \( f \in E^* \).
        \begin{solution}
            \begin{align*}
                \|f\|_{E^\ast} &= \sup_{x \in E, \|x\| = 1} |\langle f, x\rangle|\\
                &= \sup_{x \in E, \|x\| = 1} |\langle f, \sum_{i=1}^n x_i e_i\rangle|\\
                &= \sup_{x \in E, \|x\| = 1} |\sum_{i=1}^n x_i\langle f, e_i\rangle|\\
                &= \sup_{x \in E, \|x\| = 1} |\sum_{i=1}^n x_if_i|\\
                &\leq \sup_{x \in E, \|x\| = 1} |\sum_{i=1}^n x_i\max_{1\leq i \leq n}f(e_i)|\\
                &= \max_{1\leq i \leq n}|f(e_i)||\sum_{i=1}^n x_i|\\
                &\leq \max_{i\in [n]}|f_i|
            \end{align*}
            Suppose $\max_{i\in [n]}|f_i| = f(e_j)$ for some $j\in [n].$ then we have that if $x \in E$ such that $x_i = 0$ and $x_j = 1,$ then $\|x\| = 1$ and 
            \[f(e_j) = f_j = |\sum_{i=1}^n x_i f_i| \leq \sup_{x\in E, \|x\| = 1}|\sum_{i=1}^n x_i f_i| = \|f\|_{E^\ast}.\] Thus, we have that $\max_{i\in [n]}f_i = \|f\|_{E^\ast}$
        \end{solution}
        \item Determine explicitly the set \( F(x) \) (duality map) for every \( x \in E \).
        \begin{solution}
            We claim that, 
            \[F(x) = \{f \; ; \; f(e_i) = \text{sign}(x_i) \|x\|\}.\] Evidently, \[\|f\| = \max_{i\in [n]}|f_i| = \max_{i\in [n]}\|x\| = \|x\|\] and 
            \[\langle f, x \rangle = \sum_{i=1}^n x_i f_i = \sum_{i=1}^n |x_i|\|x\| = \|x\|\sum_{i=1}^n |x_i| = \|x\|^2\]
        \end{solution}
    \end{enumerate}
    \item Same questions but where \( E \) is provided with the norm
    

\[
    \|x\|_\infty = \max_{1 \leq i \leq n} |x_i|.
    \]
    \begin{solution}
        From the previous problem, we have that 
        \begin{align*}
            \|f\|_{E^\ast} &= \sup_{x \in E, \|x\| = 1} |\sum_{i=1}^n x_if_i|\\
            &\leq \sup_{x \in E, \|x\| = 1} \sum_{i=1}^n |x_i||f_i|\\
            &= \sum_{i=1}^n |f_i|
        \end{align*}
        Conversely, we have that if $x\in E$ with $x_i = \text{sign}(f_i)$ for all $i\in [n],$ then we have that 
        \[\sum_{i=1}^n |f_i| = \sum_{i=1}^n |x_i f_i| = |\sum_{i=1}^n x_if_i| \leq \sup_{x\in E, \|x\| = 1}|\sum_{i=1}^n x_if_i| = \|f\|_{E^*}\]
        \end{solution}
    \begin{solution}
        We claim that $f\in F(x)$ if 
        \[f_i = 0 \qquad \text{if}\;\; |x_i| \neq \max_{k\in [n]}|x_k|\]
        \[\sum_{i=1}^n|f_i| = \max_{k\in n}|x_i| \qquad \text{if}\;\; |x_i| \neq \max_{k\in [n]}|x_k|\]
        Thus, we obviously have that 
        \[\|f\|_{E^*} = \|x\|_\infty\] and sure enough, we have that
        \[\langle f, x\rangle = \sum_{i=1}^nx_i f_i = \sum_{x_i = \|x\|_\infty} x_i^2 = \|x\|^2_\infty\]
    \end{solution}

    \item Same questions but where \( E \) is provided with the norm
    

\[
    \|x\|_2 = \left( \sum_{i=1}^{n} |x_i|^2 \right)^{1/2},
    \]


    and more generally with the norm
    

\[
    \|x\|_p = \left( \sum_{i=1}^{n} |x_i|^p \right)^{1/p}, \quad \text{where } p \in (1, \infty).
    \]
    \begin{solution}
    Let $q\in (1,\infty)$ such that $\frac{1}{p} = \frac{1}{q} = 1.$ We use holder's inequality from last PSET
        \begin{align*}
            \|f\|_{E^\ast} &= \sup_{x \in E, \|x\| = 1} |\sum_{i=1}^n x_if_i|\\
            &= \sup_{x \in E, \|x\| = 1} \sum_{i=1}^n |x_i||f_i|\\
            &=\sup_{x \in E, \|x\| = 1} \left(\sum_{i=1}^n |x_i|^p\right)^\frac{1}{p}\left( \sum_{i=1}^n |f_i|^q\right)^\frac{1}{q}\\
            &= \left(\sum_{i=1}^n |f_i|^q\right)^\frac{1}{q}
        \end{align*}
    \end{solution}
\begin{solution}
    $f_i \in F(x)$ if 
    \[\left(\sum_{i=1}^n |x_i|^p\right)^\frac{1}{p} = \left(\sum_{i=1}^n |f_i|^q\right)^\frac{1}{q}\]
    and 
    \[\sum_{i=1}^n|f_i||x_i| = \left(\sum_{i=1}^n |x_i|^p\right)^\frac{2}{p}.\] Using H\"older's inequality on the last statement we find that applying the first equality,
    \[\sum_{i=1}^n|f_i||x_i| \leq \left(\sum_{i=1}^n |f_i|^q\right)^\frac{1}{q}\left(\sum_{i=1}^n |x_i|^p\right)^\frac{1}{p} = \left(\sum_{i=1}^n |x_i|^p\right)^\frac{2}{p},\] an so
    \[\langle f, x\rangle = \|f\|_q\|x\|_p.\] As proved in a previous PSET, an equality like this implies that 
    \[\sum_{i=1}^n |\hat{x_i}||\hat{f_i}| = 1,\] where
    \[\hat{x_i} = \frac{x_i}{\|x\|_p}, \qquad \hat{f_i} = \frac{f_i}{\|f\|_q}.\] Reversing the argument from the previous PSET, we get that
    \[f_i^q = Cx_i^p \implies f_i = C^\frac{1}{q}x_i^\frac{p}{q}.\] Using the second line, we find that 
    \[\sum_{i=1}^n |f_i||x_i| = \sum_{i=1}^n |C^\frac{1}{q}x_i^\frac{p}{q}||x_i^\frac{1}{p}| = C^\frac{1}{q} \sum_{i=1}^n |x_i^{p-1}|x_i| = C^\frac{1}{q}\sum_{i=1}^n |x_i|^p = \left(\sum_{i=1}^n |x_i|^p\right)^\frac{2}{p}.\]
    Thus, 
    \[C^\frac{1}{q} = (\|x\|_p)^{2-p} \implies f_i= (\|x\|_p)^{2-p} |x_i|^{p-1}\]

    We claim that 
    \[F(x) = \]
\end{solution}


\end{enumerate}

\end{problem}

\newpage
\section*{Problem 2}
\begin{problem}
    Consider the space \( E = c_0 \) (sequences tending to zero) with its usual norm (see Section 11.3). For every element \( u = (u_1, u_2, u_3, \ldots) \) in \( E \) define


\[ 
f(u) = \sum_{n=1}^{\infty} \frac{1}{2^n} u_n. 
\]



\begin{enumerate}
    \item Check that \( f \) is a continuous linear functional on \( E \) and compute \( \|f\|_{E^*} \).
\begin{solution}
    Obviously, $f: E \to \bbR.$ 

    Let $u, w \in E$ and $\lambda\in \bbR.$ Then we have that 
    \[f(\lambda  u + w) = \sum_{n=1}^{\infty} \frac{1}{2^n} (\lambda u_n + w_n) = \lambda \sum_{n=1}^{\infty} \frac{1}{2^n} u_n + \sum_{n=1}^{\infty} \frac{1}{2^n} w_n = \lambda f(u) + f(w)\]

    Let $u\in E.$ Then 
    \begin{align}
    \|f(u)\|_\bbR = |\sum_{n=1}^{\infty} \frac{1}{2^n} u_n| \leq \sum_{n=1}^{\infty} \frac{1}{2^n} |u_n| \leq \|u\|\sum_{n=1}^{\infty} \frac{1}{2^n}  = \|u\|    
    \end{align}
     and thus $f$ is bounded linear functional. Thus, $f$ is continuous, and so $f\in E^*.$ Using (1), we see that
    \begin{align*}
        \|f\|_{E^*} &= \sup_{u\in E, \|u\| = 1} \|f(u)\|\\
        &= \sup_{u\in E, \|u\| = 1} \|f(u)\|\\
        &= \sup_{u\in E, \|u\| = 1}\|u\|\\
        &= 1.
    \end{align*}
\end{solution}

    
    \item Can one find some \( u \in E \) such that \( \|u\| = 1 \) and \( f(u) = \|f\|_{E^*} \)?
    \begin{solution}
        No, but suppose we can. That is, there exists some $u \in E$ with $\|u\| = 1$ such that 
        \[\sum_{n=1}^{\infty} \frac{1}{2^n} u_n = 1.\] We claim that for all $n,$ $u_n = 1.$ Suppose not, that there exists some $i$ such that $u_n <1.$ Then 
        \[f(u) = \sum_{n = 1}^\infty \frac{1}{2^n}u_n = \sum_{n\neq i}\frac{1}{2^n} + \frac{1}{2^i}u_n < \sum_{n=1}^\infty \frac{1}{2^n} = 1.\] Thus, $f(u) < 1.$ 
    \end{solution}
    
\end{enumerate}

\newpage
\section*{Problem 3}
\begin{problem}
    Let $E$ be a normed vector space and let $H\subset E$ be a hyperplane. Let $V\subset E$ be an affine subspace such that $H\subset V.$
\end{problem}
\begin{enumerate}
    \item Prove that either $V = H$ or $V = E.$
    \begin{solution}
    Since $V$ is affine, there exists some linear subspace $U\subset E$ such that 
    \[V = x_0 + U.\]
        Suppose $V\neq H.$ That is, there exists some $v\in V$ such that $f(v) \neq \alpha.$ Let $x \in E.$ Either $f(x) = \alpha,$ in which case $x\in H \subset V$ and we are done, or $f(x) \neq \alpha.$ We wish to show that $x = x_0 + u$ for some $u\in U.$ 

        Suppose $f(v) = f(x).$ Then $f(v - x) = 0$ and so if $z\in H,$ we have that $f(v - x + h) = \alpha,$ and so $v-x + h \in H$ and so $v- x + h \in V.$  Consider now that $(v-x + z) \in V,$ $v\in V$ and  $z\in V,$ then there exist $u_{i}$ for $i = \{1,2,3\}$ such that
        \[x = -(x - y + z) + x + z = -(x_0 + u_1) + (x_0 + u_2) + (x_0 + u_3) = x_0 + (u_2 + u_3 - u_1).\] Since $U$ is a linear subspace, $(u_2 + u_3 - u_1)\in U$ and thus $x\in V.$

        Suppose $f(v) \neq f(x),$ and now consider 
        \[g(t) = tf(v) + (1-t)f(x)\] to be the line intersecting $f(v)$ and $f(x).$ We claim that for some $t,$ $g(t) = \alpha.$ Indeed, after some algebra, one can confirm that if 
        \[\tau = \frac{f(x) - \alpha}{f(x) - f(v)}, \implies g(\tau)= \alpha.\] Thus, $g(\tau) = \tau f(v) - (1-\tau)f(x)\in H.$ Moreover, we have that 
        \[\alpha = \tau f(v) + (1-\tau)f(x) = f(v\tau) + f((1-\tau)x) = f(v\tau + (1-\tau)x),\] and so 
        \[v\tau + (1-\tau)x = \in H \subset V. \implies v\tau + (1-\tau)x = u_1 + x_0, \quad u_1 \in U.\] Since $v\in V,$ we let $v = u_2 + x_0,$ where $u_2 \in U.$ Thus, we get that 
        \[(1-\tau)x = (1-\tau)x_0 + u_1 - \tau u_2,\] and so $(1-\tau)x,$ dividng by $(1-\tau)$ and using the fact that $U$ is a subspace gives that $x\in V.$ 

        In either case, we have that $E \subset V \subset E,$ and thus $E = V$ whenever $H \neq V.$
    \end{solution}
    \item Prove that either $H$ is closed or dense in $E.$
    \begin{solution}
        Let $x\in E$ such that $f(x) = \alpha.$ Let $v \in \ker f,$ then $f(x + v) = f(x) + f(v)= \alpha,$ and so $v + x \in H,$ and so $\ker f + x \subset H.$ We claim that $\ker f$ is a subspace of $E,$ and thus $\ker f + x$ is an affine subspace. Let $z \in H.$ Then \[f(z) = \alpha = f(x) \implies f(z - x) = 0 \implies z-x \in \ker f \implies z\in \ker f  + x \implies H\subset \ker f + x.\]
        Thus, we have that $\ker f + x = H,$ and so $H \subset \overline{\ker f} + x = V  \subset E.$ Thus, we have from part (a) that either $V = H,$ in which case $H$ is closed since $V$ is closed, or $V = E.$ Suppose the latter, then $\overline{\ker f} + x = E,$ and so since $H = \ker f  + x,$ we have that $H$ is dense in $E$ by definition.
    \end{solution}
\end{enumerate}
\end{problem}


\newpage
\section*{Problem 4}
\begin{problem}
    
Let \( E \) be an n.v.s. with norm \( \| \cdot \| \). Let \( C \subset E \) be an open convex set such that \( 0 \in C \). Let \( p \) denote the gauge of \( C \) (see Lemma 1.2).

\begin{enumerate}
    \item Assuming \( C \) is symmetric (i.e., \( -C = C \)) and \( C \) is bounded, prove that \( p \) is a norm which is equivalent to \( \| \cdot \| \).
    \begin{solution}
    Recall that 
    \[p(x):= \inf\{\alpha \; ; \; \frac{1}{\alpha}x \in C\}\]
        It was shown in class that $p$ was a Minkowski functional. That is, 
        \[p(\lambda v) = |\lambda|p(v), \qquad p(v + w) \leq p(v) + p(w),\] thus, it remains to show that $p(v) = 0$ if and only if $v = 0.$ 
        
        Suppose $v = 0,$ then we know $v\in C,$ moreover, for any $\alpha >0,$ we have that $\frac{1}{\alpha}v = 0 \in C,$ in particular, as $\alpha \to 0,$ we have that $\frac{1}{\alpha}v = 0,$ and thus $p(v) = 0.$ On the other hand, suppose $p(v) = 0,$ then for any $n \in \bbN,$ there exists an $\alpha>0$ such that $0 < \alpha < \frac{1}{n}$ and $\frac{1}{\alpha}v \in C.$ However, we have that $nv < \frac{1}{\alpha}v \in C,$ and thus $nv \in C$ for all $C,$ which implies $C$ is unbounded. Thus, $v = 0$ and so $p$ is a norm.

        We will show that for some $K \in \bbR,$ for any $v\in E$
        \[\|v\| \leq Kp(v).\] Since $C$ is bounded, then for any $v\in E,$ we have that $\|v\|< K.$ That is, if $c \in C,$ then $\|c\| < K.$ Thus, let $v\in E,$ then by definition of $p(v):$ 
        $\frac{v}{p(v)} \in C$\footnote{This is not necessarily true, since $p(v)$ could be $0,$ but we make this statement informally and note that one could talk about this using arbitrarily close $\alpha.$ Either way, the next line is correct.}, which implies that $\|v\| \leq K\|p(v)\|$

        
    \end{solution}

\item Let \( E = C([0, 1]; \mathbb{R}) \) with its usual norm


\[
\|u\| = \max_{t \in [0,1]} |u(t)|.
\]



Let


\[
C = \left\{ u \in E; \int_0^1 |u(t)|^2 \, dt < 1 \right\}.
\]



Check that \( C \) is convex and symmetric and that \( 0 \in C \). Is \( C \) bounded in \( E \)? Compute the gauge \( p \) of \( C \) and show that \( p \) is a norm on \( E \). Is \( p \) equivalent to \( \| \cdot \| \)?
\begin{solution}
    Let $f,g \in C.$ Let $t \in [0,1],$ we need to check that 
    \[tf + (1-t)g \in C.\] Consider that since we are working with real valued functions, $|u(t)|^2 = (u(t))^2$
\begin{align*}
    \int_0^1 |tf(x) - (1-t)g(x)|^2 dx &= \int_0^1 (tf(x) - (1-t)g(x))^2 dx\\
    &= \int_0^1 t^2 f^2(x)dx - \int_0^1 2t(1-t)f(x)g(x)dx + \int_0^1 (1-t)^2 g^2(x)dx\\
    &= t^2\int_0^1 f^2(x)dx - 2t(1-t)\int_0^1 f(x)g(x)dx + (1-t)^2\int_0^1  g^2(x)dx\\
    &< t^2 - 2t(1-t)\int_0^1 f(x)g(x)dx + (1-t)^2
\end{align*}
We claim that $\int_0^1 f(x)g(x)dx < 1,$ which we prove using Holder's inequality
\[\int_0^1 f(x)g(x)\leq \left(\int_0^1 f(x)dx\right)^\frac{1}{2} \left(\int_0^1 g(x)dx\right)^\frac{1}{2} < 1.\] Thus, we have that since $t\in [0,1]:$
\[\int_0^1 |tf(x) - (1-t)g(x)|^2 < t^2 - 2t(1-t) + (1-t)^2 = (t - (1-t))^2 = (2t - 1)^2 < 1,\] and so $C$ is convex. 

Let $f\in C.$ symmetry comes straight out of the fact that $(-f)^2 = f^2$ and thus they integrate to less than one.

Consider the function $f_n(x) = \frac{\sqrt{n}}{1 + nx},$ then we have that 
\[\int_0^1 \frac{\sqrt{n}}{1+ nx} dx=\sqrt{n}\int_0^1 \frac{1}{1 + nx}dx = \sqrt{n}\int_1^{1 + n} \frac{1}{u}du = \frac{\log(1 + n)}{\sqrt{n}} < 1.\] To see the final inequality, consider that for $x>1,$ we have that $4x < 1 + 2x + x^2,$ which implies that $\frac{1}{1 + x} < \frac{1}{2\sqrt{x}},$ thus, we gave that $(\log(1 + x))' < (\sqrt{x})',$ for all $x>1,$ so it suffices to show that $ \log(2) < \sqrt{1},$ which you can just look up, i don't care. So then we have that $\frac{\log(n+1)}{\sqrt{n}}< 1$ for $n>1.$ Thus, $f_n \in C$ for all $n.$ However, we have that 
\[\|f_n\| = \frac{\sqrt{n}}{1} = \sqrt{n},\] which is certainly not bounded, and so $p$ is not equivalent to $\|\cdot\|,$ since $\|f\| \geq M$ for any $M\in \bbR.$
    \end{solution}

\newpage
\section*{Problem 5}
\begin{problem}
    Let \( E \) be a finite-dimensional normed space. Let \( C \subset E \) be a nonempty convex set such that \( 0 \notin C \). We claim that there always exists some hyperplane that separates \( C \) and \(\{0\}\).

\begin{quote}
    [Note that every hyperplane is closed (why?). The main point in this exercise is that no additional assumption on \( C \) is required.]
\end{quote}
\begin{solution}
    We see that every hyperplane is closed because if $H = [f = \alpha],$ where $f: E\to \bbR$ is a linear functional, then the fact that $E$ is finite dimensional implies that $f$ is finite. 
\end{solution}

\begin{enumerate}
    \item Let \((x_n)_{n \geq 1}\) be a countable subset of \( C \) that is dense in \( C \) (why does it exist?). For every \( n \) let
    

\[
    C_n = \text{conv}\{x_1, x_2, \ldots, x_n\} = \left\{ x = \sum_{i=1}^n t_i x_i; \ t_i \geq 0 \ \forall i \ \text{and} \ \sum_{i=1}^n t_i = 1 \right\}.
    \]


    Check that \( C_n \) is compact and that \( \bigcup_{n=1}^\infty C_n \) is dense in \( C \).
\begin{solution}
    Define $\varphi: \bbR^n \to C_n$ such that if $y\in \bbR^n,$ then \[\varphi(y) = \varphi\begin{pmatrix}
        y_1 \\\vdots \\y_n
    \end{pmatrix} = \sum_{i=1}^n y_i x_i.\] Define \[X = \{t = \begin{pmatrix}
        t_1 \\\vdots\\t_n
    \end{pmatrix}\; ; \; t_i \geq 0, \quad \sum_{i=1}^n t_i = 1\},\] then we have that $c^{(n)}\in C_n$ if and only if $\varphi^{-1}(c^{(n)})\in X.$ That is, $\varphi(X) = C_n.$ Equip $X$ with the $\|y\|_1 = \sum_{i=1}^n y_i$ norm. We see that $X\subset \bbR^n,$ and thus it suffices to show that $X$ is closed. Let $t_m = \begin{pmatrix}
        t_1^{(m)}\\t_2^{(m)}\\\vdots \\t_n^{(m)}
    \end{pmatrix} \to \tilde{t} = \begin{pmatrix}
        \tilde{t_1}\\ \tilde{t_2} \\ \vdots \\\tilde{t_n}
    \end{pmatrix}$ where $(t_m)\in X.$ Then we have (by the previous PSET) that $t_i^{(m)} \to \tilde{t_i}$ at the same rate. That is, there exists an $N\in \bbN$ such that if $m \geq N,$ then 
    $|\tilde{t_i} - {t_i}^{(m)}| < \frac{\epsilon}{2^n},$ and so
    \[|1 - \sum_{i=1}^n \tilde{t_i}| \leq |1 - \sum_{i=1}^n t_i^{(m)}| + |\sum_{i=1}^n t_i^{(m)} - \sum_{i=1}^n \tilde{t_i}| < \epsilon.\] Thus, $X$ is closed and bounded (bounded by $B_2(0)$), and so $X$ is compact. Since $\varphi$ is continuous, (linear function between finite dimensional spaces), we have that $\varphi(X) = C_n$ is compact. 
    
    
    Evidently, we have that $C_n \subset C_{n+1}.$ Let $x\in C$ and $\epsilon>0.$ We claim that there is some $c \in \bigcup C_n$ such that $c\in B_\epsilon(x).$ Since $(x_n)$ is dense in $C,$ then there is some $x_k \in (x_n)$ such that $x_k \in B_\epsilon(x).$ Consider now that if $t_i = 0$ except for $t_k = 1,$ then $\sum_{i=1}^k t_i = 1$ and  \[c = x_k = \sum_{i=1}^k t_i x_i \in C_k \in \bigcup_{n=1}^\infty C_n\] 
\end{solution}

    \item Prove that there is some \( f_n \in E^\star \) such that
    

\[
    \|f_n\| = 1 \ \text{and} \ \langle f_n, x \rangle \geq 0 \ \forall x \in C_n.
    \]

\begin{solution}
    We have that $C_n$ compact and $\{0\}$ is closed, and they are disjoint. By the Hahn-Banach separation theorem (second geometric form), there exists a closed hyperplane $H_n = [f_n = \alpha_n]$ that strictly separates $C_n$ and $\{0\}.$ That is, for all $x\in C_n,$ 
    \[0 = f_n(0) \leq \alpha_n - \epsilon < \alpha_n + \epsilon \leq f_n(x) = \langle f_n, x \rangle.\] Thus, we can take 
    \[\varphi_n(x) = \frac{f_n(x)}{\|f_n(x)\|},\] and so we have that $\|\varphi_n(x)\| = 1$ and $\langle \varphi_n, x \rangle \geq 0.$
\end{solution}

    \item Deduce that there is some \( f \in E^\star \) such that
    

\[
    \|f\| = 1 \ \text{and} \ \langle f, x \rangle \geq 0 \ \forall x \in C.
    \]


    Conclude.
    \begin{solution}
        Take \[\lim_{n\to \infty}\varphi_n(x) := \varphi(x)\] Take some $x\in C.$ Let $\delta>0$ from continuity.
        By the density of the $(x_n),$ there exists some $k$ such that $x_k \in B_\frac{\delta}{2}(x).$ Thus, we have that since $C_n \subset C_{n+1},$ then by the definition of $\varphi,$ we have that $\langle \varphi, x_k \rangle \geq 0$ and $\|\varphi\| = 1.$ Thus, by continuity, we have that $\|\varphi(x)\| \geq 0$ and $\|\varphi\| = 1$ for all $x\in C.$
    \end{solution}

    \item Let \( A, B \subset E \) be nonempty disjoint convex sets. Prove that there exists some hyperplane \( H \) that separates \( A \) and \( B \).
    \begin{solution}
        Take $C = A - B.$ Since $A$ and $B$ are disjoint, we have that $0\notin C.$ Evidently, $C$ is nonempty. We just have to show that $C$ is convex. Let $x,y \in C$ and let 
        \[f(t) = tx + (1-t)y = t(a_1 - b_1) + (1-t)(a_2 - b_2) = ta_1 + (1-t)a_2 -(tb_1  +(1-t)b_2) = a - b \in C,\] by convexity. By the previous parts, there exists some $f\in E^*$ such that $\|f\| = 1$ and
        \[ 0\leq \langle f, C\rangle = \langle f, A-B\rangle = \langle f, A\rangle - \langle f, B\rangle.\] Thus, for all $a\in A,$ $b\in B,$ we have that 
        \[\langle f, b\rangle \leq \langle f, a \rangle,\] and so there exists a hyperplane separating the two. Specifically, fix a constant such that 
        \[\langle f, b \rangle \leq \alpha \leq \langle f, a\rangle\] and take the hyperplane $H = [f = \alpha].$
    \end{solution}
\end{enumerate}

\end{problem}

\newpage
\section*{Problem 6}
\begin{problem}
Let $E = \ell^1$ (see Section 11.3) and consider the two sets
\[
X = \left\{ x = (x_n)_{n \geq 1} \in E; \quad x_{2n} = 0 \quad \forall n \geq 1 \right\}
\]
and
\[
Y = \left\{ y = (y_n)_{n \geq 1} \in E; \quad y_{2n} = \frac{1}{2^n} y_{2n-1} \quad \forall n \geq 1 \right\}.
\]

\begin{enumerate}
    \item Check that $X$ and $Y$ are closed linear spaces and that $\overline{X + Y} = E$.
\begin{solution}
    Let \[
    \begin{pmatrix}
        x_1^{(1)}\\
        x_2^{(1)}\\
        \vdots\\
        x_i^{(1)}\\
        \vdots
    \end{pmatrix}, 
    \begin{pmatrix}
        x_1^{(2)}\\
        x_2^{(2)}\\
        \vdots\\
        x_i^{(2)}\\
        \vdots
    \end{pmatrix},
    \dots,
    \begin{pmatrix}
        x_1^{(n)}\\
        x_2^{(n)}\\
        \vdots\\
        x_i^{(n)}\\
        \vdots
    \end{pmatrix},
    \dots =
    x^{(n)}_i \to (x_i) = x = \begin{pmatrix}
        x_1\\x_2\\\vdots\\ x_i \\\vdots
    \end{pmatrix}\] where $(x^{(n)}_i) \in X.$ Thus, we have that for all $\epsilon>0,$ for all $i,$ for $n$ large, we have
    \[\|x^{n}_{(i)} - x\|_{\ell^1} = \sum_{i=1}^\infty |x_i^{(n)} - x_i| < \epsilon.\] Thus, we have that 
    \[\sum_{i=2k}^\infty |x_i^{(n)} - x_i|  = \sum_{i=2k}^\infty |x_i| < \epsilon.\] Because this is true for all $\epsilon >0,$ then it must be that $x_i = 0$ for all $i = 2k$ for $k\in \bbN,$ and so $x\in X.$

    Let \[
    \begin{pmatrix}
        y_1^{(1)}\\
        y_2^{(1)}\\
        \vdots\\
        y_i^{(1)}\\
        \vdots
    \end{pmatrix}, 
    \begin{pmatrix}
        y_1^{(2)}\\
        y_2^{(2)}\\
        \vdots\\
        y_i^{(2)}\\
        \vdots
    \end{pmatrix},
    \dots,
    \begin{pmatrix}
        y_1^{(n)}\\
        y_2^{(n)}\\
        \vdots\\
        y_i^{(n)}\\
        \vdots
    \end{pmatrix},
    \dots =
    y^{(n)}_i \to (y_i) = y = \begin{pmatrix}
        y_1\\y_2\\\vdots\\ y_i \\\vdots
    \end{pmatrix}\] where $(y^{(n)}_i) \in Y.$ Thus, we have that for all $\epsilon>0,$ for all $i,$ for $n$ large, we have
    \[\|y^{n}_{(i)} - y\|_{\ell^1} = \sum_{i=1}^\infty |y_i^{(n)} - y_i| < \epsilon.\] and thus for all $i:$
    \[|y_i^{(n)} - y_i| \leq \|y_{i}^{(n)} - y\|_{\ell^1}<\epsilon,\] and so 
    \[y_i^{(n)} = y_i\] for large $n,$ and thus
    \[y^{(n)}_{2i} = \frac{1}{2^{2i}}y^{(n)}_{2i-1}\xrightarrow{n\to \infty} \frac{1}{2^{2i}}y_{2i-1},\] and so we are done, since this shows that $y\in Y.$ 

    Evidently, $\overline{X + Y}\subset E.$ Let $e \in E.$ Then we define 
    \[x_{2n} = 0, \quad y_{2n} = e_{2n}\]
    \[x_{2n-1} =e_{2n-1} - 2^{2n}e^{2n}, \quad y_{2n-1} = 2^{2n}e_{2n},\] then obviously, $x \in X$ and $y\in Y$ and this sequence converges to $e\in E.$ Thus, $E\subset \overline{X + Y}$
    
\end{solution}
    \item Let $c \in E$ be defined by
    \[
    c_n =
    \begin{cases}
        0, & \forall n \geq 1, \quad n \text{ odd}, \\
        \frac{1}{2^n}, & \forall n \geq 1, \quad n \text{ even}.
    \end{cases}
    \]
    Check that $c \notin X + Y$.
    \begin{solution}
        Suppose it is! Then we would have (by logic from above), that 
        \[x_{2n}= 0, \qquad x_{2n-1} = \frac{1}{2^{2n-1}} 
 - 1\]
        \[y_{2n} = \frac{1}{2^{2n}} \qquad y_{2n-1} = 2^{2n}\frac{1}{2^{2n}} = 1.\]
        Thus, $c = x+y,$ But then $y\notin \ell^1$
    \end{solution}
    \item Set $Z = X - c$ and check that $Y \cap Z = \emptyset$. Does there exist a closed hyperplane in $E$ that separates $Y$ and $Z$? Compare with Theorem 1.7 and Exercise 1.9.
    \begin{solution}
    Suppose not, then let $\chi \in Y \cap Z.$ Since $\chi \in Y,$ there exists some sequence $y\in Y$ such that $\chi = y.$ Similarly, there exists some $z\in Z$ such that $\chi = z.$ Since $Z = X - c,$ then there exists some $x\in X$ such that $\chi = x-c,$ and thus $y = x-c,$ and so $c = x + (-y),$ which is a contradiction to what we have seen above, since $-y \in Y.$ 

    Suppose we could separate $Y$ and $Z$ with some closed hyperplane $H = [f = \alpha].$ Then there exists some $\alpha$ such that for all $y \in Y,$ $z\in Z,$ we have that 
    \[f(y) < \alpha < f(z) = f(x-c) = f(x) - f(c) \implies f(Y) < \alpha < f(Z).\] By the lemma below, we have that this implies that since $Y$ and $X$ are linear subspaces (easy to show), then $f(Y) = f(X) = 0.$  Thus, we have that $Y, X \subset \ker f.$ Moreover, we also know that by the strict separation, we have that $f(c) <0.$ We get that $X + Y \subset \ker f,$ and since $f$ is continuous, then $\overline{ X + Y}\subset \ker f,$ and so $E \subset \ker f.$ Thus, $f \equiv 0$ on all $E,$ which is a contradiction to the fact that $f(c) < 0.$
    \end{solution}
    \begin{reflection}
        We claim that if $W$ is subspace and $f: W\to \bbR$ is a linear functional. Then either $f \equiv 0$ or $f(W) = \bbR.$ Suppose $f\not \equiv 0,$ then there exists some $w\in W$ such that $f(w) = a >0.$ Then for any $x\in \bbR,$ we have that  $\frac{x}{a}f(w) = \frac{x}{a} a,$ and thus since $f(\frac{x}{a}w)= x$ and $\frac{x}{a}q\in W,$ then we are done. 
    \end{reflection}
\end{enumerate}
\end{problem}



\newpage
\section*{Problem 7}
\begin{problem}
Let $E $ be a normed vector space and let $f\in E^*$ with $f\neq 0.$ Let $M = [f = 0].$
    \begin{enumerate}
    \item Determine $M^\perp$.
    \begin{solution}
        We recall that 
        \[M^\perp = \{\varphi \in E^* \; ; \; \langle \varphi , x\rangle = 0 \; \forall x\in M\}.\] We claim that
        \[M^\perp = \{\sum_{i=1}^n \lambda_i f(x)\;;\; x\in M\},\] where $f$ is the functional from the hyperplane.

        Let $g\in \{\sum_{i=1}^n \lambda_i f(x)\;;\; x\in M\},$ then $g(x) = \lambda_1f(x) + \cdots \lambda_n f(x) = 0.$

        Let $g\in M^\perp,$ then for all $x\in M,$ we have that $g(x) = 0.$ Suppose that $g\notin \text{span}\{f\},$ then we have that $f$ and $g$ are linearly independent, and thus 
        \[a_1 f  + a_2 g = 0\] is satisfied if and only if $a_1, a_2 = 0.$ This is obviously not true, since $f, g = 0$ and so the equation is satisfied for any scalars, and so the vectors are linearly dependent, and so $g\in \text{span}(f).$
    \end{solution}
    \item Prove that for every $x \in E$, 
    \[
    \text{dist}(x, M) = \inf_{y \in M} \|x - y\| = \frac{|\langle f, x \rangle|}{\|f\|}.
    \]
    [Find a direct method or use Example 3 in Section 1.4.]
    \begin{solution}
        We have that for $y\in M,$ $\langle f, y \rangle = 0,$ and so 
        \[|\langle f, x\rangle| = |\langle f, x-y \rangle| \leq \|f\|\|x-y\| \implies \frac{|\langle f,x\rangle|}{\|f\|} \leq \|x-y\|\]
        Thus, 
        \[\frac{|\langle f,x\rangle|}{\|f\|} \leq \text{dist}(x,M).\] 
        For the other direction, take $v \in E\setminus M.$\footnote{We know $v \notin M$ since $f(v) = 0$ if it were, and thus we can ignore it for the operator norm}  Then 
        \[\|f\| = \sup_{v}\frac{|f(v)|}{|v|}.\] Note that we have that 
        \[d(x,M)\leq \|x - (x - \frac{f(x)}{f(u)}u)\| = \|\frac{f(x)}{f(u)}(u)\|\leq \frac{\|f\|}{|f(u)|}\|u\|,\] and so using the definition of the operator norm above, we are done.
    \end{solution}
    \item Assume now that 
    \[
    E = \{ u \in C([0,1]; \mathbb{R}) ; \quad u(0) = 0 \}
    \]
    and that
    \[
    \langle f, u \rangle = \int_0^1 u(t) dt, \quad u \in E.
    \]
    Prove that
    \[
    \text{dist}(u, M) = \left| \int_0^1 u(t) dt \right| \quad \forall u \in E.
    \]
    Show that 
    \[
    \inf_{v \in M} \|u - v\|
    \]
    is never achieved for any $u \in E \setminus M$.
\end{enumerate}
\end{problem}
\begin{solution}
    Consider that 
    \[\|f\| = \sup_{\|u\| = 1} \langle |f(u)\rangle| = \sup_{\|u\| = 1}|\int_0^1 u(t)dt| = |\int_0^1 dt| = 1.\] Thus, by part (ii), 
    \[\text{dist}(u,M) = \inf_{u\in M}\|x-u\| = \frac{|\langle f, u\rangle|}{\|f\|} = |\langle f, u\rangle| = \left| \int_0^1 u(t)dt\right|.\] Suppose $u\in E\setminus M,$ then 
    \[|\int_0^1u(t)dt)| = 0,\] then $\int_0^1 u(t)dt = 0,$ and so $\langle f, u \rangle = 0,$ and so $u \in M.$
\end{solution}

\newpage
\section*{Problem 8}
\begin{problem}
    Let $X$ be a normed vector space. Assume that for some $x,y \in X,$ we have that 
    \[\|x + y\| = \|x\| + \|y\|.\] Show that for every $\alpha, \beta >0$ we have that 
    \[\|\alpha x + \beta y\| = \alpha\| x\| +\beta \| y\|\]
    \end{problem}
\begin{solution}
    Without loss of generality, suppose $\alpha \geq \beta.$ By the triangle inequality, we have that 
    \[\|\alpha x + \beta y\| \leq \alpha\| x\| +\beta \| y\|,\] so it suffices to show that 
    \[\|\alpha x + \beta y\| \geq \alpha\| x\| +\beta \| y\|.\] Consider that by the reverse triangle inequality,
    \begin{align*}
        \|\alpha x + \beta y\| &= \|\alpha (x + y) + (\beta - \alpha)y\|\\
        &\geq \|\alpha (x + y)\| - \|(\beta - \alpha)y\|\\
        &=\alpha\|x\|+ \alpha\|y\| + (\beta - \alpha)\|y\|\\
        &= \alpha\|x\| + \beta \|y\|
    \end{align*}.
\end{solution}

\newpage
\section*{Problem 9}
\begin{problem}
    Let $1\leq p \leq q < \infty.$ Then $\|x\|_q \leq \|x\|_p.$
\end{problem}
\begin{solution}
Consider the case when $\|x\|_p = 1.$ Then we have that necessarily, $|x_i| \leq 1$ for all $i,$ and thus $|x_i|^q \leq |x_i|$ for all $i.$ Thus, 
\[\sum_{i=1}^\infty |x_i|^q \leq \sum_{i=1}^\infty |x_i| =1 \implies \|x\|_q \leq 1.\] For the general case, take 
\[y_k = \frac{x_k}{\|x\|_p},\] then $\|y\|_p = 1$ and thus 
\[\|y\|_q \leq \|y\|_p = 1,\] but we have that 
\[\|y\|_q = \frac{\|x\|_q}{\|x\|_p} \leq 1,\] and so we are done.
\end{solution}

\end{enumerate}

\newpage
\section*{Problem 10}
\begin{problem}
    Show that 
    \[\lim_{p\to \infty}\|x\|_p = \|x\|_\infty\]
\end{problem}
\begin{solution}
    Consider that for any $n,$ we have that by taking limits
    \[\sum_{k=1}^\infty |a_k|^p \geq |a_n|^p \implies \|a\|_\infty \leq \|a\|_p.\] On the other hand, we have that since $a\in \ell^p,$ then
    \[\sum_{k=1}^\infty |a_k|^p < \infty,\] Thus, there must exist some $C$ such that \[\sum_{k=1}^\infty |\frac{a_k}{A}|^p  \leq C \implies \left(\sum_{k=1}^\infty |\frac{a_k}{A}|^p\right)^\frac{1}{p}  = C^{\frac{1}{p}}.\] By homogeneity, we have that 
    \[\|a\|_p \leq C^\frac{1}{p}\|a\|_\infty.\] Since for large $p,$ we have that $|C^\frac{1}{p} - 1|< \epsilon,$ then 
    \[\|a\|_p \leq C^\frac{1}{p}\|a\|_\infty \to \|a\|_\infty\] Thus, as $p\to \infty,$ we have that $\|a\|_p = \|a\|_\infty.$
\end{solution}

\newpage

\section*{Problem 11}
\begin{problem}
    Show that a normed space $X$ is Banach if and only if $\sum x_n$ converges in $X$ whenever $\|x_n\|< \frac{1}{2^n}$ for every $n.$
\end{problem}
\begin{solution}
    Suppose $X$ is Banach. Suppose $\|x_n\|< \frac{1}{2^n},$ then 
    \[\sum_{n=1}^\infty \|x_n\| < \sum_{n=1}^\infty \frac{1}{2^n} = 1,\] and thus since $X$ is Banach and the series is absolutely summable, then it is summable, and thus 
    \[\sum_{n=1}^\infty x_n < \infty.\]

    Let $(x_n) \in X$ be Cauchy. Thus, for all $n, m\geq N_k,$ we have that  
    \[\|x_n - x_m\| < \frac{1}{2^k}.\] Let $n_K = N_1 + N_2 + \dots + N_k.$ Obviously, $n_{k+1} \geq n_k \geq N_K$ and thus our subsequence $(x_{n_k})$ is Cauchy by
    \[\|x_{n_k} - x_{n_{k+1}}\|< \frac{1}{2^k}.\] Thus, we have by assumption that our series is summable or that
    \[\sum_{k=1}^K x_{n_k} - x_{n_{k+1}} = x_{n_K} - x_{n_1} \xrightarrow{n\to \infty} x - x_{n_1} \] as $K \to \infty,$ and thus our Cauchy subsequence converges to a limit. Thus, it suffices to show that if $(a_n)$ is Cauchy and it has a convergent subsequence, then $(x_n)$ must converge. Observe that since $(x_n)$ is Cauchy, then there exists some $N_1\in \bbN$ such that $n>N_1,$ then $d(a_n,a_{n_k})<\frac{\epsilon}{2}$ (since $n_k$ is implicitly greater than $n$). Since $(x_{n_k})$ converges to $x,$ then there exists some $N_2,$ such that if $n>N_2,$ we have that $d(x_{n_k},x)<\frac{\epsilon}{2}.$ Thus if $N = \max\{N_1, N_2\}$ and $n\geq N,$
    \[d(x_n, x)\leq d(x_n, a_{x_k}) + d(x_{n_k}, x)\leq \epsilon.\]
\end{solution}


\end{problem}


\newpage
\section*{Problem 12}
Let $X$ be a normed vector space obtained from 
\[c_0 = \{(x_n) \; ; \; \lim_{n\to \infty}x_n = 0\}\] equipped with the norm $\|x\|_0 = \sum_{n=1}^\infty \frac{1}{2^k}|x_k|.$ Show that $X$ is not Banach.

\begin{solution}
    Consider the following sequence in $c_0:$
    \[(x_{n}^{(i)}) = 
    \begin{pmatrix}
        1 & 1 & 1 & \cdots \\
        0 & 1 & 1 & \cdots \\
        0 & 0 & 1 & \cdots 
    \end{pmatrix},\] where the columns are the sequences in $(x_{n}^{(i)}).$ Let $\epsilon>0,$ we can find a $K$ such that $\frac{1}{2^K} < \epsilon$ and thus if $n,m \geq K,$ then
    \[\|x_{n}^{(i)} - x_{m}^{(i)}\|_0 \leq \sum_{n+1}^m\frac{1}{2^k} < \frac{1}{2^n} < \frac{1}{2^K},\] and thus $(x_{n}^{(i)})$ is Cauchy. Evidently, we have that 
    \[(x_{n}^{(i)}) \to (1,1,1,\dots),\] which is not in $c_0.$
\end{solution}
\newpage

\section*{Problem 13}
\begin{problem}
    Let $X$ be a normed vector space and suppose $C\subset X$ is nonempty and convex. Then $\text{int}(C)$ and $\overline{C}$ are both convex.
\end{problem}
\begin{solution}
    Suppose not. Let $x, y \in \text{int}(C)$ and let
    \[z = tx + (1-t)y\] such that $z\notin \text{int}(C).$ Thus, for all $\epsilon>0,$ $B_\epsilon(z)\not\subset C.$ Specifically, let $z_0$ be such a point. Since $x, y \in \text{int} (C),$ then there exist $\epsilon>0$ such that 
    \begin{align}
    B_\epsilon(x) \subset \text{int}(C) \qquad C_\epsilon(y) \subset \text{int}(C).    
    \end{align}
    Consider now that
    \[f(t) = t(x - (z - z_0)) + (1-t)(y - (z - z_0)) = z_0,\] and so it suffices to show that $z - (z-z_0)$ and $y - (z - z_0)$ are in $C,$ since this would imply that there is a convex combination of them that are not in $C.$
    Since $z_0 \in B_\epsilon(z),$ we have that 
    \[\|z - z_0\|< \frac{\epsilon}{2}.\] Thus, by (2),
    \[\|x - (z - z_0)\| \leq \|x\| + \|z - z_0\|< \|x\| + \frac{\epsilon}{2} \implies (x- (z - z_0))\in C.\] Similarly, $y - (z - z_0)\in C.$
\end{solution}
\begin{solution}
    Let $x,y \in \overline{C}$ and suppose there exists 
    \[z= tx - (1-t)y\notin \overline{C}\] Thus, there exists some $\epsilon>0$ such that $B_\epsilon(z)$ contains no points of $C.$ Specifically, let $z_0 \in B_\epsilon(z)$ such that $z_0 \notin C.$ Again, we do the exact same thing as above to conclude that we can express $z_0$ as convex combinations of $x_0 \in B_\epsilon(x)$ and $y_0 \in B_\epsilon(y),$ where $x_0 \in C$ and $y_0 \in C,$ and thus $C$ is not convex. 
\end{solution}

\newpage
\section*{Problem 14}
\begin{problem}
    Let $X$ be a normed vector space. If $A\subset X$ is open, then $\text{conv}(A)$ is open. Is the convex hull of closed sets in $\bbR^n$ closed?
\end{problem}
\begin{solution}
    Recall that 
    \[\text{conv}(A) = \{\sum_{i=1}^nt_i a_i, \; a_i \in A \;\forall i, \; t_i \geq 0\; \forall i, \;\sum_{i=1}^n t_i = 1\}.\]  Let $a_0 \in \text{conv(A)},$ then 
    \[a_0 = \sum_{i=1}^nt_i a_i.\] Since each $a_i \in A,$ then there exists some $r_i>0$ such that 
    \[B_{r_i}(a_i)\subset U.\] Let $r = \min_{i\in [n]} \{r_i\}.$ We can then conclude that
    \[\sum_{i=1}^n t_iB_{r}(a_i) = \sum_{i=1}^n(t_ia_i + t_i B_r(0)) = B_r(0) + \sum_{i=1}^nt_ia_i = B_r(a_0),\] and thus any $a\in B_r(a_0)$ can be written as a convex combination of $a_i'\in A.$

    The convex hull of closed sets in $\bbR^n$ is not necessarily closed. Let $f: \bbR\setminus\{0\} \to \bbR$ such that $f(x) = \frac{1}{x^2}.$ Then consider the graph of $f$ as
    \[G(f) = \{(x, f(x))\}.\] Obviously, $G(f)$ is closed. It is not hard to see that the convex hull is simply the upper plane, which is not closed. 
\end{solution}

\newpage
\section*{Problem 15}
\begin{problem}
    Suppose $X,Y$ are nonempty normed vector spaces and $T\in \mathcal{L}(X, Y),$
    then 
    \[\|T\| = \sup_{\|x\| = 1}\{\|T(x)\}\} = \sup_{\|x\| \leq 1}\{\|T(x)\}\}\]
\end{problem}
\begin{solution}
    Consider that \[\|Tx\| \leq \|T\|\|x\| \leq \|T\|,\] and thus \[\sup_{\|x\| \leq 1}\{\|T(x)\|\} \leq \sup_{\|x\| = 1}\{\|T(x)\|\}\]
    The other direction is obvious, since we can just take $\|x|\ = 1$ and thus $\|x\| \leq 1$ and so
    \[\sup_{\|x\| = 1}\{\|T(x)\|\} \leq \sup_{\|x\| \leq 1}\{\|T(x)\|\}\]
\end{solution}
\newpage
\section*{Problem 16}
\begin{problem}
    Let \( X, Y \) be Banach spaces and \( T \in \mathcal{L}(X, Y) \). If there exists \( \delta > 0 \) such that \( \|T(x)\| \geq \delta \|x\| \) for all \( x \in X \), then \( T(X) \) is closed in \( Y \).
\end{problem}
\begin{solution}
    Suppose that $(w_n) \in T(X)$ with $(w_n)\to w.$ We want to show that there exists some $v\in X$ such that $T(v) = w.$ Because $(w_n) \in T(X)$ for all $n,$ then there exists $(v_n)$ such that $T(v_i) = w_i.$ Since $(w_n) \to w,$ then $(w_n)$ is Cauchy. There exists some $N$ such that if $n, m \geq N,$ then $\|w_n - w_m\|  = \|T(v_n) - T(v_m)\| < \epsilon \delta.$ 
    \[\delta\|v_n - v_m\|\leq \|T(v_n - T_m)\| = \|T(v_n) - T(v_m)\| = \|v_n - v_m\|< \epsilon \delta,\] and thus 
    \[\|v_n - v_m\| < \epsilon.\] Since $(v_n) \in X$ is Cauchy and $X$ is Banach, we have that there exists some $v\in X$ such that $v_n \to v.$ Since $T\in \mathcal{L}(V,W),$ then $T$ is continuous, and thus 
    \[T(v_n) \to T(v) = w\in Y.\]
\end{solution}

\newpage
\section*{Problem 17}
\begin{problem}
    Show that the norm \( \|x\| = \sum_{i=1}^{\infty} 2^{-i} |x_i| \) in \( \ell^2 \) is not equivalent to the \( \|\cdot\|_2 \) norm.
\end{problem}
\begin{solution}
    Let $(x_n) = \frac{1}{\sqrt{n}}.$ Since $(x_n)$ is monotonic and bounded, then by the claim in the reflection, we have that $\|x\| = \sum_{n=1}^\infty \frac{1}{2^n}\frac{1}{\sqrt{n}} < \infty.$ However, in the other norm, we have that 
    \[\|x\|_2 = \left(\sum_{n=1}^\infty |\frac{1}{\sqrt{n}}|^2\right)^\frac{1}{2} =\left(\sum_{n=1}^\infty \frac1n\right)^\frac{1}{2} = \infty\] since the harmonic series diverges. Thus, the norms are not equivalent.
\end{solution}
\begin{reflection}
\begin{problem}
If $\sum a_n$ converges and $(b_n)$ is monotonic and bounded, prove that $\sum a_nb_n$ converges.
\end{problem}
    We claim that if $(c_n)\to 0$ is monotone decreasing, then $\sum a_k c_n$ converges. By Luis' hint during office hours, we have that:
    \begin{align*}
    \sum_{k=1}^{n-1}A_k(c_k - c_{c+1}) &= A_1(c_1 -c_2) + A_2(c_2 - c_3)+ \dots\\
    &= \sum_{k=1}^{n-1}a_kc_k -  A_{n-1}c_{n},
    \end{align*} which can be proved by induction.
    We have that since $A_n$ converges, then it is bounded by some $|A_k|\leq A.$ Thus, we note that since $c_{k} - c_{k+1}\geq 0$ for all $k,$ then
    \begin{align*}
        \left|\sum_{k=m+1}^n a_k c_k\right| &= \left|\sum_{k=1}^{n-1}A_k(c_k - c_{k+1}) + A_{n-1}c_n - (\sum_{k=1}^mA_k(c_{k} - c_{k+1}) + A_mc_{m+1})\right|\\
        &= \left| \sum_{k=m+1}^{n-1}A_k(c_{k} - c_{k+1}) + A_{n-1}c_n + A_mc_{m+1}\right|\\
        &\leq \left| \sum_{k=m+1}^{n-1}A(c_{k} - c_{k+1}) + Ac_n + Ac_{m+1}\right|\\
        &\leq A\left(\sum_{k=m+1}^{n-1}(c_k - c_{k+1}) + c_n + c_{m+1}\right)\\
        &= A \left(2c_{m+1}\right)\\
        &< 2A\frac{\epsilon}{2A}.\\
    \end{align*}
    The last inequality comes from the fact that $c_n \to 0.$ Suppose that $b_n$ is monotone increasing and bounded. Thus, it converges to some $b.$ Let $c_n = b- b_n.$ Then we have that $c_n \to 0$ and $c_n$ is monotonically decreasing. Thus, by the work above, we have that 
    \[\sum_{k=1}^n a_k(b-b_k) = b\sum_{k=1}^na_k - \sum_{k=1}^n a_kb_k\] converges. Since $b\sum_{k=1}^n a_k$ converges, then $\sum_{k=1}^n a_k b_k$ converges. Suppose $b_n$ is monotone decreasing and bounded, then $b_n -b\to 0$ and  $c_n = b_n - b$ is monotone decreasing. Thus, we can use the same strategy as above to show that $\sum a_kb_k$ converges.
\end{reflection}

\newpage
$<$3 love you (thanks gaya)
\section*{Problem 17}
\begin{problem}
Let \( E \) and \( F \) be two Banach spaces and let \( (T_n) \) be a sequence in \( \mathcal{L}(E, F) \).  
Assume that for every \( x \in E \), \( T_nx \) converges as \( n \to \infty \) to a limit denoted by \( Tx \).  
Show that if \( x_n \to x \) in \( E \), then \( T_nx_n \to Tx \) in \( F \).
\end{problem}
\begin{solution}
Since $T_nx \to Tx$ for every $x,$ we have that there exists some $N_1 $ such that if $n\geq N_1,$ then 
\[\|T_nx - Tx\|< \frac{\epsilon}{2}.\]
By the uniform boundedness theorem, we have that 
\[\|T_n(x_n - x)\|\leq c\|x_n - x\|,\] so there exists some $N_2$ such that if $n\geq N_2,$ we have that because $x_n \to x,$ then 
\[\|x_n - x\| < \frac{\epsilon}{2c}.\] Take the max between the $N_1$ and $N_2,$ then for $n$ larger,
    \begin{align*}
        \|T_nx_n - Tx\| &\leq \|T_nx_n - T_n x\| + \|T_nx -Tx\|\\
        &\leq \|T_n(x_n - x)\| + \|T_nx - Tx\|\\
        &< c\|x_n - x\| + \frac{\epsilon}{2}\\
        &< \epsilon
    \end{align*}
\end{solution}




\newpage
\section*{Problem 18}
\begin{problem}


\[
\text{ Let } E \text{ and } F \text{ be two Banach spaces and let } a : E \times F \rightarrow \mathbb{R} \text{ be a bilinear form satisfying:}
\]




\[
\begin{aligned}
(i) & \text{ for each fixed } x \in E, \text{ the map } y \mapsto a(x, y) \text{ is continuous;} \\
(ii) & \text{ for each fixed } y \in F, \text{ the map } x \mapsto a(x, y) \text{ is continuous.}
\end{aligned}
\]




\[
\text{Prove that there exists a constant } C \geq 0 \text{ such that}
\]




\[
|a(x, y)| \leq C \|x\| \|y\| \quad \forall x \in E, \quad \forall y \in F.
\]


   
\end{problem}

\begin{solution}
    We introduce new notation (because the author believes $a$ is stupid) and let $T(x,y)$ be the bilinear map denoted by $a$ in the problem. It suffices to show that 
    \[\sup_{\substack{\|x\|\leq 1\\\|y\|\leq 1}}|T(x,y)| = \|T\| < \infty,\] since this shows that $T$ is bounded in both arguments. Fix $x\in E.$ We have that $\langle T, (x,y) \rangle  = T_x(y)$ is continuous, that is:
    \[\sup_{\|y\|\leq 1} |T_x| < \infty.\] We will show that if $\overline{B_E}$ is the closed unit ball in $E,$ then $y \mapsto T(\overline{B_E}, y)$ is bounded. Obviously, $T(\overline{B_E}, y)\subset F^*.$ By corollary 2.5, it suffices to show that for every $y\in F,$ the set 
    \[T_y(B_E)\] is bounded, which comes from the fact that for fixed $y,$ $T$ is continuous, and thus 
    \[|T_y(B_E)|< \infty,\] and so by the corollary $|T(B_E, y)|< \infty.$ In particular, we have that for all $\|y\|\leq 1,$ $|T(B_E, B_F)|< \infty,$ which is equivalent to saying that 
    \[\|T\| < \infty.\]
\end{solution}

\newpage
\section*{Problem 19}

\begin{problem}

{ Let } $\alpha = (\alpha_n)$ { be a given sequence of real numbers and let } $1 \leq p \leq \infty.$ { Assume that } $\sum |\alpha_n||x_n| < \infty$ { for every element } $x = (x_n)$ { in } $\ell^p$ { (the space } $\ell^p$ { is defined in Section 11.3).}

{Prove that } $\alpha \in \ell^{p'}.$
\end{problem}
\begin{solution}
    Define 
    \[\langle \alpha_n x \rangle = \sum_{i=1}^n |\alpha_i| |x_i|.\] We see that by assumption, each $\alpha_n$ is a bounded linear map and that 
    \[\langle \alpha_n, x\rangle \to \langle \alpha, x\rangle = \sum_{i=1}^\infty|a_i||x_i| < \infty.\] and so $\alpha \in \mathcal{L}(\ell^p, \bbR).$ It remains to show that $\alpha \in \ell^{p'},$ that is, there exists some $C$ such that
    \[\|\alpha\|_{\ell^{p'}}  = \left(\sum_{n=1}^\infty |\alpha_n|^{p'}\right)^\frac{1}{p'} \leq C.\] By the uniform boundedness principle, there exists some $c$ such that since
    \[\|\alpha_n x\|_\bbR = |\alpha_n x|  = \sum_{i=1}^n |\alpha_i||x_i|\leq c \|x\|_{\ell^p} = c\left(\sum_{i=1}^\infty |x_i|^p\right)^\frac{1}{p},\] for all $n,$ then 
    \[\sum_{i=1}^\infty |\alpha_i||x_i| \leq c \|x\|_{\ell^p},\] and so $\alpha$ is a bounded linear map.   

    Choose $x$ such that $x_i = \frac{|\alpha_i|^{p'}}{|a_i|}.$ Thus, we find that 
    \[\sum_{i=1}^\infty |\alpha_i||x_i| = \sum_{i=1}^\infty |\alpha_i||\frac{|\alpha_i|^{p'}}{|a_i|}| = \sum_{i=1}^\infty |\alpha_i|^{p'} < \infty,\] and thus $\alpha \in \ell^{p'}$ 

\end{solution}

\newpage
\section*{Problem 20}
\begin{problem}
{ Let } $E$ \text{ and } $F$ { be two Banach spaces and let } $T \in \mathcal{L}(E, F)$ { be surjective.}

1. Let \( M \) be any subset of \( E \). Prove that \( T(M) \) is closed in \( F \) if and only if \( M + N(T) \) is closed in \( E \).
\begin{solution}
    Suppose $T(M)$ is closed in $F.$ Let $(x_n)\in M+ N(T),$ such that $x_n \to x.$
    
    then $x_n = m_n + n_n$ \footnote{I would be sorry about notation but then I remember how I am not sorry about notation} Thus, we have that $T(x_n) = T(m_n + T(n_n)) = T(m_n).$ Since $T(M)$ is closed we have that the graph of $T$ is closed and so $(x_n, T(x_n)) \to (x, y).$ Thus $T(x_n) \to y$ and $y \in T(M)$ and thus $Tx = y.$ In other words, we have that $M + N(T) = T^{-1}(T(M))$ is closed since by the closed graph theorem, $T$ is continuous.

    Suppose $M+ N(T)$ is closed. Then we have that $E \setminus (M + N(T))$ is open. Thus, we have by the open mapping theorem that \[T(E \setminus (M + N(T))) = T(E)\setminus T(M + N(T)) = F \setminus (T(M) + T(N(T)))= F \setminus T(M)\] is open, and thus $T(M)$ is closed.
\end{solution}

2. Deduce that if \( M \) is a closed vector space in \( E \) and \( \dim N(T) < \infty \), then \( T(M) \) is closed.
\begin{solution}
    It suffices to show that $M + N(T)$ is closed.  Take a sequence $(x_n)\in M + N(T)$ such that $x_n = m_n + \eta_n$ where $(m_n)\in M$ and $(\eta_n)\in N(T)$ and $(x_n) \to x.$ We want to show $m_n \to m \in M$, and that $\eta_n \to \eta \in N(T).$ 
    
    
    Suppose $M \cap N(T) = \{0\}.$ Define $\eta_n' = \frac{\eta_n}{\|\eta_n\|}.$ Then $\eta' = (\eta_n')$ is in the closed unit ball in $N(T)$ since $\|\eta'\| = 1,$ and so it has a convergent subsequence $ \eta_{n_k}'\to \eta' \in N(T)$ since $N(T)$ is finite dimensional\footnote{The closed unit ball is compact if and only if the space is finite dimensional is a big theorem from last year}. 
    Suppose $\|\eta_{n_k}\|\to \infty,$ then 
    \[m_{n_k}' = x_{n_k}' - \eta_{n_k}' \to -\eta'.\] Since $M$ is closed, we have that $-\eta' \in M,$ and so $\eta' \in M,$ and thus we must have that $\eta' = 0,$ which is a contradiction to the fact that $\|\eta_{n_k}\|\to \infty.$ Thus, by scaling, $\eta_{n_k} \to \eta \in N(T),$ and so 
    \[m_{n_k} = x_{n_k} - \eta_{n_0} \to x-\eta \in M,\] and thus $x \in M + N(T).$ 

    For the general case, let $\alpha$ be the basis of $M\cap N(T).$ Let $\beta$ be a basis of $N(T)$ such that $\alpha\setminus\{0\}\subset  \beta.$ We see that 
    \[M + N(T) = M + N(T)\setminus\{\text{span}(\alpha)\}\] and that $M\cap N(T)\setminus\{\alpha\} = \{0\},$ and so $M + N(T)\setminus\{\text{span}(\alpha)\}$ is closed, and so $M + N(T)$ is closed.
\end{solution}

\end{problem}



\newpage
\section*{Problem 21}
\begin{problem}
    \textit{Let } $E$ \textit{ be a Banach space, } $F = \ell^1$, \textit{ and let } $T \in \mathcal{L}(E, F)$ \textit{ be surjective. Prove that there exists } $S \in \mathcal{L}(F, E)$ \textit{ such that } $T \circ S = I_F$, \textit{ i.e., } $S$ \textit{ has a right inverse of } $T$. 

\textbf{[Hint:} Do not apply Theorem 2.12; try to define $S$ explicitly using the canonical basis of $\ell^1$.]

\end{problem}
\begin{solution}
    $T$ is an open map by the open mapping theorem, and thus 
    \[c B_0^{\ell^1}(1)\subset T(B_0^{E}(1)).\] Thus, if $(e_n) = (0,0,\dots, 1, 0, 0, \dots),$ where $1$ is in the $n$th spot is the canonical $\ell^1$ basis, then since $e_n \in B_0^{\ell^1}(1),$ then there exists some $v_n \in B_0^{E}(\frac{1}{c})$ such that $e_n = T(v_n).$ Define $S: \ell^1 \to E$ such that if $y = (y_1, y_2, \dots) \in \ell^1,$ then
    \[S(y) = \sum_{n=1}^\infty v_ny_n.\] Then we have that 
    \[T(S(y)) = T\sum_{n=1}^\infty v_n y_n = \sum_{n=1}^\infty T(v_n)y_n = \sum_{n=1}^\infty e_ny_n = y.\] We might need to justify the fact that linear transforms are linear under an infinite sum, but if we really wanted to, we would just define $S_n$ as a limit series up to $S$ and take limits then, which would be fine since $S_n$ is continuous. 
\end{solution}

\newpage
\section*{Problem 22}
\begin{problem}
 \text{ Let } E \text{ be a Banach space. Let } G \text{ and } L \text{ be two closed subspaces of } E. \text{ Assume that there exists a constant } C \text{ such that}


\[
\text{dist}(x, G \cap L) \leq C \, \text{dist}(x, L), \quad \forall x \in G.
\]


\text{Prove that } G + L \text{ is closed.}

\end{problem}
\begin{solution}
    (From answer key) Let $T: G\to E\setminus L$ such that $Tx = \pi x,$ where $\pi: E\to E/ L$ is the canonical surjection. Thus, we have that 
    \[\text{dist}(x, \ker T) = \text{dist}(x, G\cap L) \leq C\text{dist}(x,L) = C \|Tx\|.\] Thus, we have that $\pi(G)$ is closed, and so 
    \[\pi^{-1}(\pi(G)) = G\cap L\] is closed.
\end{solution}


\newpage
\section*{Problem 23}
\begin{problem}
    Let \( E = C([0,1]) \) with its usual norm. Consider the operator \( A : D(A) \subset E \to E \) defined by
\[
D(A) = C^1([0,1]) \quad \text{and} \quad (Au)(t) = u'(t).
\]
\begin{enumerate}
    \item Check that \( \overline{D(A)} \neq E \).
    \begin{solution}
        Obviously, we have that $D(A)\subset E,$ and thus 
        \[\overline{D(A)}\subset \overline{E}.\] Under the $\sup$ norm, we know that $C^1([]0,1])$ is closed. Thus, we have that 
        \[\overline{D(A)}\subset E.\] Let $f\in E.$ By the Stone-Weirstrass theorem, for all $n,$ there exists a polynomial $P_n \in C^1([0,1])$ such that 
        \[\|P_n - f\| \leq \frac{1}{n},\] and so $P_n \to f,$ and thus $f \in LP(D(A)),$ and so $f\in \overline{D(A)}.$ Thus, we have that 
        \[E \subset \overline{D(A)}.\] 
    \end{solution}
    \item Is \( A \) closed?
    \begin{solution}
        Ee know that a linear operator $T$ is closed if and only if its graph is closed. Thus, consider if $f_n \to f$ with $(f_n, f_n') \to (f, g).$ We have that
    \[f_n = \int f_n' \to f = \int g \implies f' = g,\] and so $g = Af.$ To see that $f\in D(A),$ we see that $g$ must be continuous since it is the uniform limit of continuous functions, and thus $f'$ is continuous (and evidently $f$ is differentiable).
    \end{solution}
    \item Consider the operator \( B : D(B) \subset E \to E \) defined by
    \[
    D(B) = C^2([0,1]) \quad \text{and} \quad (Bu)(t) = u'(t).
    \]
    Is \( B \) closed?
    \begin{solution}
        No, it is not. Let $(f_n)\in C^2([0,1])$ be a sequence of functions converging to $f(x) = x|x|.$ Then $f_n' \to g,$ where $g = |x| = f'.$ However, $f$ is obviously not in $C^2,$ and so we are done.
    \end{solution}
\end{enumerate}
\end{problem}

\newpage
\section*{Problem 24}
\begin{problem}
    Let \( E \) and \( F \) be two Banach spaces. Let \( T \in \mathcal{L}(E, F) \) and let \( A : D(A) \subset E \to F \) be an unbounded operator that is densely defined and closed. Consider the operator \( B : D(B) \subset E \to F \) defined by  
\[
D(B) = D(A), \quad B = A + T.
\]

\begin{enumerate}
    \item Prove that \( B \) is closed.
\begin{solution}
Let $(x_n) \in D(B)$ such that $x_n \to x$ and $B(x_n) \to y.$ First, we know that 
\[B(x_n) = (A + T)(x_n) = A(x_n)+ T(x_n).\] Since $(x_n) \in D(B)$ and $D(A)= D(B),$ then $(x_n)\in D(A),$ and so  $A$ being closed implies that 1) $x\in D(A) = D(B)$ and that $A(x_n) \to A(x).$ Since $T$ is continuous, we have that $T(x_n) \to T(x).$ Thus, 
\[B(x_n) \to A(x) + T(x) = Bx,\] and so we have proved that $y = Bx$ and that $x\in D(B).$
\end{solution}

    \item Prove that \( D(B^\star) = D(A^\star) \) and \( B^\star = A^\star + T^\star \).\begin{solution}
    
Let $c_A, c_B>0.$
By definition, we have that 
\[D(A^*) = \{f\in F^* \; |\langle f, Ax \rangle| \leq c_A\|x\| \quad \forall x\in D(A)\}\]
\[D(B^*) = \{f\in F^* \; |\langle f, Bx \rangle| \leq c_B\|x\| \quad \forall x\in D(B)\}\]
Let $f\in D(A^*),$ then for all $x\in D(A) = D(B),$ we have that 
\[|\langle f, Bx \rangle|  = |\langle f, (A + T)x \rangle| = |\langle f, Ax + Tx \rangle| \leq |\langle f, Ax \rangle| + |\langle f, Tx \rangle |, \qquad \forall \; x\ni D(B) = D(A)\] The first term is bounded by $c_A\|x\|.$ As for the second, we know that $T$ is a bounded linear operator and that $f \in F^*,$, and thus there is some $c_T, c_f >0$ such that for all $x\in E,$ 
\[\|\langle f, Tx\rangle\| \leq \|f\|\|Tx\| \leq \|f\|c_T\|x\| \leq c_Fc_T\|x\|.\] Thus, we have that $f\in D(B^*).$ 

Let $f\in D(B^*).$ Then there exists some $c_B$ such that for all  
\[|\langle f ,  Bx\rangle| \leq c_B \|x\|.\] Thus, we have that 
\[|\langle f ,  Bx\rangle| = |\langle f ,  Ax\rangle| + |\langle f, Tx\rangle| \leq c_B \|x\|.\] In particular, we have that 
\[\|\langle f, Ax\rangle \| \leq c_B\|x\|, \qquad \forall \; x \in D(A) = D(B).\] and so $f\in D(A^*).$

Thus, we have that $D(A^*) = D(B^*).$ To show that $B^* = A^* + T^*,$ then since for all $x\in D(B),$ $f\in D(B^*)$
\[\langle f, Bx\rangle = \langle B^*f, x\rangle,\] then by the previous parts, it will suffice to show that 
\[\langle f, Bx\rangle = \langle (A^* + T^*)f, x\rangle.\] We use bilinearity and the definition of the adjoints:
\[\langle (A^* + T^*)f, x\rangle = \langle A^*f +T^*f, x\rangle = \langle A^*f, x\rangle + \langle T^*f, x\rangle = \langle f, Ax\rangle + \langle f, Tx \rangle = \langle f, (A + T)x\rangle = \langle f, Bx\rangle.\] If you have any issues with the forall $x\in D(A)$ or $x\in E$ definition of the adjoint, consider that $D(A) = D(B)\subset E$ and $D(A^*)= D(B^*)\subset F^*.$   
\end{solution}
\end{enumerate}
\end{problem}

\newpage
\section*{Problem 25}
\begin{problem}
    The purpose of this exercise is to construct an unbounded operator \( A : D(A) \subset E \to E \) that is densely defined, closed, and such that \( \overline{D(A^\star)} \neq E^\star \).

Let \( E = \ell^1 \), so that \( E^\star = \ell^\infty \). Consider the operator \( A : D(A) \subset E \to E \) defined by


\[
D(A) = \left\{ u = (u_n) \in \ell^1; \ (nu_n) \in \ell^1 \right\} \text{ and } Au = (nu_n).
\]



1. Check that \( A \) is densely defined and closed.
\begin{solution}
    Let $x = (x_n)\in \ell^1$ and $\epsilon>0.$ Since $x\in \ell^1,$ we have that 
    \[\sum_{n=1}^\infty x_n < \infty \implies \sum_{n = N_\epsilon}^\infty x_n < \epsilon\] for some $N_\epsilon.$ Take 
    \[u = (u_n) = \begin{cases}
        x_n, \quad n \leq N\\
        0, \quad n > N
    \end{cases}.\] Evidently, we have that $u \in \ell^1$ and $(nu_n)\in \ell^1.$ Moreover, we have that 
    \[\sum_{n=1}^\infty |u_n - x_n| = \sum_{N+1}^\infty |x_n|< \epsilon,\] and so we have that for any $B_\epsilon(x),$ there is some $u \in D(A)$ such that $u \in B_\epsilon(x),$ implying that $\overline{D(A)} = \ell^1.$

    Let $(x_n) \to x$ with $(x_n)\in D(A)$ and suppose $A(x_n) \to y.$ We want to show that $y = A(x),$ where $x\in D(A).$ For the latter, fix $\epsilon = 1.$ Then we have that there exists some $N$ such that $\|x_n - x\|< 1,$ and so since $x_N \in \ell^1:$
    \[\|x\| \leq \|x - x_N\| + \|x_N\|| < 1 + \|x_N\| < \infty.\] Thus, $x\in \ell^1.$ To show that $(nx_n)\in \ell^1,$ we let $(kx_n^{(k)})  = x_n'$ and $(kx^{(k)}) = x'.$ We have that $x_n' \in \ell'$ by assumption, and 
    \[\|x'\|_1 \leq \|x' - x_n'\| + \|x_n'|\|,\] so it remains to bound the first term. For all $\epsilon>0,$ there exists some $k>0$ so that $\frac{1}{k2^k}< \epsilon.$ Thus, for $n$ large enough, we have that that for all $k,$ \[|x_n^{(k)} - x^{(k)}| \leq \|x_n - x\|< \frac{1}{k2^k}\] and thus 
    \[\|x_n' - x'\| = \sum_{k=1}^\infty |kx_n^{(k)} - kx^{k}| = 1,\] and so we are done, since this also shows that $(nx_n) = y.$
\end{solution}
2. Determine \( D(A^\star) \), \( A^\star \), and \( \overline{D(A^\star)} \).
\end{problem}
\begin{solution}

    We have that 
    \begin{align*}
        D(A^*) = \{x \in \ell^\infty \; ; \; \langle x, Au\rangle \leq c\|u\|, \quad \forall \; u\in \ell^1\} &= \{x \in \ell^\infty \; ; \; \langle x, (nu_n)\rangle \leq c\|u\|, \quad \forall \; u\in \ell^1\}\\ &= \{x\in \ell^\infty \; ; \; (nx_n)\in \ell^\infty\}
    \end{align*}
    Using the fact that if $x\in \ell^\infty$ and $u\in \ell^1,$ then 
    \begin{align}
\langle x, Au\rangle = \langle x, (nu_n) \rangle  = \sum_{n=1}^\infty nx_nu_n = \langle A^* x, u\rangle.
    \end{align}
 
    And clearly, we define $A^*: D(A^*) \subset \ell^\infty \to \ell^\infty$ such that if $x\in \ell^\infty,$ then 
    \[A^*(x) =  (nx_n),\] which satisfies  (3). 
Since for $x\in D(A^*)$ we need that \[\limsup_{n\to \infty} (nx_n) < \infty,\] then we must have that $x_n \to 0.$ Thus, we claim that $c_0 = \overline{D(A^*)}.$ One inclusion is clear as previously stated. Let $y\in c_0.$ Then we have that $y_n \to 0$ as $n\to \infty.$ We claim that there exists some sequence $(x_n)\in D(A^*)$ such that $x_n \to y.$ To see this, just let 
\[x_1 = (0, y_2 , y_3, \dots),\] which is evidently in $\ell^\infty$ and so is $(nx_n).$ Then we have that 
\[\|x_n - y\| = \sup_{n}|y_n - x_n| = y_n \to 0.\]
\end{solution}

\newpage
\section*{Problem 27}
\begin{problem}
    Let \( E = \ell^1 \), so that \( E^\star = \ell^\infty \). Consider the operator \( T \in \mathcal{L}(E, E) \) defined by



\[ Tu = \left( \frac{1}{n} u_n \right)_{n \geq 1} \text{ for every } u = (u_n)_{n \geq 1} \text{ in } \ell^1. \]



Determine \( N(T) \), \( N(T)^\perp \), \( T^\star \), \( R(T^\star) \), and \( \overline{R(T^\star)} \). Compare with Corollary 2.18.
\end{problem}
\begin{solution}
Obviously, 
\[N(T) = \{0\},\] since the only map to zero is the zero vector. We have that 
\[N(T)^\perp = \{x \in \ell^\infty \; ; \;  \langle y, x\rangle = 0, \quad \forall \; y \in N(T)\}.\] Since $\{0\} = N(T),$ we have that for any $x\in \ell^\infty,$ 
\[\langle 0, x\rangle = 0 \implies N(T)^\perp = \ell^\infty.\]
Let $y\in D(T^*),$ (evidently, we have that 
\[D(T^*) = \{y \in \ell^\infty \;  ; \; (\frac{1}{n}y_n)\in \ell^1\}\]), then 
\begin{align*}
    \langle y, Tx \rangle &=  \langle y, (\frac{1}{n}x_n)\rangle\\
    &= \sum_{n=1}^\infty |y_n||\frac{1}{n}x_n|\\
    &= \sum_{n=1}^\infty |\frac{1}{n}y_n||x_n|\\
    &= \langle (\frac{1}{n}y_n), x\rangle\\
    &= \langle T^*y, x\rangle,
\end{align*}
an so $T^*: D(T^*)\subset \ell^\infty \to \ell^\infty$ is defined by 
\[T^*(y)= (\frac{1}{n}y_n).\] This is also the range, $R(T^*),$ all the sequences that can be expressed as $\frac{1}{n}y_n.$ 

It is clear that $\overline{R(T^*)} = c_0$ from similar work as in the previous problem. 
\end{solution}
\end{document}


