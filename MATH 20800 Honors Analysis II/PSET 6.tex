\documentclass[11pt]{article}

% NOTE: Add in the relevant information to the commands below; or, if you'll be using the same information frequently, add these commands at the top of paolo-pset.tex file. 
\newcommand{\name}{Agustín Esteva}
\newcommand{\email}{aesteva@uchicago.edu}
\newcommand{\classnum}{207}
\newcommand{\subject}{Honors Analysis in $\bbR^n$}
\newcommand{\instructors}{Panagiotis E. Souganidis}
\newcommand{\assignment}{Problem Set 6}
\newcommand{\semester}{Winter 2024}
\newcommand{\duedate}{\today}
\newcommand{\bA}{\mathbf{A}}
\newcommand{\bB}{\mathbf{B}}
\newcommand{\bC}{\mathbf{C}}
\newcommand{\bD}{\mathbf{D}}
\newcommand{\bE}{\mathbf{E}}
\newcommand{\bF}{\mathbf{F}}
\newcommand{\bG}{\mathbf{G}}
\newcommand{\bH}{\mathbf{H}}
\newcommand{\bI}{\mathbf{I}}
\newcommand{\bJ}{\mathbf{J}}
\newcommand{\bK}{\mathbf{K}}
\newcommand{\bL}{\mathbf{L}}
\newcommand{\bM}{\mathbf{M}}
\newcommand{\bN}{\mathbf{N}}
\newcommand{\bO}{\mathbf{O}}
\newcommand{\bP}{\mathbf{P}}
\newcommand{\bQ}{\mathbf{Q}}
\newcommand{\bR}{\mathbf{R}}
\newcommand{\bS}{\mathbf{S}}
\newcommand{\bT}{\mathbf{T}}
\newcommand{\bU}{\mathbf{U}}
\newcommand{\bV}{\mathbf{V}}
\newcommand{\bW}{\mathbf{W}}
\newcommand{\bX}{\mathbf{X}}
\newcommand{\bY}{\mathbf{Y}}
\newcommand{\bZ}{\mathbf{Z}}

%% blackboard bold math capitals
\newcommand{\bbA}{\mathbb{A}}
\newcommand{\bbB}{\mathbb{B}}
\newcommand{\bbC}{\mathbb{C}}
\newcommand{\bbD}{\mathbb{D}}
\newcommand{\bbE}{\mathbb{E}}
\newcommand{\bbF}{\mathbb{F}}
\newcommand{\bbG}{\mathbb{G}}
\newcommand{\bbH}{\mathbb{H}}
\newcommand{\bbI}{\mathbb{I}}
\newcommand{\bbJ}{\mathbb{J}}
\newcommand{\bbK}{\mathbb{K}}
\newcommand{\bbL}{\mathbb{L}}
\newcommand{\bbM}{\mathbb{M}}
\newcommand{\bbN}{\mathbb{N}}
\newcommand{\bbO}{\mathbb{O}}
\newcommand{\bbP}{\mathbb{P}}
\newcommand{\bbQ}{\mathbb{Q}}
\newcommand{\bbR}{\mathbb{R}}
\newcommand{\bbS}{\mathbb{S}}
\newcommand{\bbT}{\mathbb{T}}
\newcommand{\bbU}{\mathbb{U}}
\newcommand{\bbV}{\mathbb{V}}
\newcommand{\bbW}{\mathbb{W}}
\newcommand{\bbX}{\mathbb{X}}
\newcommand{\bbY}{\mathbb{Y}}
\newcommand{\bbZ}{\mathbb{Z}}

%% script math capitals
\newcommand{\sA}{\mathscr{A}}
\newcommand{\sB}{\mathscr{B}}
\newcommand{\sC}{\mathscr{C}}
\newcommand{\sD}{\mathscr{D}}
\newcommand{\sE}{\mathscr{E}}
\newcommand{\sF}{\mathscr{F}}
\newcommand{\sG}{\mathscr{G}}
\newcommand{\sH}{\mathscr{H}}
\newcommand{\sI}{\mathscr{I}}
\newcommand{\sJ}{\mathscr{J}}
\newcommand{\sK}{\mathscr{K}}
\newcommand{\sL}{\mathscr{L}}
\newcommand{\sM}{\mathscr{M}}
\newcommand{\sN}{\mathscr{N}}
\newcommand{\sO}{\mathscr{O}}
\newcommand{\sP}{\mathscr{P}}
\newcommand{\sQ}{\mathscr{Q}}
\newcommand{\sR}{\mathscr{R}}
\newcommand{\sS}{\mathscr{S}}
\newcommand{\sT}{\mathscr{T}}
\newcommand{\sU}{\mathscr{U}}
\newcommand{\sV}{\mathscr{V}}
\newcommand{\sW}{\mathscr{W}}
\newcommand{\sX}{\mathscr{X}}
\newcommand{\sY}{\mathscr{Y}}
\newcommand{\sZ}{\mathscr{Z}}


\renewcommand{\emptyset}{\O}

\newcommand{\abs}[1]{\lvert #1 \rvert}
\newcommand{\norm}[1]{\lVert #1 \rVert}
\newcommand{\sm}{\setminus}


\newcommand{\sarr}{\rightarrow}
\newcommand{\arr}{\longrightarrow}

% NOTE: Defining collaborators is optional; to not list collaborators, comment out the line below.
%\newcommand{\collaborators}{Alyssa P. Hacker (\texttt{aphacker}), Ben Bitdiddle (\texttt{bitdiddle})}

% Copyright 2021 Paolo Adajar (padajar.com, paoloadajar@mit.edu)
% 
% Permission is hereby granted, free of charge, to any person obtaining a copy of this software and associated documentation files (the "Software"), to deal in the Software without restriction, including without limitation the rights to use, copy, modify, merge, publish, distribute, sublicense, and/or sell copies of the Software, and to permit persons to whom the Software is furnished to do so, subject to the following conditions:
%
% The above copyright notice and this permission notice shall be included in all copies or substantial portions of the Software.
% 
% THE SOFTWARE IS PROVIDED "AS IS", WITHOUT WARRANTY OF ANY KIND, EXPRESS OR IMPLIED, INCLUDING BUT NOT LIMITED TO THE WARRANTIES OF MERCHANTABILITY, FITNESS FOR A PARTICULAR PURPOSE AND NONINFRINGEMENT. IN NO EVENT SHALL THE AUTHORS OR COPYRIGHT HOLDERS BE LIABLE FOR ANY CLAIM, DAMAGES OR OTHER LIABILITY, WHETHER IN AN ACTION OF CONTRACT, TORT OR OTHERWISE, ARISING FROM, OUT OF OR IN CONNECTION WITH THE SOFTWARE OR THE USE OR OTHER DEALINGS IN THE SOFTWARE.

\usepackage{fullpage}
\usepackage{enumitem}
\usepackage{amsfonts, amssymb, amsmath,amsthm}
\usepackage{mathtools}
\usepackage[pdftex, pdfauthor={\name}, pdftitle={\classnum~\assignment}]{hyperref}
\usepackage[dvipsnames]{xcolor}
\usepackage{bbm}
\usepackage{graphicx}
\usepackage{mathrsfs}
\usepackage{pdfpages}
\usepackage{tabularx}
\usepackage{pdflscape}
\usepackage{makecell}
\usepackage{booktabs}
\usepackage{natbib}
\usepackage{caption}
\usepackage{subcaption}
\usepackage{physics}
\usepackage[many]{tcolorbox}
\usepackage{version}
\usepackage{ifthen}
\usepackage{cancel}
\usepackage{listings}
\usepackage{courier}

\usepackage{tikz}
\usepackage{istgame}

\hypersetup{
	colorlinks=true,
	linkcolor=blue,
	filecolor=magenta,
	urlcolor=blue,
}

\setlength{\parindent}{0mm}
\setlength{\parskip}{2mm}

\setlist[enumerate]{label=({\alph*})}
\setlist[enumerate, 2]{label=({\roman*})}

\allowdisplaybreaks[1]

\newcommand{\psetheader}{
	\ifthenelse{\isundefined{\collaborators}}{
		\begin{center}
			{\setlength{\parindent}{0cm} \setlength{\parskip}{0mm}
				
				{\textbf{\classnum~\semester:~\assignment} \hfill \name}
				
				\subject \hfill \href{mailto:\email}{\tt \email}
				
				Instructor(s):~\instructors \hfill Due Date:~\duedate	
				
				\hrulefill}
		\end{center}
	}{
		\begin{center}
			{\setlength{\parindent}{0cm} \setlength{\parskip}{0mm}
				
				{\textbf{\classnum~\semester:~\assignment} \hfill \name\footnote{Collaborator(s): \collaborators}}
				
				\subject \hfill \href{mailto:\email}{\tt \email}
				
				Instructor(s):~\instructors \hfill Due Date:~\duedate	
				
				\hrulefill}
		\end{center}
	}
}

\renewcommand{\thepage}{\classnum~\assignment \hfill \arabic{page}}

\makeatletter
\def\points{\@ifnextchar[{\@with}{\@without}}
\def\@with[#1]#2{{\ifthenelse{\equal{#2}{1}}{{[1 point, #1]}}{{[#2 points, #1]}}}}
\def\@without#1{\ifthenelse{\equal{#1}{1}}{{[1 point]}}{{[#1 points]}}}
\makeatother

\newtheoremstyle{theorem-custom}%
{}{}%
{}{}%
{\itshape}{.}%
{ }%
{\thmname{#1}\thmnumber{ #2}\thmnote{ (#3)}}

\theoremstyle{theorem-custom}

\newtheorem{theorem}{Theorem}
\newtheorem{lemma}[theorem]{Lemma}
\newtheorem{example}[theorem]{Example}

\newenvironment{problem}[1]{\color{black} #1}{}

\newenvironment{solution}{%
	\leavevmode\begin{tcolorbox}[breakable, colback=green!5!white,colframe=green!75!black, enhanced jigsaw] \proof[\scshape Solution:] \setlength{\parskip}{2mm}%
	}{\renewcommand{\qedsymbol}{$\blacksquare$} \endproof \end{tcolorbox}}

\newenvironment{reflection}{\begin{tcolorbox}[breakable, colback=black!8!white,colframe=black!60!white, enhanced jigsaw, parbox = false]\textsc{Reflections:}}{\end{tcolorbox}}

\newcommand{\qedh}{\renewcommand{\qedsymbol}{$\blacksquare$}\qedhere}

\definecolor{mygreen}{rgb}{0,0.6,0}
\definecolor{mygray}{rgb}{0.5,0.5,0.5}
\definecolor{mymauve}{rgb}{0.58,0,0.82}

% from https://github.com/satejsoman/stata-lstlisting
% language definition
\lstdefinelanguage{Stata}{
	% System commands
	morekeywords=[1]{regress, reg, summarize, sum, display, di, generate, gen, bysort, use, import, delimited, predict, quietly, probit, margins, test},
	% Reserved words
	morekeywords=[2]{aggregate, array, boolean, break, byte, case, catch, class, colvector, complex, const, continue, default, delegate, delete, do, double, else, eltypedef, end, enum, explicit, export, external, float, for, friend, function, global, goto, if, inline, int, local, long, mata, matrix, namespace, new, numeric, NULL, operator, orgtypedef, pointer, polymorphic, pragma, private, protected, public, quad, real, return, rowvector, scalar, short, signed, static, strL, string, struct, super, switch, template, this, throw, transmorphic, try, typedef, typename, union, unsigned, using, vector, version, virtual, void, volatile, while,},
	% Keywords
	morekeywords=[3]{forvalues, foreach, set},
	% Date and time functions
	morekeywords=[4]{bofd, Cdhms, Chms, Clock, clock, Cmdyhms, Cofc, cofC, Cofd, cofd, daily, date, day, dhms, dofb, dofC, dofc, dofh, dofm, dofq, dofw, dofy, dow, doy, halfyear, halfyearly, hh, hhC, hms, hofd, hours, mdy, mdyhms, minutes, mm, mmC, mofd, month, monthly, msofhours, msofminutes, msofseconds, qofd, quarter, quarterly, seconds, ss, ssC, tC, tc, td, th, tm, tq, tw, week, weekly, wofd, year, yearly, yh, ym, yofd, yq, yw,},
	% Mathematical functions
	morekeywords=[5]{abs, ceil, cloglog, comb, digamma, exp, expm1, floor, int, invcloglog, invlogit, ln, ln1m, ln, ln1p, ln, lnfactorial, lngamma, log, log10, log1m, log1p, logit, max, min, mod, reldif, round, sign, sqrt, sum, trigamma, trunc,},
	% Matrix functions
	morekeywords=[6]{cholesky, coleqnumb, colnfreeparms, colnumb, colsof, corr, det, diag, diag0cnt, el, get, hadamard, I, inv, invsym, issymmetric, J, matmissing, matuniform, mreldif, nullmat, roweqnumb, rownfreeparms, rownumb, rowsof, sweep, trace, vec, vecdiag, },
	% Programming functions
	morekeywords=[7]{autocode, byteorder, c, _caller, chop, abs, clip, cond, e, fileexists, fileread, filereaderror, filewrite, float, fmtwidth, has_eprop, inlist, inrange, irecode, matrix, maxbyte, maxdouble, maxfloat, maxint, maxlong, mi, minbyte, mindouble, minfloat, minint, minlong, missing, r, recode, replay, return, s, scalar, smallestdouble,},
	% Random-number functions
	morekeywords=[8]{rbeta, rbinomial, rcauchy, rchi2, rexponential, rgamma, rhypergeometric, rigaussian, rlaplace, rlogistic, rnbinomial, rnormal, rpoisson, rt, runiform, runiformint, rweibull, rweibullph,},
	% Selecting time-span functions
	morekeywords=[9]{tin, twithin,},
	% Statistical functions
	morekeywords=[10]{betaden, binomial, binomialp, binomialtail, binormal, cauchy, cauchyden, cauchytail, chi2, chi2den, chi2tail, dgammapda, dgammapdada, dgammapdadx, dgammapdx, dgammapdxdx, dunnettprob, exponential, exponentialden, exponentialtail, F, Fden, Ftail, gammaden, gammap, gammaptail, hypergeometric, hypergeometricp, ibeta, ibetatail, igaussian, igaussianden, igaussiantail, invbinomial, invbinomialtail, invcauchy, invcauchytail, invchi2, invchi2tail, invdunnettprob, invexponential, invexponentialtail, invF, invFtail, invgammap, invgammaptail, invibeta, invibetatail, invigaussian, invigaussiantail, invlaplace, invlaplacetail, invlogistic, invlogistictail, invnbinomial, invnbinomialtail, invnchi2, invnF, invnFtail, invnibeta, invnormal, invnt, invnttail, invpoisson, invpoissontail, invt, invttail, invtukeyprob, invweibull, invweibullph, invweibullphtail, invweibulltail, laplace, laplaceden, laplacetail, lncauchyden, lnigammaden, lnigaussianden, lniwishartden, lnlaplaceden, lnmvnormalden, lnnormal, lnnormalden, lnwishartden, logistic, logisticden, logistictail, nbetaden, nbinomial, nbinomialp, nbinomialtail, nchi2, nchi2den, nchi2tail, nF, nFden, nFtail, nibeta, normal, normalden, npnchi2, npnF, npnt, nt, ntden, nttail, poisson, poissonp, poissontail, t, tden, ttail, tukeyprob, weibull, weibullden, weibullph, weibullphden, weibullphtail, weibulltail,},
	% String functions 
	morekeywords=[11]{abbrev, char, collatorlocale, collatorversion, indexnot, plural, plural, real, regexm, regexr, regexs, soundex, soundex_nara, strcat, strdup, string, strofreal, string, strofreal, stritrim, strlen, strlower, strltrim, strmatch, strofreal, strofreal, strpos, strproper, strreverse, strrpos, strrtrim, strtoname, strtrim, strupper, subinstr, subinword, substr, tobytes, uchar, udstrlen, udsubstr, uisdigit, uisletter, ustrcompare, ustrcompareex, ustrfix, ustrfrom, ustrinvalidcnt, ustrleft, ustrlen, ustrlower, ustrltrim, ustrnormalize, ustrpos, ustrregexm, ustrregexra, ustrregexrf, ustrregexs, ustrreverse, ustrright, ustrrpos, ustrrtrim, ustrsortkey, ustrsortkeyex, ustrtitle, ustrto, ustrtohex, ustrtoname, ustrtrim, ustrunescape, ustrupper, ustrword, ustrwordcount, usubinstr, usubstr, word, wordbreaklocale, worcount,},
	% Trig functions
	morekeywords=[12]{acos, acosh, asin, asinh, atan, atanh, cos, cosh, sin, sinh, tan, tanh,},
	morecomment=[l]{//},
	% morecomment=[l]{*},  // `*` maybe used as multiply operator. So use `//` as line comment.
	morecomment=[s]{/*}{*/},
	% The following is used by macros, like `lags'.
	morestring=[b]{`}{'},
	% morestring=[d]{'},
	morestring=[b]",
	morestring=[d]",
	% morestring=[d]{\\`},
	% morestring=[b]{'},
	sensitive=true,
}

\lstset{ 
	backgroundcolor=\color{white},   % choose the background color; you must add \usepackage{color} or \usepackage{xcolor}; should come as last argument
	basicstyle=\footnotesize\ttfamily,        % the size of the fonts that are used for the code
	breakatwhitespace=false,         % sets if automatic breaks should only happen at whitespace
	breaklines=true,                 % sets automatic line breaking
	captionpos=b,                    % sets the caption-position to bottom
	commentstyle=\color{mygreen},    % comment style
	deletekeywords={...},            % if you want to delete keywords from the given language
	escapeinside={\%*}{*)},          % if you want to add LaTeX within your code
	extendedchars=true,              % lets you use non-ASCII characters; for 8-bits encodings only, does not work with UTF-8
	firstnumber=0,                % start line enumeration with line 1000
	frame=single,	                   % adds a frame around the code
	keepspaces=true,                 % keeps spaces in text, useful for keeping indentation of code (possibly needs columns=flexible)
	keywordstyle=\color{blue},       % keyword style
	language=Octave,                 % the language of the code
	morekeywords={*,...},            % if you want to add more keywords to the set
	numbers=left,                    % where to put the line-numbers; possible values are (none, left, right)
	numbersep=5pt,                   % how far the line-numbers are from the code
	numberstyle=\tiny\color{mygray}, % the style that is used for the line-numbers
	rulecolor=\color{black},         % if not set, the frame-color may be changed on line-breaks within not-black text (e.g. comments (green here))
	showspaces=false,                % show spaces everywhere adding particular underscores; it overrides 'showstringspaces'
	showstringspaces=false,          % underline spaces within strings only
	showtabs=false,                  % show tabs within strings adding particular underscores
	stepnumber=2,                    % the step between two line-numbers. If it's 1, each line will be numbered
	stringstyle=\color{mymauve},     % string literal style
	tabsize=2,	                   % sets default tabsize to 2 spaces
%	title=\lstname,                   % show the filename of files included with \lstinputlisting; also try caption instead of title
	xleftmargin=0.25cm
}

% NOTE: To compile a version of this pset without problems, solutions, or reflections, uncomment the relevant line below.

%\excludeversion{problem}
%\excludeversion{solution}
%\excludeversion{reflection}

\begin{document}	
	
	% Use the \psetheader command at the beginning of a pset. 
	\psetheader

\section*{Problem 1}
\begin{problem}
    Define 
    \[c_0 := \{(a_n)\in \ell^\infty \; : \; a_n \to 0\}\]
    Show that $c_0^* = \ell^1.$
\end{problem}
\begin{solution}
    Let $x\in c_0.$ We represent $x$ via a Schauder basis:
    \[x = \sum_{n=1}^\infty x_ne_n,\] where $e_n = \delta_{ni}$ (kronecker-delta) and $x_n \in \bbR.$ Evidently, this representation is unique for each $x.$
    Thus, we have that for any $f\in c_0^*,$ 
    \[\langle f, x\rangle = \langle f, \sum_{n=1}^\infty x_n e_n \rangle = \sum_{n=1}^\infty x_n f(e_n).\] We define $f(e_n)= a_n.$ Thus, 
    \[f(x) = \sum_{n=1}^\infty x_na_n.\] Uniqueness comes from the uniqueness of the Schauder basis. We note that 
    \begin{align}
        |f(x)| \leq \|f\|_{c_0^*}\|x\|_\infty
    \end{align}
   
    We will next show that $(a_n)\in \ell^1.$  Define 
    \[x_i = 
    \begin{cases}
        \text{sign}(a_n), \qquad n\leq N\\
        0, \qquad \qquad \quad n>N
    \end{cases}.\] Evidently, $x\in c_0$ with $\|x\|_\infty = 1$
    We use (1) and see that
    \begin{align}
      \sum_{n=1}^N |a_n| = |f(x)| = \left|\sum_{n=1}^\infty x_na_n\right| \leq \|f\|_{c_0^*}\|x\|_\infty  = \|f\|_{c_0^*}   
    \end{align}
        Thus, because $f\in c_0^*$ then $\|f\|< \infty$ and because $x\in c_0.$ Thus, because (2) holds for any $N,$ we see that $a\in \ell^1.$ To see that $f \to a$ is an isometry, we need to show that $\|f\|_{c_0^*} = \|a\|_1.$
    We use H\"older's inequality for one side:
    \[|f(x)| = \sum_{n=1}^\infty x_n a_n \leq \|x\|_\infty\|a\|_1 \implies \sup_{\|x\|_\infty =1}|f(x)| = \|f\|_{c_0^*} \leq \|a\|_1.\] We use (2) to directly show that $\|a\|_1 \leq \|f\|_{c_0^*}.$

    Finally, we see that $f$ is bounded. $f$ linear is obvious from the definition.
\end{solution}

\newpage
\section*{Problem 2}
\begin{problem}
    Show that $(\ell^1)^* = \ell^\infty.$ 
\end{problem}
\begin{solution}
    Let $x\in \ell^1.$ We represent $x$ via a Schauder basis:
    \[x = \sum_{n=1}^\infty x_n e_n\] as in the problem above. Thus, for any $f\in (\ell^1)^*:$
    \[\langle f, x\rangle = \langle f, \sum_{n=1}^\infty x_n e_n \rangle = \sum_{n=1}^\infty x_n f(e_n).\] We define $f(e_n)= a_n.$ Thus, 
    \[f(x) = \sum_{n=1}^\infty x_na_n.\] Uniqueness comes from the uniqueness of the Schauder basis. We note that 
    \begin{align*}
        |f(x)| \leq \|f\|_{c_0^*}\|x\|_1
    \end{align*}
    To show that $a = (a_n)\in \ell^\infty,$ simply notice that 
    \begin{align}
|a_n| = |f(e_n)| \leq \|f\|\|e_n\|_1 = \|f\| \implies \|a\|_\infty = \sup_{n}|a_n| \leq \|f\|        
    \end{align}

Thus, $a \in \ell^\infty.$ 

To see that $f\mapsto a$ is an isometry, we use H\"older's inequality:
\[|f(x)| = \left|\sum_{n=1}^\infty x_n a_n\right|\leq \|x\|_1\|a\|_\infty \implies \|f\| = \sup_{\|x\|_1 = 1}|f(x)| \leq \|a\|_\infty.\] We use (3) to see the other side of the inequality.
\end{solution}

\newpage
\section*{Problem 3}
\begin{problem}
    Suppose $p\in (1, \infty)$ and $q\in \bbR$ with $\frac{1}{p} + \frac{1}{q} = 1.$ Prove that $(\ell^p)^* = \ell^q.$
\end{problem}
\begin{solution}
    Let $x\in \ell^p.$ We represent $x$ via a Schauder basis:
    \[x = \sum_{n=1}^\infty x_n e_n\] as in the problem above. Let $f\in (\ell^p)^*.$ Then
    \[\langle f, x\rangle = \langle f, \sum_{n=1}^\infty x_n e_n \rangle = \sum_{n=1}^\infty x_n f(e_n).\] We define $f(e_n)= a_n.$ Thus, 
    \[f(x) = \sum_{n=1}^\infty x_na_n,\] and so 
    \[|f(x)|\leq \|f\|\|x\|_p.\] To show that $a\in \ell^q,$ we define 
    \[x_n = 
    \begin{cases}
    \frac{|a_n|^q}{a_n}, \qquad a_n \neq 0, n\leq N\\
    0, \qquad \quad \: \: a_n = 0, n> N
    \end{cases}\]
    Thus, we see that 
    \[|f(x)| = \left|\sum_{n=1}^\infty x_na_n\right| = \sum_{n=1}^N |a_n|^q\leq \|f\|\|x\|_p= \|f\|\left(\sum_{n=1}^\infty|a_n|^{(q-1)p}\right)^\frac{1}{p} = \|f\| \left(\sum_{n=1}^\infty |a_n|^q\right)^\frac{1}{p},\] and so taking $N\to \infty$ and dividing by the very last term, we see that $a\in \ell^q$ since $\|a\|_q \leq \|f\|.$ 

    We use H\"older again:
    \[|f(x)| \leq \sum_{n=1}^\infty |x_n| |a_n| \leq \|x\|_p \|a\|_q \implies \|f\| = \sup_{\|x\|_p = 1} \leq \|a\|_q.\]
\end{solution}

\newpage
\section*{Problem 4}
\begin{problem}
    Let $C$ be a convex symmetric subset of a Banach space $X.$ Assume that the linear functional $f$ on $X$ is continuous at $0$ when restricted to $C.$ Show that $f|_C$ is uniformly continuous.
\end{problem}
\begin{solution}
    Since $f|_C$ is continuous at $0,$ then for all $\epsilon>0,$ there exists some $\delta>0$ such that if $v\in C$ with $\|v\|< \delta,$ then $\|f(v)\|< \epsilon.$ Take $x,y \in C$ with $\|x-y\|< \delta.$ Since $y \in C$ and $C$ is symmetric, then $-y \in C.$ Take $t = \frac{1}{2},$ then by convexity,
    \[g(t) := tx + (1-t)(-y) \implies g(\frac{1}{2}) = \frac{1}{2}x - \frac{1}{2}y = \frac{1}{2}(x-y)\in C.\] Thus, 
    \[\|\frac{1}{2}(x-y)\| < \delta \implies \|f(\frac{1}{2}(x-y))\| = \frac{1}{2}\|f(x) - f(y)\|< \epsilon,\] and so $f$ is uniformly continuous on $C.$
\end{solution}

\newpage
\section*{Problem 5}
\begin{problem}
    Let $f$ be a linear functional on a Banach space $X.$ Suppose $f\not \equiv 0,$ then the following are equivalent:
    \begin{enumerate}
        \item $f$ is continuous.
        \item $f$ is continuous at $0.$
        \item $f^{-1}(\{0\})$ is closed.
    \end{enumerate}
\end{problem}

\begin{solution}
    (i $\mapsto$ ii) is obvious. 

    (ii $\mapsto$ iii) Let $(x_n)\in f^{-1}(0)$ with $x_n \to x.$ We want to show that $f(x) = 0.$ Since $f$ is continuous at $0,$ then for all $\epsilon>0,$ there exists a $\delta>0$ such that if $v\in X$ with $\|v\|< \delta,$ then $\|f(v)\|< \epsilon.$
    
    For $n$ large, we have that $\|x_n - x\|<\delta,$ and thus since $f(x_n) = 0:$
    \[\|f(x)\| = \|f(x) - f(x_n)\| = \|f(x - x_n)\| < \epsilon,\] and thus $f(x) = 0,$ and so $x\in f^{-1}(0).$

    (iii $\mapsto$ i) Suppose that for each $n,$ there exists some $x_n \in X$ with $x_n \neq 0$ such that 
    \[\|f(x_n)\| \geq n\|x\_n|,\] that is $X$ is not bounded. Let 
    \[z_n ;= x_1 - f(x_1)\frac{x_n}{f(x_n)}.\] Thus, $f(z_n) = 0,$ and so $z_n \in f^{-1}(\{0\}).$ Moreover, $z_n \to x_1$ since 
    \[\|z_n - x_1\| = \| x_1 - f(x_1)\frac{x_n}{f(x_n)} - x_1\| = \|f(x_1) \frac{x_n}{f(x_n)}\| = |f(x_1)|\frac{\|x_n\|}{|f(x_n)|} \leq |f(x_1)|\frac{1}{n}\]
    Thus, $z_n \to x_1$ but $f(x_1) \geq |x_1|>0$ and so $x_1 \notin f^{-1}(\{0\}),$ and thus $f^{-1}(\{0\})$ is not closed, a contradiction. Thus, $f$ is bounded and so $f$ is continuous.
\end{solution}

\newpage
\section*{Problem 6}
\begin{problem}
    Suppose $X$ is finite dimensional Banach space and $C$ is a convex subset that is dense in $X.$ Then $C = X.$
\end{problem}
\begin{solution}
    Suppose not. Then let $x_0 \in C\setminus C.$ Since $C$ and $\{x_0\}$ are convex, disjoint, and nonempty, then the finite dimensional Hahn-Banach states that there exists some closed hyperplane $H = [f = \alpha]$ such that 
    \[f(C) \leq \alpha \leq f(x_0).\] Let $z_0 = x_0 + \frac{x_0}{f(x_0)}\in X.$ Then 
    \begin{align}
    f(C) \leq \alpha \leq f(x_0) < f(z_0) = f(x_0) + 1.  
    \end{align}
     Let $\epsilon = \frac{f(z_0)}{2}.$ By continuity of $f,$ there exists some $\delta>0$ such that if $\|x-z_0\| < \delta,$ we have that $\|f(x) - f(z_0)\|< \frac{f(z_0)}{2}.$ Since $C$ is dense in $X,$ we can find some $c \in C$ with $\|c - z_0\|< \delta,$ a contradiction to (4).
\end{solution}

\newpage
\section*{Problem 7}
\begin{problem}
    Suppose $C\subset X$ with $X$ Banach and $f: C\to \bbR$ Lipshitz. Show that $f$ can be extended to all of $X.$
\end{problem}
\begin{solution}
    Define $F: X\to \bbR$ such that 
    \[F(x):= \inf_{c\in C}\left[f(c) + L\|x-c\|\right].\] If $c_0\in C,$ then 
    \[F(c_0) = \inf_{c\in C}\left[f(c) + L\|c_0-c\|\right] \leq f(c_0) + L\|c_0-c_0\| = f(c_0).\]
    We also have that for any $c\in C:$
    \[f(c_0) - f(c) \leq L\|c_0 - c\| \implies f(c_0) \leq f(c) + L\|x_0 - c\| \implies f(c_0) \leq F(x).\]
    We see that $F|_C= f.$ 
    

    Suppose $x,y \in X.$ By definition, for all $\epsilon>0,$ there exists some $c_x \in C$ such that 
    \[F(x) \geq f(c_x) + L\|x - c_x\| - \epsilon.\] Since $c_x \in C,$ then  
    \[F(y) \leq f(c_x) + L\|y-c_x\|.\]
    Using the reverse triangle inequality:
    \begin{align*}
        F(y) - F(x) &\leq f(c_x) +L\|y - c_x\| - f(c_x) - L\|x - c_x\| + \epsilon\\
        &\leq L\|y - c_x\| - L\|c_x - x\| + \epsilon\\
        &\leq L\|y - x\| + \epsilon
    \end{align*}
    Because this is true for all $\epsilon>0,$ we see that $F(y) - F(x) \leq L\|y - x\|.$ 

    For all $\epsilon>0,$ there exists some $c_y \in C$ such that 
    \[F(y) \geq f(c_y) + L\|y - c_y\| - \epsilon.\] Since $c_y \in C,$ we have that 
    \[F(x) \leq f(c_y) + L\|x- c_y\|.\] Using the reverse triangle inequality:
    \begin{align*}
        F(x) - F(y) &\leq f(c_y) + L\|x- c_y\| - [f(c_y) + L\|y - c_y\| - \epsilon]\\
        &= f(c_y) + L\|x- c_y\| - f(c_y) - L\|y - c_y\| + \epsilon\\
        &= L(\|x - c_y\| - \|c_y - y\|) + \epsilon\\
        &\leq L(\|x - y\|) + \epsilon,
    \end{align*}
    and so $F(x) - F(y) \leq L\|x - y\|.$ Putting it together, we see that 
    \[|F(x) - F(y)| \leq L\|x-y\|\]
\end{solution}



\newpage
\section*{Problem 8}
\begin{problem}
    Let $X = \bbR^2$ with \[\|x\| = (|x_1|^{4} + |x_2|^{4})^\frac{1}{4}.\] Calculate directly the dual norm on $X^*.$ 
\end{problem}
\begin{solution}
   Let $x\in \bbR^2.$
    \[f(x) = f(x_1e_1 + x_2e_2) = x_1f(e_1)+ x_2f(e_2).\] Thus, using H\"older's inequality:
    \begin{align*}
    |f(x)| &= |x_1f(e_1)+ x_2f(e_2)|\\
    &\leq |x_1||f(e_1)| + |x_2||f(e_2)|\\
    &\leq (|x_1|^4 + |x_2|^4)^\frac{1}{4}(|f(e_1)|^\frac{4}{3} + |f(e_2)|^\frac{4}{3})^\frac{3  }{4}.
    \end{align*}
    Thus, 
    \[\sup_{\|x\| = 1} |f(x)| \leq (|f(e_1)|^\frac{4}{3} + |f(e_2)|^\frac{4}{3})^\frac{3  }{4}.\]
    Let $x\in \bbR^2$ with 
    \[x_n = \begin{cases}
        \frac{|f_n|^\frac{4}{3}}{f_n}, \qquad f_n \neq 0\\
        0, \qquad \;\;\:\: \quad f_n = 0
    \end{cases}.\] Thus, we see that 
    \[|f(x)| = \left|\sum_{n=1}^2x_nf_n\right| = \sum_{n=1}^2 |f_n|^\frac{4}{3} \leq \|f\|\|x\| = \|f\|\left(\sum_{n=1}^2|x_n|^4\right)^\frac{1}{4} = \|f\|\left(\sum_{n=1}^2 |f_n|^\frac{4}{3}\right)^\frac{1}{4}.\] Dividing by the very last term, we see that 
    \[\left(\sum_{n=1}^2 |f_n|^\frac{4}{3}\right)^\frac{3}{4} \leq \|f\|.\] Thus, 
    \[\|f\| = \left(\sum_{n=1}^2 |f_n|^\frac{4}{3}\right)^\frac{3}{4}\]
\end{solution}

\newpage
\section*{Problem 9}
\begin{problem}
    Let $X,Y$ be Banach spaces and suppose $T\in \mathcal{L}(X,Y).$ Show the following are equivalent:
    \begin{enumerate}
        \item $T(X)$ is closed.
        \item $T$ is an open mapping from $X$ unto $T(X)$
        \item There exists some $M>0$ such that for all $y\in T(X),$ there exists some $x\in T^{-1}(y)$ such that $\|x\|_X \leq M \|y\|_Y.$
    \end{enumerate}
\end{problem}
\begin{solution}
    (i $\mapsto$ ii) Suppose $T(X)$ is closed. Since $T(X)\subset Y$ and $Y$ is Banach, then $T(X)$ is Banach. Thus, $T$ is surjective unto $T(X),$ and so by the open mapping theorem, $T$ is an open map from $X$ unto $T(X).$

    (ii $\mapsto$ iii) Suppose $T$ is an open map from $X$ unto $T(X).$ Thus, there exists some $c>0$ such that $B^F_c(0)\subset T(B_E).$ Let $y \in T(X).$ Then $\frac{1}{c}y \in B_c^F(0),$ and so $\frac{1}{c}y \in T(B_E).$ Thus, there exists some $x\in B_E$ such that $T(x) = \frac{1}{c}y.$ That is, $x = T^{-1}(\frac{1}{c}y).$ Thus, 
    \[\|x\| = \|T^{-1}\frac{1}{c}y\| \leq \frac{1}{c}\|T\|\|y\|_Y.\] Let $M = \frac{1}{c}\|T^{-1}\|.$ It suffices to show that $T^{-1}: T(X)\to Y$ is bounded, but this comes from the fact that $T^{-1}(B_c^F(0))\subset B_E.$

    (iii $\mapsto$ i) Define $\pi: E \to E/\ker T$ to be the canonical surjection. We know that 
    \[\|\pi x\| = \text{dist}\|x- N(T)\|,\] and that $T = \tilde{T}\circ \pi,$ where $\tilde{T}: E/\ker T \to F$ is a bijection from $E/\ker T$ to $R(T)$ with $R(T) = R(\tilde{T})$ Moreover, $\|T\| = \|\tilde{T}\|.$ Since $\tilde{T}$ is bijective and it is known that $E/\ker T$ is a Banach Space, then $R(\tilde{T})\subset F$ is closed if and only if $R(\tilde{T})$ is a Banach space if and only if (by a corollary of the open mapping theorem) $\tilde{T}^{-1}$ is continuous if and only if there exists some $M>0$ such that for all $y\in R(T),$ we have that
    \[\|\tilde{T}^{-1}y\| = \|x\|\leq M\|y\|.\] Without using the open mapping theorem, we have that $\tilde{T}^{-1}$ is continuous since $\tilde{T}$ is an isomorphism. Thus, we see that if $y_n \to y$ with $(y_n)\in T(X),$ then $(y_n)$ is Cauchy and so for large enough $n,$ we get that if $x_n = \tilde{T}^{-1}y_n$
    \[\|x_n - x_m\| \leq M\|T(x_n - x_m)\| = M\|Tx_n - Tx_m\|= M\|y_n - y_m\|< \epsilon,\] and so $(x_n)\in X$ is Cauchy and thus converges to some $x$ via the completeness of $X,$ and we claim that $Tx = y:$
    \[\|Tx - y\|\leq \|Tx - Tx_n\| + \|Tx_n - y\| \leq \|T\|\|x - x_n\| + \|y_n - y\| < \epsilon.\]
\end{solution}

\newpage
\section*{Problem 10}
\begin{problem}
Let $X,Y$  be Banach and suppose $T\in \mathcal{L}(X,Y).$ Show that if $T$ maps closed sets in $X$ unto closed sets on $Y,$ then $T(X)$ is closed in $Y.$     
\end{problem}

\begin{solution}
    By the previous problem, it suffices to show that $T$ is an open mapping from $X$ unto $T(X).$ By class, it suffices to show that there exists some $c>0$ such that $B_c^F(0)\subset T(B_E).$ We will show this result for $\tilde{T}.$ That is, $T = \tilde{T}\circ \pi,$ where $\tilde{T}: X/\ker T \to Y$ is a bijection from $E/\ker T$ to $R(T)$ with $R(T) = R(\tilde{T})$ Moreover, $\|T\| = \|\tilde{T}\|$ and $\tilde{T}$ is bijective.

    To show that $\tilde{T}$ maps closed sets (in $X/\ker T$ with the norm $\|[x]\|_{X/\ker T} = \text{dist}(x - \ker T)$) to closed sets in $Y,$ let $K$ be closed in $X/\ker T.$ Let $(y_n)\in \tilde{T}(K)$ with $y_n \to y.$ Since $\tilde{T}$ is bijective, then the inverse exists and is continuous, and thus there exists $(x_n)\in X\setminus \ker T$ such that $x_n= y_n$ and $\|x\| = \|\tilde{T}^{-1}y_n\| \leq \|T^{-1}\|\|y_n\| \leq C\|y_n\|.$ Thus, since $(y_n)$ is Cauchy, then for large $n,m:$
    \[\|x_n - x_m\| \leq C\|y_n - y_m\|< \epsilon,\] and thus $(x_n)$ is Cauchy. We wish to show that $x_n \to x \in K,$ so it suffices to notice that $K$ is closed and $X/\ker T$ is Banach, and thus $K$ is Banach. Of course, it remains to check that $X/\ker T$ is actually Banach:

    Let $M\subset X$ be closed and suppose $\pi: X \to X/M$ is the natural surjection. Then let $(\pi(x_n))\in X/M$ be Cauchy. We can assume by passing unto a subsequence that 
    \[\|\pi(x_{n_{k+1}}) - \pi(x_{n_{k}})\|_{X/M} < \frac{1}{2^{k}}\] By the definition of the norm, there exists some $(m_{n_k})\in M$ such that 
    \[\|x_{n_{k+1}} - x_{n_{k}} - m_{n_k}\|_{X} < \frac{1}{2^{k}}.\] Without any issue, we write $m_{n_k} = u_{n_{k+1}} - u_{n_k}$ and ($u_{n_1} = 0$) to see that $x_{n_k} - u_{n_k}$ is Cauchy in $X,$ and thus converges to a limit in $X,$ and so $\pi(x_{n_k} - u_{n_k}) = \pi(x_{n_k})$ also converges to a limit, and since the sequence is Cauchy, then the entire sequence converges to a limit. 

    Thus, we find that since $\ker T$ is closed, then $X/\ker T$ is Banach, and so we have found that $x\in K.$ To see that $Tx= y,$ notice that for large $n:$
    \[\|Tx - y\|\leq \|Tx - Tx_n\| + \|Tx_n - y\| \leq \|T\|\|x - x_n\| + \|y_n - y\| < \epsilon.\] Thus, $\tilde{T}$ maps closed maps into closed maps. 

    Suppose $\tilde{T}$ is not an open mapping. Then for all $n>0,$ there exists some $y_n\in R(\tilde{T})$ with $\|y_n\| \leq \frac{1}{n}$ such that $y_n \notin \tilde{T}(B_{x/\ker T}).$ By the bijectivity of $\tilde{T},$ there exist $x_n \in X/\ker T$ such that $x_n = \tilde{T}^{-1}y_n$ and $\|x_n\|\geq 1.$ Since $\tilde{T}^{-1}$ is continuous, then $\|x_n\|\leq C\|y_n\|$ for all $n,$ and thus if 
    \[\hat{y}_n = \frac{y_n}{\|x_n\|}\] and $\hat{x}_n = \tilde{T}^{-1}\hat{y}_n,$ then $\hat{x}_n \in \overline{B_{X/\ker T}}$ for any $n$ since $\|\hat{x}_n\| = 1.$ Notice how we have that $\|\hat{y}_n\| < \frac{1}{n},$ and so $y_n \to y,$ where $(\hat{y}_n)\in \tilde{T}(\overline{B_{X/\ker T}}).$ Since $\tilde{T}$ is a closed mapping, then $\hat{y}\in \tilde{T}(B_{X/\ker T}),$ and since $\|\hat{y}_n\|< \frac{1}{n},$ $\hat{y} = 0.$
    There exists some $\hat{x}\in B_{X/\ker T}$ such that $\tilde{T}\hat{x} = \hat{y}.$ To see that $\hat{x}_n \to x,$ consider that 
    \[\|\hat{x}_n - \hat{x}\|= \|T^{-1}(\hat{y}_n - \hat{y})\| \leq C\|\hat{y}_n - \hat{y}\| < \epsilon.\] We have that $\hat{x} \in B_{X/\ker T}$ and in fact, $\|\hat{x}\| = 1.$ However, 
    \[T\hat{x}_n = \hat{y}_n \to \hat{y} = 0\] But 
    \[T\hat{x}_n \to T\hat{x} \neq 0,\] a contradiction! 
    
    Thus, $\tilde{T}$ is an open mapping unto $R(\tilde{T}).$ By the previous problem, $R(\tilde{T})$ is closed. Finally, we use that $R(\tilde{T}) = R(T)$ and we conclude.
    
\end{solution}




\newpage
\section*{Problem 11}
\begin{problem}
    Let $T\in \mathcal{L}(X,Y).$ Prove the following:
    \begin{enumerate}
        \item $\ker(T) = R(T^*)^\perp$
        \item $\ker(T^*) = R(T)^\perp$
        \item $\overline{R(T)} = \ker(T^*)^\perp$
        \item $\overline{R(T^*)} \subset \ker(T)^\perp$
    \end{enumerate}
\end{problem}
\begin{solution}
    (a) We begin by noting the definitions: 
    \[\ker(T) = \{x \in X \; : \; Tx = 0\}\]
    \[R(T^*)^\perp = \{x\in X \; : \; \langle T^*v, x\rangle = 0, \; \forall \: v \in Y^*\}.\]
    Let $x\in \ker T,$ then $Tx = 0,$ and so 
    \[\langle T^* v, x\rangle = \langle v, Tx\rangle=  \langle v, 0\rangle = 0,\] and so $x\in R(T^*)^\perp.$ Thus $\ker T \subset R(T^*)^\perp.$

    Let $x\in R(T^*)^\perp$ and suppose $x\not \in \ker(T).$ Thus, $(x,0)\not \in G(T).$ $T$ is continuous, and so $G(T)$ is closed. Thus, we use Hahn-Banach. There exist $(f,g)\in E^* \times F^*$ such that 
    \[\langle f, x\rangle + \langle g, Tx\rangle < \alpha < \langle f, u \rangle + \langle g, 0\rangle = \langle f, u \rangle, \quad \forall x\in D(T).\] Because $G(T)$ is a linear subspace, then 
    \begin{align}
    \langle f, x\rangle + \langle g, Tx\rangle = 0    
    \end{align}
     and thus $|\langle g, Tx \rangle| \leq \|f\|\|x\|,$ and so $g\in D(T^*).$ Using (5), we see that 
     \[\langle f + T^* g, x\rangle = 0, \quad \forall x\in D(T) \implies f = T^*(g).\] Thus, $\langle f, u \rangle = \langle T^*g, u \rangle >0,$ but $u\in R(T^*)^\perp,$ and so $\langle T^*g, u\rangle = 0.$ A contradiction! Thus, we get (a).

     To see (b), we let $v\in \ker(T^*).$ Thus, $T^*v = 0,$ and so for any $x\in D(T),$ we have that 
     \[0 = \langle T^* v, x \rangle =\langle v, Tx\rangle ,\] and so $v\in R(T)^\perp.$

     Let $v\in R(T)^\perp.$ Then for all $x\in D(T),$ $\langle v, Tx \rangle = 0.$ Thus, $|\langle T^* v, x| \leq 0 \|x\|.$ If $v\in D(T^*),$ then $0 = \langle v, Ax \rangle = \langle T^*v, x \rangle  =0$ for all $x\in D(T)$ and so $T*v = 0$ and so $v\in \ker(T^*.)$ This shows (b).

     For (c), it suffices to see that $(M^\perp)^\perp = \overline{M},$ where $M\subset X$ is a linear subspace.
     \[M^\perp = \{f\in Y^* \; : \; \langle f, x\rangle = 0, \; \forall\: x\in X\}\subset Y^*.\] Thus, 
     \[(M^\perp)^\perp =\{x\in X \; : \; \langle x, f \rangle = 0, \quad \forall \; f\in M^\perp\} \subset X\] Evidently, $M \subset (M^\perp)^\perp.$ $(M^\perp)^\perp$ is closed, and thus $\overline{M}\subset (M^\perp)^\perp.$

     Suppose there is some $x_0\in \overline{M}$ such that $x\notin (M^\perp)^\perp.$ We use Hahn Banach to separate $\{x_0\}$ and $\overline{M}.$ There exists some closed $f = [H = \alpha]$ such that 
     \begin{align}
     f(\overline{M}) < \alpha < f(x_0),    
     \end{align}
      and thus $f(M) = 0,$ and so for all $x\in M,$ we have that $\langle f, x \rangle = 0,$ and so $f\in M^\perp$ and thus $\langle f, x_0 \rangle = 0,$ a contradiction to (6). Thus, since $R(T)\subset Y$ is a linear subspace, then $(R(T)^\perp)^\perp = \overline{R(T)},$ and by (c), we see that $\overline{R(T)} = \ker(T^*)^\perp.$

      Suppose that $M\subset E^*.$ We want to show that $\overline{M}\subset (M^\perp)^\perp.$ This is clear from the definition and the above. Thus the result follows directly from (a).
\end{solution}

\newpage
\section*{Problem 12}
\begin{problem}
    Let $X,Y$ be Banach with $T\in \mathcal{L}(X,Y).$ Show that $T$ maps $X$ unto a dense set if and only if $T^*$ is injective. 
\end{problem}
\begin{solution}
    $T^*$ is injective if and only if $\ker T^* = \{0\}$ if and only if $\ker(T^*)^\perp = Y$ if and only if $\overline{R(T)} = Y.$
\end{solution}


\end{document}