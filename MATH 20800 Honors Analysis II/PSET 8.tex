\documentclass[11pt]{article}

% NOTE: Add in the relevant information to the commands below; or, if you'll be using the same information frequently, add these commands at the top of paolo-pset.tex file. 
\newcommand{\name}{Agustín Esteva}
\newcommand{\email}{aesteva@uchicago.edu}
\newcommand{\classnum}{207}
\newcommand{\subject}{Honors Analysis in $\bbR^n$}
\newcommand{\instructors}{Panagiotis E. Souganidis}
\newcommand{\assignment}{Problem Set 8}
\newcommand{\semester}{Winter 2024}
\newcommand{\duedate}{\today}
\newcommand{\bA}{\mathbf{A}}
\newcommand{\bB}{\mathbf{B}}
\newcommand{\bC}{\mathbf{C}}
\newcommand{\bD}{\mathbf{D}}
\newcommand{\bE}{\mathbf{E}}
\newcommand{\bF}{\mathbf{F}}
\newcommand{\bG}{\mathbf{G}}
\newcommand{\bH}{\mathbf{H}}
\newcommand{\bI}{\mathbf{I}}
\newcommand{\bJ}{\mathbf{J}}
\newcommand{\bK}{\mathbf{K}}
\newcommand{\bL}{\mathbf{L}}
\newcommand{\bM}{\mathbf{M}}
\newcommand{\bN}{\mathbf{N}}
\newcommand{\bO}{\mathbf{O}}
\newcommand{\bP}{\mathbf{P}}
\newcommand{\bQ}{\mathbf{Q}}
\newcommand{\bR}{\mathbf{R}}
\newcommand{\bS}{\mathbf{S}}
\newcommand{\bT}{\mathbf{T}}
\newcommand{\bU}{\mathbf{U}}
\newcommand{\bV}{\mathbf{V}}
\newcommand{\bW}{\mathbf{W}}
\newcommand{\bX}{\mathbf{X}}
\newcommand{\bY}{\mathbf{Y}}
\newcommand{\bZ}{\mathbf{Z}}

%% blackboard bold math capitals
\newcommand{\bbA}{\mathbb{A}}
\newcommand{\bbB}{\mathbb{B}}
\newcommand{\bbC}{\mathbb{C}}
\newcommand{\bbD}{\mathbb{D}}
\newcommand{\bbE}{\mathbb{E}}
\newcommand{\bbF}{\mathbb{F}}
\newcommand{\bbG}{\mathbb{G}}
\newcommand{\bbH}{\mathbb{H}}
\newcommand{\bbI}{\mathbb{I}}
\newcommand{\bbJ}{\mathbb{J}}
\newcommand{\bbK}{\mathbb{K}}
\newcommand{\bbL}{\mathbb{L}}
\newcommand{\bbM}{\mathbb{M}}
\newcommand{\bbN}{\mathbb{N}}
\newcommand{\bbO}{\mathbb{O}}
\newcommand{\bbP}{\mathbb{P}}
\newcommand{\bbQ}{\mathbb{Q}}
\newcommand{\bbR}{\mathbb{R}}
\newcommand{\bbS}{\mathbb{S}}
\newcommand{\bbT}{\mathbb{T}}
\newcommand{\bbU}{\mathbb{U}}
\newcommand{\bbV}{\mathbb{V}}
\newcommand{\bbW}{\mathbb{W}}
\newcommand{\bbX}{\mathbb{X}}
\newcommand{\bbY}{\mathbb{Y}}
\newcommand{\bbZ}{\mathbb{Z}}

%% script math capitals
\newcommand{\sA}{\mathscr{A}}
\newcommand{\sB}{\mathscr{B}}
\newcommand{\sC}{\mathscr{C}}
\newcommand{\sD}{\mathscr{D}}
\newcommand{\sE}{\mathscr{E}}
\newcommand{\sF}{\mathscr{F}}
\newcommand{\sG}{\mathscr{G}}
\newcommand{\sH}{\mathscr{H}}
\newcommand{\sI}{\mathscr{I}}
\newcommand{\sJ}{\mathscr{J}}
\newcommand{\sK}{\mathscr{K}}
\newcommand{\sL}{\mathscr{L}}
\newcommand{\sM}{\mathscr{M}}
\newcommand{\sN}{\mathscr{N}}
\newcommand{\sO}{\mathscr{O}}
\newcommand{\sP}{\mathscr{P}}
\newcommand{\sQ}{\mathscr{Q}}
\newcommand{\sR}{\mathscr{R}}
\newcommand{\sS}{\mathscr{S}}
\newcommand{\sT}{\mathscr{T}}
\newcommand{\sU}{\mathscr{U}}
\newcommand{\sV}{\mathscr{V}}
\newcommand{\sW}{\mathscr{W}}
\newcommand{\sX}{\mathscr{X}}
\newcommand{\sY}{\mathscr{Y}}
\newcommand{\sZ}{\mathscr{Z}}


\renewcommand{\emptyset}{\O}

\newcommand{\abs}[1]{\lvert #1 \rvert}
\newcommand{\norm}[1]{\lVert #1 \rVert}
\newcommand{\sm}{\setminus}


\newcommand{\sarr}{\rightarrow}
\newcommand{\arr}{\longrightarrow}

% NOTE: Defining collaborators is optional; to not list collaborators, comment out the line below.
%\newcommand{\collaborators}{Alyssa P. Hacker (\texttt{aphacker}), Ben Bitdiddle (\texttt{bitdiddle})}

% Copyright 2021 Paolo Adajar (padajar.com, paoloadajar@mit.edu)
% 
% Permission is hereby granted, free of charge, to any person obtaining a copy of this software and associated documentation files (the "Software"), to deal in the Software without restriction, including without limitation the rights to use, copy, modify, merge, publish, distribute, sublicense, and/or sell copies of the Software, and to permit persons to whom the Software is furnished to do so, subject to the following conditions:
%
% The above copyright notice and this permission notice shall be included in all copies or substantial portions of the Software.
% 
% THE SOFTWARE IS PROVIDED "AS IS", WITHOUT WARRANTY OF ANY KIND, EXPRESS OR IMPLIED, INCLUDING BUT NOT LIMITED TO THE WARRANTIES OF MERCHANTABILITY, FITNESS FOR A PARTICULAR PURPOSE AND NONINFRINGEMENT. IN NO EVENT SHALL THE AUTHORS OR COPYRIGHT HOLDERS BE LIABLE FOR ANY CLAIM, DAMAGES OR OTHER LIABILITY, WHETHER IN AN ACTION OF CONTRACT, TORT OR OTHERWISE, ARISING FROM, OUT OF OR IN CONNECTION WITH THE SOFTWARE OR THE USE OR OTHER DEALINGS IN THE SOFTWARE.

\usepackage{fullpage}
\usepackage{enumitem}
\usepackage{amsfonts, amssymb, amsmath,amsthm}
\usepackage{mathtools}
\usepackage[pdftex, pdfauthor={\name}, pdftitle={\classnum~\assignment}]{hyperref}
\usepackage[dvipsnames]{xcolor}
\usepackage{bbm}
\usepackage{graphicx}
\usepackage{mathrsfs}
\usepackage{pdfpages}
\usepackage{tabularx}
\usepackage{pdflscape}
\usepackage{makecell}
\usepackage{booktabs}
\usepackage{natbib}
\usepackage{caption}
\usepackage{subcaption}
\usepackage{physics}
\usepackage[many]{tcolorbox}
\usepackage{version}
\usepackage{ifthen}
\usepackage{cancel}
\usepackage{listings}
\usepackage{courier}

\usepackage{tikz}
\usepackage{istgame}

\hypersetup{
	colorlinks=true,
	linkcolor=blue,
	filecolor=magenta,
	urlcolor=blue,
}

\setlength{\parindent}{0mm}
\setlength{\parskip}{2mm}

\setlist[enumerate]{label=({\alph*})}
\setlist[enumerate, 2]{label=({\roman*})}

\allowdisplaybreaks[1]

\newcommand{\psetheader}{
	\ifthenelse{\isundefined{\collaborators}}{
		\begin{center}
			{\setlength{\parindent}{0cm} \setlength{\parskip}{0mm}
				
				{\textbf{\classnum~\semester:~\assignment} \hfill \name}
				
				\subject \hfill \href{mailto:\email}{\tt \email}
				
				Instructor(s):~\instructors \hfill Due Date:~\duedate	
				
				\hrulefill}
		\end{center}
	}{
		\begin{center}
			{\setlength{\parindent}{0cm} \setlength{\parskip}{0mm}
				
				{\textbf{\classnum~\semester:~\assignment} \hfill \name\footnote{Collaborator(s): \collaborators}}
				
				\subject \hfill \href{mailto:\email}{\tt \email}
				
				Instructor(s):~\instructors \hfill Due Date:~\duedate	
				
				\hrulefill}
		\end{center}
	}
}

\renewcommand{\thepage}{\classnum~\assignment \hfill \arabic{page}}

\makeatletter
\def\points{\@ifnextchar[{\@with}{\@without}}
\def\@with[#1]#2{{\ifthenelse{\equal{#2}{1}}{{[1 point, #1]}}{{[#2 points, #1]}}}}
\def\@without#1{\ifthenelse{\equal{#1}{1}}{{[1 point]}}{{[#1 points]}}}
\makeatother

\newtheoremstyle{theorem-custom}%
{}{}%
{}{}%
{\itshape}{.}%
{ }%
{\thmname{#1}\thmnumber{ #2}\thmnote{ (#3)}}

\theoremstyle{theorem-custom}

\newtheorem{theorem}{Theorem}
\newtheorem{lemma}[theorem]{Lemma}
\newtheorem{example}[theorem]{Example}

\newenvironment{problem}[1]{\color{black} #1}{}

\newenvironment{solution}{%
	\leavevmode\begin{tcolorbox}[breakable, colback=green!5!white,colframe=green!75!black, enhanced jigsaw] \proof[\scshape Solution:] \setlength{\parskip}{2mm}%
	}{\renewcommand{\qedsymbol}{$\blacksquare$} \endproof \end{tcolorbox}}

\newenvironment{reflection}{\begin{tcolorbox}[breakable, colback=black!8!white,colframe=black!60!white, enhanced jigsaw, parbox = false]\textsc{Reflections:}}{\end{tcolorbox}}

\newcommand{\qedh}{\renewcommand{\qedsymbol}{$\blacksquare$}\qedhere}

\definecolor{mygreen}{rgb}{0,0.6,0}
\definecolor{mygray}{rgb}{0.5,0.5,0.5}
\definecolor{mymauve}{rgb}{0.58,0,0.82}

% from https://github.com/satejsoman/stata-lstlisting
% language definition
\lstdefinelanguage{Stata}{
	% System commands
	morekeywords=[1]{regress, reg, summarize, sum, display, di, generate, gen, bysort, use, import, delimited, predict, quietly, probit, margins, test},
	% Reserved words
	morekeywords=[2]{aggregate, array, boolean, break, byte, case, catch, class, colvector, complex, const, continue, default, delegate, delete, do, double, else, eltypedef, end, enum, explicit, export, external, float, for, friend, function, global, goto, if, inline, int, local, long, mata, matrix, namespace, new, numeric, NULL, operator, orgtypedef, pointer, polymorphic, pragma, private, protected, public, quad, real, return, rowvector, scalar, short, signed, static, strL, string, struct, super, switch, template, this, throw, transmorphic, try, typedef, typename, union, unsigned, using, vector, version, virtual, void, volatile, while,},
	% Keywords
	morekeywords=[3]{forvalues, foreach, set},
	% Date and time functions
	morekeywords=[4]{bofd, Cdhms, Chms, Clock, clock, Cmdyhms, Cofc, cofC, Cofd, cofd, daily, date, day, dhms, dofb, dofC, dofc, dofh, dofm, dofq, dofw, dofy, dow, doy, halfyear, halfyearly, hh, hhC, hms, hofd, hours, mdy, mdyhms, minutes, mm, mmC, mofd, month, monthly, msofhours, msofminutes, msofseconds, qofd, quarter, quarterly, seconds, ss, ssC, tC, tc, td, th, tm, tq, tw, week, weekly, wofd, year, yearly, yh, ym, yofd, yq, yw,},
	% Mathematical functions
	morekeywords=[5]{abs, ceil, cloglog, comb, digamma, exp, expm1, floor, int, invcloglog, invlogit, ln, ln1m, ln, ln1p, ln, lnfactorial, lngamma, log, log10, log1m, log1p, logit, max, min, mod, reldif, round, sign, sqrt, sum, trigamma, trunc,},
	% Matrix functions
	morekeywords=[6]{cholesky, coleqnumb, colnfreeparms, colnumb, colsof, corr, det, diag, diag0cnt, el, get, hadamard, I, inv, invsym, issymmetric, J, matmissing, matuniform, mreldif, nullmat, roweqnumb, rownfreeparms, rownumb, rowsof, sweep, trace, vec, vecdiag, },
	% Programming functions
	morekeywords=[7]{autocode, byteorder, c, _caller, chop, abs, clip, cond, e, fileexists, fileread, filereaderror, filewrite, float, fmtwidth, has_eprop, inlist, inrange, irecode, matrix, maxbyte, maxdouble, maxfloat, maxint, maxlong, mi, minbyte, mindouble, minfloat, minint, minlong, missing, r, recode, replay, return, s, scalar, smallestdouble,},
	% Random-number functions
	morekeywords=[8]{rbeta, rbinomial, rcauchy, rchi2, rexponential, rgamma, rhypergeometric, rigaussian, rlaplace, rlogistic, rnbinomial, rnormal, rpoisson, rt, runiform, runiformint, rweibull, rweibullph,},
	% Selecting time-span functions
	morekeywords=[9]{tin, twithin,},
	% Statistical functions
	morekeywords=[10]{betaden, binomial, binomialp, binomialtail, binormal, cauchy, cauchyden, cauchytail, chi2, chi2den, chi2tail, dgammapda, dgammapdada, dgammapdadx, dgammapdx, dgammapdxdx, dunnettprob, exponential, exponentialden, exponentialtail, F, Fden, Ftail, gammaden, gammap, gammaptail, hypergeometric, hypergeometricp, ibeta, ibetatail, igaussian, igaussianden, igaussiantail, invbinomial, invbinomialtail, invcauchy, invcauchytail, invchi2, invchi2tail, invdunnettprob, invexponential, invexponentialtail, invF, invFtail, invgammap, invgammaptail, invibeta, invibetatail, invigaussian, invigaussiantail, invlaplace, invlaplacetail, invlogistic, invlogistictail, invnbinomial, invnbinomialtail, invnchi2, invnF, invnFtail, invnibeta, invnormal, invnt, invnttail, invpoisson, invpoissontail, invt, invttail, invtukeyprob, invweibull, invweibullph, invweibullphtail, invweibulltail, laplace, laplaceden, laplacetail, lncauchyden, lnigammaden, lnigaussianden, lniwishartden, lnlaplaceden, lnmvnormalden, lnnormal, lnnormalden, lnwishartden, logistic, logisticden, logistictail, nbetaden, nbinomial, nbinomialp, nbinomialtail, nchi2, nchi2den, nchi2tail, nF, nFden, nFtail, nibeta, normal, normalden, npnchi2, npnF, npnt, nt, ntden, nttail, poisson, poissonp, poissontail, t, tden, ttail, tukeyprob, weibull, weibullden, weibullph, weibullphden, weibullphtail, weibulltail,},
	% String functions 
	morekeywords=[11]{abbrev, char, collatorlocale, collatorversion, indexnot, plural, plural, real, regexm, regexr, regexs, soundex, soundex_nara, strcat, strdup, string, strofreal, string, strofreal, stritrim, strlen, strlower, strltrim, strmatch, strofreal, strofreal, strpos, strproper, strreverse, strrpos, strrtrim, strtoname, strtrim, strupper, subinstr, subinword, substr, tobytes, uchar, udstrlen, udsubstr, uisdigit, uisletter, ustrcompare, ustrcompareex, ustrfix, ustrfrom, ustrinvalidcnt, ustrleft, ustrlen, ustrlower, ustrltrim, ustrnormalize, ustrpos, ustrregexm, ustrregexra, ustrregexrf, ustrregexs, ustrreverse, ustrright, ustrrpos, ustrrtrim, ustrsortkey, ustrsortkeyex, ustrtitle, ustrto, ustrtohex, ustrtoname, ustrtrim, ustrunescape, ustrupper, ustrword, ustrwordcount, usubinstr, usubstr, word, wordbreaklocale, worcount,},
	% Trig functions
	morekeywords=[12]{acos, acosh, asin, asinh, atan, atanh, cos, cosh, sin, sinh, tan, tanh,},
	morecomment=[l]{//},
	% morecomment=[l]{*},  // `*` maybe used as multiply operator. So use `//` as line comment.
	morecomment=[s]{/*}{*/},
	% The following is used by macros, like `lags'.
	morestring=[b]{`}{'},
	% morestring=[d]{'},
	morestring=[b]",
	morestring=[d]",
	% morestring=[d]{\\`},
	% morestring=[b]{'},
	sensitive=true,
}

\lstset{ 
	backgroundcolor=\color{white},   % choose the background color; you must add \usepackage{color} or \usepackage{xcolor}; should come as last argument
	basicstyle=\footnotesize\ttfamily,        % the size of the fonts that are used for the code
	breakatwhitespace=false,         % sets if automatic breaks should only happen at whitespace
	breaklines=true,                 % sets automatic line breaking
	captionpos=b,                    % sets the caption-position to bottom
	commentstyle=\color{mygreen},    % comment style
	deletekeywords={...},            % if you want to delete keywords from the given language
	escapeinside={\%*}{*)},          % if you want to add LaTeX within your code
	extendedchars=true,              % lets you use non-ASCII characters; for 8-bits encodings only, does not work with UTF-8
	firstnumber=0,                % start line enumeration with line 1000
	frame=single,	                   % adds a frame around the code
	keepspaces=true,                 % keeps spaces in text, useful for keeping indentation of code (possibly needs columns=flexible)
	keywordstyle=\color{blue},       % keyword style
	language=Octave,                 % the language of the code
	morekeywords={*,...},            % if you want to add more keywords to the set
	numbers=left,                    % where to put the line-numbers; possible values are (none, left, right)
	numbersep=5pt,                   % how far the line-numbers are from the code
	numberstyle=\tiny\color{mygray}, % the style that is used for the line-numbers
	rulecolor=\color{black},         % if not set, the frame-color may be changed on line-breaks within not-black text (e.g. comments (green here))
	showspaces=false,                % show spaces everywhere adding particular underscores; it overrides 'showstringspaces'
	showstringspaces=false,          % underline spaces within strings only
	showtabs=false,                  % show tabs within strings adding particular underscores
	stepnumber=2,                    % the step between two line-numbers. If it's 1, each line will be numbered
	stringstyle=\color{mymauve},     % string literal style
	tabsize=2,	                   % sets default tabsize to 2 spaces
%	title=\lstname,                   % show the filename of files included with \lstinputlisting; also try caption instead of title
	xleftmargin=0.25cm
}

% NOTE: To compile a version of this pset without problems, solutions, or reflections, uncomment the relevant line below.

%\excludeversion{problem}
%\excludeversion{solution}
%\excludeversion{reflection}

\begin{document}	
	
	% Use the \psetheader command at the beginning of a pset. 
	\psetheader

\section*{Problem 1}
\begin{problem}
    Let $(u_n) \in H$ and $(t_n) \in (0,\infty)$ such that 
    \[(t_nu_n - t_mu_m, u_n - u_m) \leq 0,\]
    \begin{enumerate}
        \item Suppose $(t_n)$ is nondecreasing. Prove that $(u_n)$ converges to a limit.
        \begin{solution}
       Consider that \[|u_n - u_m|^2 = |u_n|^2 + |u_m|^2 - 2(u_n, u_m).\]
            Using bilinearity and properties of the inner product, we have that by plugging the above at a convenient spot:
            \begin{align*}
               (t_nu_n - t_mu_m, u_n - u_m) &= (t_nu_n, u_n) - (t_nu_n, u_m)- (t_mu_m, u_n) + (t_mu_m, u_m)\\
               &= t_n|u_n|^2 - t_n(u_n, u_m) - t_m(u_n, u_m) + t_m|v_m|^2\\
               &= t_n|u_n|^2 - (t_n + t_m)(u_n, u_m) + t_m|v_m|^2\\
               &= t_n|u_n|^2 - \frac{1}{2}(t_n + t_m)(-|u_n - u_m|^2 + |u_n|^2 + |u_m|^2) + t_m|v_m|^2\\
               &= \frac{1}{2}(t_n - t_m)|u_n|^2 +\frac{1}{2}(t_n + t_m)(|u_n - u_m|^2) - \frac{1}{2}(t_n - t_m)|u_m|^2\\
               &= \frac{1}{2}(t_n - t_m)(|u_n|^2 - |u_m|^2) + \frac{1}{2}(t_n + t_m)(|u_n - u_m|^2)\\
               &\leq 0
            \end{align*}
            The first term must necessarily be negative since the second is positive, and so
            \begin{align}
                (t_n - t_m)(|u_n|^2 - |u_m|^2) \leq 0
            \end{align}
            Since $(t_n)$ is nondecreasing, then for $n>m,$ we have that $t_n - t_m \geq 0,$ and thus by (1) we have that 
            \[|u_n|^2 - |u_m|^2 \leq 0 \implies |u_n|\leq |u_m|,\] and so $|u_n|$ is nonincreasing. Moreover, we have that 
            \[\frac{1}{2}(t_n + t_m)(|u_n - u_m|^2) \leq -\frac{1}{2}(t_n - t_m)(|u_n|^2 - |u_m|^2) = \frac{1}{2}(t_n - t_m)(|u_m|^2- |u_n|^2)\leq \frac{1}{2}t_n(|u_m|^2- |u_n|^2),\] and so since $t_n \leq t_n + t_m,$ we have that
            \[|u_n - u_m|^2 \leq |u_m|^2 - |u_n|^2.\] Letting $n\to \infty,$  we see that since $|u_n|$ is nonincreasing and bounded below, that $|u_n|\to L,$ and so $|u_n|^2 \to L^2$ Thus, taking letting $m\to \infty$ and letting $n = m+1,$ we see that 
            \[|u_n - u_m|^2 \leq |u_m|^2 - |u_n|^2 \to L^2 - L^2 = 0,\] and so $|u_n - u_m|$ is Cauchy. Thus, since $(u_n)\in H,$ we have that $(u_n)$ converges.
        \end{solution}
        \begin{problem}
            Assume that $(t_n)$ is non increasing. Then either:
            \begin{enumerate}
                \item $|u_n|\to \infty$
                \item or $(u_n)$ converges
            \end{enumerate}
            If $t_n \to t >0,$ show that $(u_n)$ converges. 
        \end{problem}
        \begin{solution}
            Let $n>m,$ then by (1) we have that since $t_n - t_m \leq 0,$ then $(|u_n|^2 - |u_m|^2)\geq 0 \implies |u_n|^2 \leq |u_m|^2,$ and so $|u_n|$ is non decreasing. From the above calculations, we have that 
            \[|u_n - u_m|^2 \leq |u_n|^2 - |u_m|^2.\] Thus, if $|u_n| \to L \leq \infty,$ then by the above reasoning, $(u_n)$ converges. If $|u_n| \to \infty,$ then evidently $(u_n) \to \infty.$ 

            Suppose $t_n \to t.$ As per the hint, we let $v_n = t_nu_n$ and so $s_n = \frac{1}{t_n},$ then 
            \[(s_nv_n - s_mv_m, v_n - v_m)\leq 0,\] then by the work above, $v_n$ converges and so $u_n$ converges. 
        \end{solution}
    \end{enumerate}
\end{problem}

\newpage
\section*{Problem 2}
\begin{problem}
    Let $K\subset H$ be a closed convex set and let $f\in H.$ If $u = P_Kf,$ then show that for any $v\in K$
    \[|v-u|^2 \leq |v-f|^2 - |f-u|^2,\] deduce that 
    \[|v-u|\leq |v-f|\] and give a geometric interpretation.
\end{problem}
\begin{solution}
    Consider that 
    \begin{align*}
        |v - u|^2 &= |v-f + f-u|^2\\
        &= (v-f + f-u ,v-f + f-u)\\
        &= (v-f, v-f) + 2(v-f, f-u) + (f-u, f-u)\\
        &= |v-f|^2 + |f-u|^2 + 2(v - f, f-u)\\
        &= |u-f|^2 + |v-f|^2 + 2(f-u, v-f)\\
        &= |u-f|^2 + |v-f|^2 + 2(f-u, v-u + u -f)\\
        &= |u-f|^2 + |v-f|^2 + 2(f - u, v-u) -2 (f-u, f-u)\\
        &= |u-f|^2 + |v-f|^2 + 2(f - u, v-u) -2 |f-u|^2\\
        &= |v-f|^2 - |f-u|^2 + 2(f-u, v-u)\\
        &\leq |v-f|^2 - |f-u|^2
    \end{align*}
    Suppose not for the second part, then 
    \[|v-u| > |v-f| \implies |v-u|^2 > |v-f|^2 \implies |v-u|^2 > |v-f|^2 - |f-u|^2,\] which is a contradiction.
    A geometric interpretation:
    
\end{solution}

\newpage
\section*{Problem 3}
\begin{problem}
\begin{enumerate}
    \item Let \( (K_n) \) be a nonincreasing sequence of closed convex sets in \( H \) such that \( \bigcap_n K_n \neq \emptyset \).
   Prove that for every \( f \in H \) the sequence \( u_n = P_{K_n} f \) converges (strongly) to a limit and identify the limit.
   \begin{solution}
       There really is only one natural candidate for the limit. We claim that 
       \[u_n \to u, \quad u = P_{K_\infty}f.\] Where $K_\infty = \bigcap_{n=1}^\infty K_n.$ We know that $K_\infty$ is is closed since it is the intersection of closed sets, and we know it is convex since the intersection of nested convex sets is convex. We have that since 
       \[K_1 \supset K_2 \cdots,\] then since $\bigcap K_n \subset K_n$ for any $n,$ $u\in K_n$ for all $n.$ Let $d_n = |f - u_n|.$ Since $K_{n+1}\subset K_n,$ and each $K_n$ is closed, then
       \[d_n = \inf_{v\in K_n} |f-v| \leq \inf_{v\in K_{n+1}}|f-v|= d_{n+1} \leq \cdots \leq \inf_{v\in K_\infty}|f-v| = d.\] Thus, $d_n$ is monotonic increasing and bounded above, and thus converges to some $d_n \to L.$
       
       
       Apply the parallelogram law to $a = f-u_n$ and $b = f - u_m,$ then 
       \[\left|\frac{f-u_n + f - u_m}{2}\right|^2 + \left|\frac{u_n - u_m}{2}\right|^2 = \frac{1}{4}|(f - u_n) + (f-u_m)|^2 + \frac{1}{4}|u_n - u_m|^2 = \frac{1}{4}(d_n + d_m)^2 +\frac{1}{4}|u_n - u_m|^2\] is equal to 
       \[\frac{1}{2}(|d_n|^2 + |d_m|^2).\] Thus, we have that since $d_n \geq 0$ for any $n,$ then 
       \[|u_n - u_m|^2  = 2|d_n|^2 + 2|d_m|^2 - (d_n + d_m)^2 = |d_n|^2 + 2d_nd_m + |d_m|^2  = (d_m -d_n)^2.\] Thus, since $(d_n)$ converges, then it is Cauchy and thus
       \[|u_n - u_m| < d_m - d_n \to 0,\]
       and so $(u_n)$ is Cauchy in a Hilbert space and thus converges to some $u$. We claim that $u\in K_\infty.$ Since $(u_n)\in K_1$ for all $n$ and $K_1$ is closed, we have that $u\in K_1,$ similarly, since $(u_n)\in K_2$ for all $n$ except for possible $u_1,$ then $u\in K_2.$ Because this holds for all $n,$ then $u\in K_\infty.$ Similarly, we have that $|f-u_1|\leq |f-v|$ for all $v\in K_\infty,$ and $|f-u_2| \leq |f-v|$ for all $v\in K_\infty,$ and taking the limit we see that $|f-u|\leq |f-v|$ for all $v\in K_\infty.$
   \end{solution}
   \item Let \( (K_n) \) be a nondecreasing sequence of nonempty closed convex sets in \( H \).
   Prove that for every \( f \in H \) the sequence \( u_n = P_{K_n} f \) converges (strongly) to a limit and identify the limit.
   \begin{solution}
       Since $K_1 \subset K_2 \cdots,$ then either $\bigcup K_n = H$ or $\bigcup K_n = K_\infty \neq H.$ Suppose the first case, then $f\in H,$ and so $f\in K_n$ for some $n,$ and thus $P_{K_n}f = f$ Since $K_n \subset K_m$ for all $m\geq n,$ then $P_{K_m} = f$ for all $m \geq n,$ and thus $u_n \to f.$ 

       Consider the second case now, then either $f\in K_\infty,$ in which case we revert back to the first case, or $f\notin K_\infty.$ If the latter, then consider that $d_n = P_{K_n}f$ is a decreasing sequence bounded below by $0,$ and thus converges to a limit. By the reasoning above, we have that $(u_n)$ converges to some $u.$ We claim that $u = P_{\overline{K_\infty}}f,$ where $\overline{K_\infty}$ is obviously closed. To see that $\overline{K_\infty}$ convex, it suffices to notice that $\bigcup K_n$ is convex (since the closure of a convex set is convex). Since $u_n \in K_m$ for all $m> n,$ which implies $u \in \overline{K_\infty}.$ Moreover, let $v\in \overline{K_\infty},$ then either $v\in K_n$ for some $n$ or $v\in LP(\bigcup K_n).$ Suppose the former, then
       \[|f - u|\leq |f-u_n| \leq |f-v| \quad \forall v\in K_n.\] Now suppose the latter, then there exist some $(v_n)\in \bigcup K_n$ such that $v_n \to v.$ But we have that for any $n,$
       \[|f-u| \leq |f - v_n| \implies |f-u| \leq \liminf_{n\to \infty}|f-v_n| = |f-v|,\]
        the conclusion from both of these cases is that while we first showed that $u_n \to u,$ now this shows that $u  = P_{\overline{K_\infty}}f$
   \end{solution}
   
   \item Let \( \varphi : H \to \mathbb{R} \) be a continuous function that is bounded from below and let $K_n$ be as above in part (b). Prove that the sequence \( \alpha_n = \inf_{K_n} \varphi \) converges and identify the limit. 
   \begin{solution}
       Consider that $\alpha_n = \inf\{\varphi(x) \; | \; x\in K_n\}.$ Thus, if $n< m,$ then $K_n \subseteq K_m,$ and thus $\alpha_n \geq \alpha_m.$ Thus, $(\alpha_n)$ is non-increasing and bounded below, and so we let 
       \[\alpha_n \to \alpha_\infty,\] and claim that 
       \[\alpha_\infty = \inf_{\overline{\bigcup_{n=1}^\infty K_n}}\varphi.\] Let $u \in \overline{\bigcup K_n},$ then by part (b), we have that $u_n = P_{K_n}u \to u.$ Thus, since $\alpha_n(x) \leq \varphi(x)$ for any $x\in K_n,$ then since $u_n \in K_n,$ we have that $\alpha_n(u_n)\leq \varphi(u_n).$ Since both are continuous, we have that $\alpha_n(u) \leq \varphi(u)$ for all $n,$ and so $\alpha_\infty(u)\leq \varphi(u).$ Because this holds for any $u \in K_\infty,$ then 
       \[\alpha_\infty \leq \inf_{\overline{\bigcup K_n}}\varphi\]
       Since for any $n,$ we have that $K_n \subset \overline{\bigcup}_{i=1}^\infty K_i,$ then $\alpha_n \geq \alpha_\infty,$ and so \[\alpha_\infty \geq \inf_{\overline{\bigcup_{n=1}^\infty K_n}}\varphi.\] 
       
   \end{solution}
\end{enumerate}

\newpage
\section*{Problem 4}
    \begin{problem}
        Let $F: H \to \bbR$ be convex and $C^1.$ Let $K\subset H$ be convex and let $u\in H.$ Show the following are equivalent:
        \begin{enumerate}[label=(\roman*)]
            \item $F(u) \leq F(v), \qquad \forall v\in K$
            \item $(F'(u), v-u) \geq 0 \qquad \forall v\in K.$
        \end{enumerate}
    \end{problem}
    \begin{solution}
        ($a \mapsto b$) Suppose $F(u)\leq F(v),$ then if we let $v' = tu + (1-t)v,$ where $v \in K,$ we have that since $K$ is convex, $v' \in K,$ and thus
        \[F(u)\leq F(v') = F((1-t)u + tv),\] and so 
        \[0 \leq \frac{F(u + t(u-v)) - F(u)}{t} \xrightarrow[]{t\to 0} F_{u-v}'(u) = (F'(u), u-v)\]

        ($b\mapsto a$) We claim that a a continuously differentiable function is convex if and only if its graph lies above all its tangents. That is, since $u\in H,$ then\footnote{MVT} 
        \[F(v) \geq F(u) + F'(u)\cdot(v-u).\] By assumption, we have that 
        \[F(v)- F(u) \geq (F'(u), (v-u)) \geq 0 \implies F(v) \geq F(u)\]
    \end{solution}

\newpage
\section*{Problem 5}
\begin{problem}
    Let \( G \subset H \) be a linear subspace of a Hilbert space \( H \); \( G \) is equipped with the norm of \( H \). Let \( F \) be a Banach space. Let \( S : G \to F \) be a bounded linear operator. Prove that there exists a bounded linear operator \( T : H \to F \) that extends \( S \) and such that


\[
\|T\|_{\mathcal{L}(H,F)} = \|S\|_{\mathcal{L}(G,F)}.\]
\end{problem}
\begin{solution}
    Since $\overline{G}$ is a closed linear subspace, then $P_{\overline{G}}: H \to \overline{G}$ is a continuous function since for any $f_1, f_2 \in H,$ we have that 
    \[\|P_{\overline{G}}f_2 - P_{\overline{G}}f_2\|\leq \|f_1 - f_2\|.\] Define $\overline{S}: \overline{G}\to F$ as an extension of $S$ such that if $v\in \overline{G}\sm G$ and $(v_n)\in G$ with $v_n \to v,$ then 
    \[\overline{S}(v) = \lim_{n\to \infty} S(v_n).\] If $v\in G,$ then let $\overline{S}(v) = S(v).$ To show that $\overline{S}$ is continuous, let $s_1, s_2 \in \overline{G},$ such that $\|s_1 - s_2\|\leq \epsilon,$ then for $n$ large enough, we have that if $s_n^1 \to s_1$ and $s_n^2 \to s_2,$ then by continuity:
    \[\|s_n^1 - s_n^2\|\leq \|s_n^1 - s_1\| + \|s_1 - s_2\| + \|s_2  - s_n^2\|< \delta \implies \|S(s_n^1) - S(s_n^2)\|< \frac{\epsilon}{3}\]
    Since $S(s_n^1) \to \overline{S}(s_1)$ and $S(s_n^2) \to \overline{S}(s_2),$ then 
    \[\|\overline{S}(s_1) - \overline{S}(s_2)\|\leq \|\overline{S}(s_1) - S(s_n^1)\| + \|S(s_n^1) - S(s_n^2)\| + \|S(s_n^2 - \overline{S}(s_n^2))\|< \epsilon,\] and thus $\overline{S}$ is continuous. $\overline{S}$ is clearly linear by the linearity of limits. Since the composition of continuous functions is continuous, then 
    \[T = \overline{S}\circ P_{\overline{G}}: H \to F \] extends $S$ and is a bounded linear operator. 

    Consider that since $T = \overline{S}\circ P_{\overline{G}},$ then since for any $\|x\| = 1,$ we have that $\|P_{\overline{G}}\| = 1$
    \[\|T\| \leq \|\overline{S}\|\|P_{\overline{G}}\| = \|S\|\|P_{\overline{G}}\| \leq \|S\|.\] The other inequality is clear since $T$ is just an extension of $S.$
\end{solution}

\end{problem}
\newpage

\section*{Problem 6}
\begin{problem}
    Let $M, N \subset H$ be two closed linear subspaces. Assume that $(u,v) = 0$ for all $u \in M,$ and $v\in N.$ Show that $M+ N$ is closed.
\end{problem}
\begin{solution}
    Suppose $f\in H$ with $(u_k)\in M + N$ such that $u_k \to f.$ Since $(u_k)\in M+ N,$ then there exist $m_k \in M$ and $n_k \in N$ such that $m_k + n_k  = u_k \to f.$ 
    Thus, for any $k,$ we have that 
    \[\|m_k + n_k\|^2 = \|m_k\|^2 + \|n_k\|^2 + 2(m_k,n_k) = \|m_k\|^2 + \|n_k\|^2\geq \|m_k\|^2 \geq 0.\] That is, 
    \[\|m_k + n_k\| \geq \|m_k\|\geq 0,\] and so $m_k$ is bounded (since $m_k + n_k$ converges and is thus bounded). Thus, there exists some convergent subsequence $m_{k_j} \to m.$ Since $m$ is closed, we have that $m \in M.$ 
    
    Consider now that 
    \[n_{k_j} = u_{k_j} - m_{k_j} \to f - m,\] where again, $f-m \in N$ by the closedness of $N.$ Thus, $f = f-m + m \in M+ N,$ and we are done.
\end{solution}

\newpage
\section*{Problem 7}
\begin{problem}
    Let $C\subset H$ be a nonempty closed convex set and suppose $T: C \to C$ is a non-linear contraction such that for any $u,v \in C,$
    \[|Tu - Tv|\leq |u-v|\]
\end{problem}
\begin{enumerate}
    \item Let $(u_n)\in C$ such that $u_n \rightharpoonup u$ and $(u_n - Tu_n)\to f.$ Prove that 
    \[u - Tu = f\]
    \begin{solution}
    Let $g: C \to H$ such that $g(v) = v - Tv.$ Since $T$ is continuous (since it is a contraction), then $g$ is continuous.  Since $C$ is convex and strongly closed, then $C$ is weakly closed, and so $u_n \rightharpoonup u \in C.$ 
    \begin{align*}
    \|u- u_n\|^2  &\geq \|Tu - Tu_n\|^2\\
    &=\|(g(u_n) - g(u) - u_n + u\|^2\\
    &=\|(g(u_n) - g(u)) + (u - u_n)\|^2\\
    &= \|g(u_n) - g(u)\|^2 + \|u-u_n\|^2  +2(g(u_n) - g(u), (u-u_n))
    \end{align*}
    and so 
    \[0 \geq \|g(u_n) - g(u)\|^2 + 2(g(u_n) - g(u), (u-u_n))\] Consider $\varphi: C \to \bbR$ defined by 
    \[\varphi(v) = (g(v) - g(u),u - v).\] Then $\varphi \in C^*$ since the inner product is bilinear and thus $\varphi(u_n) \to \varphi(u),$ and so 
    \[2 \varphi(u_n) = (g(u_n) - g(u), (u-u_n)) \to (g(u) - g(u), (u-u)) = \varphi(u) = 0.\] Thus, for large enough $n,$ we have that $\|g(u_n) - g(u)\|^2 \leq 0,$ and so $g(u_n) = g(u),$ but this then implies that 
    \[u_n - Tu_n = u - Tu, \qquad n \text{ large}\]
    \end{solution}
\item If $C$ is bounded with $T(C)\subset C,$ then $T$ has a fixed point.
\begin{solution}
    As per the hint, fix $a\in C$ and consider $T_\epsilon: C \to C$
    \[T_\epsilon(u) = (1-\epsilon)Tu + \epsilon a.\] Consider that for any $u,v \in C,$
    \[|T_\epsilon u - T_\epsilon v| = |(1-\epsilon)Tu + \epsilon a - (1-\epsilon)Tv + \epsilon a| = (1-\epsilon) |Tu - Tv| \leq (1-\epsilon)|u-v|.\] Thus, $T_\epsilon$ is a contraction and $C$ is Banach (closed subset of a Hilbert space), and Banach contraction principle tells us that for all $1>\epsilon>0,$ $T_\epsilon$ has a fixed point at some $p_\epsilon.$ Thus, 
    \[p_\epsilon = (1-\epsilon)Tp_\epsilon + \epsilon a\] Consider letting $\epsilon = \frac{1}{n}$ for each $n,$ then \[p_\frac{1}{n}  = (1-\frac{1}{n})Tp_\frac{1}{n} + \frac{1}{n} a = Tp_{\frac{1}{n} } - \frac{1}{n}Tp_\frac{1}{n} + \frac{1}{n}a,\] and so as $n\to \infty,$
    \[p_\frac{1}{n} - Tp_\frac{1}{n} \to 0.\] Since $(p_\frac{1}{n}) \in C$ and $C$ is bounded and closed and convex, then we claim that $C$ is compact in the weak topology. Consider that since $H$ is reflexive, then $B_E$ is weakly compact, and thus there exists some $K$ such that $C \subset KB_E,$ where $KB_E$ is weakly compact. Since $C$ is convex and strongly closed, then it is weakly closed, and thus $C$ is weakly compact. Thus, there exists some subsequence $(p_\frac{1}{n_k}) \rightharpoonup p_0,$ where $p_0 \in C.$ Note that we still have that 
    
    then by part 1, we have that $p_\frac{1}{n_k} - Tp_\frac{1}{n_k} \to 0.$ By part 1, we have that
    \[p_0 - Tp_0  = 0 \implies p_0 = Tp_0.\]
\end{solution}
\end{enumerate}

\newpage
\section*{Problem 8}
\begin{problem}
    Let \( D \subset H \) be a subset such that the linear space spanned by \( D \) is dense in \( H \). 
Let \( (E_n)_{n \geq 1} \) be a sequence of closed subspaces in \( H \) that are mutually orthogonal. 
Assume that 
\begin{align}
\sum_{n=1}^{\infty} |P_{E_n} u|^2 = |u|^2 \quad \forall u \in D    
\end{align}
 
Prove that \( H \) is the Hilbert sum of the \( E_n \)'s.

\end{problem}
\begin{solution}
    Let $E = \text{span}\bigcup E_n.$
    
    We know that for any $v\in H,$ if $v_k = P_{E_k}v,$ then if $S_n = \sum_{k=1}^n P_{E_k},$ we have that 
    \[S_n \to S = P_{\overline{E}}v.\] Thus, by Parseval's identity, we have that
    \begin{align}
    \sum_{k=1}^\infty |v_k|^2= |P_{\overline{E}}v|^2
    \end{align}
    
    Now let $u \in D,$ combining (2) and (3), we see that $|u|^2 = |P_{\overline{E}}u|^2.$ First, we note that if $M$ is a closed subspace of a Hilbert space $H$ and $f \in H,$ then
    \[f = P_Mf + P_{M^\perp}f.\] To see this, consider that $M \cap M^\perp = \{0\}$ and $M + M^\perp = H,$ then 
    \[f = P_Mf + (f - P_Mf) = P_Mf + P_{M^\perp}f.\] Thus, since $u\in D\subset H$ and $\overline{E}$ is a closed subspace, then by orthogonality
    \[u = P_{\overline{E}}u + P_{\overline{E}^\perp}u \implies |u|^2 = |P_{\overline{E}}u|^2 + |P_{\overline{E}}u|^2 + 2|(P_{\overline{E}}u, P_{\overline{E}^\perp} u)| = |P_{\overline{E}}u|^2 + |P_{\overline{E}^\perp}u|^2\] From our conclusion above, we see that 
    \[0 = |P_{\overline{E}^\perp}u|^2 \implies P_{\overline{E}^\perp}u = 0.\] Because this holds for any $u \in D,$ and $D$ is dense in $H,$ then $\overline{E}^\perp = \{0\},$ and so $\overline{E} = H.$
\end{solution}


\newpage
\section*{Problem 9}
\begin{problem}
\begin{enumerate}
    \item Suppose $H$ is separable. Let $V\subset H$ be a linear subspace that is dense in $H,$ then $V$ contains an orthonormal basis of $H.$
    \begin{solution}
        Since $H$ is separable and $V\subset H,$ then $V$ is separable. Let $(v_n)$ be a countably dense subset of $V,$ and let $F_k = \text{span}\{v_1, \dots, v_k\}.$ $F_k$ is finite and $\bigcup F_k$ is dense in $V.$ Since $F_1$ is finite, then for any $x\in F_1,$ $x = x_1 e_1,$ where $\|e_1\| = 1.$ Thus, take $e_1.$ If $F_2 \neq F_1,$ then let 
        $u_2 \in F_2 \setminus F_1,$ and define 
        \[e_2 = \frac{u_2 - (u_2, e_1)e_1}{\|u_2 - (u_2, e_1)e_1\|},\] and so 
        \[(e_1, e_2) = (e_1, \frac{u_2 - (u_2, e_1)e_1}{\|u_2 - (u_2, e_1)e_1\|}) = \frac{1}{\|u_2 - (u_2, e_1)e_1\|}[(e_1, u_2) - (u_2, e_1)(e_1, e_1)] = 0.\] Continue this process, then $(e_n)$ is a an orthonormal basis of $V,$ and since $\overline{V} = H,$ then
        \[\overline{\text{span}\{e_1, \dots\}} = V \implies \overline{\text{span}\{e_1, \dots\}} = H,\] and so $(e_n)$ is an orthonormal basis of $H.$
    \end{solution}
    \item Let $(e_n)$ be an orthonormal sequence in $H$ such that $(e_i, e_j)= \delta_{ij}.$ Prove there exists some orthonormal basis of $H$ that contains $\bigcup_{i=1}^\infty e_n.$
    \begin{solution}
        Consider $E_k = \text{span}\{e_1, \dots, e_k\}.$ Define $F_k$ as above, then if
        \[\overline{E} = \overline{\bigcup_{k=1}^\infty E_k} \neq \bigcup F_k = F,\] we let $w_k \in F\setminus \overline{E},$ then let $k$ be such that $w\in F_k$ but $w\notin F_{k-1}.$
        
        Define $e_1'$ such that \[e_1' = \frac{w_k - \sum_{j=1}^{k-1}(w_k, e_j)e_j}{\|w_k - \sum_{j=1}^{k-1}(w_k, e_j)e_j\|}.\] Then for any $e_i \in E,$ we have that by the same reasoning as the first problem, $(e_i, e_1')=0.$ Then redefine $E =(e_n)\cup e_1'$ which is an orthonormal sequence in $H.$ Continue this procedure until $\overline{E} = F,$ and then $E$ is an orthonormal basis of $H$ containing $\bigcup e_n.$
    \end{solution}
\end{enumerate}
    
    
\end{problem}









\end{document}