\documentclass[10pt, oneside]{article} 
\usepackage{amsmath, amsthm, amssymb, calrsfs, wasysym, verbatim, bbm, color, graphics, geometry, esint, float}


\geometry{tmargin=.75in, bmargin=.75in, lmargin=.75in, rmargin = .75in}  

\newcommand{\bbR}{\mathbb{R}}
\newcommand{\bbC}{\mathbb{C}}
\newcommand{\bbZ}{\mathbb{Z}}
\newcommand{\bbP}{\mathbb{P}}
\newcommand{\bbN}{\mathbb{N}}
\newcommand{\bbQ}{\mathbb{Q}}
\newcommand{\Cdot}{\boldsymbol{\cdot}}
\newcommand{\scA}{\mathscr{A}}
\newcommand{\curl}{\text{curl}}

\theoremstyle{definition}
\newtheorem{exmp}{Example}[section]
\newtheorem{thm}{Theorem}
\newtheorem{defn}{Definition}
\newtheorem{prop}{Proposition}
\newtheorem{conv}{Convention}
\newtheorem{rem}{Remark}
\newtheorem{lem}{Lemma}
\newtheorem{cor}{Corollary}
\input{paolo-pset.tex}



\title{UChicago Markov Chains, Martingales, and Brownian Motion Analysis Notes: 23500}
\author{Notes by Agustín Esteva, Lectures by Stephen Yearwood, Books by }
\date{Academic Year 2024-2025}

\begin{document}

\maketitle
\tableofcontents

\vspace{.25in}


\newpage
\section{Lectures}

\subsection{Monday, Mar 24: Markov Processes: Basic Definitions and Examples}
He is British, I am fucked.
\begin{defn}
    A \textbf{stochastic process}  is a collection of random variables $\{X_t\}_{t \in T}$ indexed by time, where each $X_t$ takes values in $S$
\end{defn}
We call $S$ our \textit{state space}. In discrete time, it should be obvious that $T = \bbN_0^*$ (scale the naturals by a constant) and $T = [0, \bbR)$ in continuous time. We say that $S$ is a discrete space if it is countable and continuous if it is $\bbR^d.$
\begin{rem}
    In order to characterize the distribution of $\{X_n\},$ we must specify $\mathbb{P}\{X_0 = S_0, \dots, X_n = S_n\}$ for all $n \in \bbN$ and for all $S_0, \dots, S_n \in S.$ It is much easier work with conditional probability. We will see that with Markov Chains, the Markov Property guarantees that we only need to worry about the distribution of $X_{n-1}$ to figure out $X_n.$
\end{rem}
\begin{defn}
    If $E,F$ are events with $\mathbb{P}\{F\} >0,$ then the \textbf{conditional probability} of $E$ given $F$ is 
    \[\mathbb{P}(E | F) = \frac{\mathbb{P}(E \cap F)}{\mathbb{P}(F)}\]
\end{defn}

\begin{defn}
    A stochastic process is called a \textbf{Markov Chain} if for all $n \in \bbN,$ and for all $S_0, \dots, S_n \in S,$ we have that 
    \[\bbP(X_n = S_n |  X_0= S_0, \dots, X_{n-1}= S_{n-1}) = \bbP(X_n = S_n | X_{n-1} =S_{n-1})\]
\end{defn}
\begin{defn}
    A Markov Chain $\{X_n\}$ is \textbf{time-homogeneous} if for all $n\in N,$ for all $x,y \in S,$ 
    \[\bbP(X_n = y | X_{n-1} = x) = \bbP(X_1 = y | X_0 = x)\]
\end{defn}
Thus, it does not matter when you get to the states, and so we can just specify the distribution of $X_0$ and the transition probability;
\[p(x,y) := \bbP(X_1 = y | X_0 = x)\]
and then scale to find the rest.
\begin{exmp}
    Let $S = \{0,1\}.$ A Markov chain taking values in $S$ specified by $p = p(0,1)$ and $q = p(1,0).$ It is obvious that $p(0,0) = 1-p$ and $q(1,1) = 1-q.$
\end{exmp}
\begin{exmp}
    Let $S = \bbZ.$ Let $\{X_n\}_{n\geq 0}$ be defined by $X_0 = X_1 = X_2 =0$ and for $n\geq 3,$ then 
    \[X_n = \begin{cases}
        X_{n-1} + 1,\\
        X_{n-1} - 1,\\
        X_{n-3}
    \end{cases}\]
    each with with probability $1/3.$ This is NOT a Markov Process since $X_n$ can depend on $X_{n-3}.$
\end{exmp}

\begin{defn}
    The $n-$step transition probabilities 
    \[p^n(x,y) = \bbP(X_n = y | X_0 = x).\]
\end{defn}
That is, if we start at $x,$ what is the probability that the $n$th step is $y$?
\begin{prop}
    For all $m,n \in \bbN$ and for all $x,y \in S,$ then 
    \[p^{n + m}(x,y) = \sum_{z\in S}p^n(x,z)p^m(z,y)\]
\end{prop}
\begin{proof}
    Induction?
\end{proof}
\begin{defn}
    The \textbf{transition matrix} of a Markov Chain the the $N \times N$ matrix $P$ whose $ij$ entry is $p(i,j).$
\end{defn}
\begin{prop}
    For each $n \in \bbN,$ $P^n$ is the matrix whose $ij$ entry is $p^n(i,j).$
\end{prop}
\begin{proof}
    We prove by induction. It is trivial for $n = 1.$ Assume it is true for any general $n-1.$ Thus, the $ij$ entry of $P^n = P^{n-1}P$ is \[\sum_{k=1}^n P^{n-1}_{ik}P_{kj} = \sum_{k=1}^n p^{n-1}(i,k)p(k,j) = p^n(i,j),\] where the last equality comes from Proposition 1.
\end{proof}

\newpage
\subsection{Wednesday, Mar 26: Recurrence and Transience for Finite State Space}
We illustrate the stuff from last class with a simple example:
\begin{exmp}
    Consider the two state Markov chain with $S = \{0,1\}$ and 
    \[p(1,0) = \frac{1}{3}, \quad p(1,0) = \frac{1}{2},\] then 
    \[P = \begin{pmatrix}
        p(0,0) & p(0,1)\\
        p(1,0) & p(1,1) 
    \end{pmatrix} = \begin{pmatrix}
        \frac{2}{3} & \frac{1}{3}\\
        \frac{1}{2} & \frac{1}{2}
    \end{pmatrix},\] then 
    \[\bbP\{X_3 = 0 | X_0 = 0\} = p^3(0,0) = \frac{65}{108}\]
\end{exmp}
For the following, we consider a Markov chain $\{X_n\}$ with state space $S.$
\begin{defn}
    Two states $x,y \in S$ \textbf{communicate} if there exists $m,n >0$ such that $p^n(x,y) >0$ and $p^m(y,x) > 0 .$ We write $x \leftrightarrow y.$
\end{defn}
\begin{rem}
    Communication is an equivalence relation, and so we can partition $S$ into a disjoint union of communication classes by modding out the communication classes.
\end{rem}
\begin{defn}
    A communication class $c$ is \textbf{recurrent} if $p(x,y) = 0$ for all $x\in C$ and $y \in S\setminus\{C\}.$ Otherwise, we say that the communication class is \textbf{transient}.
\end{defn}

\begin{defn}
    A Markov chain is \textbf{irreducible} if there is only one communication class.
\end{defn}

\begin{exmp}
    Consider a Markov chain with $S  = \{1,2,3,4,5\}$ and 
    \[P = \begin{pmatrix}
        \frac{1}{5} & \frac{1}{5} & 0 & 0 & \frac{3}{5}\\
        0 & 0 & 0 & 0 & 0 \\
        0 & 0 & \frac{1}{2} & \frac{1}{2} & 0\\
        0 & 0 & \frac{1}{4} & \frac{3}{4} & 0\\
        \frac{1}{2} & \frac{1}{4} & \frac{1}{4} & 0 & 0
    \end{pmatrix}\] then 
    \[C_1 = \{1,2,5\}, \qquad C_2 = \{3,4\}\] and $C_1$ is transient and $C_2$ is recurrent
\end{exmp}

\begin{exmp} (Gambler's ruin)
    Consider the random walk on $S = \{0,1,2, \dots, N\}$ with absorbing boundary, and transition probability
    \[p(x,x+1) = p(x, x-1) = \frac{1}{2}, \quad x\in \{\1,2,\dots, N-1\}\] and 
    \[p(0,0) = p(N,N) = 1\]
\end{exmp}

\begin{prop}
    Suppose $S$ is finite. If $C$ is a recurrent communication class, then if $\{X_n\}$ starts in $C,$ with probability $1,$ $\{X_n\}$ visits every state in $C$ infinitely often. That is, for each $x,y \in C,$ $\bbP\{X_n = y \text{ i.o.} \; | \; X_0 = x\} = 1.$
\end{prop}
\begin{proof}
    Since $C$ is a communication class, then for every $z \in C,$ there exists some $n_z \in \bbN$ such that $p^{n_z}(z,y) >0.$ Let $n = \max\{n_z\; \forall z \in C\},$ and $q = \min\{p^{n_z}(z,y)\; | \; z\in C\}.$ Note that this quantities necessarily exist because $S$ is finite. Let $E_k = \{X_i = y, \; | \; i \in \{(k-1)n + 1, (k-1)n + 2\, \dots (k-1)nk\}\}.$ Then for states $s_0, s_1, \dots, s_{nk} \in S,$ we have that 
    \[\bbP\{E_{k+1} \: | \: X_0 = s_0, X_1 = s_1, \dots, X_{nk} = S_{nk}\} = \bbP\{E_1 \; | \; X_0 = s_{nk}\} \geq q.\] For $M, N \in \bbN$ with $M > N,$ we have that 
    \begin{align*}
      \bbP\{E_k \text{ does not occur for any any } k \in \{N, N + 1, \dots, M\}\} &= \bbP\{\bigcap_{k=N}^M E_k^c\}  \\
      &= \bbP\{E_M^c \:  | \: \bigcap_{k = N}^{M-1}E_k\}\bbP\{\bigcap_{K=N}^{M-1}E_k\}\\
      &\leq (1-q)\bbP\{\bigcap_{K=N}^{M-1}E_k\} \\
      &\leq\dots\leq (1-q)^{M-N} \to 0
    \end{align*}
\end{proof}
\begin{prop}
    Suppose $S$ is finite. If $C$ is a transient communication class, then w.p. 1, $X$ eventually leaves $C$ and never returns.
\end{prop}

\newpage
\subsection{Friday, Mar 28: The Strong Markov Property}

\begin{defn}
    A random time $\tau \in \bbN_0 \cup \{\infty\}$ is called a \textbf{stopping time} if, for all $n \in \bbN,$ the event $\{\tau = n\}$ is determined by $X_0, X_1, \dots, X_n.$ 
\end{defn}
\begin{exmp}
\begin{enumerate}
    \item (The Hitting Time) \[\tau = \min\{n \mid X_n = x\}, \quad x\in S.\]
    In the Gambler's ruin model, where the gambler starts with $k-$dollars and gambles all the way to $N$ or $0$ dollars, the hitting time is the first the the gambler reaches $\$1$.
    \item $\tau = k^\text{th}$ time for which $X_n \in A,$ $k\in \bbN,$ $A \subset S.$
    \item $\tau = \min\{\tau_1, \tau_2\},$ where $\tau_1$ and $\tau_2$ are stopping times. 
    \item Let $N \in \bbN,$ $x\in S,$ $\tau$ be the last time $n \leq N$ for which $X_n = x,$ and $\tau  = 0$ if no such time exists. $\tau$ is NOT a stopping time because it  depends on stuff in the future.
\end{enumerate}
\end{exmp}

\begin{thm}
    (Strong Markov Property) Let $\tau$ be a stopping time for $\{X_n\}.$ Let $n \geq 0,$ $m\geq 1,$ $x_n \dots, x_n \in S$ such that 
    \[\bbP\{X_0 = x_0, \dots, X_\tau = X_n\} >0,\] and 
    let $y_1, \dots, y_m \in S.$ Then 
    \[\bbP\{X_{\tau + 1} = y, \dots, X_{\tau + m} = y_m\mid X_0 = x_0, \dots, X-\tau = x_n\} =\bbP\{X_{\tau + 1} = y, \dots, X_{\tau + m} = y_m\mid X_0 = x_n\}\]
\end{thm}
\begin{proof}
    The event $\{X_0 = x_0, \dots, X_\tau = x_n\}$ is the same as the event $\{X_0 = x_0, \dots, X_n = x_n\}$ and $\{\tau = n\}.$ Since $\tau$ is a stopping time determined where the event $\{\tau = n\}$ by $\{X_0, \dots, X_n\}.$ Thus, we get that the events $\{X_0 = x_0, \dots, X_\tau = x_n\} = \{X_0 = x_0, \dots, X_n = n\}\cap \{\tau = n\},$ then 
    \begin{align*}
        \bbP\{X_{\tau + 1} = y_1, \dots, X_{\tau  + m} = y_m \mid X_0 = x_0, \dots, X_\tau = x_n\} &= \bbP\{X_{n + 1} = y_1, \dots, X_{n  + m} = y_m \mid X_0 = x_0, \dots, X_n = x_n\}\\
        &= \bbP\{X_{n + 1} = y_1, \dots, X_{n  + m} = y_m \mid X_n = x_n\}
    \end{align*} and conclude with the regular markov property and $n-$step invariance.
\end{proof}

\begin{exmp}
    Let $x\in S$ with $S$ finite and let $\tau = \min\{n \geq 0 \mid X_n = x\}.$ Assume that $\bbP\{\tau < \infty\} = 1.$ For any $x_0, \dotss, x_n \in S$ such that 
    $\bbP\{X_0 = x_0, \dots, X_\tau = x_n\} >0,$ we have $x_n = x,$ and so the conditional distribution of $\{X_{\tau + j}\}_{j \geq 0}$ given everything up to $\tau$ is the same as the original chain starting at $x.$ 

    Thus, the Strong Markov property essentially guarantees that the Markov chain resets after hitting $\tau$!
\end{exmp}
    
\begin{defn}
    Let $X_n$ be a Markov chain on a countable state space $S.$ For any $x\in S,$ the \textbf{period} of $x$ is the greatest common divisor of $J_x = \{n \geq 1 \mid p^n(x,x) >0\}.$
\end{defn}
\begin{exmp}
    If $p(x,x) >0,$ then $1 \in J_x,$ and so the period of $x$ is $1.$
\end{exmp}

\begin{thm}
    If $x\leftrightarrow y,$ the periods of $x$ and $y,$ denoted by $d_x$ and $d_y$ respectively, are the same. 
\end{thm}
\begin{proof}
    Choose $n, m$ such that $p^n(x,y) >0$ and $p^m(y,x) >0.$ Then $p^{n +m}(x,x) >0$ and $p^{m + n}(y,y) >0.$ Thus, $ n + m \in J_x \cap J_y.$ Assume that $d_x < d_y.$ Then there exists some $k \in K_x$ not divisible by $d_y.$ Then $n + m + k \in J_y,$ and so $d_y$ divides $n + m + k$ and $n + m,$ and so $d_y$ divides $k,$ which is a contradiction.    
\end{proof}

\begin{defn}
    A Markov chain is aperiodic if every state has period $1.$
\end{defn}

\begin{exmp}
    A knight on an $8\times 8$ chessboard selects one of the next legal moves with equal probability, independently of the past. 
    \[S = \{(x,y) \in \bbR^2 \mid x \in \{1, \dots, 8\}, y \in \{1, \dots, 8\}\}.\]  The period is $2.$ 
\end{exmp}






\end{document}