\documentclass[10pt, oneside]{article} 
\usepackage{amsmath, amsthm, amssymb, calrsfs, wasysym, verbatim, bbm, color, graphics, geometry, esint, float}


\geometry{tmargin=.75in, bmargin=.75in, lmargin=.75in, rmargin = .75in}  

\newcommand{\bbR}{\mathbb{R}}
\newcommand{\bbC}{\mathbb{C}}
\newcommand{\bbZ}{\mathbb{Z}}
\newcommand{\bbP}{\mathbb{P}}
\newcommand{\bbN}{\mathbb{N}}
\newcommand{\bbE}{\mathbb{E}}
\newcommand{\bbQ}{\mathbb{Q}}
\newcommand{\Cdot}{\boldsymbol{\cdot}}
\newcommand{\scA}{\mathscr{A}}
\newcommand{\curl}{\text{curl}}

\theoremstyle{definition}
\newtheorem{exmp}{Example}[section]
\newtheorem{thm}{Theorem}
\newtheorem{defn}{Definition}
\newtheorem{prop}{Proposition}
\newtheorem{conv}{Convention}
\newtheorem{rem}{Remark}
\newtheorem{lem}{Lemma}
\newtheorem{cor}{Corollary}
% Copyright 2021 Paolo Adajar (padajar.com, paoloadajar@mit.edu)
% 
% Permission is hereby granted, free of charge, to any person obtaining a copy of this software and associated documentation files (the "Software"), to deal in the Software without restriction, including without limitation the rights to use, copy, modify, merge, publish, distribute, sublicense, and/or sell copies of the Software, and to permit persons to whom the Software is furnished to do so, subject to the following conditions:
%
% The above copyright notice and this permission notice shall be included in all copies or substantial portions of the Software.
% 
% THE SOFTWARE IS PROVIDED "AS IS", WITHOUT WARRANTY OF ANY KIND, EXPRESS OR IMPLIED, INCLUDING BUT NOT LIMITED TO THE WARRANTIES OF MERCHANTABILITY, FITNESS FOR A PARTICULAR PURPOSE AND NONINFRINGEMENT. IN NO EVENT SHALL THE AUTHORS OR COPYRIGHT HOLDERS BE LIABLE FOR ANY CLAIM, DAMAGES OR OTHER LIABILITY, WHETHER IN AN ACTION OF CONTRACT, TORT OR OTHERWISE, ARISING FROM, OUT OF OR IN CONNECTION WITH THE SOFTWARE OR THE USE OR OTHER DEALINGS IN THE SOFTWARE.

\usepackage{fullpage}
\usepackage{enumitem}
\usepackage{amsfonts, amssymb, amsmath,amsthm}
\usepackage{mathtools}
\usepackage[pdftex, pdfauthor={\name}, pdftitle={\classnum~\assignment}]{hyperref}
\usepackage[dvipsnames]{xcolor}
\usepackage{bbm}
\usepackage{graphicx}
\usepackage{mathrsfs}
\usepackage{pdfpages}
\usepackage{tabularx}
\usepackage{pdflscape}
\usepackage{makecell}
\usepackage{booktabs}
\usepackage{natbib}
\usepackage{caption}
\usepackage{subcaption}
\usepackage{physics}
\usepackage[many]{tcolorbox}
\usepackage{version}
\usepackage{ifthen}
\usepackage{cancel}
\usepackage{listings}
\usepackage{courier}

\usepackage{tikz}
\usepackage{istgame}

\hypersetup{
	colorlinks=true,
	linkcolor=blue,
	filecolor=magenta,
	urlcolor=blue,
}

\setlength{\parindent}{0mm}
\setlength{\parskip}{2mm}

\setlist[enumerate]{label=({\alph*})}
\setlist[enumerate, 2]{label=({\roman*})}

\allowdisplaybreaks[1]

\newcommand{\psetheader}{
	\ifthenelse{\isundefined{\collaborators}}{
		\begin{center}
			{\setlength{\parindent}{0cm} \setlength{\parskip}{0mm}
				
				{\textbf{\classnum~\semester:~\assignment} \hfill \name}
				
				\subject \hfill \href{mailto:\email}{\tt \email}
				
				Instructor(s):~\instructors \hfill Due Date:~\duedate	
				
				\hrulefill}
		\end{center}
	}{
		\begin{center}
			{\setlength{\parindent}{0cm} \setlength{\parskip}{0mm}
				
				{\textbf{\classnum~\semester:~\assignment} \hfill \name\footnote{Collaborator(s): \collaborators}}
				
				\subject \hfill \href{mailto:\email}{\tt \email}
				
				Instructor(s):~\instructors \hfill Due Date:~\duedate	
				
				\hrulefill}
		\end{center}
	}
}

\renewcommand{\thepage}{\classnum~\assignment \hfill \arabic{page}}

\makeatletter
\def\points{\@ifnextchar[{\@with}{\@without}}
\def\@with[#1]#2{{\ifthenelse{\equal{#2}{1}}{{[1 point, #1]}}{{[#2 points, #1]}}}}
\def\@without#1{\ifthenelse{\equal{#1}{1}}{{[1 point]}}{{[#1 points]}}}
\makeatother

\newtheoremstyle{theorem-custom}%
{}{}%
{}{}%
{\itshape}{.}%
{ }%
{\thmname{#1}\thmnumber{ #2}\thmnote{ (#3)}}

\theoremstyle{theorem-custom}

\newtheorem{theorem}{Theorem}
\newtheorem{lemma}[theorem]{Lemma}
\newtheorem{example}[theorem]{Example}

\newenvironment{problem}[1]{\color{black} #1}{}

\newenvironment{solution}{%
	\leavevmode\begin{tcolorbox}[breakable, colback=green!5!white,colframe=green!75!black, enhanced jigsaw] \proof[\scshape Solution:] \setlength{\parskip}{2mm}%
	}{\renewcommand{\qedsymbol}{$\blacksquare$} \endproof \end{tcolorbox}}

\newenvironment{reflection}{\begin{tcolorbox}[breakable, colback=black!8!white,colframe=black!60!white, enhanced jigsaw, parbox = false]\textsc{Reflections:}}{\end{tcolorbox}}

\newcommand{\qedh}{\renewcommand{\qedsymbol}{$\blacksquare$}\qedhere}

\definecolor{mygreen}{rgb}{0,0.6,0}
\definecolor{mygray}{rgb}{0.5,0.5,0.5}
\definecolor{mymauve}{rgb}{0.58,0,0.82}

% from https://github.com/satejsoman/stata-lstlisting
% language definition
\lstdefinelanguage{Stata}{
	% System commands
	morekeywords=[1]{regress, reg, summarize, sum, display, di, generate, gen, bysort, use, import, delimited, predict, quietly, probit, margins, test},
	% Reserved words
	morekeywords=[2]{aggregate, array, boolean, break, byte, case, catch, class, colvector, complex, const, continue, default, delegate, delete, do, double, else, eltypedef, end, enum, explicit, export, external, float, for, friend, function, global, goto, if, inline, int, local, long, mata, matrix, namespace, new, numeric, NULL, operator, orgtypedef, pointer, polymorphic, pragma, private, protected, public, quad, real, return, rowvector, scalar, short, signed, static, strL, string, struct, super, switch, template, this, throw, transmorphic, try, typedef, typename, union, unsigned, using, vector, version, virtual, void, volatile, while,},
	% Keywords
	morekeywords=[3]{forvalues, foreach, set},
	% Date and time functions
	morekeywords=[4]{bofd, Cdhms, Chms, Clock, clock, Cmdyhms, Cofc, cofC, Cofd, cofd, daily, date, day, dhms, dofb, dofC, dofc, dofh, dofm, dofq, dofw, dofy, dow, doy, halfyear, halfyearly, hh, hhC, hms, hofd, hours, mdy, mdyhms, minutes, mm, mmC, mofd, month, monthly, msofhours, msofminutes, msofseconds, qofd, quarter, quarterly, seconds, ss, ssC, tC, tc, td, th, tm, tq, tw, week, weekly, wofd, year, yearly, yh, ym, yofd, yq, yw,},
	% Mathematical functions
	morekeywords=[5]{abs, ceil, cloglog, comb, digamma, exp, expm1, floor, int, invcloglog, invlogit, ln, ln1m, ln, ln1p, ln, lnfactorial, lngamma, log, log10, log1m, log1p, logit, max, min, mod, reldif, round, sign, sqrt, sum, trigamma, trunc,},
	% Matrix functions
	morekeywords=[6]{cholesky, coleqnumb, colnfreeparms, colnumb, colsof, corr, det, diag, diag0cnt, el, get, hadamard, I, inv, invsym, issymmetric, J, matmissing, matuniform, mreldif, nullmat, roweqnumb, rownfreeparms, rownumb, rowsof, sweep, trace, vec, vecdiag, },
	% Programming functions
	morekeywords=[7]{autocode, byteorder, c, _caller, chop, abs, clip, cond, e, fileexists, fileread, filereaderror, filewrite, float, fmtwidth, has_eprop, inlist, inrange, irecode, matrix, maxbyte, maxdouble, maxfloat, maxint, maxlong, mi, minbyte, mindouble, minfloat, minint, minlong, missing, r, recode, replay, return, s, scalar, smallestdouble,},
	% Random-number functions
	morekeywords=[8]{rbeta, rbinomial, rcauchy, rchi2, rexponential, rgamma, rhypergeometric, rigaussian, rlaplace, rlogistic, rnbinomial, rnormal, rpoisson, rt, runiform, runiformint, rweibull, rweibullph,},
	% Selecting time-span functions
	morekeywords=[9]{tin, twithin,},
	% Statistical functions
	morekeywords=[10]{betaden, binomial, binomialp, binomialtail, binormal, cauchy, cauchyden, cauchytail, chi2, chi2den, chi2tail, dgammapda, dgammapdada, dgammapdadx, dgammapdx, dgammapdxdx, dunnettprob, exponential, exponentialden, exponentialtail, F, Fden, Ftail, gammaden, gammap, gammaptail, hypergeometric, hypergeometricp, ibeta, ibetatail, igaussian, igaussianden, igaussiantail, invbinomial, invbinomialtail, invcauchy, invcauchytail, invchi2, invchi2tail, invdunnettprob, invexponential, invexponentialtail, invF, invFtail, invgammap, invgammaptail, invibeta, invibetatail, invigaussian, invigaussiantail, invlaplace, invlaplacetail, invlogistic, invlogistictail, invnbinomial, invnbinomialtail, invnchi2, invnF, invnFtail, invnibeta, invnormal, invnt, invnttail, invpoisson, invpoissontail, invt, invttail, invtukeyprob, invweibull, invweibullph, invweibullphtail, invweibulltail, laplace, laplaceden, laplacetail, lncauchyden, lnigammaden, lnigaussianden, lniwishartden, lnlaplaceden, lnmvnormalden, lnnormal, lnnormalden, lnwishartden, logistic, logisticden, logistictail, nbetaden, nbinomial, nbinomialp, nbinomialtail, nchi2, nchi2den, nchi2tail, nF, nFden, nFtail, nibeta, normal, normalden, npnchi2, npnF, npnt, nt, ntden, nttail, poisson, poissonp, poissontail, t, tden, ttail, tukeyprob, weibull, weibullden, weibullph, weibullphden, weibullphtail, weibulltail,},
	% String functions 
	morekeywords=[11]{abbrev, char, collatorlocale, collatorversion, indexnot, plural, plural, real, regexm, regexr, regexs, soundex, soundex_nara, strcat, strdup, string, strofreal, string, strofreal, stritrim, strlen, strlower, strltrim, strmatch, strofreal, strofreal, strpos, strproper, strreverse, strrpos, strrtrim, strtoname, strtrim, strupper, subinstr, subinword, substr, tobytes, uchar, udstrlen, udsubstr, uisdigit, uisletter, ustrcompare, ustrcompareex, ustrfix, ustrfrom, ustrinvalidcnt, ustrleft, ustrlen, ustrlower, ustrltrim, ustrnormalize, ustrpos, ustrregexm, ustrregexra, ustrregexrf, ustrregexs, ustrreverse, ustrright, ustrrpos, ustrrtrim, ustrsortkey, ustrsortkeyex, ustrtitle, ustrto, ustrtohex, ustrtoname, ustrtrim, ustrunescape, ustrupper, ustrword, ustrwordcount, usubinstr, usubstr, word, wordbreaklocale, worcount,},
	% Trig functions
	morekeywords=[12]{acos, acosh, asin, asinh, atan, atanh, cos, cosh, sin, sinh, tan, tanh,},
	morecomment=[l]{//},
	% morecomment=[l]{*},  // `*` maybe used as multiply operator. So use `//` as line comment.
	morecomment=[s]{/*}{*/},
	% The following is used by macros, like `lags'.
	morestring=[b]{`}{'},
	% morestring=[d]{'},
	morestring=[b]",
	morestring=[d]",
	% morestring=[d]{\\`},
	% morestring=[b]{'},
	sensitive=true,
}

\lstset{ 
	backgroundcolor=\color{white},   % choose the background color; you must add \usepackage{color} or \usepackage{xcolor}; should come as last argument
	basicstyle=\footnotesize\ttfamily,        % the size of the fonts that are used for the code
	breakatwhitespace=false,         % sets if automatic breaks should only happen at whitespace
	breaklines=true,                 % sets automatic line breaking
	captionpos=b,                    % sets the caption-position to bottom
	commentstyle=\color{mygreen},    % comment style
	deletekeywords={...},            % if you want to delete keywords from the given language
	escapeinside={\%*}{*)},          % if you want to add LaTeX within your code
	extendedchars=true,              % lets you use non-ASCII characters; for 8-bits encodings only, does not work with UTF-8
	firstnumber=0,                % start line enumeration with line 1000
	frame=single,	                   % adds a frame around the code
	keepspaces=true,                 % keeps spaces in text, useful for keeping indentation of code (possibly needs columns=flexible)
	keywordstyle=\color{blue},       % keyword style
	language=Octave,                 % the language of the code
	morekeywords={*,...},            % if you want to add more keywords to the set
	numbers=left,                    % where to put the line-numbers; possible values are (none, left, right)
	numbersep=5pt,                   % how far the line-numbers are from the code
	numberstyle=\tiny\color{mygray}, % the style that is used for the line-numbers
	rulecolor=\color{black},         % if not set, the frame-color may be changed on line-breaks within not-black text (e.g. comments (green here))
	showspaces=false,                % show spaces everywhere adding particular underscores; it overrides 'showstringspaces'
	showstringspaces=false,          % underline spaces within strings only
	showtabs=false,                  % show tabs within strings adding particular underscores
	stepnumber=2,                    % the step between two line-numbers. If it's 1, each line will be numbered
	stringstyle=\color{mymauve},     % string literal style
	tabsize=2,	                   % sets default tabsize to 2 spaces
%	title=\lstname,                   % show the filename of files included with \lstinputlisting; also try caption instead of title
	xleftmargin=0.25cm
}



\title{UChicago Markov Chains, Martingales, and Brownian Motion Analysis Notes: 23500}
\author{Notes by Agustín Esteva, Lectures by Stephen Yearwood, Books by }
\date{Academic Year 2024-2025}

\begin{document}

\maketitle
\tableofcontents

\vspace{.25in}


\newpage
\section{Lectures}

\subsection{Monday, Mar 24: Markov Processes: Basic Definitions and Examples}
He is from Trinidad and Tobago. 
\begin{defn}
    A \textbf{stochastic process}  is a collection of random variables $\{X_t\}_{t \in T}$ indexed by time, where each $X_t$ takes values in $S$
\end{defn}
We call $S$ our \textit{state space}. In discrete time, it should be obvious that $T = \bbN_0^*$ (where $\bbN_0^*$ is when on goscale the naturals by a constant) and $T = [0, \bbR)$ in continuous time. We say that $S$ is a discrete space if it is countable and continuous if it is $\bbR^d.$
\begin{rem}
    In order to characterize the distribution of $\{X_n\},$ we must specify $\mathbb{P}\{X_0 = S_0, \dots, X_n = S_n\}$ for all $n \in \bbN$ and for all $S_0, \dots, S_n \in S.$ It is much easier work with conditional probability. We will see that with Markov Chains, the Markov Property guarantees that we only need to worry about the distribution of $X_{n-1}$ to figure out $X_n.$
\end{rem}
\begin{defn}
    If $E,F$ are events with $\mathbb{P}\{F\} >0,$ then the \textbf{conditional probability} of $E$ given $F$ is 
    \[\mathbb{P}(E | F) = \frac{\mathbb{P}(E \cap F)}{\mathbb{P}(F)}\]
\end{defn}
\begin{prop}
    (Law of Total Probability) Recall that if $(B_n) \in S$ is a sequence of mutually exclusive and exhaustive events, then 
    \[P(A) = \sum_{n=1}^\infty P(A \cap B_n) = \sum_{n=1}^\infty P(A \mid B_n)P(B_n)\]
\end{prop}

\begin{defn}
    A stochastic process is called a \textbf{Markov Chain} if for all $n \in \bbN,$ and for all $S_0, \dots, S_n \in S,$ we have that 
    \[\bbP(X_n = S_n |  X_0= S_0, \dots, X_{n-1}= S_{n-1}) = \bbP(X_n = S_n | X_{n-1} =S_{n-1})\]
\end{defn}
\begin{defn}
    A Markov Chain $\{X_n\}$ is \textbf{time-homogeneous} if for all $n\in N,$ for all $x,y \in S,$ 
    \[\bbP(X_n = y | X_{n-1} = x) = \bbP(X_1 = y | X_0 = x)\]
\end{defn}
Thus, it does not matter when you get to the states, and so we can just specify the distribution of $X_0$ and the transition probability;
\[p(x,y) := \bbP(X_1 = y | X_0 = x)\]
and then scale to find the rest.
\begin{exmp}
    Let $S = \{0,1\}.$ A Markov chain taking values in $S$ specified by $p = p(0,1)$ and $q = p(1,0).$ It is obvious that $p(0,0) = 1-p$ and $q(1,1) = 1-q.$
\end{exmp}
\begin{exmp}
    Let $S = \bbZ.$ Let $\{X_n\}_{n\geq 0}$ be defined by $X_0 = X_1 = X_2 =0$ and for $n\geq 3,$ then 
    \[X_n = \begin{cases}
        X_{n-1} + 1,\\
        X_{n-1} - 1,\\
        X_{n-3}
    \end{cases}\]
    each with with probability $1/3.$ This is NOT a Markov Process since $X_n$ can depend on $X_{n-3}.$
\end{exmp}

\begin{defn}
    The $n-$step transition probabilities 
    \[p^n(x,y) = \bbP(X_n = y | X_0 = x).\]
\end{defn}
That is, if we start at $x,$ what is the probability that the $n$th step is $y$?
\begin{prop}
    For all $m,n \in \bbN$ and for all $x,y \in S,$ then 
    \[p^{n + m}(x,y) = \sum_{z\in S}p^n(x,z)p^m(z,y)\]
\end{prop}
\begin{proof}
    We have that by time homogeneity, 
    \[p^n(x,z)p^m(z,y) = \bbP\{X_n = z \mid X_0 = x\} \bbP\{X_{n + m} = y \mid X_n = z\}\] By the Markov Property,
    \[\bbP\{X_{n + m} = y \mid X_n = z\} = \bbP\{X_{n + m} = y \mid X_n = z, X_0 = x\}\] By Bayes' rule:
    \[\bbP\{X_n = z \mid X_0 = x\}\bbP\{X_{n + m} = y \mid X_n = z, X_0 = x\} = \bbP\{X_{n +m} = y, X_n = z \mid X_0 = x\}\] Thus, 
    \[\sum_{z \in S}p^n(x,z)p^m(z,y) = \sum_{z\in S}\bbP\{X_{n +m} = y, X_n = z \mid X_0 = x\} = \bbP\{X_{n + m} = y \mid X_0 = x\} = p^{n + m}(x,y)\]
\end{proof}
\begin{defn}
    The \textbf{transition matrix} of a Markov Chain the the $N \times N$ matrix $P$ whose $ij$ entry is $p(i,j).$
\end{defn}
\begin{prop}
    For each $n \in \bbN,$ $P^n$ is the matrix whose $ij$ entry is $p^n(i,j).$
\end{prop}
\begin{proof}
    We prove by induction. It is trivial for $n = 1.$ Assume it is true for any general $n-1.$ Thus, the $ij$ entry of $P^n = P^{n-1}P$ is \[\sum_{k=1}^n P^{n-1}_{ik}P_{kj} = \sum_{k=1}^n p^{n-1}(i,k)p(k,j) = p^n(i,j),\] where the last equality comes from Proposition 1.
\end{proof}

\newpage
\subsection{Wednesday, Mar 26: Recurrence and Transience for Finite State Space}
We illustrate the stuff from last class with a simple example:
\begin{exmp}
    Consider the two state Markov chain with $S = \{0,1\}$ and 
    \[p(1,0) = \frac{1}{3}, \quad p(1,0) = \frac{1}{2},\] then 
    \[P = \begin{pmatrix}
        p(0,0) & p(0,1)\\
        p(1,0) & p(1,1) 
    \end{pmatrix} = \begin{pmatrix}
        \frac{2}{3} & \frac{1}{3}\\
        \frac{1}{2} & \frac{1}{2}
    \end{pmatrix},\] then 
    \[\bbP\{X_3 = 0 | X_0 = 0\} = p^3(0,0) = \frac{65}{108}\]
\end{exmp}
For the following, we consider a Markov chain $\{X_n\}$ with state space $S.$
\begin{defn}
    Two states $x,y \in S$ \textbf{communicate} if there exists $m,n >0$ such that $p^n(x,y) >0$ and $p^m(y,x) > 0 .$ We write $x \leftrightarrow y.$
\end{defn}
\begin{rem}
    Communication is an equivalence relation, and so we can partition $S$ into a disjoint union of communication classes by modding out the communication classes.
\end{rem}
\begin{defn}
    A communication class $c$ is \textbf{recurrent} if $p(x,y) = 0$ for all $x\in C$ and $y \in S\setminus\{C\}.$ Otherwise, we say that the communication class is \textbf{transient}.
\end{defn}
In other words, if $C$ is recurrent, then the chain never leaves. If it is transcient, then there is no problem leaving.


\begin{defn}
    A Markov chain is \textbf{irreducible} if there is only one communication class.
\end{defn}

\begin{exmp}
    Consider a Markov chain with $S  = \{1,2,3,4,5\}$ and 
    \[P = \begin{pmatrix}
        \frac{1}{5} & \frac{1}{5} & 0 & 0 & \frac{3}{5}\\
        0 & 0 & 0 & 0 & 0 \\
        0 & 0 & \frac{1}{2} & \frac{1}{2} & 0\\
        0 & 0 & \frac{1}{4} & \frac{3}{4} & 0\\
        \frac{1}{2} & \frac{1}{4} & \frac{1}{4} & 0 & 0
    \end{pmatrix}\] then 
    \[C_1 = \{1,2,5\}, \qquad C_2 = \{3,4\}\] and $C_1$ is transient and $C_2$ is recurrent
\end{exmp}

\begin{exmp} (Gambler's ruin)
    Consider the random walk on $S = \{0,1,2, \dots, N\}$ with absorbing boundary, and transition probability
    \[p(x,x+1) = p(x, x-1) = \frac{1}{2}, \quad x\in \{1,2,\dots, N-1\}\] and 
    \[p(0,0) = p(N,N) = 1\]

    $\{1,\dots, N-1\}$ is a transient communication class, while $\{0\}, \{N\}$ are both recurrent communication classes.
\end{exmp}

\begin{prop}
    Suppose $S$ is finite. If $C$ is a recurrent communication class, then if $\{X_n\}$ starts in $C,$ with probability $1,$ $\{X_n\}$ visits every state in $C$ infinitely often. That is, for each $x,y \in C,$ $\bbP\{X_n = y \text{ i.o.} \; | \; X_0 = x\} = 1.$
\end{prop}
\begin{proof}
    Since $C$ is a communication class, then for every $z \in C,$ there exists some $n_z \in \bbN$ such that $p^{n_z}(z,y) >0.$ Let $n = \max\{n_z\; \forall z \in C\},$ and $q = \min\{p^{n_z}(z,y)\; | \; z\in C\}.$ Note that this quantities necessarily exist because $S$ is finite. Let $E_k = \{X_i = y, \; | \; i \in \{(k-1)n + 1, (k-1)n + 2\, \dots (k-1)nk\}\}.$ Then for states $s_0, s_1, \dots, s_{nk} \in S,$ we have that 
    \[\bbP\{E_{k+1} \: | \: X_0 = s_0, X_1 = s_1, \dots, X_{nk} = S_{nk}\} = \bbP\{E_1 \; | \; X_0 = s_{nk}\} \geq q.\] For $M, N \in \bbN$ with $M > N,$ we have that 
    \begin{align*}
      \bbP\{E_k \text{ does not occur for any any } k \in \{N, N + 1, \dots, M\}\} &= \bbP\{\bigcap_{k=N}^M E_k^c\}  \\
      &= \bbP\{E_M^c \:  | \: \bigcap_{k = N}^{M-1}E_k\}\bbP\{\bigcap_{K=N}^{M-1}E_k\}\\
      &\leq (1-q)\bbP\{\bigcap_{K=N}^{M-1}E_k\} \\
      &\leq\dots\leq (1-q)^{M-N} \to 0
    \end{align*}
\end{proof}
\begin{prop}
    Suppose $S$ is finite. If $C$ is a transient communication class, then w.p. 1, $X$ eventually leaves $C$ and never returns.
\end{prop}

\newpage
\subsection{Friday, Mar 28: The Strong Markov Property}

\begin{defn}
    A random time $\tau \in \bbN_0 \cup \{\infty\}$ is called a \textbf{stopping time} if, for all $n \in \bbN,$ the event $\{\tau = n\}$ is determined by $X_0, X_1, \dots, X_n.$ 
\end{defn}
\begin{exmp}
\begin{enumerate}
    \item (The Hitting Time) \[\tau = \min\{n \mid X_n = x\}, \quad x\in S.\]
    In the Gambler's ruin model, where the gambler starts with $k-$dollars and gambles all the way to $N$ or $0$ dollars, the hitting time is the first the the gambler reaches $\$1$.
    \item $\tau = k^\text{th}$ time for which $X_n \in A,$ $k\in \bbN,$ $A \subset S.$
    \item $\tau = \min\{\tau_1, \tau_2\},$ where $\tau_1$ and $\tau_2$ are stopping times. 
    \item Let $N \in \bbN,$ $x\in S,$ $\tau$ be the last time $n \leq N$ for which $X_n = x,$ and $\tau  = 0$ if no such time exists. $\tau$ is NOT a stopping time because it  depends on stuff in the future.
\end{enumerate}
\end{exmp}

\begin{thm}
    (Strong Markov Property) Let $\tau$ be a stopping time for $\{X_n\}.$ Let $n \geq 0,$ $m\geq 1,$ $x_n \dots, x_n \in S$ such that 
    \[\bbP\{X_0 = x_0, \dots, X_\tau = X_n\} >0,\] and 
    let $y_1, \dots, y_m \in S.$ Then 
    \[\bbP\{X_{\tau + 1} = y, \dots, X_{\tau + m} = y_m\mid X_0 = x_0, \dots, X-\tau = x_n\} =\bbP\{X_{\tau + 1} = y, \dots, X_{\tau + m} = y_m\mid X_0 = x_n\}\]
\end{thm}
\begin{proof}
    The event $\{X_0 = x_0, \dots, X_\tau = x_n\}$ is the same as the event $\{X_0 = x_0, \dots, X_n = x_n\}$ and $\{\tau = n\}.$ Since $\tau$ is a stopping time determined where the event $\{\tau = n\}$ by $\{X_0, \dots, X_n\}.$ Thus, we get that the events $\{X_0 = x_0, \dots, X_\tau = x_n\} = \{X_0 = x_0, \dots, X_n = n\}\cap \{\tau = n\},$ then 
    \begin{align*}
        \bbP\{X_{\tau + 1} = y_1, \dots, X_{\tau  + m} = y_m \mid X_0 = x_0, \dots, X_\tau = x_n\} &= \bbP\{X_{n + 1} = y_1, \dots, X_{n  + m} = y_m \mid X_0 = x_0, \dots, X_n = x_n\}\\
        &= \bbP\{X_{n + 1} = y_1, \dots, X_{n  + m} = y_m \mid X_n = x_n\}
    \end{align*} and conclude with the regular markov property and $n-$step invariance.
\end{proof}

\begin{exmp}
    Let $x\in S$ with $S$ finite and let $\tau = \min\{n \geq 0 \mid X_n = x\}.$ Assume that $\bbP\{\tau < \infty\} = 1.$ For any $x_0, \dotss, x_n \in S$ such that 
    $\bbP\{X_0 = x_0, \dots, X_\tau = x_n\} >0,$ we have $x_n = x,$ and so the conditional distribution of $\{X_{\tau + j}\}_{j \geq 0}$ given everything up to $\tau$ is the same as the original chain starting at $x.$ 

    Thus, the Strong Markov property essentially guarantees that the Markov chain resets after hitting $\tau$!
\end{exmp}
    
\begin{defn}
    Let $X_n$ be a Markov chain on a countable state space $S.$ For any $x\in S,$ the \textbf{period} of $x$ is the greatest common divisor of $J_x = \{n \geq 1 \mid p^n(x,x) >0\}.$
\end{defn}
\begin{exmp}
    If $p(x,x) >0,$ then $1 \in J_x,$ and so the period of $x$ is $1.$
\end{exmp}

\begin{thm}
    If $x\leftrightarrow y,$ the periods of $x$ and $y,$ denoted by $d_x$ and $d_y$ respectively, are the same. 
\end{thm}
\begin{proof}
    Choose $n, m$ such that $p^n(x,y) >0$ and $p^m(y,x) >0.$ Then $p^{n +m}(x,x) >0$ and $p^{m + n}(y,y) >0.$ Thus, $ n + m \in J_x \cap J_y.$ Assume that $d_x < d_y.$ Then there exists some $k \in K_x$ not divisible by $d_y.$ Then $n + m + k \in J_y,$ and so $d_y$ divides $n + m + k$ and $n + m,$ and so $d_y$ divides $k,$ which is a contradiction.    
\end{proof}

\begin{defn}
    A Markov chain is aperiodic if every state has period $1.$
\end{defn}

\begin{exmp}
    A knight on an $8\times 8$ chessboard selects one of the next legal moves with equal probability, independently of the past. 
    \[S = \{(x,y) \in \bbR^2 \mid x \in \{1, \dots, 8\}, y \in \{1, \dots, 8\}\}.\]  The period is $2.$ 
\end{exmp}

\newpage
\subsection{Monday, Mar 31: Stationary Distributions}
\begin{defn}
    Let $\tilde{v} = (\tilde{v}_1, \tilde{v}_2, \dots, \tilde{v}_n)$ be a vector such that $\tilde{v}_j = \bbP\{X_0 = j\}.$ We say that $\tilde{v}$ is the \textbf{initial distribution} of the Markov chain.
\end{defn}
\begin{prop}
    For each $i \in [n],$ then $i$th entry of the row vector $\tilde{v} P$ is $\bbP\{X_1 = i\}.$
\end{prop}
\begin{exmp}
    Consider the Markov chain with $S = \{1,2,3,\}$ and 
    \[P = \begin{pmatrix}
        \frac{1}{2} & \frac{1}{4} & \frac{1}{4}\\
        \frac{1}{4} & \frac{1}{2} & \frac{1}{4}\\
        0 & \frac{1}{4} & \frac{3}{4}
    \end{pmatrix}\]
    For $n$ large, 
        \[P^n = \begin{pmatrix}
        \frac{1}{6} & \frac{1}{3} & \frac{1}{2}\\
        \frac{1}{6} & \frac{1}{3} & \frac{1}{2}\\
        \frac{1}{6} & \frac{1}{3} & \frac{1}{2}
    \end{pmatrix}.\] We encounter the phenomena that the limit of $P^n,$ if it exists, has identical rows. 

    Call this row $\pi.$ For any probability vector $\Tilde{v},$ $\lim_{n\to \infty} \Tilde{v} P^n = \pi$
\end{exmp}

Suppose $\pi$ is a limiting probability vector. That is, for any initial distribution $\Tilde{v},$ $\lim_{n\to \infty}\Tilde{v}P^n = \pi.$ Then 
\[\pi = \lim_{n\to \infty}\Tilde{v}P^{n + 1} = (\lim_{n\to \infty}\Tilde{v}P^n)P = \pi P\] We say that $\pi$ is a {stationary/invariant/equilibrium/steady-state} distribution for the Markov chain. 

\begin{defn}
    Let $\pi: S \to [0,1]$ be a probability distribution on $S$ such that $\sum_{x\in S} \pi_x = 1.$ We say that $\pi$ is a \textbf{stationary distribution} of $\{X_n\}$ if 
    \[\pi_y = \sum_{x\in S}\pi_x p(x,y), \quad \forall y \in S.\] That is, $\pi P =  \pi.$
\end{defn}

\begin{thm}
    If $S$ is finite and $\{X_n\}$ is an irreducible and aperiodic chain, then there exists a unique stationary distribution $\pi$ for $\{X_n\}.$ Moreover, for any $x,y \in S,$ \[\lim_{n\to \infty}p^n(y,x) = \pi_y\]
\end{thm}

\begin{proof}
    (Existence) Fix $z \in S.$ Suppose $X_0 = z.$ Let $\tau = \min\{n \geq 1 \mid X_n = z\}$ be the first return time to $z.$ Note that $\tau < \infty $ by proposition $4.$ For any $x\in S,$ define
    \[\Tilde{\pi}_x := \mathbb{E}[ \#\{n \in \{0,1, \dots \tau - 1\} \mid X_n = x\}]\] 
    We claim that 
    \[x \mapsto \frac{\pi_x}{\bbE[\tau]}\] is a stationary distribution for $\{X_n\}.$ We want to show that for all $y \in S,$ $\tilde{\pi}_y = \sum_{x\in S} \Tilde{\pi}_x p(x,y).$ Evidently, 
    \[\tilde{\pi}_x = \mathbb{E}\left[\sum_{x\in S}\mathbbm{1}_{\{X_n = x\}}\right] = \mathbb{E} \left[\sum_{n = 0}^\infty \mathbbm{1}_{\{X_n = x, \tau >n\}}\right] = \sum_{n =0}^\infty \bbP\{X_n = x, \tau >n\}.\] Thus, 
    \[\sum_{x\in S}\tilde{\pi}_x p(x,y) = \sum_{x\in S}\sum_{n = 0}^\infty \bbP\{X_n = x, \tau >n\} p(x,y) = \sum_{n = 0}^\infty \sum_{x\in S} \bbP\{X_n = x, T >n X_{n + 1}  = y\} = \sum_{n = 0}^\infty \bbP\{\tau >{n + 1}X_{n + 1} = y\} = \tilde{\pi}_y\]
    In order to find a stationary probability distribution, we need that 
    \[\sum_{x\in S} \frac{\tilde{\pi}_x}{\mathbb{E}[\tau]} = 1.\] But then 
    \[\tilde{pi}_x = \sum_{n = 0}^\infty \bbP\{X_n = x , \tau >n\} = \sum_{n = 0}^\infty \bbP\{\tau > n, X_n = x \mid X_n = x\}.\] SO then 
    \[\tilde{\pi}_x  = \sum_{n  = 0}^\infty \sum_{x\in S}\bbP\{\tau > n  \mid X_n = x\}\bbP\{\tau  > n \mid X_n  = x\}\bbP\{X_n = x\} = \sum_{n = 0}^\infty \bbP\{\tau >n\} = \mathbbE[\tau]\]
\end{proof}

\begin{exmp}
    Consider a Markov chain $\{X_n\}$ with 
    \[P = \begin{pmatrix}
        \frac{1}{3} & \frac{1}{3} & \frac{1}{3}\\
        \frac{1}{2} & 0 & \frac{1}{2}\\
        \frac{1}{5} & \frac{1}{5} & \frac{3}{5}
    \end{pmatrix}.\] $\{X_n\}$ is clearly aperiodic and irreducible. To compute $\pi,$ we solve the system $\pi P  = \pi,$ with $\pi_1 + \pi_2  + \pi_3 = 1.$ 
    \[(\pi_1 , \pi_2 ,\pi_3) P  = \pi \iff \frac{1}{3}\pi_1 + \frac{1}{2}\pi_2 + \frac{1}{5}\pi_3 = \pi_1,\quad  \frac{1}{3}\pi_1 + \frac{1}{5}\pi_3  = \pi_2, \quad  \frac{1}{3}\pi_1 + \frac{1}{2}\pi_2 + \frac{3}{5}\pi_3\]
    Thus, we find that solving for the RREF
    \[\begin{bmatrix}
        \frac{1}{3} & \frac{1}{3} & \frac{1}{5} & | & 1\\
        \frac{1}{3} & 0 & \frac{1}{5}& | & 1\\
        \frac{1}{3} & \frac{1}{2} & \frac{3}{5}& | & 1\\
        1 & 1 & 1 & | & 1
    \end{bmatrix}\]
    Thus, 
    \[\pi = (\frac{3}{10}, \frac{1}{5}, \frac{1}{2})\]
\end{exmp}



\newpage
\section{Problem Sessions}
\subsection{Monday, Mar 31: Problem Session 1}
When do we run into the issue that $\lim_{n\to \infty}\pi P^n$ does not exist? 
\begin{enumerate}
    \item[(1)] If Period $> 1$
    \item[(2)] If there are multiple recurrence classes that are transient.
\end{enumerate}

Problems:
\begin{enumerate}
    \item Let $X_0, X_1, \dots$ be a Markov chain with state space $S = \{0, 1, 2, 3\}$, and with transition matrix
    \[
    \begin{bmatrix}
        \frac{1}{4} & 0 & \frac{1}{2} & \frac{1}{4} \\
        0 &  \frac{1}{5} & 0 & \frac{1}{5} \\
        0 & 1 & 0 & 0 \\
        \frac{1}{3} & \frac{1}{3} & 0 & \frac{1}{3}
    \end{bmatrix}.
    \]
    A new process is defined by $Z_n = 0$ if $X_n = 0$ or $1$ and $Z_n = X_n$ if $X_n = 2$ or $3$. Find $P(Z_{n+1} = 2 | Z_n = 0, Z_{n-1} = 2)$ and $P(Z_{n+1} = 2 | Z_n = 0, Z_{n-1} = 3)$. Is $Z_n$ a Markov chain?
    \begin{solution}
    \begin{align*}
        \bbP\{Z_{n + 1}= 2 \mid Z_n = 0, Z_{n-1} = 2\} &= \frac{\bbP\{Z_{n + 1} = 2, Z_{n} = 0, Z_{n-1} = 2\}}{\bbP\{Z_n = 0, Z_{n-1 } = 2\}}\\
        &= \frac{\bbP\{Z_{n + 1} = 2, Z_{n} = 0 \mid Z_{n-1} =2\}}{\bbP\{Z_n = 0 \mid Z_{n-1} = 2\}}\\
        &= 0
    \end{align*}
    \begin{align*}
        \bbP\{Z_{n + 1} = 2 \mid Z_n = 0, Z_{n-1} = 3\} &= \frac{\bbP\{Z_{n + 1} = 2, Z_{n} = 0 \mid Z_{n-1} =3\}}{\bbP\{Z_n = 0 \mid Z_{n-1} = 3\}}\\
        &= \frac{\bbP\{X_n = 2, X_n = 0 \mid X_{n-1} = 3\} +\bbP\{X_n = 2, X_n = 1 \mid X_{n-1} = 3\}}{\bbP\{X_n = 0 \mid X_{n-1} =3\} + \bbP\{X_{n} = 1 \mid X_{n-1} = 3\}}\\
        &= \frac{\frac{1}{3}\frac{1}{2} + 0}{\frac{1}{6} + \frac{1}{6}}\\
        &= \frac{1}{4}
    \end{align*}
    Thus, $\{Z_n\}$ is not a Markov chain
    \end{solution}

    \item We repeatedly roll two four-sided dice with numbers 1, 2, 3, and 4 on them. Let $Y_k$ be the sum on the $k$-th roll, $S_n = Y_1 + Y_2 + \dots + Y_n$ be the total of the first $n$ rolls, and $X_n = S_n (\text{ mod } 6)$. Find the transition matrix for $\{X_n\}$.
    \begin{solution}
        $\{X_n\}$ is a Markov process because it only depends on the current state $S_n,$ as you can figure out the next turn only from this, since 
        \[X_{n + 1} = S_{n + 1}\text{ mod } 6 = (S_{n} + Y_{n + 1})\text{ mod } 6= S_n \text{ mod } 6 + Y_{n + 1}\text{ mod } 6 = X_n + Y_{n+1}\text{ mod } 6\] 
        \[P = \begin{bmatrix}
            \frac{3}{16} & \frac{1}{8} & \frac{1}{8} & \frac{1}{8} & \frac{3}{16} & \frac{1}{4}\\
             \frac{1}{4} & \frac{3}{16}& \frac{1}{8} & \frac{1}{8} & \frac{1}{8} & \frac{3}{16} \\
             \frac{3}{16} & \frac{1}{4} & \frac{3}{16}& \frac{1}{8} & \frac{1}{8} & \frac{1}{8} \\
             \frac{1}{8} & \frac{3}{16} & \frac{1}{4} & \frac{3}{16}& \frac{1}{8} & \frac{1}{8} \\
             \frac{1}{8} &\frac{1}{8} & \frac{3}{16} & \frac{1}{4} & \frac{3}{16}& \frac{1}{8} \\
             \frac{1}{8} &\frac{1}{8} &\frac{1}{8} & \frac{3}{16} & \frac{1}{4} & \frac{3}{16} 
        \end{bmatrix}\]
    \end{solution}

    \item Consider a Markov chain with states $S = \{0, \dots, N\}$ and transition probabilities $p(i, i+1) = p$, $p(i, i-1) = q$, for $1 \leq i \leq N-1$, where $p+q = 1$, $0 < p < 1$. Assume $p(0,1) = p(N, N-1) = 1$.
    \begin{enumerate}
        \item Draw a transition diagram for this chain.
        \begin{solution}
        In the following diagram, if you the probability  is above, then it is going to the right:
            \begin{center}
\begin{tikzpicture}[scale=1.2, every node/.style={draw, circle, inner sep=1pt}]
    \node (A) at (-4,0) {0};
    \node (B) at (-2,0) {1};
    \node (C) at (0,0) {$\cdots$}
    \node (D) at (2,0) {$N-1$};
    \node (E) at (4, 0) {$N$};

    \draw[->] (A) -- (B) node[midway, above] {\textcolor{red}{$1$}};
    \draw[->] (B) -- (A) node[midway, below] {\textcolor{red}{$q$}};
    \draw[->] (B) -- (C) node[midway, above] {\textcolor{red}{$p$}};
    \draw[->] (C) -- (B) node[midway, below] {\textcolor{red}{$q$}};
    \draw[->] (C) -- (D) node[midway, above] {\textcolor{red}{$p$}};
    \draw[->] (D) -- (C) node[midway, below] {\textcolor{red}{$q$}};
    \draw[->] (D) -- (E) node[midway, above] {\textcolor{red}{$p$}};
    \draw[->] (E) -- (D) node[midway, below] {\textcolor{red}{$1$}};
\end{tikzpicture}
\end{center}            
        \end{solution}
        \item Is the Markov chain irreducible? 
        \begin{solution}
Let $i,j \in [N],$ then we claim that $i \leftrightarrow j.$ Without loss of generality, suppose that $i < j.$ Suppose $j-i = N.$ Then $p(i,j)\geq p^N >0$ and $p(j,i) \geq q^N >0.$
        \end{solution}
        \item What is the period of this chain? 
        \begin{solution}
            The period has to be $2.$
        \end{solution}
    \end{enumerate}

    \item A taxicab driver moves between the airport $A$ and two hotels $B$ and $C$ according to the following rules. If he is at the airport, he will be at one of the two hotels next with equal probability. If at a hotel, then he returns to the airport with probability $\frac{3}{4}$ and goes to the other hotel with probability $\frac{1}{4}$.
    \begin{enumerate}
        \item Find the transition matrix for the chain.
        \item Suppose the driver begins at the airport at time 0. Find the probability for each of his three possible locations at time 2 and the probability he is at hotel $B$ at time 3.
    \end{enumerate}

    \item At time $n = 0$, two ladybirds are placed at vertices $i$ and $j$ of a regular hexagon, whose vertices are labeled $1, \dots, 6$. At time $n = 1$, each of them moves, independently of the other, to one of the two adjacent vertices with probability $\frac{1}{2}$, and so on at each time $n = 2, 3, \dots$.
    \begin{enumerate}
        \item Denote $X_n$ the distance between the two ladybirds at time $n \geq 0$, i.e., the minimum number of edges between them. Find the transition matrix for $\{X_n\}$.
        \item Identify the communication classes. Are they recurrent or transient?
    \end{enumerate}

    \item Find the invariant (stationary) distributions for the following Markov chains with given transition matrices:
    \[
    \begin{bmatrix}
        \frac{1}{2} & \frac{2}{5} & \frac{1}{10} \\
        \frac{1}{5} & \frac{1}{2} & \frac{3}{10} \\
        \frac{1}{10} & \frac{3}{10} & \frac{3}{5}
    \end{bmatrix}.
    \]
    \begin{solution}
        We must find a solution vector to $\pi P = \pi \iff P^T \pi^T = \pi^T$ Then 
        \[P^T = 
        \frac{1}{10}\begin{bmatrix}
            5 & 2 & 1\\
            4 & 5 & 3\\
            1 & 3 & 6
        \end{bmatrix}\] and call $\pi^T = \textbf{v}.$ Then 
        the eigenvalue for $\lambda = 10$ is given by 
        \[\ker(A - 10I)  = \ker(\begin{bmatrix}
            -5 & 2 & 1\\
            4 & -5 & 3\\
            1 & 3 & -4
        \end{bmatrix})\] The RREF of this matrix is given by 
        \[
            \begin{bmatrix}
                1 &  0 & 0\\
                0 & 1 & 0\\
                0 & 0 & 1
            \end{bmatrix}\]
        So then 
        \[v_1 = 
        \frac{1}{10}\begin{bmatrix}
            \frac{11}{17}\\
            \frac{19}{17}\\
            1
        \end{bmatrix}\]
    \end{solution}
    (a)
    \[
    \begin{bmatrix}
        \frac{1}{2} & \frac{2}{5} & \frac{1}{10} \\
        \frac{3}{10} & \frac{2}{5} & \frac{3}{10} \\
        \frac{1}{5} & \frac{1}{5} & \frac{3}{5}
    \end{bmatrix}.
    \]
    (b)
\end{enumerate}
\end{document}