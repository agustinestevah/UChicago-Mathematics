\documentclass[11pt]{article}
\usepackage{float}
\usepackage{tikz}
\usetikzlibrary{automata, positioning}
% NOTE: Add in the relevant information to the commands below; or, if you'll be using the same information frequently, add these commands at the top of paolo-pset.tex file. 
\newcommand{\name}{Agustín Esteva}
\newcommand{\email}{aesteva@uchicago.edu}
\newcommand{\classnum}{23500}
\newcommand{\subject}{Markov Chains, Martingales, and Brownian Motion}
\newcommand{\instructors}{Stephen Yearwood}
\newcommand{\assignment}{Problem Set 3}
\newcommand{\semester}{Spring 2025}
\newcommand{\duedate}{04/18/2025}
\newcommand{\bA}{\mathbf{A}}
\newcommand{\bB}{\mathbf{B}}
\newcommand{\bC}{\mathbf{C}}
\newcommand{\bD}{\mathbf{D}}
\newcommand{\bE}{\mathbf{E}}
\newcommand{\bF}{\mathbf{F}}
\newcommand{\bG}{\mathbf{G}}
\newcommand{\bH}{\mathbf{H}}
\newcommand{\bI}{\mathbf{I}}
\newcommand{\bJ}{\mathbf{J}}
\newcommand{\bK}{\mathbf{K}}
\newcommand{\bL}{\mathbf{L}}
\newcommand{\bM}{\mathbf{M}}
\newcommand{\bN}{\mathbf{N}}
\newcommand{\bO}{\mathbf{O}}
\newcommand{\bP}{\mathbf{P}}
\newcommand{\bQ}{\mathbf{Q}}
\newcommand{\bR}{\mathbf{R}}
\newcommand{\bS}{\mathbf{S}}
\newcommand{\bT}{\mathbf{T}}
\newcommand{\bU}{\mathbf{U}}
\newcommand{\bV}{\mathbf{V}}
\newcommand{\bW}{\mathbf{W}}
\newcommand{\bX}{\mathbf{X}}
\newcommand{\bY}{\mathbf{Y}}
\newcommand{\bZ}{\mathbf{Z}}
\newcommand{\Vol}{\text{Vol}}

%% blackboard bold math capitals
\newcommand{\bbA}{\mathbb{A}}
\newcommand{\bbB}{\mathbb{B}}
\newcommand{\bbC}{\mathbb{C}}
\newcommand{\bbD}{\mathbb{D}}
\newcommand{\bbE}{\mathbb{E}}
\newcommand{\bbF}{\mathbb{F}}
\newcommand{\bbG}{\mathbb{G}}
\newcommand{\bbH}{\mathbb{H}}
\newcommand{\bbI}{\mathbb{I}}
\newcommand{\bbJ}{\mathbb{J}}
\newcommand{\bbK}{\mathbb{K}}
\newcommand{\bbL}{\mathbb{L}}
\newcommand{\bbM}{\mathbb{M}}
\newcommand{\bbN}{\mathbb{N}}
\newcommand{\bbO}{\mathbb{O}}
\newcommand{\bbP}{\mathbb{P}}
\newcommand{\bbQ}{\mathbb{Q}}
\newcommand{\bbR}{\mathbb{R}}
\newcommand{\bbS}{\mathbb{S}}
\newcommand{\bbT}{\mathbb{T}}
\newcommand{\bbU}{\mathbb{U}}
\newcommand{\bbV}{\mathbb{V}}
\newcommand{\bbW}{\mathbb{W}}
\newcommand{\bbX}{\mathbb{X}}
\newcommand{\bbY}{\mathbb{Y}}
\newcommand{\bbZ}{\mathbb{Z}}

%% script math capitals
\newcommand{\sA}{\mathscr{A}}
\newcommand{\sB}{\mathscr{B}}
\newcommand{\sC}{\mathscr{C}}
\newcommand{\sD}{\mathscr{D}}
\newcommand{\sE}{\mathscr{E}}
\newcommand{\sF}{\mathscr{F}}
\newcommand{\sG}{\mathscr{G}}
\newcommand{\sH}{\mathscr{H}}
\newcommand{\sI}{\mathscr{I}}
\newcommand{\sJ}{\mathscr{J}}
\newcommand{\sK}{\mathscr{K}}
\newcommand{\sL}{\mathscr{L}}
\newcommand{\sM}{\mathscr{M}}
\newcommand{\sN}{\mathscr{N}}
\newcommand{\sO}{\mathscr{O}}
\newcommand{\sP}{\mathscr{P}}
\newcommand{\sQ}{\mathscr{Q}}
\newcommand{\sR}{\mathscr{R}}
\newcommand{\sS}{\mathscr{S}}
\newcommand{\sT}{\mathscr{T}}
\newcommand{\sU}{\mathscr{U}}
\newcommand{\sV}{\mathscr{V}}
\newcommand{\sW}{\mathscr{W}}
\newcommand{\sX}{\mathscr{X}}
\newcommand{\sY}{\mathscr{Y}}
\newcommand{\sZ}{\mathscr{Z}}


\renewcommand{\emptyset}{\O}

\newcommand{\abs}[1]{\lvert #1 \rvert}
\newcommand{\norm}[1]{\lVert #1 \rVert}
\newcommand{\sm}{\setminus}


\newcommand{\sarr}{\rightarrow}
\newcommand{\arr}{\longrightarrow}

% NOTE: Defining collaborators is optional; to not list collaborators, comment out the line below.
%\newcommand{\collaborators}{Alyssa P. Hacker (\texttt{aphacker}), Ben Bitdiddle (\texttt{bitdiddle})}

% Copyright 2021 Paolo Adajar (padajar.com, paoloadajar@mit.edu)
% 
% Permission is hereby granted, free of charge, to any person obtaining a copy of this software and associated documentation files (the "Software"), to deal in the Software without restriction, including without limitation the rights to use, copy, modify, merge, publish, distribute, sublicense, and/or sell copies of the Software, and to permit persons to whom the Software is furnished to do so, subject to the following conditions:
%
% The above copyright notice and this permission notice shall be included in all copies or substantial portions of the Software.
% 
% THE SOFTWARE IS PROVIDED "AS IS", WITHOUT WARRANTY OF ANY KIND, EXPRESS OR IMPLIED, INCLUDING BUT NOT LIMITED TO THE WARRANTIES OF MERCHANTABILITY, FITNESS FOR A PARTICULAR PURPOSE AND NONINFRINGEMENT. IN NO EVENT SHALL THE AUTHORS OR COPYRIGHT HOLDERS BE LIABLE FOR ANY CLAIM, DAMAGES OR OTHER LIABILITY, WHETHER IN AN ACTION OF CONTRACT, TORT OR OTHERWISE, ARISING FROM, OUT OF OR IN CONNECTION WITH THE SOFTWARE OR THE USE OR OTHER DEALINGS IN THE SOFTWARE.

\usepackage{fullpage}
\usepackage{enumitem}
\usepackage{amsfonts, amssymb, amsmath,amsthm}
\usepackage{mathtools}
\usepackage[pdftex, pdfauthor={\name}, pdftitle={\classnum~\assignment}]{hyperref}
\usepackage[dvipsnames]{xcolor}
\usepackage{bbm}
\usepackage{graphicx}
\usepackage{mathrsfs}
\usepackage{pdfpages}
\usepackage{tabularx}
\usepackage{pdflscape}
\usepackage{makecell}
\usepackage{booktabs}
\usepackage{natbib}
\usepackage{caption}
\usepackage{subcaption}
\usepackage{physics}
\usepackage[many]{tcolorbox}
\usepackage{version}
\usepackage{ifthen}
\usepackage{cancel}
\usepackage{listings}
\usepackage{courier}

\usepackage{tikz}
\usepackage{istgame}

\hypersetup{
	colorlinks=true,
	linkcolor=blue,
	filecolor=magenta,
	urlcolor=blue,
}

\setlength{\parindent}{0mm}
\setlength{\parskip}{2mm}

\setlist[enumerate]{label=({\alph*})}
\setlist[enumerate, 2]{label=({\roman*})}

\allowdisplaybreaks[1]

\newcommand{\psetheader}{
	\ifthenelse{\isundefined{\collaborators}}{
		\begin{center}
			{\setlength{\parindent}{0cm} \setlength{\parskip}{0mm}
				
				{\textbf{\classnum~\semester:~\assignment} \hfill \name}
				
				\subject \hfill \href{mailto:\email}{\tt \email}
				
				Instructor(s):~\instructors \hfill Due Date:~\duedate	
				
				\hrulefill}
		\end{center}
	}{
		\begin{center}
			{\setlength{\parindent}{0cm} \setlength{\parskip}{0mm}
				
				{\textbf{\classnum~\semester:~\assignment} \hfill \name\footnote{Collaborator(s): \collaborators}}
				
				\subject \hfill \href{mailto:\email}{\tt \email}
				
				Instructor(s):~\instructors \hfill Due Date:~\duedate	
				
				\hrulefill}
		\end{center}
	}
}

\renewcommand{\thepage}{\classnum~\assignment \hfill \arabic{page}}

\makeatletter
\def\points{\@ifnextchar[{\@with}{\@without}}
\def\@with[#1]#2{{\ifthenelse{\equal{#2}{1}}{{[1 point, #1]}}{{[#2 points, #1]}}}}
\def\@without#1{\ifthenelse{\equal{#1}{1}}{{[1 point]}}{{[#1 points]}}}
\makeatother

\newtheoremstyle{theorem-custom}%
{}{}%
{}{}%
{\itshape}{.}%
{ }%
{\thmname{#1}\thmnumber{ #2}\thmnote{ (#3)}}

\theoremstyle{theorem-custom}

\newtheorem{theorem}{Theorem}
\newtheorem{lemma}[theorem]{Lemma}
\newtheorem{example}[theorem]{Example}

\newenvironment{problem}[1]{\color{black} #1}{}

\newenvironment{solution}{%
	\leavevmode\begin{tcolorbox}[breakable, colback=green!5!white,colframe=green!75!black, enhanced jigsaw] \proof[\scshape Solution:] \setlength{\parskip}{2mm}%
	}{\renewcommand{\qedsymbol}{$\blacksquare$} \endproof \end{tcolorbox}}

\newenvironment{reflection}{\begin{tcolorbox}[breakable, colback=black!8!white,colframe=black!60!white, enhanced jigsaw, parbox = false]\textsc{Reflections:}}{\end{tcolorbox}}

\newcommand{\qedh}{\renewcommand{\qedsymbol}{$\blacksquare$}\qedhere}

\definecolor{mygreen}{rgb}{0,0.6,0}
\definecolor{mygray}{rgb}{0.5,0.5,0.5}
\definecolor{mymauve}{rgb}{0.58,0,0.82}

% from https://github.com/satejsoman/stata-lstlisting
% language definition
\lstdefinelanguage{Stata}{
	% System commands
	morekeywords=[1]{regress, reg, summarize, sum, display, di, generate, gen, bysort, use, import, delimited, predict, quietly, probit, margins, test},
	% Reserved words
	morekeywords=[2]{aggregate, array, boolean, break, byte, case, catch, class, colvector, complex, const, continue, default, delegate, delete, do, double, else, eltypedef, end, enum, explicit, export, external, float, for, friend, function, global, goto, if, inline, int, local, long, mata, matrix, namespace, new, numeric, NULL, operator, orgtypedef, pointer, polymorphic, pragma, private, protected, public, quad, real, return, rowvector, scalar, short, signed, static, strL, string, struct, super, switch, template, this, throw, transmorphic, try, typedef, typename, union, unsigned, using, vector, version, virtual, void, volatile, while,},
	% Keywords
	morekeywords=[3]{forvalues, foreach, set},
	% Date and time functions
	morekeywords=[4]{bofd, Cdhms, Chms, Clock, clock, Cmdyhms, Cofc, cofC, Cofd, cofd, daily, date, day, dhms, dofb, dofC, dofc, dofh, dofm, dofq, dofw, dofy, dow, doy, halfyear, halfyearly, hh, hhC, hms, hofd, hours, mdy, mdyhms, minutes, mm, mmC, mofd, month, monthly, msofhours, msofminutes, msofseconds, qofd, quarter, quarterly, seconds, ss, ssC, tC, tc, td, th, tm, tq, tw, week, weekly, wofd, year, yearly, yh, ym, yofd, yq, yw,},
	% Mathematical functions
	morekeywords=[5]{abs, ceil, cloglog, comb, digamma, exp, expm1, floor, int, invcloglog, invlogit, ln, ln1m, ln, ln1p, ln, lnfactorial, lngamma, log, log10, log1m, log1p, logit, max, min, mod, reldif, round, sign, sqrt, sum, trigamma, trunc,},
	% Matrix functions
	morekeywords=[6]{cholesky, coleqnumb, colnfreeparms, colnumb, colsof, corr, det, diag, diag0cnt, el, get, hadamard, I, inv, invsym, issymmetric, J, matmissing, matuniform, mreldif, nullmat, roweqnumb, rownfreeparms, rownumb, rowsof, sweep, trace, vec, vecdiag, },
	% Programming functions
	morekeywords=[7]{autocode, byteorder, c, _caller, chop, abs, clip, cond, e, fileexists, fileread, filereaderror, filewrite, float, fmtwidth, has_eprop, inlist, inrange, irecode, matrix, maxbyte, maxdouble, maxfloat, maxint, maxlong, mi, minbyte, mindouble, minfloat, minint, minlong, missing, r, recode, replay, return, s, scalar, smallestdouble,},
	% Random-number functions
	morekeywords=[8]{rbeta, rbinomial, rcauchy, rchi2, rexponential, rgamma, rhypergeometric, rigaussian, rlaplace, rlogistic, rnbinomial, rnormal, rpoisson, rt, runiform, runiformint, rweibull, rweibullph,},
	% Selecting time-span functions
	morekeywords=[9]{tin, twithin,},
	% Statistical functions
	morekeywords=[10]{betaden, binomial, binomialp, binomialtail, binormal, cauchy, cauchyden, cauchytail, chi2, chi2den, chi2tail, dgammapda, dgammapdada, dgammapdadx, dgammapdx, dgammapdxdx, dunnettprob, exponential, exponentialden, exponentialtail, F, Fden, Ftail, gammaden, gammap, gammaptail, hypergeometric, hypergeometricp, ibeta, ibetatail, igaussian, igaussianden, igaussiantail, invbinomial, invbinomialtail, invcauchy, invcauchytail, invchi2, invchi2tail, invdunnettprob, invexponential, invexponentialtail, invF, invFtail, invgammap, invgammaptail, invibeta, invibetatail, invigaussian, invigaussiantail, invlaplace, invlaplacetail, invlogistic, invlogistictail, invnbinomial, invnbinomialtail, invnchi2, invnF, invnFtail, invnibeta, invnormal, invnt, invnttail, invpoisson, invpoissontail, invt, invttail, invtukeyprob, invweibull, invweibullph, invweibullphtail, invweibulltail, laplace, laplaceden, laplacetail, lncauchyden, lnigammaden, lnigaussianden, lniwishartden, lnlaplaceden, lnmvnormalden, lnnormal, lnnormalden, lnwishartden, logistic, logisticden, logistictail, nbetaden, nbinomial, nbinomialp, nbinomialtail, nchi2, nchi2den, nchi2tail, nF, nFden, nFtail, nibeta, normal, normalden, npnchi2, npnF, npnt, nt, ntden, nttail, poisson, poissonp, poissontail, t, tden, ttail, tukeyprob, weibull, weibullden, weibullph, weibullphden, weibullphtail, weibulltail,},
	% String functions 
	morekeywords=[11]{abbrev, char, collatorlocale, collatorversion, indexnot, plural, plural, real, regexm, regexr, regexs, soundex, soundex_nara, strcat, strdup, string, strofreal, string, strofreal, stritrim, strlen, strlower, strltrim, strmatch, strofreal, strofreal, strpos, strproper, strreverse, strrpos, strrtrim, strtoname, strtrim, strupper, subinstr, subinword, substr, tobytes, uchar, udstrlen, udsubstr, uisdigit, uisletter, ustrcompare, ustrcompareex, ustrfix, ustrfrom, ustrinvalidcnt, ustrleft, ustrlen, ustrlower, ustrltrim, ustrnormalize, ustrpos, ustrregexm, ustrregexra, ustrregexrf, ustrregexs, ustrreverse, ustrright, ustrrpos, ustrrtrim, ustrsortkey, ustrsortkeyex, ustrtitle, ustrto, ustrtohex, ustrtoname, ustrtrim, ustrunescape, ustrupper, ustrword, ustrwordcount, usubinstr, usubstr, word, wordbreaklocale, worcount,},
	% Trig functions
	morekeywords=[12]{acos, acosh, asin, asinh, atan, atanh, cos, cosh, sin, sinh, tan, tanh,},
	morecomment=[l]{//},
	% morecomment=[l]{*},  // `*` maybe used as multiply operator. So use `//` as line comment.
	morecomment=[s]{/*}{*/},
	% The following is used by macros, like `lags'.
	morestring=[b]{`}{'},
	% morestring=[d]{'},
	morestring=[b]",
	morestring=[d]",
	% morestring=[d]{\\`},
	% morestring=[b]{'},
	sensitive=true,
}

\lstset{ 
	backgroundcolor=\color{white},   % choose the background color; you must add \usepackage{color} or \usepackage{xcolor}; should come as last argument
	basicstyle=\footnotesize\ttfamily,        % the size of the fonts that are used for the code
	breakatwhitespace=false,         % sets if automatic breaks should only happen at whitespace
	breaklines=true,                 % sets automatic line breaking
	captionpos=b,                    % sets the caption-position to bottom
	commentstyle=\color{mygreen},    % comment style
	deletekeywords={...},            % if you want to delete keywords from the given language
	escapeinside={\%*}{*)},          % if you want to add LaTeX within your code
	extendedchars=true,              % lets you use non-ASCII characters; for 8-bits encodings only, does not work with UTF-8
	firstnumber=0,                % start line enumeration with line 1000
	frame=single,	                   % adds a frame around the code
	keepspaces=true,                 % keeps spaces in text, useful for keeping indentation of code (possibly needs columns=flexible)
	keywordstyle=\color{blue},       % keyword style
	language=Octave,                 % the language of the code
	morekeywords={*,...},            % if you want to add more keywords to the set
	numbers=left,                    % where to put the line-numbers; possible values are (none, left, right)
	numbersep=5pt,                   % how far the line-numbers are from the code
	numberstyle=\tiny\color{mygray}, % the style that is used for the line-numbers
	rulecolor=\color{black},         % if not set, the frame-color may be changed on line-breaks within not-black text (e.g. comments (green here))
	showspaces=false,                % show spaces everywhere adding particular underscores; it overrides 'showstringspaces'
	showstringspaces=false,          % underline spaces within strings only
	showtabs=false,                  % show tabs within strings adding particular underscores
	stepnumber=2,                    % the step between two line-numbers. If it's 1, each line will be numbered
	stringstyle=\color{mymauve},     % string literal style
	tabsize=2,	                   % sets default tabsize to 2 spaces
%	title=\lstname,                   % show the filename of files included with \lstinputlisting; also try caption instead of title
	xleftmargin=0.25cm
}

% NOTE: To compile a version of this pset without problems, solutions, or reflections, uncomment the relevant line below.

%\excludeversion{problem}
%\excludeversion{solution}
%\excludeversion{reflection}

\begin{document}	
	
	% Use the \psetheader command at the beginning of a pset. 
	\psetheader

\section*{Problem 1}
Consider the queuing model as discussed in class (section 1.2 of the week 3 notes on canvas).

\begin{enumerate}[label=(\alph*)]
    \item For the transient case (i.e., when \(q<p\)) compute
    \[
    \alpha(x):=\mathbb{P}\{\text{starting at $x$ the queue ever reaches state 0}\}.
    \]
\begin{solution}
    Let $x \geq 0,$ We use the law of total probability and the Markov property to compute:
\begin{align*}
    \alpha(x) &= \alpha(x-1) q(1-p) + \alpha(x)(pq + (1-q)(1-p)) + \alpha(x + 1)p(1-q)\\
    &= \alpha(x-1) q(1-p) + \alpha(x)\big(1 - p(1-q) - q(1-p)\big) + \alpha(x + 1)p(1-p)\\
    &= \alpha(x-1) q(1-p) + \alpha(x) - \alpha(x)\big(p(1-q) + q(1-p)\big) + \alpha(x + 1)p(1-q)\\
    &= \alpha(x-1) \frac{q(1-p)}{p(1-q) + q(1-q)} + \alpha(x + 1) \frac{p(1-q)}{p(1-q) + q(1-p)}\\
    &= \alpha(x-1)A + \alpha(x+1)B
\end{align*}
So then after some algebra and using the general formula that 
\[\alpha_{\pm} = \frac{1 \pm \sqrt{1 - 4AB}}{2A} \implies \alpha \in \{1, \frac{q(1-p)}{p(1-q)}\}\] Thus, 
\[\alpha(x) = \lambda_1  + \lambda_2\left(\frac{q(1-p)}{p(1-q)}\right)^x\]
We have two boundary conditions:
\[\alpha(0) = 1, \qquad \lim_{n \to \infty}\alpha(n) = 0\] From the first, we see that $\lambda_1 + \lambda_2 = 1.$ From the second, we see that since $q<p,$ then 
\[q< p \iff q - qp < p - qp \iff q(1-p) < p(1-q) \iff \frac{q(1-p)}{p(1-q)} <1 \implies \left(\frac{q(1-p)}{p(1-q)}\right)^n \to0,\] and so $\lambda_1 = 0.$ Thus, $\lambda_2 = 1$ and so 
\[\alpha(x) = \left(\frac{q(1-p)}{p(1-q)}\right)^x\]
\end{solution}
    
    \item For which values of \(p,q\) is the chain null/positive recurrent? In the positive recurrent case, give the stationary distribution.
\begin{solution}
    The chain is positive recurrent if and only if a stationary distribution exists, so it suffices to find a condition for which the stationary distribution exists. A stationary distribution must satisfy 
    \[\pi_0 = (1-p)\pi_0 + q(1-p)\pi_1\]
    \[\pi_1 = p \pi_0 + \big(pq +(1-p)(1-q)\big) \pi_1 + q(1-p)\pi_2\]
    \[\pi_n = p(1-q) \pi_{n-1} + \big(pq + (1-p)(1-q)\big)\pi_n + q(1-p)\pi_{n+1}, \quad n\geq 2\]
    We have already solved the recursive relation. 
    \[\pi_n = \lambda_1 + \lambda_2\left(\frac{p(1-q)}{q(1-p)}\right)^n\]
    Solving for the constants, we see that 
    \[1 = \sum_{n=0}^\infty \pi_n = \sum_{n=1}^\infty \lambda_1 + \lambda_2\left(\frac{p(1-q)}{q(1-p)}\right)^n \implies \lambda_1 = 0.\] Thus, we see that 
    \[\lambda_2\sum_{n=0}^\infty \left(\frac{p(1-q)}{q(1-p)}\right)^n < \infty \iff p < q.\] Thus, the chain is null recurrent if, and only if, $p = q.$ It is positive recurrent if $p<q.$ From the above, we see that if $p<q,$ then the series is geometric and thus 
    \[1 = \lambda_2\sum_{n=0}^\infty \left(\frac{p(1-q)}{q(1-p)}\right)^n = \frac{\lambda_2}{1 - (\frac{p(1-q)}{q(1-p)})} \implies \lambda_2 = 1 - \frac{p(1-q)}{q(1-p)}.\] Thus, 
    \[\pi_n = (1 - \frac{p(1-q)}{q(1-p)})\left(\frac{p(1-q)}{q(1-p)}\right)^n\]
\end{solution}
    
    \item What is the average length of the queue in equilibrium (i.e., the long-run average length of the queue)?
    \begin{solution}
        Clearly, if $q\geq p,$ then the average length is infinity. If $q<p,$ then we see that 
\begin{align*}
\bbE[\pi] &= \sum_{n=0}^\infty n\pi_n\\ &= (1 - \frac{p(1-q)}{q(1-p)})\sum_{n=0}^\infty n\left(\frac{p(1-q)}{q(1-p)}\right)^n\\ &= (1 - \frac{p(1-q)}{q(1-p)})\left(\frac{\frac{p(1-q)}{q(1-p)}}{(1 - \frac{p(1-q)}{q(1-p)})^2}\right)\\ &= \frac{\frac{p(1-q)}{q(1-p)}}{1 - \frac{p(1-q)}{q(1-p)}}\\ &= \frac{p(1-q)}{q-p}    
\end{align*}

    \end{solution}
\end{enumerate}


\newpage
\section*{Problem 2}
Consider the following Markov chain with state space \(S=\{0,1,\ldots\}\). A sequence of positive numbers \(p_{1},p_{2},\ldots\) is given with \(\sum_{i=1}^{\infty}p_{i}=1\). Whenever the chain reaches state 0 it chooses a new state according to the \(p_{i}\). Whenever the chain is at a state other than 0 it proceeds deterministically, one step at a time, toward 0. In other words, the chain has transition probability
\[
p(x,x-1)=1,\quad x>0, \quad p(0,x)=p_{x},\quad x>0.
\]
This is a recurrent chain since the chain keeps returning to 0. Under what conditions on the \(p_{x}\) is the chain positive recurrent? In this case, what is the limiting probability distribution \(\pi\)? \textit{[Hint: it may be easier to compute \(\mathbb{E}(T)\) directly where \(T\) is the time of first return to 0 starting at 0.]}
\begin{solution}
Let $x\in S,$ then define
    \[T_x:= \min\{n\geq 1 :X_n = x \mid X_0 = x\}.\] Clearly, since the first move doesn't matter (unless $x = 0$), we have that $\bbE[T_x] = 1 + \bbE[\bbE[T_x \mid X_2]].$ Evidently, $\bbE[T_x \mid X_2 = x] = 2.$ But since the chain resets every two moves, then by the markov property
    \[\bbE[T_x \mid X_2 = y] = \bbE[T_x \mid X_2 = z] = \bbE[T_x \mid X_0 = x], \quad y\neq z\neq x.\] Thus, for any $x\in S$ such that $x\neq 0,$ we have that
    \begin{align*}
        \bbE[T_x] &= 1 + \bbE[\bbE[T_x \mid X_2]]\\
        &= 1 + (1 + \bbE[T_x \mid X_2 = 1])p_1  + \dots + (1 + \bbE[T_x \mid X_2 = x-1])p_{x-1} + (1)p_x + (1 + \bbE[T_x \mid X_2 =x+1])p_{x+1} + \cdots\\
        &= 1 + \sum_{n=1}^\infty p_n + \bbE[T_x \mid X_0 = x]\sum_{n\neq x} p_n\\
        &= 2 + \bbE[T_x \mid X_0 = x](1-p_x)
    \end{align*}
    Thus, \[\bbE[T_x \mid X_0 = x] = \frac{2}{p_x}\]
    Which is finite as long as $p_x >0.$ For $x = 0,$ we have that 
    \[\bbE[T_0 \mid X_0 = 0] = 2.\] Thus, $\bbE[T_x \mid X_0 = x] < \infty$ as long as $p_x < \infty$ for all $x\in S.$ Thus, if this condition is met, then 
    \[\pi_x = \begin{cases}
        \frac{p_x}{2}, \quad x\geq 1\\
        \frac{1}{2}, \quad \:\:x = 1
    \end{cases}\]
\end{solution}

\section*{Problem 3 (Optional)}
A \textit{diagonal lattice path} is a "curve" in the plane made up of line segments that go from a point \((i,j)\) to either \((i+1,j+1)\) (an up step) or \((i+1,j-1)\) (a down step). A \textit{Dyck path of length \(2n\)} is a diagonal lattice path from \((0,0)\) to \((2n,0)\) that does not go below the \(x\)-axis.

\begin{enumerate}[label=(\alph*)]
    \item Prove that the diagonal lattice paths from \((0,0)\) to \((2n,0)\) that go below the \(x\)-axis are in bijection with the diagonal lattice paths from \((0,0)\) to \((2n,-2)\). \textit{(Hint: Given a path P from \((0,0)\) to \((0,2n)\) that goes below the \(x\)-axis, consider the first edge \(e\) that crosses \(y=0.\) Switch the directions of every edge after \(e\), i.e., an up edge becomes down, and a down edge becomes up.)}
    
    \item Show that the number of Dyck paths from \((0,0)\) to \((2n,0)\) is given by
    \[
    C_{n}:=\frac{1}{n+1}\binom{2n}{n}.
    \]
    The quantity \(C_{n}\) is called the \(n^{\text{th}}\) \textit{Catalan number}, and appears very frequently in enumerative combinatorics.
    
    \item Let \(\{X_{n}\}\) be a simple random walk on \(\mathbb{Z}\) starting at \(0\), and let \(T:=\min\{n\geq 1:X_{n}=0\}\).
    \begin{enumerate}[label=(\roman*)]
        \item Let \(E_{k}:=\{T=2k\}\) be the event that the walk first returns to \(0\) at time \(2k\). Use the previous parts to find \(\mathbb{P}\{E_{k}\}\) in terms of \(k\).
        \item Use Stirling's approximation to show that \(E[T]=\infty\).
    \end{enumerate}
\end{enumerate}

\section*{Problem 4 (20 points)}
For each of the following Markov chains, determine whether the chain is positive recurrent, null recurrent, or transient. In the positive recurrent case, find the stationary distribution.

\begin{enumerate}[label=(\alph*)]
    \item For \(x\in\mathbb{Z}\) with \(x\geq 0\), \(p(x,0)=(x+1)/(x+2)\) and \(p(x,x+1)=1/(x+2)\) (\(p(x,y)=0\) for all other \(y\)).
\begin{solution}
We claim that this process is positive recurrent. It suffices to find a stationary distribution. The stationary distribution must satisfy
\[\pi_0 = \sum_{n=0}^\infty \frac{n+1}{n+2}\pi_n, \quad \pi_n = \frac{1}{(n+1)!}\pi_0, \quad \sum_{n=0}^\infty \pi_n = 1.\] From the second and third, we see that 
\begin{align*}
    1 &= \sum_{n=0}^\infty \pi_n\\
    &= \sum_{n=0}^\infty \frac{1}{(n+1)!}\pi_0\\
    &= \pi_0(e-1)
\end{align*}
and so $\pi_0 = (e-2)!.$ Thus, 
\[\pi_n = \frac{1}{(n+1)!}\frac{1}{(e-1)}\]
\end{solution}
    
    \item For \(x\in\mathbb{Z}\) with \(x\geq 0\), \(p(x,0)=1/(x+2)^{2}\) and \(p(x,x+1)=1-1/(x+2)^{2}\) (\(p(x,y)=0\) for all other \(y\)).
\end{enumerate}

\section*{Problem 5 (20 points)}
Consider the Markov chain with state space \(S=\{0,1,2,\cdots\}\) with transition probabilities
\[
p(0,0)=\frac{2}{3},\quad p(0,1)=\frac{1}{3},
\]
\[
p(x,x-1)=\frac{2}{3},\quad p(x,x+1)=\frac{1}{3},\quad x>0.
\]

\begin{enumerate}[label=(\alph*)]
    \item Show that this is positive recurrent by giving the invariant probability.
    \begin{solution}
A stationary probability satisfies the recursive relation
\[\pi_n = \frac{1}{3}\pi_{n-1} + \frac{2}{3}\pi_{n+1}.\] Then 
\[\alpha = \frac{1\pm \sqrt{1 - 4(\frac{1}{3}\cdot\frac{2}{3})}}{\frac{4}{3}} \in \{1, \frac{1}{2}\}.\]
Thus, 
\[\pi_n = \lambda_1 + \lambda_2\frac{1}{2^n}.\] We have that 
\[1 = \sum_{n=0}^\infty \pi_n = \sum_{n=0}^\infty \lambda_1 + \lambda_2\sum_{n=0}^\infty \frac{1}{2^n}\implies \lambda_1 = 0, \lambda_2 = \frac{1}{2}.\] Thus, 
\[\pi_n = \frac{1}{2^{n+1}}.\]
    \end{solution}
    
    \item For \(x>0\), let \(E_{x}\) denote the expected number of steps in the chain until it reaches the origin assuming that \(X_{0}=x\). Find \(E_{1}\). \textit{(Hint: first consider the expected return time starting at the origin and write \(E_{1}\) in terms of this.)}
\begin{solution}
    From above, since $\pi_0 = \frac{1}{2},$ then 
    \[E_0 = 2.\] However, we can also write 
    \[E_0 = \bbE[n >0: X_n = 0 \mid X_0 = x] = \bbE[\bbE[n>0 : X_n = 0 \mid X_1]] = (1 + 0)\frac{2}{3} + (1 + E_1)\frac{1}{3} = 1 + \frac{1}{3}E_1.\] Thus, 
    \[1 + \frac{1}{3}E_1 = 2 \implies E_1 = 3.\]
\end{solution}
    
    \item Find \(E_{x}\) for all \(x>0\).
\begin{solution}
    \[3 = E_1 = 1  + \frac{1}{3}E_{2} \implies E_2 = 6.\] In general, for $x \geq 2,$ 
    \[E_2 = 1 + \frac{2}{3}E_{1} + \frac{1}{3}E_{x+1}\implies 3E_n - 3 - 2E_{n-1} = E_{n+1}\]
\end{solution}
    
    \item Suppose we modify the chain so that
    \[
    p(0,1)=\frac{1}{4},\quad p(0,2)=\frac{1}{4},\quad p(0,3)=\frac{1}{4},\quad p(0,4)=\frac{1}{4}.
    \]
    The transitions for \(x>0\) are the same as before. Let \(\pi\) denote the invariant probability for this new chain. Find \(\pi(0)\).
    
    \item Find \(\pi(1)\) for this new chain.
\end{enumerate}

\section*{Problem 6 (10 points)}
Let $\{Y_j\}_{j\in\bbN}$ be independent, identically distributed integer-valued random variables which are not identically equal to zero. For a given value of \(X_{0}\in\mathbb{Z}\), let \(X_{n}=X_{0}+\sum_{j=1}^{n}Y_{j}\) for each \(n\geq 1\). We view \(\{X_{n}\}\) as a Markov chain taking values in \(\mathbb{Z}\). Show that \(\{X_{n}\}\) does not have a stationary distribution. Conclude that \(\{X_{n}\}\) is either null recurrent or transient, not positive recurrent. \textit{(Hint: assume for contradiction that there is a stationary distribution \(\pi\), and look at a value of \(n\in\mathbb{Z}\) such that \(\pi(n)\) is maximal).}

\section*{Problem 7 (Optional)}
In this exercise, we will establish Stirling's formula. Let \(X_{1},X_{2},\ldots\) be independent Poisson random variables with mean 1 and let \(Y_{n}=X_{1}+\cdots+X_{n}\), which is a Poisson random variable with mean \(n\). Let
\[
p(n,k)=\mathbb{P}\{Y_{n}=k\}=e^{-n}\frac{n^{k}}{k!}.
\]

\begin{enumerate}[label=(\alph*)]
    \item Use the central limit theorem to show that if \(a>0\),
    \[
    \lim_{n\to\infty}\sum_{n\leq k<n+a\sqrt{n}}p(n,k)=\int_{0}^{a}\frac{1}{\sqrt{2\pi}}e^{-x^{2}/2}\,dx.
    \]
    
    \item Show that if \(a>0\), \(n\) is a positive integer, and \(n\leq k<n+a\sqrt{n}\), then
    \[
    e^{-a^{2}}p(n,n)\leq p(n,k)\leq p(n,n).
    \]
    
    \item Use (a) and (b) to conclude that
    \[
    p(n,n)\sim\frac{1}{\sqrt{2\pi n}}.
    \]
    Stirling's formula follows immediately.
\end{enumerate}



\end{document}