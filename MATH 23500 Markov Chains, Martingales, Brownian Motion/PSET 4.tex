\documentclass[11pt]{article}
\usepackage{float}
\usepackage{tikz}
\usetikzlibrary{automata, positioning}
% NOTE: Add in the relevant information to the commands below; or, if you'll be using the same information frequently, add these commands at the top of paolo-pset.tex file. 
\newcommand{\name}{Agustín Esteva}
\newcommand{\email}{aesteva@uchicago.edu}
\newcommand{\classnum}{23500}
\newcommand{\subject}{Markov Chains, Martingales, and Brownian Motion}
\newcommand{\instructors}{Stephen Yearwood}
\newcommand{\assignment}{Problem Set 4}
\newcommand{\semester}{Spring 2025}
\newcommand{\duedate}{05/02/2025}
\newcommand{\bA}{\mathbf{A}}
\newcommand{\bB}{\mathbf{B}}
\newcommand{\bC}{\mathbf{C}}
\newcommand{\bD}{\mathbf{D}}
\newcommand{\bE}{\mathbf{E}}
\newcommand{\bF}{\mathbf{F}}
\newcommand{\bG}{\mathbf{G}}
\newcommand{\bH}{\mathbf{H}}
\newcommand{\bI}{\mathbf{I}}
\newcommand{\bJ}{\mathbf{J}}
\newcommand{\bK}{\mathbf{K}}
\newcommand{\bL}{\mathbf{L}}
\newcommand{\bM}{\mathbf{M}}
\newcommand{\bN}{\mathbf{N}}
\newcommand{\bO}{\mathbf{O}}
\newcommand{\bP}{\mathbf{P}}
\newcommand{\bQ}{\mathbf{Q}}
\newcommand{\bR}{\mathbf{R}}
\newcommand{\bS}{\mathbf{S}}
\newcommand{\bT}{\mathbf{T}}
\newcommand{\bU}{\mathbf{U}}
\newcommand{\bV}{\mathbf{V}}
\newcommand{\bW}{\mathbf{W}}
\newcommand{\bX}{\mathbf{X}}
\newcommand{\bY}{\mathbf{Y}}
\newcommand{\bZ}{\mathbf{Z}}
\newcommand{\Vol}{\text{Vol}}

%% blackboard bold math capitals
\newcommand{\bbA}{\mathbb{A}}
\newcommand{\bbB}{\mathbb{B}}
\newcommand{\bbC}{\mathbb{C}}
\newcommand{\bbD}{\mathbb{D}}
\newcommand{\bbE}{\mathbb{E}}
\newcommand{\bbF}{\mathbb{F}}
\newcommand{\bbG}{\mathbb{G}}
\newcommand{\bbH}{\mathbb{H}}
\newcommand{\bbI}{\mathbb{I}}
\newcommand{\bbJ}{\mathbb{J}}
\newcommand{\bbK}{\mathbb{K}}
\newcommand{\bbL}{\mathbb{L}}
\newcommand{\bbM}{\mathbb{M}}
\newcommand{\bbN}{\mathbb{N}}
\newcommand{\bbO}{\mathbb{O}}
\newcommand{\bbP}{\mathbb{P}}
\newcommand{\bbQ}{\mathbb{Q}}
\newcommand{\bbR}{\mathbb{R}}
\newcommand{\bbS}{\mathbb{S}}
\newcommand{\bbT}{\mathbb{T}}
\newcommand{\bbU}{\mathbb{U}}
\newcommand{\bbV}{\mathbb{V}}
\newcommand{\bbW}{\mathbb{W}}
\newcommand{\bbX}{\mathbb{X}}
\newcommand{\bbY}{\mathbb{Y}}
\newcommand{\bbZ}{\mathbb{Z}}

%% script math capitals
\newcommand{\sA}{\mathscr{A}}
\newcommand{\sB}{\mathscr{B}}
\newcommand{\sC}{\mathscr{C}}
\newcommand{\sD}{\mathscr{D}}
\newcommand{\sE}{\mathscr{E}}
\newcommand{\sF}{\mathscr{F}}
\newcommand{\sG}{\mathscr{G}}
\newcommand{\sH}{\mathscr{H}}
\newcommand{\sI}{\mathscr{I}}
\newcommand{\sJ}{\mathscr{J}}
\newcommand{\sK}{\mathscr{K}}
\newcommand{\sL}{\mathscr{L}}
\newcommand{\sM}{\mathscr{M}}
\newcommand{\sN}{\mathscr{N}}
\newcommand{\sO}{\mathscr{O}}
\newcommand{\sP}{\mathscr{P}}
\newcommand{\sQ}{\mathscr{Q}}
\newcommand{\sR}{\mathscr{R}}
\newcommand{\sS}{\mathscr{S}}
\newcommand{\sT}{\mathscr{T}}
\newcommand{\sU}{\mathscr{U}}
\newcommand{\sV}{\mathscr{V}}
\newcommand{\sW}{\mathscr{W}}
\newcommand{\sX}{\mathscr{X}}
\newcommand{\sY}{\mathscr{Y}}
\newcommand{\sZ}{\mathscr{Z}}


\renewcommand{\emptyset}{\O}

\newcommand{\abs}[1]{\lvert #1 \rvert}
\newcommand{\norm}[1]{\lVert #1 \rVert}
\newcommand{\sm}{\setminus}


\newcommand{\sarr}{\rightarrow}
\newcommand{\arr}{\longrightarrow}

% NOTE: Defining collaborators is optional; to not list collaborators, comment out the line below.
%\newcommand{\collaborators}{Alyssa P. Hacker (\texttt{aphacker}), Ben Bitdiddle (\texttt{bitdiddle})}

% Copyright 2021 Paolo Adajar (padajar.com, paoloadajar@mit.edu)
% 
% Permission is hereby granted, free of charge, to any person obtaining a copy of this software and associated documentation files (the "Software"), to deal in the Software without restriction, including without limitation the rights to use, copy, modify, merge, publish, distribute, sublicense, and/or sell copies of the Software, and to permit persons to whom the Software is furnished to do so, subject to the following conditions:
%
% The above copyright notice and this permission notice shall be included in all copies or substantial portions of the Software.
% 
% THE SOFTWARE IS PROVIDED "AS IS", WITHOUT WARRANTY OF ANY KIND, EXPRESS OR IMPLIED, INCLUDING BUT NOT LIMITED TO THE WARRANTIES OF MERCHANTABILITY, FITNESS FOR A PARTICULAR PURPOSE AND NONINFRINGEMENT. IN NO EVENT SHALL THE AUTHORS OR COPYRIGHT HOLDERS BE LIABLE FOR ANY CLAIM, DAMAGES OR OTHER LIABILITY, WHETHER IN AN ACTION OF CONTRACT, TORT OR OTHERWISE, ARISING FROM, OUT OF OR IN CONNECTION WITH THE SOFTWARE OR THE USE OR OTHER DEALINGS IN THE SOFTWARE.

\usepackage{fullpage}
\usepackage{enumitem}
\usepackage{amsfonts, amssymb, amsmath,amsthm}
\usepackage{mathtools}
\usepackage[pdftex, pdfauthor={\name}, pdftitle={\classnum~\assignment}]{hyperref}
\usepackage[dvipsnames]{xcolor}
\usepackage{bbm}
\usepackage{graphicx}
\usepackage{mathrsfs}
\usepackage{pdfpages}
\usepackage{tabularx}
\usepackage{pdflscape}
\usepackage{makecell}
\usepackage{booktabs}
\usepackage{natbib}
\usepackage{caption}
\usepackage{subcaption}
\usepackage{physics}
\usepackage[many]{tcolorbox}
\usepackage{version}
\usepackage{ifthen}
\usepackage{cancel}
\usepackage{listings}
\usepackage{courier}

\usepackage{tikz}
\usepackage{istgame}

\hypersetup{
	colorlinks=true,
	linkcolor=blue,
	filecolor=magenta,
	urlcolor=blue,
}

\setlength{\parindent}{0mm}
\setlength{\parskip}{2mm}

\setlist[enumerate]{label=({\alph*})}
\setlist[enumerate, 2]{label=({\roman*})}

\allowdisplaybreaks[1]

\newcommand{\psetheader}{
	\ifthenelse{\isundefined{\collaborators}}{
		\begin{center}
			{\setlength{\parindent}{0cm} \setlength{\parskip}{0mm}
				
				{\textbf{\classnum~\semester:~\assignment} \hfill \name}
				
				\subject \hfill \href{mailto:\email}{\tt \email}
				
				Instructor(s):~\instructors \hfill Due Date:~\duedate	
				
				\hrulefill}
		\end{center}
	}{
		\begin{center}
			{\setlength{\parindent}{0cm} \setlength{\parskip}{0mm}
				
				{\textbf{\classnum~\semester:~\assignment} \hfill \name\footnote{Collaborator(s): \collaborators}}
				
				\subject \hfill \href{mailto:\email}{\tt \email}
				
				Instructor(s):~\instructors \hfill Due Date:~\duedate	
				
				\hrulefill}
		\end{center}
	}
}

\renewcommand{\thepage}{\classnum~\assignment \hfill \arabic{page}}

\makeatletter
\def\points{\@ifnextchar[{\@with}{\@without}}
\def\@with[#1]#2{{\ifthenelse{\equal{#2}{1}}{{[1 point, #1]}}{{[#2 points, #1]}}}}
\def\@without#1{\ifthenelse{\equal{#1}{1}}{{[1 point]}}{{[#1 points]}}}
\makeatother

\newtheoremstyle{theorem-custom}%
{}{}%
{}{}%
{\itshape}{.}%
{ }%
{\thmname{#1}\thmnumber{ #2}\thmnote{ (#3)}}

\theoremstyle{theorem-custom}

\newtheorem{theorem}{Theorem}
\newtheorem{lemma}[theorem]{Lemma}
\newtheorem{example}[theorem]{Example}

\newenvironment{problem}[1]{\color{black} #1}{}

\newenvironment{solution}{%
	\leavevmode\begin{tcolorbox}[breakable, colback=green!5!white,colframe=green!75!black, enhanced jigsaw] \proof[\scshape Solution:] \setlength{\parskip}{2mm}%
	}{\renewcommand{\qedsymbol}{$\blacksquare$} \endproof \end{tcolorbox}}

\newenvironment{reflection}{\begin{tcolorbox}[breakable, colback=black!8!white,colframe=black!60!white, enhanced jigsaw, parbox = false]\textsc{Reflections:}}{\end{tcolorbox}}

\newcommand{\qedh}{\renewcommand{\qedsymbol}{$\blacksquare$}\qedhere}

\definecolor{mygreen}{rgb}{0,0.6,0}
\definecolor{mygray}{rgb}{0.5,0.5,0.5}
\definecolor{mymauve}{rgb}{0.58,0,0.82}

% from https://github.com/satejsoman/stata-lstlisting
% language definition
\lstdefinelanguage{Stata}{
	% System commands
	morekeywords=[1]{regress, reg, summarize, sum, display, di, generate, gen, bysort, use, import, delimited, predict, quietly, probit, margins, test},
	% Reserved words
	morekeywords=[2]{aggregate, array, boolean, break, byte, case, catch, class, colvector, complex, const, continue, default, delegate, delete, do, double, else, eltypedef, end, enum, explicit, export, external, float, for, friend, function, global, goto, if, inline, int, local, long, mata, matrix, namespace, new, numeric, NULL, operator, orgtypedef, pointer, polymorphic, pragma, private, protected, public, quad, real, return, rowvector, scalar, short, signed, static, strL, string, struct, super, switch, template, this, throw, transmorphic, try, typedef, typename, union, unsigned, using, vector, version, virtual, void, volatile, while,},
	% Keywords
	morekeywords=[3]{forvalues, foreach, set},
	% Date and time functions
	morekeywords=[4]{bofd, Cdhms, Chms, Clock, clock, Cmdyhms, Cofc, cofC, Cofd, cofd, daily, date, day, dhms, dofb, dofC, dofc, dofh, dofm, dofq, dofw, dofy, dow, doy, halfyear, halfyearly, hh, hhC, hms, hofd, hours, mdy, mdyhms, minutes, mm, mmC, mofd, month, monthly, msofhours, msofminutes, msofseconds, qofd, quarter, quarterly, seconds, ss, ssC, tC, tc, td, th, tm, tq, tw, week, weekly, wofd, year, yearly, yh, ym, yofd, yq, yw,},
	% Mathematical functions
	morekeywords=[5]{abs, ceil, cloglog, comb, digamma, exp, expm1, floor, int, invcloglog, invlogit, ln, ln1m, ln, ln1p, ln, lnfactorial, lngamma, log, log10, log1m, log1p, logit, max, min, mod, reldif, round, sign, sqrt, sum, trigamma, trunc,},
	% Matrix functions
	morekeywords=[6]{cholesky, coleqnumb, colnfreeparms, colnumb, colsof, corr, det, diag, diag0cnt, el, get, hadamard, I, inv, invsym, issymmetric, J, matmissing, matuniform, mreldif, nullmat, roweqnumb, rownfreeparms, rownumb, rowsof, sweep, trace, vec, vecdiag, },
	% Programming functions
	morekeywords=[7]{autocode, byteorder, c, _caller, chop, abs, clip, cond, e, fileexists, fileread, filereaderror, filewrite, float, fmtwidth, has_eprop, inlist, inrange, irecode, matrix, maxbyte, maxdouble, maxfloat, maxint, maxlong, mi, minbyte, mindouble, minfloat, minint, minlong, missing, r, recode, replay, return, s, scalar, smallestdouble,},
	% Random-number functions
	morekeywords=[8]{rbeta, rbinomial, rcauchy, rchi2, rexponential, rgamma, rhypergeometric, rigaussian, rlaplace, rlogistic, rnbinomial, rnormal, rpoisson, rt, runiform, runiformint, rweibull, rweibullph,},
	% Selecting time-span functions
	morekeywords=[9]{tin, twithin,},
	% Statistical functions
	morekeywords=[10]{betaden, binomial, binomialp, binomialtail, binormal, cauchy, cauchyden, cauchytail, chi2, chi2den, chi2tail, dgammapda, dgammapdada, dgammapdadx, dgammapdx, dgammapdxdx, dunnettprob, exponential, exponentialden, exponentialtail, F, Fden, Ftail, gammaden, gammap, gammaptail, hypergeometric, hypergeometricp, ibeta, ibetatail, igaussian, igaussianden, igaussiantail, invbinomial, invbinomialtail, invcauchy, invcauchytail, invchi2, invchi2tail, invdunnettprob, invexponential, invexponentialtail, invF, invFtail, invgammap, invgammaptail, invibeta, invibetatail, invigaussian, invigaussiantail, invlaplace, invlaplacetail, invlogistic, invlogistictail, invnbinomial, invnbinomialtail, invnchi2, invnF, invnFtail, invnibeta, invnormal, invnt, invnttail, invpoisson, invpoissontail, invt, invttail, invtukeyprob, invweibull, invweibullph, invweibullphtail, invweibulltail, laplace, laplaceden, laplacetail, lncauchyden, lnigammaden, lnigaussianden, lniwishartden, lnlaplaceden, lnmvnormalden, lnnormal, lnnormalden, lnwishartden, logistic, logisticden, logistictail, nbetaden, nbinomial, nbinomialp, nbinomialtail, nchi2, nchi2den, nchi2tail, nF, nFden, nFtail, nibeta, normal, normalden, npnchi2, npnF, npnt, nt, ntden, nttail, poisson, poissonp, poissontail, t, tden, ttail, tukeyprob, weibull, weibullden, weibullph, weibullphden, weibullphtail, weibulltail,},
	% String functions 
	morekeywords=[11]{abbrev, char, collatorlocale, collatorversion, indexnot, plural, plural, real, regexm, regexr, regexs, soundex, soundex_nara, strcat, strdup, string, strofreal, string, strofreal, stritrim, strlen, strlower, strltrim, strmatch, strofreal, strofreal, strpos, strproper, strreverse, strrpos, strrtrim, strtoname, strtrim, strupper, subinstr, subinword, substr, tobytes, uchar, udstrlen, udsubstr, uisdigit, uisletter, ustrcompare, ustrcompareex, ustrfix, ustrfrom, ustrinvalidcnt, ustrleft, ustrlen, ustrlower, ustrltrim, ustrnormalize, ustrpos, ustrregexm, ustrregexra, ustrregexrf, ustrregexs, ustrreverse, ustrright, ustrrpos, ustrrtrim, ustrsortkey, ustrsortkeyex, ustrtitle, ustrto, ustrtohex, ustrtoname, ustrtrim, ustrunescape, ustrupper, ustrword, ustrwordcount, usubinstr, usubstr, word, wordbreaklocale, worcount,},
	% Trig functions
	morekeywords=[12]{acos, acosh, asin, asinh, atan, atanh, cos, cosh, sin, sinh, tan, tanh,},
	morecomment=[l]{//},
	% morecomment=[l]{*},  // `*` maybe used as multiply operator. So use `//` as line comment.
	morecomment=[s]{/*}{*/},
	% The following is used by macros, like `lags'.
	morestring=[b]{`}{'},
	% morestring=[d]{'},
	morestring=[b]",
	morestring=[d]",
	% morestring=[d]{\\`},
	% morestring=[b]{'},
	sensitive=true,
}

\lstset{ 
	backgroundcolor=\color{white},   % choose the background color; you must add \usepackage{color} or \usepackage{xcolor}; should come as last argument
	basicstyle=\footnotesize\ttfamily,        % the size of the fonts that are used for the code
	breakatwhitespace=false,         % sets if automatic breaks should only happen at whitespace
	breaklines=true,                 % sets automatic line breaking
	captionpos=b,                    % sets the caption-position to bottom
	commentstyle=\color{mygreen},    % comment style
	deletekeywords={...},            % if you want to delete keywords from the given language
	escapeinside={\%*}{*)},          % if you want to add LaTeX within your code
	extendedchars=true,              % lets you use non-ASCII characters; for 8-bits encodings only, does not work with UTF-8
	firstnumber=0,                % start line enumeration with line 1000
	frame=single,	                   % adds a frame around the code
	keepspaces=true,                 % keeps spaces in text, useful for keeping indentation of code (possibly needs columns=flexible)
	keywordstyle=\color{blue},       % keyword style
	language=Octave,                 % the language of the code
	morekeywords={*,...},            % if you want to add more keywords to the set
	numbers=left,                    % where to put the line-numbers; possible values are (none, left, right)
	numbersep=5pt,                   % how far the line-numbers are from the code
	numberstyle=\tiny\color{mygray}, % the style that is used for the line-numbers
	rulecolor=\color{black},         % if not set, the frame-color may be changed on line-breaks within not-black text (e.g. comments (green here))
	showspaces=false,                % show spaces everywhere adding particular underscores; it overrides 'showstringspaces'
	showstringspaces=false,          % underline spaces within strings only
	showtabs=false,                  % show tabs within strings adding particular underscores
	stepnumber=2,                    % the step between two line-numbers. If it's 1, each line will be numbered
	stringstyle=\color{mymauve},     % string literal style
	tabsize=2,	                   % sets default tabsize to 2 spaces
%	title=\lstname,                   % show the filename of files included with \lstinputlisting; also try caption instead of title
	xleftmargin=0.25cm
}

% NOTE: To compile a version of this pset without problems, solutions, or reflections, uncomment the relevant line below.

%\excludeversion{problem}
%\excludeversion{solution}
%\excludeversion{reflection}

\begin{document}	
	
	% Use the \psetheader command at the beginning of a pset. 
	\psetheader
\section*{Problem 1}
\begin{problem}
    Consider the following variant of the branching process. At each time \( n \), each individual produces offspring independently with offspring distribution \(\{p_j\}_{j \geq 0}\), then dies with probability \( q \in (0,1) \). So, each individual reproduces \( k \) times where \( k \) is its lifetime (\( k \) is a random positive integer). Note that \( q = 1 \) for the version of the branching process discussed in class. Assuming that \( X_0 = 1 \), show that if \( \phi(t) = \sum_{j=0}^\infty t^j p_j \) is the generating function, then the extinction probability is the smallest positive solution to
\[
q\phi(t) + (1-q)t\phi(t) = t.
\]

\end{problem}
\begin{solution}
Conditioning, 
\begin{align*}
  a &= \bbP\{\text{extinction} \mid X_0 = 1\}\\ &= \sum_{k=0}^\infty \bbP\{X_1 = k\}\bbP\{\text{extinction} \mid X_1 = k \}  \\
  &= \sum_{k=0}^\infty \bbP\{X_1 = k \mid \text{fucking dies}\}\bbP\{\text{fucking dies}\} + \bbP\{X_1 = k \mid \text{lives}\}\bbP\{\text{lives}\})a^k\\
  &= qp_0 + \sum_{k=1}^\infty (p_kq + p_{k-1}(1-q))a^k\\
  &= \sum_{k=0}^\infty (p_kq + p_{k-1}(1-q))a^k\\
  &= \sum_{k=1}^\infty p_k qa^k + \sum_{k=1}^\infty (1-q)p_{k-1}a^k\\
  &= \sum_{k=1}^\infty p_k qa^k + \sum_{k=0}^\infty (1-q)p_{k}a^{k+1}\\
  &= q\phi(a) + (1-q) a \phi(a)
\end{align*} We claim that the extinction probability is the smallest possible solution to $\varphi(a) = a.$ We note  by the fourth line above that the generating function is
\[\varphi(a) = qp_0 + \sum_{k=1}^\infty (p_kq + p_{k-1}(1-q))a^k,\] and so by a result in class, the extinction probability is given by the smallest positive solution to 
$a=  \varphi(a) = \cdots = q\phi(a) + (1-q)a\phi(a)$
\end{solution} 



\newpage
\section*{Problem 2}
\begin{problem}
    Let $\{X_n\}$ be a branching process with offspring distribution $\{p_j\}_{j \geq 0}$ and let $\phi(a)$ be the generating function $\sum_{j=0}^\infty p_ka^k.$ We let $\phi^{(n)} = \phi \circ \phi \circ \cdots \circ \phi$ composed $n$ times. Show that for $n\geq 1,$ 
    \[\bbP\{X_n = 0 \mid X_{n-1} \neq 0\} = \frac{\phi^{(n-1)}(p_0) - \phi^{(n-1)}(0)}{1 - \phi^{(n-1)}(0)}.\]
\end{problem}
\begin{solution}
By the law of total probability, we have that 
\[\bbP\{X_n = 0\} = \bbP\{X_n = 0, X_{n-1} = 0\} + \bbP\{X_{n} = 0, X_{n-1} \neq 0\}\]
    So we use Bayes rule:
    \begin{align*}
        \bbP\{X_n = 0 \mid X_{n-1} \neq 0\} &= \frac{\bbP\{X_n = 0, X_{n-1} \neq 0\}}{\bbP\{X_{n-1} \neq 0\}}\\
        &= \frac{\bbP\{X_n = 0, X_{n-1} \neq 0\}}{1 - \bbP\{X_{n-1} = 0\}}\\
        &= \frac{\bbP\{X_n = 0\} - \bbP\{X_n = 0, X_{n-1} = 0\}}{1 - \bbP\{X_{n-1} = 0\}}\\
        &=\frac{\bbP\{X_n = 0\} - \bbP\{ X_{n-1} = 0\}}{1 - \bbP\{X_{n-1} = 0\}}\\
        &= \frac{\phi^{(n)}(0) - \phi^{(n-1)}(0)}{1 - \phi^{(n-1)}(0)},
    \end{align*}
    and by a result in class
    \[\phi^{(n)}(0) = \phi^{(n-1)}(\phi(0)) = \phi^{(n-1)}(p_0),\] and so we are done.
    
\end{solution}


\newpage
\section*{Problem 3}
Let \(\{X_t\}\) and \(\{Y_t\}\) be independent Poisson processes with rates \(\lambda_1\) and \(\lambda_2\), respectively. We imagine that \(\{X_t\}\) and \(\{Y_t\}\) count the number of calls on two different phone lines, with the time \(t\) measured in hours.

\begin{enumerate}
    \item[(a)] Find the probability that there were 5 calls on line 1 between times 0 and 1 and 5 calls on line 1 between times 1 and 2.
    \begin{solution}
    We are asked to find $\bbP\{X_1 - X_0 = 5, X_2 - X_1 = 5\}.$ By the memorylessness of the Poisson-Process, these events are independent and are both distributed Poisson with parameter $\lambda_1.$ Thus
    \[\bbP\{X_1 - X_0 = 5, X_2 - X_1 = 5\}= \bbP\{X_1 = 5\}\bbP\{X_2 - X_1 = 5\} = \boxed{(\frac{e^{\lambda_1}\lambda_1^5}{5!})^2}\]
    \end{solution}
    \item[(b)] Given that there were 10 total calls (on both lines) between time 1 and time 2, find the conditional probability that all 10 calls were on line 1.
    \begin{solution}
        We use the formula derived in problem session to see that 
        \begin{align*}
        \bbP\{X_2 - X_1 = 10 \mid (X_2 - X_1) + (Y_2 - Y_1) = 10\}& = \bbP\{X_1 = 10 \mid X_1 + Y_1 = 10\}\\ &= \binom{10}{10}(\frac{\lambda_1}{\lambda_1 + \lambda_2})^{10}(\frac{\lambda_2}{\lambda_1 + \lambda_2})^{0}\\ &= \boxed{(\frac{\lambda_1}{\lambda_1 + \lambda_2})^{10}    }
        \end{align*}
        
    \end{solution}
    \item[(c)] Let \(T_1\) (resp. \(T_2\)) be the time of the first call on line 1 (resp. line 2). Find \(\mathbb{E}[\min\{T_1, T_2\}]\).
    \begin{solution}
        From class, we know that if 
        \[\tau := \min\{T_1, T_2\},\] then since $T_1 \sim \exp{\lambda_1}$ and $T_2 \sim \exp{\lambda_2},$ then $\tau \sim \exp{\lambda_1 + \lambda_2},$ and thus 
        \[\bbE[\tau] = \boxed{\frac{1}{\lambda_1 + \lambda_2}}\]
    \end{solution}
    \item[(d)] Find the distribution of the number of calls on line 1 before time \(T_2\) (i.e., find \(\mathbb{P}[X_{T_2} = k]\) for each \(k\)).
    \begin{solution}
We use the Gamma function,
\begin{align*}
    \bbP\{X_{T_2} = k\} &= \int_0^\infty \bbP\{X_t = k \mid T_2 = t\}\bbP\{T_2 = t\}\, dt\\
    &= \int_0^\infty \bbP\{X_t = k \mid T_2 = t\}(\lambda_2e^{-\lambda_2 t})\, dt\\
    &= \int_0^\infty \frac{e^{-\lambda_1 t}(\lambda_1t)^k}{k!}\lambda_2 e^{-\lambda_2 t}\, dt\\
    &= \frac{\lambda_2}{k!}\int_0^\infty (\lambda_1 t)^k e^{-(\lambda_1 + \lambda_2)t}\, dt\\
    &= \frac{\lambda_2 \lambda_1^k}{k!}\int_0^\infty  t^k e^{-(\lambda_1 + \lambda_2)t}\, dt\\
    &= \frac{\lambda_2 \lambda_1^k}{(\lambda_1 + \lambda_2)k!}\int_0^\infty (\frac{u}{\lambda_1 + \lambda_2})^k e^{-u}\, du\\
    &= \frac{\lambda_2 \lambda_1^k}{k!(\lambda_1 + \lambda_2)^{k+1}}\Gamma(k + 1)\\
    &= \frac{k!}{k!} \frac{\lambda_2}{\lambda_1 + \lambda_2} \left(\frac{\lambda_1}{\lambda_1 + \lambda_2}\right)^k
\end{align*}
Thus, 
\[\boxed{X_{T_2} \sim \text{Geometric}( \frac{\lambda_2}{\lambda_1 + \lambda_2})}\]
    \end{solution}
\end{enumerate}

\newpage
\section*{Problem 4}
\begin{problem}
    Suppose that traffic on a road follows a Poisson process with rate \(\lambda > 0\) cars per minute. A chicken needs a (time) gap of length at least \(c\) minutes in the traffic to cross the road. In this problem, we wish to compute the time the chicken will have to wait to cross the road. To that end, we let \(\tau_j\) be the arrival time of the \(j\)-th car, and let 
\[
J := \min \{ j : \tau_j - \tau_{j-1} > c \}.
\]
Ideally, the chicken will start to cross the road at time \(\tau_{J-1}\) and complete its journey at time \(\tau_{J-1} + c\).

\begin{enumerate}
    \item[(a)] Suppose \(T\) is exponentially distributed with rate \(\lambda > 0\). Compute \(\mathbb{E}[T \mid T < c]\).
    \begin{solution}
    First, note that 
    \[\bbP\{T < c\} = \int_0^c \bbP\{T = t\}\, dt\]
        Using Bayes rule and the law of total expectation, we have that 
        \begin{align*}
            \bbE[T \mid T< c] &= \int_0^\infty t \bbP\{T = t\mid T<c\}\, dt\\
             &=\int_0^\infty t \frac{\bbP\{T = t ,T< c\}}{\bbP\{T< c\}}\, dt\\
             &= \int_0^c t\frac{\bbP\{T = t\}}{\bbP\{T < c\}}\,dt\\
             &= \int_0^c t \frac{f_T(t)}{F_T(t)}\, dt\\
             &= \int_0^c t \frac{\lambda e^{-\lambda t}}{1- e^{-\lambda c}}\, dt\\
             &= \frac{1}{1- e^{-\lambda c}}\int_0^c t\lambda e^{-\lambda t}\, dt\\
             &= \frac{1}{1- e^{-\lambda c}} \frac{1 - e^{-c\lambda}(c\lambda + 1)}{\lambda}\\
             &= \boxed{\frac{1}{\lambda} - \frac{c}{1- e^{-\lambda c}}}
        \end{align*}
    \end{solution}
    \item[(b)] Use part (a) to find \(\mathbb{E}[\tau_{J-1} + c]\).
\begin{solution}
Consider that $J \sim \text{Geometric}$ since we need to 'fail' $J-1$ times, that is, the first $J-1$ cars must come within $c$ of each other, and succeed once, the $J$th car will be $>c$ time from the last. Using various properties of the Poisson process, such as memorylessness and independence of arrival times, then
\begin{align*}
\bbP\{J = j\} &= \bbP\{\tau_{j-1} - \tau_{j-2} \leq  c, \tau_{j-2} - \tau_{j-3} \leq c, \dots, \tau_{1}\leq  c\} \bbP\{\tau_{j} - \tau_{j-1} >c\}\\
&= \bbP\{\tau_{1} \leq c\}^{j-1} \bbP\{\tau_{1} >c\}\\
&= (1-e^{-\lambda c})^{j-1}(e^{-\lambda c})
\end{align*}
and so 
\[\bbE[J - 1] = \frac{1- e^{-\lambda c}}{e^{-\lambda c}}\]
Using the fact that the arrival times are independent, wenote that
    \begin{align*}
        \bbE[\tau_{J-1} + c] &= \bbE[\tau_{J-1}] + c\\
        &= \bbE[\bbE[\tau_{j-1} \mid J = j]]+ c\\
        &= c + \sum_{j=1}^\infty \bbE[\tau_{j-1} \mid J = j] \bbP\{J = j\}\\
        &= c + \sum_{j=1}^\infty \bbP\{J = j\}\left( \bbE[\tau_{j-1} - \tau_{j-2} \mid J = j] + \bbE[\tau_{j-2} - \tau_{j-3} \mid J = j] + \cdots + \bbE[\tau_{1} - \tau_{0} \mid J = j]\right)\\
        &= c + \sum_{j=1}^\infty \bbP\{J = j\}\left( \bbE[\tau_{j-1} - \tau_{j-2} \mid \tau_{j-1} - \tau_{j-2} <c] + \cdots + \bbE[\tau_{1} - \tau_{0} \mid \tau_{1} - \tau_{0} <c]\right)\\
        &= c +  \sum_{j=1}^\infty \bbP\{J = j\}\left( (j-1)\bbE[\tau_{1}\mid \tau_{1}<c]\right)\\
        &= c + \bbE[\tau_{1}\mid \tau_{1}<c]\sum_{j=1}^\infty \bbP\{J-1 = j-1\} (j-1)\\
        &= c +\bbE[\tau_{1}\mid \tau_{1}<c] \bbE[J-1]\\
        &=\boxed{ c + \frac{1- e^{-\lambda c}}{e^{-\lambda c}} (\frac{1}{\lambda}- \frac{c}{1- e^{-\lambda c}})}
    \end{align*}
\end{solution}
\end{enumerate}

\end{problem}





\newpage
\section*{Problem 5}
Consider the continuous-time Markov chain \(\{X_t\}_{t \geq 0}\) with state space \(\{1, 2, 3\}\) and infinitesimal generator matrix:
\[
A = \begin{pmatrix}
-1 & 1 & 0 \\
4 & -5 & 1 \\
0 & 4 & -4
\end{pmatrix}
\]

\begin{enumerate}
    \item[(a)] If we start in state 1, what is the expected time that we move to a different state?
    \begin{solution}
        The rate of leaving state $1$ is given by $A_{1,2} + A_{1,3} = 1,$ and thus if $T$ denotes the time when we leave, we have that $T\sim \exp{\alpha(1)}$ and thus \[T \sim \exp{1} \implies \boxed{\bbE[T] = 1}\]
    \end{solution}
    \item[(b)] If we start in state 2, what is the expected time that we move to a different state?
\begin{solution}
    Similarly to above, if we define $T$ as leaving state two, then $T\sim \exp{\alpha(2)},$ and so  \[\boxed{\bbE[T] = \frac{1}{5}}\]
\end{solution}
    \item[(c)] Explain why this Markov chain is irreducible and find the stationary distribution \(\pi\).
\begin{solution}
    \begin{align*}
         0 &= \det(A - \lambda I)\\
         &= \det(\begin{pmatrix}
             -1 - \lambda & 1 & 0\\
             4 & -5 - \lambda & 1\\
             0 & 4 & -4 - \lambda
         \end{pmatrix})\\
         &= -\lambda^3 - 10\lambda^2 - 21\lambda \\
         &= \lambda(\lambda + 7)(\lambda +3)
    \end{align*}
    Solving for the null spaces we find that 
    \[v_1 = \begin{pmatrix}
        1\\1\\1
    \end{pmatrix}, \quad v_2 = \begin{pmatrix}
        \frac{1}{8} \\
        \frac{-3}{4}\\
        1
    \end{pmatrix}, \quad v_3 = \begin{pmatrix}
        -\frac{1}{8}\\
        \frac{1}{4}\\
        1
    \end{pmatrix},\] and so 
    \begin{align*}
        P_t &= e^{tA}\\
        &= \sum_{n=0}^\infty \frac{(tA)^n}{n!}\\
        &= \begin{pmatrix}
            1 & \frac{1}{8} & \frac{-1}{8}\\
            1 & \frac{-3}{4} & \frac{1}{4}\\
            1 & 1 & 1
        \end{pmatrix}\left(\sum_{n=1}^\infty \frac{1}{n!}\begin{pmatrix}
            0 & 0 & 0\\
            0 & -7t & 0\\
            0 & 0 &-3t
        \end{pmatrix}^n\right) \begin{pmatrix}
            1 & \frac{1}{8} & \frac{-1}{8}\\
            1 & \frac{-3}{4} & \frac{1}{4}\\
            1 & 1 & 1
        \end{pmatrix}^{-1}\\
        &= \begin{pmatrix}
            1 & \frac{1}{8} & \frac{-1}{8}\\
            1 & \frac{-3}{4} & \frac{1}{4}\\
            1 & 1 & 1
        \end{pmatrix}\begin{pmatrix}
            1 & 0 & 0\\
            0 & e^{-7t} & 0\\
            0 & 0 &e^{-3t}
        \end{pmatrix}\begin{pmatrix}
            1 & \frac{1}{8} & \frac{-1}{8}\\
            1 & \frac{-3}{4} & \frac{1}{4}\\
            1 & 1 & 1
        \end{pmatrix}^{-1}\\
        &= \frac{e^{-7t}}{84}
\begin{pmatrix}
2\left(7 e^{4t} + 32 e^{7t} + 3\right) & -7 e^{4t} + 16 e^{7t} - 9 & -7 e^{4t} + 4 e^{7t} + 3 \\
-4\left(7 e^{4t} - 16 e^{7t} + 9\right) & 2\left(7 e^{4t} + 8 e^{7t} + 27\right) & -2\left(-7 e^{4t} - 2 e^{7t} + 9\right) \\
16\left(-7 e^{4t} + 4 e^{7t} + 3\right) & -8\left(-7 e^{4t} - 2 e^{7t} + 9\right) & 4\left(14 e^{4t} + e^{7t} + 6\right)
\end{pmatrix}
    \end{align*}
    Thus, the Markov chain is irreducible because $p_t(x,y) >0$ for any $x,y \in S.$ Finding the stationary distribution, we send $t\to \infty$ in $P_t$ above, noting that any term with a power less than $7t$ is going to get obliterated:
\begin{align*}
    \lim_{t\to \infty}P_t &= \lim_{t\to \infty} \frac{e^{-7t}}{84}
\begin{pmatrix} 
2\left(32 e^{7t}\right) & 16 e^{7t}& 4e^{7t} \\
-4\left( - 16 e^{7t}\right) & 2\left(8 e^{7t} \right) & -2\left(- 2 e^{7t} \right) \\
16\left(4 e^{7t}\right) & -8\left(- 2 e^{7t} \right) & 4\left( e^{7t}\right)
\end{pmatrix}\\
&= \begin{pmatrix}
    \frac{64}{84} & \frac{16}{84} & \frac{4}{84}\\
    \frac{64}{84} & \frac{16}{84} & \frac{4}{84}\\
    \frac{64}{84} & \frac{16}{84} & \frac{4}{84}\\
\end{pmatrix}\\
&\implies \boxed{\pi = \begin{pmatrix}
    \frac{16}{21} & \frac{4}{21} & \frac{1}{21}
\end{pmatrix}}
\end{align*}
\end{solution}
    \item[(d)] Find \(p_t(1,3)\) for each \(t > 0\).
    \begin{solution}
        From the above, one can see that 
        \[\boxed{p_t (1,3) = \frac{e^{-7t}}{84}(-7 e^{4t} + 4 e^{7t} + 3 )}\]
    \end{solution}
    \item[(e)] Let \(\tau_1, \tau_2, \ldots\) be the times of the successive jumps of \(\{X_t\}_{t \geq 0}\). Let \(\tilde{X}_n = X_{\tau_n}\). Find the transition matrix for the discrete-time Markov chain \(\{\tilde{X}_n\}\).
\begin{solution}
    We have that the state space is given by $S = \{1,2,3\},$ but since $\{\tilde{X}_n\}$ is antsy, then $p(i,i) = 0$ for $i = 1,2,3.$ From $A,$ we note that $\alpha(1,3) = \alpha(3,1),$ and so $p(3,1) = p(1,3) = 0,$ since the rate at which $X_n$ traverses both is $0.$ Thus, 
    \[P = \begin{pmatrix}
        0 & \cdot & 0\\
        \cdot & 0 & \cdot\\
        0 & \cdot & 0 
    \end{pmatrix} \implies P = \begin{pmatrix}
        0 & 1 & 0\\
        \cdot & 0 & \cdot\\
        0 & 1 & 0
    \end{pmatrix}\]
    Note that 
    \[\bbP\{\tilde{X}_1 = 1 \mid \tilde{X}_0 = 2\} = \bbP\{\tau(2,1) < \tau(2,3)\},\] where 
    \[\tau(i,j):= \min\{t \geq 0 : X_\tau = j \mid X_{0} = i \}.\] That is, the probability our antsy Markov chain makes it to $1$ is the probability that it switches to $1$ before it switches to $3.$  Since $\tau(2,1) \sim \exp\{4\}$ and $\tau(2,3)\sim \exp{1}.$ Thus, we see that since both of these r.v. are independent, then
    \begin{align*}
        \bbP\{\tau(2,1) < \tau(2,3)\} &= \int_0^\infty \bbP\{\tau(2,1) = t, \tau(2,3) >t\}\, dt\\
        &= \int_0^\infty \bbP\{\tau(2,1) = t\}\bbP\{\tau(2,3) > t\}\,dt\\
        &= \int_0^\infty 4e^{-4 t}e^{-t}\, dt\\
        &= 4 \int_0^\infty  e^{-5t}\, dt\\
        &= \frac{4}{5}
    \end{align*}
    Thus, 
    \[\boxed{P = \begin{pmatrix}
        0 & 1 & 0\\
        \frac{4}{5} & 0 & \frac{1}{5}\\
        0 & 1 & 0
    \end{pmatrix}}\]
\end{solution}
\end{enumerate}




\newpage
\section*{Problem 6}
\begin{problem}
    Let $G$ be a finite connected graph. The continuous random walk on $G$ with state space $V(G)$ is given by rates $\alpha(x,y) = 1$ if $x$ and $y$ share an edge and zero else. Find the stationary distribution.
\end{problem}
\begin{solution}
    Suppose $V(G) = \{1,2,\dots, N\}.$ Then 
    \[A = \begin{pmatrix}
        \alpha(1,1) & \alpha(1,2) & \cdots & \alpha(1,N)\\
        \alpha(2,1) & \alpha(2,2) & \cdots & \alpha(2,N)\\
    \vdots & \vdots &\ddots & \vdots\\
    \alpha(N,1) & \alpha(N,2) & \cdots & \alpha(N,N)
    \end{pmatrix} = \begin{pmatrix}
        -\deg(1) & \alpha(1,2) & \cdots & \alpha(1,N)\\
        \alpha(2,1) & -\deg(2) & \cdots & \alpha(2,N)\\
    \vdots & \vdots &\ddots & \vdots\\
    \alpha(N,1) & \alpha(N,2) & \cdots & -\deg(N)
    \end{pmatrix}.\] But since sharing edges is an equivalence class, then $\alpha(i,j) = \alpha(j,i),$ and thus 
    \[A = \begin{pmatrix}
        -\deg(1) & \alpha(1,2) & \cdots & \alpha(1,N)\\
        \alpha(1,2) & -\deg(2) & \cdots & \alpha(2,N)\\
    \vdots & \vdots &\ddots & \vdots\\
    \alpha(1,N) & \alpha(2,N) & \cdots & -\deg(N)
    \end{pmatrix}.\]
    
    Since the graph is finite and connected, the Markov chain is clearly irreducible, and so the unique stationary distribution will satisfy $\pi A = 0.$ That is,
    \begin{align*}
        -\deg(1)\pi_1 + \alpha(2,1)\pi_2 + \dots + \alpha(N,1)\pi_N &= 0\\
        \alpha(1,2)\pi_1 -\deg(2)\pi_2 + \dots + \alpha(N,2)\pi_N &= 0\\
        &\vdots\\
        \alpha(1,N)\pi_1 + \alpha(2,N)\pi_2 + \dots -\deg(N)\pi_N &= 0\\
        \pi_1 + \pi_2 + \cdots +\pi_N &=1
    \end{align*}
    Consider the candidate distribution 
    \[\pi_1 = \pi_2 = \cdots = \pi_N.\] Then
    \begin{align*}
        -\deg(1)\pi_1 + \alpha(2,1)\pi_1 + \dots + \alpha(N,1)\pi_N = -\deg(1)\pi_1 + \deg(1)\pi_1&=  0\\
        &\vdots\\
        \alpha(1,N)\pi_N + \alpha(2,N)\pi_N + \dots -\deg(N)\pi_N  = \deg(N)\pi_N - \deg(N)\pi_N&= 0
    \end{align*}
    and thus plugging into the final constraint, we see that 
    \[\boxed{\pi_i = \frac{1}{N}}\] for all $i = 1,2,\dots N$
\end{solution}
















\end{document}