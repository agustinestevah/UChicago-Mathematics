\documentclass[11pt]{article}
\usepackage{float}
\usepackage{tikz}
\usetikzlibrary{automata, positioning}
% NOTE: Add in the relevant information to the commands below; or, if you'll be using the same information frequently, add these commands at the top of paolo-pset.tex file. 
\newcommand{\name}{Agustín Esteva}
\newcommand{\email}{aesteva@uchicago.edu}
\newcommand{\classnum}{23500}
\newcommand{\subject}{Markov Chains, Martingales, and Brownian Motion}
\newcommand{\instructors}{Stephen Yearwood}
\newcommand{\assignment}{Problem Set 5}
\newcommand{\semester}{Spring 2025}
\newcommand{\duedate}{05/09/2025}
\newcommand{\bA}{\mathbf{A}}
\newcommand{\bB}{\mathbf{B}}
\newcommand{\bC}{\mathbf{C}}
\newcommand{\bD}{\mathbf{D}}
\newcommand{\bE}{\mathbf{E}}
\newcommand{\bF}{\mathbf{F}}
\newcommand{\bG}{\mathbf{G}}
\newcommand{\bH}{\mathbf{H}}
\newcommand{\bI}{\mathbf{I}}
\newcommand{\bJ}{\mathbf{J}}
\newcommand{\bK}{\mathbf{K}}
\newcommand{\bL}{\mathbf{L}}
\newcommand{\bM}{\mathbf{M}}
\newcommand{\bN}{\mathbf{N}}
\newcommand{\bO}{\mathbf{O}}
\newcommand{\bP}{\mathbf{P}}
\newcommand{\bQ}{\mathbf{Q}}
\newcommand{\bR}{\mathbf{R}}
\newcommand{\bS}{\mathbf{S}}
\newcommand{\bT}{\mathbf{T}}
\newcommand{\bU}{\mathbf{U}}
\newcommand{\bV}{\mathbf{V}}
\newcommand{\bW}{\mathbf{W}}
\newcommand{\bX}{\mathbf{X}}
\newcommand{\bY}{\mathbf{Y}}
\newcommand{\bZ}{\mathbf{Z}}
\newcommand{\Vol}{\text{Vol}}

%% blackboard bold math capitals
\newcommand{\bbA}{\mathbb{A}}
\newcommand{\bbB}{\mathbb{B}}
\newcommand{\bbC}{\mathbb{C}}
\newcommand{\bbD}{\mathbb{D}}
\newcommand{\bbE}{\mathbb{E}}
\newcommand{\bbF}{\mathbb{F}}
\newcommand{\bbG}{\mathbb{G}}
\newcommand{\bbH}{\mathbb{H}}
\newcommand{\bbI}{\mathbb{I}}
\newcommand{\bbJ}{\mathbb{J}}
\newcommand{\bbK}{\mathbb{K}}
\newcommand{\bbL}{\mathbb{L}}
\newcommand{\bbM}{\mathbb{M}}
\newcommand{\bbN}{\mathbb{N}}
\newcommand{\bbO}{\mathbb{O}}
\newcommand{\bbP}{\mathbb{P}}
\newcommand{\bbQ}{\mathbb{Q}}
\newcommand{\bbR}{\mathbb{R}}
\newcommand{\bbS}{\mathbb{S}}
\newcommand{\bbT}{\mathbb{T}}
\newcommand{\bbU}{\mathbb{U}}
\newcommand{\bbV}{\mathbb{V}}
\newcommand{\bbW}{\mathbb{W}}
\newcommand{\bbX}{\mathbb{X}}
\newcommand{\bbY}{\mathbb{Y}}
\newcommand{\bbZ}{\mathbb{Z}}

%% script math capitals
\newcommand{\sA}{\mathscr{A}}
\newcommand{\sB}{\mathscr{B}}
\newcommand{\sC}{\mathscr{C}}
\newcommand{\sD}{\mathscr{D}}
\newcommand{\sE}{\mathscr{E}}
\newcommand{\sF}{\mathscr{F}}
\newcommand{\sG}{\mathscr{G}}
\newcommand{\sH}{\mathscr{H}}
\newcommand{\sI}{\mathscr{I}}
\newcommand{\sJ}{\mathscr{J}}
\newcommand{\sK}{\mathscr{K}}
\newcommand{\sL}{\mathscr{L}}
\newcommand{\sM}{\mathscr{M}}
\newcommand{\sN}{\mathscr{N}}
\newcommand{\sO}{\mathscr{O}}
\newcommand{\sP}{\mathscr{P}}
\newcommand{\sQ}{\mathscr{Q}}
\newcommand{\sR}{\mathscr{R}}
\newcommand{\sS}{\mathscr{S}}
\newcommand{\sT}{\mathscr{T}}
\newcommand{\sU}{\mathscr{U}}
\newcommand{\sV}{\mathscr{V}}
\newcommand{\sW}{\mathscr{W}}
\newcommand{\sX}{\mathscr{X}}
\newcommand{\sY}{\mathscr{Y}}
\newcommand{\sZ}{\mathscr{Z}}


\renewcommand{\emptyset}{\O}

\newcommand{\abs}[1]{\lvert #1 \rvert}
\newcommand{\norm}[1]{\lVert #1 \rVert}
\newcommand{\sm}{\setminus}


\newcommand{\sarr}{\rightarrow}
\newcommand{\arr}{\longrightarrow}

% NOTE: Defining collaborators is optional; to not list collaborators, comment out the line below.
%\newcommand{\collaborators}{Alyssa P. Hacker (\texttt{aphacker}), Ben Bitdiddle (\texttt{bitdiddle})}

% Copyright 2021 Paolo Adajar (padajar.com, paoloadajar@mit.edu)
% 
% Permission is hereby granted, free of charge, to any person obtaining a copy of this software and associated documentation files (the "Software"), to deal in the Software without restriction, including without limitation the rights to use, copy, modify, merge, publish, distribute, sublicense, and/or sell copies of the Software, and to permit persons to whom the Software is furnished to do so, subject to the following conditions:
%
% The above copyright notice and this permission notice shall be included in all copies or substantial portions of the Software.
% 
% THE SOFTWARE IS PROVIDED "AS IS", WITHOUT WARRANTY OF ANY KIND, EXPRESS OR IMPLIED, INCLUDING BUT NOT LIMITED TO THE WARRANTIES OF MERCHANTABILITY, FITNESS FOR A PARTICULAR PURPOSE AND NONINFRINGEMENT. IN NO EVENT SHALL THE AUTHORS OR COPYRIGHT HOLDERS BE LIABLE FOR ANY CLAIM, DAMAGES OR OTHER LIABILITY, WHETHER IN AN ACTION OF CONTRACT, TORT OR OTHERWISE, ARISING FROM, OUT OF OR IN CONNECTION WITH THE SOFTWARE OR THE USE OR OTHER DEALINGS IN THE SOFTWARE.

\usepackage{fullpage}
\usepackage{enumitem}
\usepackage{amsfonts, amssymb, amsmath,amsthm}
\usepackage{mathtools}
\usepackage[pdftex, pdfauthor={\name}, pdftitle={\classnum~\assignment}]{hyperref}
\usepackage[dvipsnames]{xcolor}
\usepackage{bbm}
\usepackage{graphicx}
\usepackage{mathrsfs}
\usepackage{pdfpages}
\usepackage{tabularx}
\usepackage{pdflscape}
\usepackage{makecell}
\usepackage{booktabs}
\usepackage{natbib}
\usepackage{caption}
\usepackage{subcaption}
\usepackage{physics}
\usepackage[many]{tcolorbox}
\usepackage{version}
\usepackage{ifthen}
\usepackage{cancel}
\usepackage{listings}
\usepackage{courier}

\usepackage{tikz}
\usepackage{istgame}

\hypersetup{
	colorlinks=true,
	linkcolor=blue,
	filecolor=magenta,
	urlcolor=blue,
}

\setlength{\parindent}{0mm}
\setlength{\parskip}{2mm}

\setlist[enumerate]{label=({\alph*})}
\setlist[enumerate, 2]{label=({\roman*})}

\allowdisplaybreaks[1]

\newcommand{\psetheader}{
	\ifthenelse{\isundefined{\collaborators}}{
		\begin{center}
			{\setlength{\parindent}{0cm} \setlength{\parskip}{0mm}
				
				{\textbf{\classnum~\semester:~\assignment} \hfill \name}
				
				\subject \hfill \href{mailto:\email}{\tt \email}
				
				Instructor(s):~\instructors \hfill Due Date:~\duedate	
				
				\hrulefill}
		\end{center}
	}{
		\begin{center}
			{\setlength{\parindent}{0cm} \setlength{\parskip}{0mm}
				
				{\textbf{\classnum~\semester:~\assignment} \hfill \name\footnote{Collaborator(s): \collaborators}}
				
				\subject \hfill \href{mailto:\email}{\tt \email}
				
				Instructor(s):~\instructors \hfill Due Date:~\duedate	
				
				\hrulefill}
		\end{center}
	}
}

\renewcommand{\thepage}{\classnum~\assignment \hfill \arabic{page}}

\makeatletter
\def\points{\@ifnextchar[{\@with}{\@without}}
\def\@with[#1]#2{{\ifthenelse{\equal{#2}{1}}{{[1 point, #1]}}{{[#2 points, #1]}}}}
\def\@without#1{\ifthenelse{\equal{#1}{1}}{{[1 point]}}{{[#1 points]}}}
\makeatother

\newtheoremstyle{theorem-custom}%
{}{}%
{}{}%
{\itshape}{.}%
{ }%
{\thmname{#1}\thmnumber{ #2}\thmnote{ (#3)}}

\theoremstyle{theorem-custom}

\newtheorem{theorem}{Theorem}
\newtheorem{lemma}[theorem]{Lemma}
\newtheorem{example}[theorem]{Example}

\newenvironment{problem}[1]{\color{black} #1}{}

\newenvironment{solution}{%
	\leavevmode\begin{tcolorbox}[breakable, colback=green!5!white,colframe=green!75!black, enhanced jigsaw] \proof[\scshape Solution:] \setlength{\parskip}{2mm}%
	}{\renewcommand{\qedsymbol}{$\blacksquare$} \endproof \end{tcolorbox}}

\newenvironment{reflection}{\begin{tcolorbox}[breakable, colback=black!8!white,colframe=black!60!white, enhanced jigsaw, parbox = false]\textsc{Reflections:}}{\end{tcolorbox}}

\newcommand{\qedh}{\renewcommand{\qedsymbol}{$\blacksquare$}\qedhere}

\definecolor{mygreen}{rgb}{0,0.6,0}
\definecolor{mygray}{rgb}{0.5,0.5,0.5}
\definecolor{mymauve}{rgb}{0.58,0,0.82}

% from https://github.com/satejsoman/stata-lstlisting
% language definition
\lstdefinelanguage{Stata}{
	% System commands
	morekeywords=[1]{regress, reg, summarize, sum, display, di, generate, gen, bysort, use, import, delimited, predict, quietly, probit, margins, test},
	% Reserved words
	morekeywords=[2]{aggregate, array, boolean, break, byte, case, catch, class, colvector, complex, const, continue, default, delegate, delete, do, double, else, eltypedef, end, enum, explicit, export, external, float, for, friend, function, global, goto, if, inline, int, local, long, mata, matrix, namespace, new, numeric, NULL, operator, orgtypedef, pointer, polymorphic, pragma, private, protected, public, quad, real, return, rowvector, scalar, short, signed, static, strL, string, struct, super, switch, template, this, throw, transmorphic, try, typedef, typename, union, unsigned, using, vector, version, virtual, void, volatile, while,},
	% Keywords
	morekeywords=[3]{forvalues, foreach, set},
	% Date and time functions
	morekeywords=[4]{bofd, Cdhms, Chms, Clock, clock, Cmdyhms, Cofc, cofC, Cofd, cofd, daily, date, day, dhms, dofb, dofC, dofc, dofh, dofm, dofq, dofw, dofy, dow, doy, halfyear, halfyearly, hh, hhC, hms, hofd, hours, mdy, mdyhms, minutes, mm, mmC, mofd, month, monthly, msofhours, msofminutes, msofseconds, qofd, quarter, quarterly, seconds, ss, ssC, tC, tc, td, th, tm, tq, tw, week, weekly, wofd, year, yearly, yh, ym, yofd, yq, yw,},
	% Mathematical functions
	morekeywords=[5]{abs, ceil, cloglog, comb, digamma, exp, expm1, floor, int, invcloglog, invlogit, ln, ln1m, ln, ln1p, ln, lnfactorial, lngamma, log, log10, log1m, log1p, logit, max, min, mod, reldif, round, sign, sqrt, sum, trigamma, trunc,},
	% Matrix functions
	morekeywords=[6]{cholesky, coleqnumb, colnfreeparms, colnumb, colsof, corr, det, diag, diag0cnt, el, get, hadamard, I, inv, invsym, issymmetric, J, matmissing, matuniform, mreldif, nullmat, roweqnumb, rownfreeparms, rownumb, rowsof, sweep, trace, vec, vecdiag, },
	% Programming functions
	morekeywords=[7]{autocode, byteorder, c, _caller, chop, abs, clip, cond, e, fileexists, fileread, filereaderror, filewrite, float, fmtwidth, has_eprop, inlist, inrange, irecode, matrix, maxbyte, maxdouble, maxfloat, maxint, maxlong, mi, minbyte, mindouble, minfloat, minint, minlong, missing, r, recode, replay, return, s, scalar, smallestdouble,},
	% Random-number functions
	morekeywords=[8]{rbeta, rbinomial, rcauchy, rchi2, rexponential, rgamma, rhypergeometric, rigaussian, rlaplace, rlogistic, rnbinomial, rnormal, rpoisson, rt, runiform, runiformint, rweibull, rweibullph,},
	% Selecting time-span functions
	morekeywords=[9]{tin, twithin,},
	% Statistical functions
	morekeywords=[10]{betaden, binomial, binomialp, binomialtail, binormal, cauchy, cauchyden, cauchytail, chi2, chi2den, chi2tail, dgammapda, dgammapdada, dgammapdadx, dgammapdx, dgammapdxdx, dunnettprob, exponential, exponentialden, exponentialtail, F, Fden, Ftail, gammaden, gammap, gammaptail, hypergeometric, hypergeometricp, ibeta, ibetatail, igaussian, igaussianden, igaussiantail, invbinomial, invbinomialtail, invcauchy, invcauchytail, invchi2, invchi2tail, invdunnettprob, invexponential, invexponentialtail, invF, invFtail, invgammap, invgammaptail, invibeta, invibetatail, invigaussian, invigaussiantail, invlaplace, invlaplacetail, invlogistic, invlogistictail, invnbinomial, invnbinomialtail, invnchi2, invnF, invnFtail, invnibeta, invnormal, invnt, invnttail, invpoisson, invpoissontail, invt, invttail, invtukeyprob, invweibull, invweibullph, invweibullphtail, invweibulltail, laplace, laplaceden, laplacetail, lncauchyden, lnigammaden, lnigaussianden, lniwishartden, lnlaplaceden, lnmvnormalden, lnnormal, lnnormalden, lnwishartden, logistic, logisticden, logistictail, nbetaden, nbinomial, nbinomialp, nbinomialtail, nchi2, nchi2den, nchi2tail, nF, nFden, nFtail, nibeta, normal, normalden, npnchi2, npnF, npnt, nt, ntden, nttail, poisson, poissonp, poissontail, t, tden, ttail, tukeyprob, weibull, weibullden, weibullph, weibullphden, weibullphtail, weibulltail,},
	% String functions 
	morekeywords=[11]{abbrev, char, collatorlocale, collatorversion, indexnot, plural, plural, real, regexm, regexr, regexs, soundex, soundex_nara, strcat, strdup, string, strofreal, string, strofreal, stritrim, strlen, strlower, strltrim, strmatch, strofreal, strofreal, strpos, strproper, strreverse, strrpos, strrtrim, strtoname, strtrim, strupper, subinstr, subinword, substr, tobytes, uchar, udstrlen, udsubstr, uisdigit, uisletter, ustrcompare, ustrcompareex, ustrfix, ustrfrom, ustrinvalidcnt, ustrleft, ustrlen, ustrlower, ustrltrim, ustrnormalize, ustrpos, ustrregexm, ustrregexra, ustrregexrf, ustrregexs, ustrreverse, ustrright, ustrrpos, ustrrtrim, ustrsortkey, ustrsortkeyex, ustrtitle, ustrto, ustrtohex, ustrtoname, ustrtrim, ustrunescape, ustrupper, ustrword, ustrwordcount, usubinstr, usubstr, word, wordbreaklocale, worcount,},
	% Trig functions
	morekeywords=[12]{acos, acosh, asin, asinh, atan, atanh, cos, cosh, sin, sinh, tan, tanh,},
	morecomment=[l]{//},
	% morecomment=[l]{*},  // `*` maybe used as multiply operator. So use `//` as line comment.
	morecomment=[s]{/*}{*/},
	% The following is used by macros, like `lags'.
	morestring=[b]{`}{'},
	% morestring=[d]{'},
	morestring=[b]",
	morestring=[d]",
	% morestring=[d]{\\`},
	% morestring=[b]{'},
	sensitive=true,
}

\lstset{ 
	backgroundcolor=\color{white},   % choose the background color; you must add \usepackage{color} or \usepackage{xcolor}; should come as last argument
	basicstyle=\footnotesize\ttfamily,        % the size of the fonts that are used for the code
	breakatwhitespace=false,         % sets if automatic breaks should only happen at whitespace
	breaklines=true,                 % sets automatic line breaking
	captionpos=b,                    % sets the caption-position to bottom
	commentstyle=\color{mygreen},    % comment style
	deletekeywords={...},            % if you want to delete keywords from the given language
	escapeinside={\%*}{*)},          % if you want to add LaTeX within your code
	extendedchars=true,              % lets you use non-ASCII characters; for 8-bits encodings only, does not work with UTF-8
	firstnumber=0,                % start line enumeration with line 1000
	frame=single,	                   % adds a frame around the code
	keepspaces=true,                 % keeps spaces in text, useful for keeping indentation of code (possibly needs columns=flexible)
	keywordstyle=\color{blue},       % keyword style
	language=Octave,                 % the language of the code
	morekeywords={*,...},            % if you want to add more keywords to the set
	numbers=left,                    % where to put the line-numbers; possible values are (none, left, right)
	numbersep=5pt,                   % how far the line-numbers are from the code
	numberstyle=\tiny\color{mygray}, % the style that is used for the line-numbers
	rulecolor=\color{black},         % if not set, the frame-color may be changed on line-breaks within not-black text (e.g. comments (green here))
	showspaces=false,                % show spaces everywhere adding particular underscores; it overrides 'showstringspaces'
	showstringspaces=false,          % underline spaces within strings only
	showtabs=false,                  % show tabs within strings adding particular underscores
	stepnumber=2,                    % the step between two line-numbers. If it's 1, each line will be numbered
	stringstyle=\color{mymauve},     % string literal style
	tabsize=2,	                   % sets default tabsize to 2 spaces
%	title=\lstname,                   % show the filename of files included with \lstinputlisting; also try caption instead of title
	xleftmargin=0.25cm
}

% NOTE: To compile a version of this pset without problems, solutions, or reflections, uncomment the relevant line below.

%\excludeversion{problem}
%\excludeversion{solution}
%\excludeversion{reflection}

\begin{document}	
	
	% Use the \psetheader command at the beginning of a pset. 
	\psetheader
\section*{Problem 1 (5 points)}
Show that if \(X\) and \(Y\) are random variables such that \(\mathbb{E}[Y \mid X] = \mathbb{E}[Y]\), then it holds that
\[
\mathbb{E}[XY] = \mathbb{E}[X] \mathbb{E}[Y],
\]
but the reverse implication does not hold.

\begin{solution}
    We use the law of total expectation to note that 
    \[\bbE[XY] = \bbE[\bbE[XY \mid X]].\] $X$ is trivially $X-$measurable, so then it acts as a constant 
    \[\bbE[\bbE[XY \mid X]] = \bbE[X \bbE[Y \mid X]] = \bbE[X \bbE[Y]].\] $\bbE[Y]$ is just a constant, not a random variable, and so 
    \[\bbE[X \bbE[Y]] = \bbE[Y]\bbE[X],\] as desired. 
\end{solution}

\newpage

\section*{Problem 2 (10 points)}
Suppose \(X \sim \text{Poi}(\lambda)\).
\begin{itemize}
    \item[(a)] Compute the expected value of \(X\) given its parity (i.e., find \(\mathbb{E}[X \mid X \text{ is odd}]\) and \(\mathbb{E}[X \mid X \text{ is even}]\)).
    \begin{solution}
    Since $X$ takes values in $\bbN_0,$ then definition of conditional expectation, 
    \begin{align*}
        \bbE[X \mid X \text{ odd}] &= \frac{\displaystyle\sum_{n = 0}^\infty n \bbP\{X= n, X \text{ odd}\}}{\displaystyle\sum_{n=0}^\infty  \bbP\{X = n, X \text{ odd}\}}\\
        &= \frac{\displaystyle\sum_{n = 0}^\infty n \bbP\{X \text{ odd } \mid X = n\}\bbP\{X= n\}}{\displaystyle\sum_{n=0}^\infty  \bbP\{X \text{ odd } \mid X = n\}\bbP\{X= n\}}\\
        &= \frac{\displaystyle\sum_{n = 0}^\infty (2n+1) \bbP\{X= 2n+1\}}{\displaystyle\sum_{n=0}^\infty  \bbP\{X= 2n+1\}}\\
        &= \frac{\displaystyle\sum_{n = 0}^\infty (2n+1) \frac{e^{-\lambda}\lambda^{2n +1}}{(2n+1)!}}{\displaystyle\sum_{n=0}^\infty  \frac{e^{-\lambda}\lambda^{2n +1}}{(2n+1)!}}\\
        &= \frac{\displaystyle\sum_{n=0}^\infty \frac{\lambda^{2n +1}}{(2n)!}}{\displaystyle\sum_{n=0}^\infty \frac{\lambda^{2n +1}}{(2n+1)!}}\\
        &= \frac{\lambda \cosh{\lambda}}{\sinh{\lambda}}\\
        &= \lambda \coth{\lambda}
    \end{align*}
    Using similar logic, one can see that 
    \[\bbE[X \mid X\text{ even}]=  \lambda \tanh{\lambda}\]
    \end{solution}
    \item[(b)] Suppose we buy \(X\) raffle tickets, each of which has a chance \(p \in (0, 1)\) of winning independently of others. Let \(Y\) be the number of prizes given out. Compute \(\mathbb{E}[Y \mid X]\) and \(\mathbb{E}[Y]\).
    \begin{solution}
$Y \mid X = k$ is binomial with probability of success $p$ and $k$ trials. Thus, $\bbE[Y \mid X = k] = k p,$ and so $\bbE[Y \mid X] = kX.$ We then use the law of total expectation to note that
\[\bbE[Y] = \bbE[\bbE[Y \mid X]] = \bbE[kX] = k\bbE[X] = k\lambda.\]

    \end{solution}
\end{itemize}

\newpage

\section*{Problem 3 (10 points)}
Let \(X_1, X_2, \ldots\) be i.i.d. random variables with \(\mathbb{P}\{X_i = 1\} = \mathbb{P}\{X_i = -1\} = \frac{1}{2}\). Let \(S_0 = 0\), and \(S_n = X_1 + X_2 + \cdots + X_n\) define a simple symmetric random walk on \(\mathbb{Z}\). As shown in class, \(S_n\) is a martingale with respect to \(\mathcal{F}_n = \sigma(X_1, \ldots, X_n)\).

\begin{itemize}
    \item[(a)] Find a deterministic sequence \(a_n \in \mathbb{R}\) such that \(M_n := S_n^3 + a_n S_n\) is a martingale with respect to \(\mathcal{F}_n\).
    \begin{solution}

Using linearity and a few other facts, we see that
\begin{align*}
    \bbE[M_n \mid \mathcal{F}_{n-1}] &= \bbE[S_{n}^3 + a_n S_n \mid \mathcal{F}_{n-1}]\\
    &= \bbE[(S_{n-1} + X_n)^3 + a_nS_{n} \mid \mathcal{F}_{n-1}]\\
    &= \bbE[S_{n-1}^3 + 3S_{n-1}^2 X_n + 3S_{n-1}X_n^2+X_n^3 + a_n S_{n} \mid \mathcal{F}_{n-1}]\\
    &= S^3_{n-1} + 3S^2_{n-1}\bbE[X_n \mid \mathcal{F}_{n-1}] + 3S_{n-1}\bbE[X_n^2 \mid \mathcal{F}_{n-1}] + \bbE[X_n^3 \mid \mathcal{F}_{n-1}] + a_n S_{n-1}\\
    &= S^3_{n-1} + 3S^2_{n-1}\bbE[X_n] + 3S_{n-1}\bbE[X_n^2 ] + \bbE[X_n^3 ] + a_n S_{n-1}\\
    &= S^3_{n-1} + 3S_{n-1} + a_nS_{n-1}
\end{align*}
and so $M_n$ is a martingale if and only if
\[S^3_{n-1} + 3S_{n-1} + a_nS_{n-1} = M_{n-1} = S_{n-1}^3 + a_{n-1}S_{n-1}\] and thus 
\[a_n = a_{n-1} - 3 \implies a_n = a_0 - 3n.\] 

    \end{solution}
    \item[(b)] Find deterministic sequences \(b_n, c_n \in \mathbb{R}\) such that \(Z_n := X_n^4 + b_n X_n^2 + c_n\) is a martingale with respect to \(\mathcal{F}_n\).
\begin{solution}
    We use independence to note that 
    \begin{align*}
        \bbE[Z_n \mid \mathcal{F}_{n-1}] &= \bbE[X_n^4 + b_nX_n^2 + c_n \mid \mathcal{F}_{n-1}]\\
        &= 1 + b_n + c_n
    \end{align*}
    But $Z_n$ is a martingale if and only if 
    \[1 + b_n + c_n = Z_{n-1} = 1 + b_{n-1} + c_{n-1},\] and thus our sequences must satisfy 
    \[b_n + c_n = b_{n-1} + c_{n-1}\] for all $n.$
\end{solution}
\end{itemize}

\newpage

\section*{Problem 4 (20 points)}
Let \(\{X_n\}\) be a biased random walk on the integers with probability \(p \in (0, 1/2)\) to move to the right and probability \(1 - p \in (1/2, 1)\) to move to the left.

\begin{itemize}
    \item[(a)] Show that \(M_n = \left(\frac{1 - p}{p}\right)^{X_n}\) is a martingale with respect to \(\mathcal{F}_n = \sigma(X_0, \ldots, X_n)\).
\begin{solution}
    Since $M_n$ depends only on $X_i$ for $i \leq n$, then clearly $M_n$ is $\mathcal{F}_n$ measurable.

    We can bound $|X_n|$ by $n$ since that is the furthest it can get in $n$ steps. Thus, since $\frac{1-p}{p} >1,$ we have that 
    \[\bbE[|M_n|]  = \bbE\left[\left|\left(\frac{1 - p}{p}\right)^{X_n}\right|\right] = \bbE\left[\left(\frac{1 - p}{p}\right)^{|X_n|}\right] \leq \bbE\left[\left(\frac{1 - p}{p}\right)^{n}\right] < \infty\]

    Finally, we have that since we can write $X_n = \sum_{i=1}^n \xi_i,$ where $\xi_i$ are i.i.d. such that
    \[\bbP\{\xi_i = 1\} = p, \quad \bbP\{\xi_i = -1\} = 1-p.\] Then
    \begin{align*}
        \bbE[M_n \mid \mathcal{F}_{n-1}] &= \bbE\left[\left(\frac{1-p}{p}\right)^{X_n} \mid \mathcal{F}_{n-1}\right]\\
        &= \bbE\left[\left(\frac{1-p}{p}\right)^{X_{n-1}}\left(\frac{1-p}{p}\right)^{\xi_{n}} \mid \mathcal{F}_{n-1}\right]\\
        &= \left(\frac{1-p}{p}\right)^{X_{n-1}}\bbE\left[\left(\frac{1-p}{p}\right)^{\xi_{n}}\right]\\
        &=\left(\frac{1-p}{p}\right)^{X_{n-1}} (p\left(\frac{1-p}{p}\right)^{1} + (1-p)\left(\frac{1-p}{p}\right)^{-1}) \\
        &= \left(\frac{1-p}{p}\right)^{X_{n-1}}\\
        &= M_{n-1}
    \end{align*}
\end{solution}
    \item[(b)] Use the optional stopping theorem to compute, for any \(x \in \{0, \ldots, N\}\), the probability that the walk reaches 0 before \(N\) if \(X_0 = x\).
    \begin{solution}
        Define
        \[\tau:= \min\{ n \geq 0 : X_n \in \{0, N\} \mid X_0 = x\}\] be the first time $X_n$ reaches $0$ or $N$ given that it begins at $X_0 = x.$ Assuming we can use the OST, we have that 
        \[\bbE[M_\tau] = \bbE[M_0] = \left(\frac{1-p}{p}\right)^x\] and thus if we call $p_L$ the probability we 'lose' (reach $0$) and $p_W = 1-p_L$ the probability we 'win' (reach $N$), we see that
        \[\left(\frac{1-p}{p}\right)^x = \bbE[M_\tau] = p_L (1) + p_W\left(\frac{1-p}{p}\right)^N \implies p_W = \frac{1 - \left(\frac{1-p}{p}\right)^x}{1 - \left(\frac{1-p}{p}\right)^N},\] and $p_L = 1-p_W.$ 

        Thus, it suffices to notice that $M_n$ satisfies the conditions for the OST:
        \begin{enumerate}
            \item The state $\{1,2,\dots, N-1\}$ is transient, and thus since $\tau$ is the first time we leave the state, then a result from Markov chains states that
            \[\bbP\{\tau  < \infty\} =1 \]
            \item We can bound the expectation by 
            \[\bbE[|M_n|] \leq \left( \frac{1-p}{p}\right)^N < \infty\]
            \item We have by a result in class that
            \[\bbE[M_n \mathbbm{1}_{\tau > n}] \leq (\frac{1-p}{p})^ne^{-cn} \to 0.\]
        \end{enumerate}
    \end{solution}
    \item[(c)] Show that \(\widetilde{M}_n = X_n + (1 - 2p)n\) is a martingale with respect to \(\mathcal{F}_n\).
    \begin{solution}
        $\widetilde{M}_n$ is clearly $\mathcal{F}_n$ measurable.

        Again, we bound $|X_n|$ by $n$ and so 
        \[\bbE[|\widetilde{M}_n|] \leq n + (1-2p)n < \infty\]

        \begin{align*}
            \bbE[\widetilde{M}_n \mid \mathcal{F}_{n-1}]&= \bbE[X_n + (1-2p)n \mid \mathcal{F}_{n-1}]\\
            &= \bbE[X_n\mid \mathcal{F}_{n-1}] + (1-2p)n\\
            &= p(X_{n-1} +1) + (1-p)(X_{n-1} - 1) + (1-2p)n\\
            &= pX_{n-1} + (1-p)X_{n-1} + p - (1-p) + (1-2p)n\\
            &= X_{n-1} -(1 - 2p) + (1-2p)n\\
            &= X_{n-1} + (1-2p)(n-1)\\
            &= \widetilde{M}_{n-1}
        \end{align*}
    \end{solution}
    \item[(d)] Use the optional stopping theorem to compute, for any \(x \in \{0, \ldots, N\}\), the expectation of the first time that \(X_n \in \{0, N\}\) if \(X_0 = x\).
    \begin{solution}
        Define
        \[\tau:= \min\{ n \geq 0 : X_n \in \{0, N\} \mid X_0 = x\}\] be the first time $X_n$ reaches $0$ or $N$ given that it begins at $X_0 = x.$ Assuming we can use the OST, we have that 
        \[\bbE[\widetilde{M}_\tau] = \bbE[\widetilde{M}_0] = x\] and 
        \[\bbE[\widetilde{M}_\tau] = \bbE[X_\tau + (1-2p)\tau] = \bbE[X_\tau]+ (1-2p)\bbE[\tau] = p_L(0) + p_W(N) + (1-2p)\bbE[\tau].\]
        Thus, 
        \[\bbE[\tau] = \frac{x- Np_W}{1-2p}\]
    \end{solution}
\end{itemize}

\newpage

\section*{Problem 5 (10 points)}
Let \(\{M_n\}_{n \geq 0}\) be a martingale. Suppose that \(M_0 = 0\) and \(\mathbb{P}[|M_n| \leq 1] = 1\) for every \(n \geq 1\).

\begin{itemize}
    \item[(a)] Let \(\tau\) be a stopping time for \(\{M_n\}_{n \geq 0}\) such that \(\mathbb{P}[\tau < \infty] = 1\). Explain why \(\mathbb{E}[M_\tau] = 0\).
    \begin{solution}
        Since $M_n$ is almost surely bounded, then 
        \begin{align*}
            \bbE[|M_n|\mathbbm{1}_{\tau >n}] &=\bbE[\bbE[|M_n|\mathbbm{1}_{\tau >n} \mid |M_n|]]\\
            &= \bbE[|M_n| \mathbbm{1}_{\tau>n} \mid |M_n| >1]\bbP\{|M_n| >1 \} + \bbE[|M_n|\mathbbm{1}_{\tau>n} \mid |M_n| \leq 1]\bbP\{|M_n| \leq 1 \}]\\
            &= \bbE[|M_n|\mathbbm{1}_{\tau>n} \mid |M_n| \leq 1]\\
            &\leq \bbE[\mathbbm{1}_{\tau >n}]\\
            &= \bbP\{\tau >n\}\\
            &= 1- \bbP\{\tau \leq n\}\\
            &\to 1- \bbP\{\tau < \infty\}\\
            &= 0
        \end{align*}
        Also we have that since $\tau = n$ for some $n \in \bbN,$
        \[\bbE[|M_\tau|] \leq 1.\] Thus, we can apply the optional stopping theorem and say that 
        \[\bbE[M_\tau]= \bbE[M_0] = 0\]
    \end{solution}
    \item[(b)] Show that for each \(r \in (0, 1]\),
    \[
    \mathbb{P}[M_n \leq r, \forall n \geq 0] > 0.
    \]
    \begin{solution}
        Suppose not, that for some $r \in (0,1],$ we have that 
        \[\bbP\{M_n \leq r, \,\forall  n \geq 0\} = 0.\] Let $\tau:= \min\{n \geq 0 : M_n >r\}.$ By our contradiction, we have that $\bbP\{\tau < \infty\} = 1.$ By the optional stopping theorem, we have that 
        \[0 = \bbE[M_\tau],\] but by definition, 
        \[\bbE[M_\tau] > r \bbP\{\tau < \infty\} = r,\] which is a contradiction.
    \end{solution}
\end{itemize}

\newpage

\section*{Problem 6 (10 points)}
Let \(X_n\) be a Markov chain on the two-dimensional integer lattice \(\mathbb{Z}^2\) with the following transition probabilities:
\[
\mathbb{P}(X_{n+1} = (i \pm 1, j) \mid X_n = (i, j)) = \frac{1}{8}, \quad
\mathbb{P}(X_{n+1} = (i, j \pm 1) \mid X_n = (i, j)) = \frac{1}{8},
\]
\[
\mathbb{P}(X_{n+1} = (i \pm 1, j \pm 1) \mid X_n = (i, j)) = \frac{1}{8}.
\]

\begin{itemize}
    \item[(a)] Prove that \(M_n := |X_n|^2 - \frac{3}{2}n\) is a martingale with respect to the natural filtration of the process. (We denote by \(|x|\) the Euclidean norm of \(x \in \mathbb{Z}^2\).)
    \begin{solution}
It is clear that $M_n$ is $\mathcal{F}_n$ measurable. We can bound the expectation by the fact that $|X_n|^2 \leq 2n^2$ (since the farthest $X_n$ can travel is diagonally all the way, which is $n\sqrt{2}$ distance from the origin)
\[\bbE[|M_n|] \leq  \bbE[|X_n|^2 + \frac{3}{2}n] = \bbE[|X_n|^2] + \frac{3}{2} = 2n^2 + \frac{3}{2} <\infty\] For the martingale property, we note that $X_n = \sum_{i=1}^n \xi_i,$ where $\xi_i$ is the $8-$sided die that determines what the next step of the random walk is. Then 
\begin{align*}
    \bbE[M_n \mid \mathcal{F}_{n-1}] &= \bbE[|X_n|^2 \mid \mathcal{F}_{n-1}] - \frac{3}{2}n\\
    &= \bbE[|X_n - X_{n-1} + X_{n-1}|^2 \mid \mathcal{F}_n] - \frac{3}{2}n\\
    &= \bbE[|\xi_n + X_{n-1}|^2\mid \mathcal{F}_n] - \frac{3}{2}n\\
    &= \bbE[|\xi_n|^2 + |X_{n-1}|^2 + 2\langle\xi_n, X_{n-1}\rangle \mid \mathcal{F}_{n-1}] - \frac{3}{2}n\\
     &= \bbE[|\xi_n|^2] + |X_{n-1}|^2 + 2\bbE[\langle \xi_n, X_{n-1}\rangle \mid \mathcal{F}_{n-1]}]- \frac{3}{2}n\\
     &= |X_{n-1}|^2 + \frac{3}{2} - \frac{3}{2}n + 2\bbE[\langle \xi_n, X_{n-1}\rangle\mid \mathcal{F}_{n-1]}]\\
     &= |X_{n-1}|^2 - \frac{3}{2}(n-1) + 2\bbE[\langle \xi_n, X_{n-1}\rangle \mid \mathcal{F}_{n-1]}].
\end{align*}

Moreover, we note that by linearity and symmetry, we have that 
\[\bbE[\langle \xi_n, X_{n-1} \rangle  \mid \mathcal{F}_{n-1]}]= \langle \bbE[\xi_n\mid \mathcal{F}_{n-1]}], \bbE[X_{n-1}\mid \mathcal{F}_{n-1]}]\rangle = \langle \bbE[\xi_n], X_{n-1}\rangle = \langle 0, X_{n-1}\rangle = 0,\]and so we are done.
    \end{solution}


    
    \item[(b)] For \(R \in \mathbb{R}_+\), define the stopping time
    \[
    T_R := \inf\{n \geq 0 : |X_n|^2 \geq R^2\}.
    \]
    Give sharp lower and upper bounds for \(\mathbb{E}[T_R \mid X_0 = (0, 0)]\).
\begin{solution}
    We apply the OST to $M_n,$ and thus 
    \[\bbE[M_{T_R}] = \bbE[M_0] = 0,\] but we also have that 
    \[\bbE[M_{T_R}] = (\bbE[|X_{T_R}|^2]-\frac{3}{2}\bbE[T_R])\bbP\{T_R < \infty\} = (\bbE[|X_{T_R}|^2]-\frac{3}{2}\bbE[T_R])\]
\end{solution}
\end{itemize}

\newpage

\section*{Problem 7 (15 points)}
Let \(G\) be a connected graph. We allow \(G\) to be infinite, but we assume that every vertex of \(G\) has finite degree. Let \(\{X_n\}_{n \geq 0}\) be the simple random walk on \(G\). A function \(f : V(G) \to \mathbb{R}\) is called harmonic at a vertex \(x \in V(G)\) if
\[
\frac{1}{\deg x} \sum_{y \sim x} f(y) = f(x),
\]
where \(\deg x\) denotes the number of neighbors of \(x\), and \(y \sim x\) means there is an edge from \(y\) to \(x\).

\begin{itemize}
    \item[(a)] Fix \(x_0 \in V(G)\) and assume that \(X_0 = x_0\). Show that if \(f\) is harmonic, then \(\{f(X_n)\}_{n \geq 0}\) is a martingale with respect to \(\sigma(X_1, \ldots, X_n)\).
    \item[(b)] Show using the martingale convergence theorem that if \(\{X_n\}_{n \geq 0}\) is recurrent, then every non-negative harmonic function on \(G\) is constant.
    \item[(c)] Show that if \(\{X_n\}_{n \in \mathbb{N}}\) is transient (in which case \(V(G)\) is infinite), then for any vertex \(x_0 \in V(G)\) there is a non-constant function on \(G\) which takes values in \([0, 1]\) and is harmonic at every vertex of \(G\) except for \(x_0\).
\end{itemize}

\newpage

\section*{Problem 8 (Optional)}
We model a sequence of gamblings as follows. Let \(\xi_1, \xi_2, \ldots\) be i.i.d. random variables with \(\mathbb{P}\{\xi_n = +1\} = p\), \(\mathbb{P}\{\xi_n = -1\} = q\), where \(p = 1 - q > \frac{1}{2}\). Define the entropy of this distribution by
\[
\alpha = p \ln\left(\frac{p}{1/2}\right) + q \ln\left(\frac{q}{1/2}\right) = p \ln p + q \ln q + \ln 2.
\]
A gambler starts playing with initial fortune \(Y_0 > 0\), and her fortune after round \(n\) is
\[
Y_n = Y_{n-1} + C_n \xi_n,
\]
where \(C_n\) is the amount she bets in this round. The bet \(C_n\) may depend on the values \(\xi_1, \xi_2, \ldots, \xi_{n-1}\), and satisfies \(0 \leq C_n < Y_{n-1}\).

The expected rate of winnings up to time \(n\) is
\[
r_n := \mathbb{E} \left[\ln\left(\frac{Y_n}{Y_0}\right)\right],
\]
which the gambler wishes to maximize.

\begin{itemize}
    \item[(a)] Prove that no matter what strategy \(C\) the gambler chooses,
    \[
    X_n := \ln Y_n - n\alpha
    \]
    is a supermartingale (i.e., \(\mathbb{E}[X_n \mid \mathcal{F}_{n-1}] \leq X_{n-1}\)), hence her expected average winning rate \(r_n / n \leq \alpha\).
    \item[(b)] Find a gambling strategy that makes the above \(X_n\) a martingale, thus achieving the expected average winning rate \(\alpha\).
\end{itemize}











\end{document}