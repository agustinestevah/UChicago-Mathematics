\documentclass[11pt]{article}
\usepackage{float}
\usepackage{tikz}
\usetikzlibrary{automata, positioning}
% NOTE: Add in the relevant information to the commands below; or, if you'll be using the same information frequently, add these commands at the top of paolo-pset.tex file. 
\newcommand{\name}{Agustín Esteva}
\newcommand{\email}{aesteva@uchicago.edu}
\newcommand{\classnum}{23500}
\newcommand{\subject}{Markov Chains, Martingales, and Brownian Motion}
\newcommand{\instructors}{Stephen Yearwood}
\newcommand{\assignment}{Problem Set 5}
\newcommand{\semester}{Spring 2025}
\newcommand{\duedate}{05/09/2025}
\newcommand{\bA}{\mathbf{A}}
\newcommand{\bB}{\mathbf{B}}
\newcommand{\bC}{\mathbf{C}}
\newcommand{\bD}{\mathbf{D}}
\newcommand{\bE}{\mathbf{E}}
\newcommand{\bF}{\mathbf{F}}
\newcommand{\bG}{\mathbf{G}}
\newcommand{\bH}{\mathbf{H}}
\newcommand{\bI}{\mathbf{I}}
\newcommand{\bJ}{\mathbf{J}}
\newcommand{\bK}{\mathbf{K}}
\newcommand{\bL}{\mathbf{L}}
\newcommand{\bM}{\mathbf{M}}
\newcommand{\bN}{\mathbf{N}}
\newcommand{\bO}{\mathbf{O}}
\newcommand{\bP}{\mathbf{P}}
\newcommand{\bQ}{\mathbf{Q}}
\newcommand{\bR}{\mathbf{R}}
\newcommand{\bS}{\mathbf{S}}
\newcommand{\bT}{\mathbf{T}}
\newcommand{\bU}{\mathbf{U}}
\newcommand{\bV}{\mathbf{V}}
\newcommand{\bW}{\mathbf{W}}
\newcommand{\bX}{\mathbf{X}}
\newcommand{\bY}{\mathbf{Y}}
\newcommand{\bZ}{\mathbf{Z}}
\newcommand{\Vol}{\text{Vol}}

%% blackboard bold math capitals
\newcommand{\bbA}{\mathbb{A}}
\newcommand{\bbB}{\mathbb{B}}
\newcommand{\bbC}{\mathbb{C}}
\newcommand{\bbD}{\mathbb{D}}
\newcommand{\bbE}{\mathbb{E}}
\newcommand{\bbF}{\mathbb{F}}
\newcommand{\bbG}{\mathbb{G}}
\newcommand{\bbH}{\mathbb{H}}
\newcommand{\bbI}{\mathbb{I}}
\newcommand{\bbJ}{\mathbb{J}}
\newcommand{\bbK}{\mathbb{K}}
\newcommand{\bbL}{\mathbb{L}}
\newcommand{\bbM}{\mathbb{M}}
\newcommand{\bbN}{\mathbb{N}}
\newcommand{\bbO}{\mathbb{O}}
\newcommand{\bbP}{\mathbb{P}}
\newcommand{\bbQ}{\mathbb{Q}}
\newcommand{\bbR}{\mathbb{R}}
\newcommand{\bbS}{\mathbb{S}}
\newcommand{\bbT}{\mathbb{T}}
\newcommand{\bbU}{\mathbb{U}}
\newcommand{\bbV}{\mathbb{V}}
\newcommand{\bbW}{\mathbb{W}}
\newcommand{\bbX}{\mathbb{X}}
\newcommand{\bbY}{\mathbb{Y}}
\newcommand{\bbZ}{\mathbb{Z}}

%% script math capitals
\newcommand{\sA}{\mathscr{A}}
\newcommand{\sB}{\mathscr{B}}
\newcommand{\sC}{\mathscr{C}}
\newcommand{\sD}{\mathscr{D}}
\newcommand{\sE}{\mathscr{E}}
\newcommand{\sF}{\mathscr{F}}
\newcommand{\sG}{\mathscr{G}}
\newcommand{\sH}{\mathscr{H}}
\newcommand{\sI}{\mathscr{I}}
\newcommand{\sJ}{\mathscr{J}}
\newcommand{\sK}{\mathscr{K}}
\newcommand{\sL}{\mathscr{L}}
\newcommand{\sM}{\mathscr{M}}
\newcommand{\sN}{\mathscr{N}}
\newcommand{\sO}{\mathscr{O}}
\newcommand{\sP}{\mathscr{P}}
\newcommand{\sQ}{\mathscr{Q}}
\newcommand{\sR}{\mathscr{R}}
\newcommand{\sS}{\mathscr{S}}
\newcommand{\sT}{\mathscr{T}}
\newcommand{\sU}{\mathscr{U}}
\newcommand{\sV}{\mathscr{V}}
\newcommand{\sW}{\mathscr{W}}
\newcommand{\sX}{\mathscr{X}}
\newcommand{\sY}{\mathscr{Y}}
\newcommand{\sZ}{\mathscr{Z}}


\renewcommand{\emptyset}{\O}

\newcommand{\abs}[1]{\lvert #1 \rvert}
\newcommand{\norm}[1]{\lVert #1 \rVert}
\newcommand{\sm}{\setminus}


\newcommand{\sarr}{\rightarrow}
\newcommand{\arr}{\longrightarrow}

% NOTE: Defining collaborators is optional; to not list collaborators, comment out the line below.
%\newcommand{\collaborators}{Alyssa P. Hacker (\texttt{aphacker}), Ben Bitdiddle (\texttt{bitdiddle})}

% Copyright 2021 Paolo Adajar (padajar.com, paoloadajar@mit.edu)
% 
% Permission is hereby granted, free of charge, to any person obtaining a copy of this software and associated documentation files (the "Software"), to deal in the Software without restriction, including without limitation the rights to use, copy, modify, merge, publish, distribute, sublicense, and/or sell copies of the Software, and to permit persons to whom the Software is furnished to do so, subject to the following conditions:
%
% The above copyright notice and this permission notice shall be included in all copies or substantial portions of the Software.
% 
% THE SOFTWARE IS PROVIDED "AS IS", WITHOUT WARRANTY OF ANY KIND, EXPRESS OR IMPLIED, INCLUDING BUT NOT LIMITED TO THE WARRANTIES OF MERCHANTABILITY, FITNESS FOR A PARTICULAR PURPOSE AND NONINFRINGEMENT. IN NO EVENT SHALL THE AUTHORS OR COPYRIGHT HOLDERS BE LIABLE FOR ANY CLAIM, DAMAGES OR OTHER LIABILITY, WHETHER IN AN ACTION OF CONTRACT, TORT OR OTHERWISE, ARISING FROM, OUT OF OR IN CONNECTION WITH THE SOFTWARE OR THE USE OR OTHER DEALINGS IN THE SOFTWARE.

\usepackage{fullpage}
\usepackage{enumitem}
\usepackage{amsfonts, amssymb, amsmath,amsthm}
\usepackage{mathtools}
\usepackage[pdftex, pdfauthor={\name}, pdftitle={\classnum~\assignment}]{hyperref}
\usepackage[dvipsnames]{xcolor}
\usepackage{bbm}
\usepackage{graphicx}
\usepackage{mathrsfs}
\usepackage{pdfpages}
\usepackage{tabularx}
\usepackage{pdflscape}
\usepackage{makecell}
\usepackage{booktabs}
\usepackage{natbib}
\usepackage{caption}
\usepackage{subcaption}
\usepackage{physics}
\usepackage[many]{tcolorbox}
\usepackage{version}
\usepackage{ifthen}
\usepackage{cancel}
\usepackage{listings}
\usepackage{courier}

\usepackage{tikz}
\usepackage{istgame}

\hypersetup{
	colorlinks=true,
	linkcolor=blue,
	filecolor=magenta,
	urlcolor=blue,
}

\setlength{\parindent}{0mm}
\setlength{\parskip}{2mm}

\setlist[enumerate]{label=({\alph*})}
\setlist[enumerate, 2]{label=({\roman*})}

\allowdisplaybreaks[1]

\newcommand{\psetheader}{
	\ifthenelse{\isundefined{\collaborators}}{
		\begin{center}
			{\setlength{\parindent}{0cm} \setlength{\parskip}{0mm}
				
				{\textbf{\classnum~\semester:~\assignment} \hfill \name}
				
				\subject \hfill \href{mailto:\email}{\tt \email}
				
				Instructor(s):~\instructors \hfill Due Date:~\duedate	
				
				\hrulefill}
		\end{center}
	}{
		\begin{center}
			{\setlength{\parindent}{0cm} \setlength{\parskip}{0mm}
				
				{\textbf{\classnum~\semester:~\assignment} \hfill \name\footnote{Collaborator(s): \collaborators}}
				
				\subject \hfill \href{mailto:\email}{\tt \email}
				
				Instructor(s):~\instructors \hfill Due Date:~\duedate	
				
				\hrulefill}
		\end{center}
	}
}

\renewcommand{\thepage}{\classnum~\assignment \hfill \arabic{page}}

\makeatletter
\def\points{\@ifnextchar[{\@with}{\@without}}
\def\@with[#1]#2{{\ifthenelse{\equal{#2}{1}}{{[1 point, #1]}}{{[#2 points, #1]}}}}
\def\@without#1{\ifthenelse{\equal{#1}{1}}{{[1 point]}}{{[#1 points]}}}
\makeatother

\newtheoremstyle{theorem-custom}%
{}{}%
{}{}%
{\itshape}{.}%
{ }%
{\thmname{#1}\thmnumber{ #2}\thmnote{ (#3)}}

\theoremstyle{theorem-custom}

\newtheorem{theorem}{Theorem}
\newtheorem{lemma}[theorem]{Lemma}
\newtheorem{example}[theorem]{Example}

\newenvironment{problem}[1]{\color{black} #1}{}

\newenvironment{solution}{%
	\leavevmode\begin{tcolorbox}[breakable, colback=green!5!white,colframe=green!75!black, enhanced jigsaw] \proof[\scshape Solution:] \setlength{\parskip}{2mm}%
	}{\renewcommand{\qedsymbol}{$\blacksquare$} \endproof \end{tcolorbox}}

\newenvironment{reflection}{\begin{tcolorbox}[breakable, colback=black!8!white,colframe=black!60!white, enhanced jigsaw, parbox = false]\textsc{Reflections:}}{\end{tcolorbox}}

\newcommand{\qedh}{\renewcommand{\qedsymbol}{$\blacksquare$}\qedhere}

\definecolor{mygreen}{rgb}{0,0.6,0}
\definecolor{mygray}{rgb}{0.5,0.5,0.5}
\definecolor{mymauve}{rgb}{0.58,0,0.82}

% from https://github.com/satejsoman/stata-lstlisting
% language definition
\lstdefinelanguage{Stata}{
	% System commands
	morekeywords=[1]{regress, reg, summarize, sum, display, di, generate, gen, bysort, use, import, delimited, predict, quietly, probit, margins, test},
	% Reserved words
	morekeywords=[2]{aggregate, array, boolean, break, byte, case, catch, class, colvector, complex, const, continue, default, delegate, delete, do, double, else, eltypedef, end, enum, explicit, export, external, float, for, friend, function, global, goto, if, inline, int, local, long, mata, matrix, namespace, new, numeric, NULL, operator, orgtypedef, pointer, polymorphic, pragma, private, protected, public, quad, real, return, rowvector, scalar, short, signed, static, strL, string, struct, super, switch, template, this, throw, transmorphic, try, typedef, typename, union, unsigned, using, vector, version, virtual, void, volatile, while,},
	% Keywords
	morekeywords=[3]{forvalues, foreach, set},
	% Date and time functions
	morekeywords=[4]{bofd, Cdhms, Chms, Clock, clock, Cmdyhms, Cofc, cofC, Cofd, cofd, daily, date, day, dhms, dofb, dofC, dofc, dofh, dofm, dofq, dofw, dofy, dow, doy, halfyear, halfyearly, hh, hhC, hms, hofd, hours, mdy, mdyhms, minutes, mm, mmC, mofd, month, monthly, msofhours, msofminutes, msofseconds, qofd, quarter, quarterly, seconds, ss, ssC, tC, tc, td, th, tm, tq, tw, week, weekly, wofd, year, yearly, yh, ym, yofd, yq, yw,},
	% Mathematical functions
	morekeywords=[5]{abs, ceil, cloglog, comb, digamma, exp, expm1, floor, int, invcloglog, invlogit, ln, ln1m, ln, ln1p, ln, lnfactorial, lngamma, log, log10, log1m, log1p, logit, max, min, mod, reldif, round, sign, sqrt, sum, trigamma, trunc,},
	% Matrix functions
	morekeywords=[6]{cholesky, coleqnumb, colnfreeparms, colnumb, colsof, corr, det, diag, diag0cnt, el, get, hadamard, I, inv, invsym, issymmetric, J, matmissing, matuniform, mreldif, nullmat, roweqnumb, rownfreeparms, rownumb, rowsof, sweep, trace, vec, vecdiag, },
	% Programming functions
	morekeywords=[7]{autocode, byteorder, c, _caller, chop, abs, clip, cond, e, fileexists, fileread, filereaderror, filewrite, float, fmtwidth, has_eprop, inlist, inrange, irecode, matrix, maxbyte, maxdouble, maxfloat, maxint, maxlong, mi, minbyte, mindouble, minfloat, minint, minlong, missing, r, recode, replay, return, s, scalar, smallestdouble,},
	% Random-number functions
	morekeywords=[8]{rbeta, rbinomial, rcauchy, rchi2, rexponential, rgamma, rhypergeometric, rigaussian, rlaplace, rlogistic, rnbinomial, rnormal, rpoisson, rt, runiform, runiformint, rweibull, rweibullph,},
	% Selecting time-span functions
	morekeywords=[9]{tin, twithin,},
	% Statistical functions
	morekeywords=[10]{betaden, binomial, binomialp, binomialtail, binormal, cauchy, cauchyden, cauchytail, chi2, chi2den, chi2tail, dgammapda, dgammapdada, dgammapdadx, dgammapdx, dgammapdxdx, dunnettprob, exponential, exponentialden, exponentialtail, F, Fden, Ftail, gammaden, gammap, gammaptail, hypergeometric, hypergeometricp, ibeta, ibetatail, igaussian, igaussianden, igaussiantail, invbinomial, invbinomialtail, invcauchy, invcauchytail, invchi2, invchi2tail, invdunnettprob, invexponential, invexponentialtail, invF, invFtail, invgammap, invgammaptail, invibeta, invibetatail, invigaussian, invigaussiantail, invlaplace, invlaplacetail, invlogistic, invlogistictail, invnbinomial, invnbinomialtail, invnchi2, invnF, invnFtail, invnibeta, invnormal, invnt, invnttail, invpoisson, invpoissontail, invt, invttail, invtukeyprob, invweibull, invweibullph, invweibullphtail, invweibulltail, laplace, laplaceden, laplacetail, lncauchyden, lnigammaden, lnigaussianden, lniwishartden, lnlaplaceden, lnmvnormalden, lnnormal, lnnormalden, lnwishartden, logistic, logisticden, logistictail, nbetaden, nbinomial, nbinomialp, nbinomialtail, nchi2, nchi2den, nchi2tail, nF, nFden, nFtail, nibeta, normal, normalden, npnchi2, npnF, npnt, nt, ntden, nttail, poisson, poissonp, poissontail, t, tden, ttail, tukeyprob, weibull, weibullden, weibullph, weibullphden, weibullphtail, weibulltail,},
	% String functions 
	morekeywords=[11]{abbrev, char, collatorlocale, collatorversion, indexnot, plural, plural, real, regexm, regexr, regexs, soundex, soundex_nara, strcat, strdup, string, strofreal, string, strofreal, stritrim, strlen, strlower, strltrim, strmatch, strofreal, strofreal, strpos, strproper, strreverse, strrpos, strrtrim, strtoname, strtrim, strupper, subinstr, subinword, substr, tobytes, uchar, udstrlen, udsubstr, uisdigit, uisletter, ustrcompare, ustrcompareex, ustrfix, ustrfrom, ustrinvalidcnt, ustrleft, ustrlen, ustrlower, ustrltrim, ustrnormalize, ustrpos, ustrregexm, ustrregexra, ustrregexrf, ustrregexs, ustrreverse, ustrright, ustrrpos, ustrrtrim, ustrsortkey, ustrsortkeyex, ustrtitle, ustrto, ustrtohex, ustrtoname, ustrtrim, ustrunescape, ustrupper, ustrword, ustrwordcount, usubinstr, usubstr, word, wordbreaklocale, worcount,},
	% Trig functions
	morekeywords=[12]{acos, acosh, asin, asinh, atan, atanh, cos, cosh, sin, sinh, tan, tanh,},
	morecomment=[l]{//},
	% morecomment=[l]{*},  // `*` maybe used as multiply operator. So use `//` as line comment.
	morecomment=[s]{/*}{*/},
	% The following is used by macros, like `lags'.
	morestring=[b]{`}{'},
	% morestring=[d]{'},
	morestring=[b]",
	morestring=[d]",
	% morestring=[d]{\\`},
	% morestring=[b]{'},
	sensitive=true,
}

\lstset{ 
	backgroundcolor=\color{white},   % choose the background color; you must add \usepackage{color} or \usepackage{xcolor}; should come as last argument
	basicstyle=\footnotesize\ttfamily,        % the size of the fonts that are used for the code
	breakatwhitespace=false,         % sets if automatic breaks should only happen at whitespace
	breaklines=true,                 % sets automatic line breaking
	captionpos=b,                    % sets the caption-position to bottom
	commentstyle=\color{mygreen},    % comment style
	deletekeywords={...},            % if you want to delete keywords from the given language
	escapeinside={\%*}{*)},          % if you want to add LaTeX within your code
	extendedchars=true,              % lets you use non-ASCII characters; for 8-bits encodings only, does not work with UTF-8
	firstnumber=0,                % start line enumeration with line 1000
	frame=single,	                   % adds a frame around the code
	keepspaces=true,                 % keeps spaces in text, useful for keeping indentation of code (possibly needs columns=flexible)
	keywordstyle=\color{blue},       % keyword style
	language=Octave,                 % the language of the code
	morekeywords={*,...},            % if you want to add more keywords to the set
	numbers=left,                    % where to put the line-numbers; possible values are (none, left, right)
	numbersep=5pt,                   % how far the line-numbers are from the code
	numberstyle=\tiny\color{mygray}, % the style that is used for the line-numbers
	rulecolor=\color{black},         % if not set, the frame-color may be changed on line-breaks within not-black text (e.g. comments (green here))
	showspaces=false,                % show spaces everywhere adding particular underscores; it overrides 'showstringspaces'
	showstringspaces=false,          % underline spaces within strings only
	showtabs=false,                  % show tabs within strings adding particular underscores
	stepnumber=2,                    % the step between two line-numbers. If it's 1, each line will be numbered
	stringstyle=\color{mymauve},     % string literal style
	tabsize=2,	                   % sets default tabsize to 2 spaces
%	title=\lstname,                   % show the filename of files included with \lstinputlisting; also try caption instead of title
	xleftmargin=0.25cm
}

% NOTE: To compile a version of this pset without problems, solutions, or reflections, uncomment the relevant line below.

%\excludeversion{problem}
%\excludeversion{solution}
%\excludeversion{reflection}

\begin{document}	
	
	% Use the \psetheader command at the beginning of a pset. 
	\psetheader
\section*{Problem 1}
Let \(\{X_n\}\) be a branching process started with a single individual, so that \(X_0 = 1\) and \(X_n\) is the number of individuals in generation \(n\). Let \(\{p_k\}_{k \geq 0}\) be the offspring distribution. Assume \(p_0 > 0\) and let \(\mu = \sum_{k=0}^\infty k p_k\) be the mean of the offspring distribution.

(a) Show that \(M_n = \mu^{-n} X_n\) is a martingale with respect to \(\mathcal{F}_n = \sigma(X_0, \ldots, X_n)\).
\begin{solution}
Note that $M_n$ is only well defined when $p_0 <1,$ so we will assume this. 

    $M_n$ is trivially $\mathcal{F}_n$ measurable. 

    Since $X_n \geq 0$ for any $n,$ we have that by a result in class about branching processes,
    \[\bbE[|M_n|] = \frac{1}{\mu^n}\bbE[|X_n|] =\frac{1}{\mu^n}\bbE[X_n]= \frac{1}{\mu^n}\mu^n \bbE[X_0] = 1  \]

    For the Martingale property, we let $\xi_i$ be the number of offspring produced by  individual $i$. We know that $\xi_i$ is independent of $X_n$ and we also infer that $X_{n} = \sum_{i=1}^{X_n}\xi_i = \sum_{i=1}^{X_n}\xi = X_n\xi.$ Thus, since $X_{n-1}$ is $\mathcal{F}_{n-1}$ measurable
    \begin{align*}
        \bbE[M_n \mid \mathcal{F}_{n-1}] &= \frac{1}{\mu^n}\bbE[X_n \mid \mathcal{F}_{n-1}]\\
        &= \frac{1}{\mu^n}\bbE[\sum_{i=1}^{X_n}\xi_i \mid \mathcal{F_{n-1}}]\\
        &=\frac{1}{\mu^n}\sum_{i=1}^{X_n}\bbE[\xi_i \mid \mathcal{F_{n-1}}]\\
        &= \frac{1}{\mu^n}X_{n-1} \bbE[\xi]\\
        &= \frac{1}{\mu^n}X_{n-1} \mu\\
        &= M_{n-1}
    \end{align*}
\end{solution}

(b) Suppose that \(\mu = 1\). For each \(K \in \mathbb{N}\), use the optional stopping theorem applied to the stopping time
\[
T_K = \min\{n \geq 1 : X_n = 0 \text{ or } X_n \geq K\}
\]
to show that the probability that the population reaches at least \(K\) individuals before going extinct is at most \(1/K\).
\begin{solution}
    Assumption of the OST:
    \begin{itemize}
        \item Since $\mu = 1$ and $p_0 \neq 0,$ we have by a result in class that with probability $1,$ the population will go extinct. Thus, $X_n = 0$ for some large $n,$ and so  
        \[\bbP\{T_K < \infty\} = 1.\]
        \item Since $X_n$ is non-negative for all $n$ and $\mu = 1,$
        \[\bbE[|M_{T_K}|] = \bbE[M_{T_K}] = \bbE[X_{T_K}]  \leq \sum_{i=1}^R \bbE[\xi_i] = R\mu = R\] 
        WRONG
        \item By the above work, we have that 
        \[\bbE[M_n \mathbbm{1}_{T_K \geq n}] \leq C \bbP\{T_K\ \geq n\} \to 0\] since the population must go extinct at some point.
    \end{itemize}
    By the OST, we have that 
    \[\bbE[M_{T_k}] = \bbE[M_0] = 1.\] But since $\mu = 1,$
    \[1 =\bbE[M_{T_K}] \geq 0P_L + KP_W \implies P_W \leq \frac{1}{K},\] where $P_W$ is the probability that we get to $K$ before $0$ and $P_L$ is the other one.
\end{solution}

(c) Use part (b) to show that the extinction probability is 1 if \(\mu = 1\).
\begin{solution}
Since $P_L = 1-P_W,$ where 
\[P_L = \bbP\{X_n= 0 \text{ before }X_n = K\} = a = 1-P_W \geq 1-\frac{1}{K},\] then as $K\to \infty,$ 
\[1 \geq P_L \geq 1 \implies a = 1.\]
\end{solution}

\newpage

\section*{Problem 2}
Let \(\{M_n\}\) be a martingale with respect to its natural filtration \(\{F_n\}\), and let \(\tau\) be a stopping time for \(M_n\) with \(\mathbb{E}[\tau] < \infty\). Suppose that there exists a constant \(K > 0\) such that \(|M_{n+1} - M_n| \leq K\) for all \(n\). Show that \(\mathbb{E}[M_\tau] = \mathbb{E}[M_0]\).
\begin{solution}
    It suffices to show we can apply the OST.
    \begin{itemize}
    \item Since $\bbE[\tau] < \infty$ and 
    \[\bbE[\tau] = \sum_{n=1}^\infty \bbP\{\tau \geq n\} < \infty,\] then the tail end of the sum must go to zero and thus $\bbP\{\tau \geq n\} \to 0,$ and so $\bbP\{\tau = \infty\} = 0$ (another way to see this is Markov's inequality)
    \item Using the triangle inequality, we have that 
    \begin{align*}
        \bbE[|M_\tau|] &= \bbE[|M_\tau - M_{\tau -1} + M_{\tau - 1} - M_{\tau - 2} + \dots M_0|]\\
        &\leq \bbE[|M_\tau -M_{\tau -1}| + |M_{\tau - 1} - M_{\tau - 2}| + \dots + |M_1 - M_0|]\\
        &\leq K\bbE[\tau] < \infty
    \end{align*} 
    \item By the first bullet point and similar logic to the second one, we have that 
    \[\bbE[M_n \mathbbm{1}_{\tau \geq n}] \leq nK\bbP\{\tau \geq n\}\to 0\]
    \end{itemize}
\end{solution}

\newpage

\section*{Problem 3 (Optional)}
Let \(N\) be a fixed positive integer, and let \(A\) be an arbitrary alphabet of size \(N\), which we view as a collection of characters \(\{s_i\}_{i \in \{1, \ldots, N\}}\). A string of length \(l \in \mathbb{N}\) is a concatenation of elements of \(A\), written as
\[
s_{j_1} s_{j_2} \ldots s_{j_l},
\]
where the indices \(j_k\) may or may not be distinct. Suppose each character \(s_i\) has probability \(p_i\) of being selected, and let \(S\) be an arbitrary (finite) string. We wish to compute the expected time until the string \(S\) is first observed, if we repeatedly sample according to the probabilities \(p_i\). To that end, let \(\{X_n\}_{n \in \mathbb{N}}\) denote the characters sampled up to time \(n\), and let

\[
T := \min\{n \geq 0 : X_{n-|S|+1} X_{n-|S|+2} \ldots X_n = S\}.
\]

Let \(L_n(S)\) be the first \(n\) (leftmost) characters of \(S\), and let \(R_n(S)\) be the last \(n\) (rightmost) characters of \(S\). Show that
\[
\mathbb{E}[T] = \sum_{i=1}^{|S|} \left(\prod_{j=1}^i p_j \right)^{-1} \mathbbm{1}_{\{R_i(S) = L_i(S)\}}.
\]

Give the analogous formula in the case of uniform sampling, and give a condition for a string \(S\) to maximize this expected time.

\newpage

\section*{Problem 4}
Suppose \(X_1, X_2, \ldots\) are independent random variables with distribution
\[
\mathbb{P}(X_j = 3) = 1 - \mathbb{P}\left(X_j = \frac{1}{3}\right) = \frac{1}{4}.
\]

Let \(M_0 = 1\) and for \(n \geq 1\),
\[
M_n = \prod_{j=1}^n X_j.
\]

(a) Show that \(M_n\) is a martingale with respect to \(F_n = \sigma(X_1, \ldots, X_n)\).
\begin{solution}
\begin{itemize}
    \item We see that $M_n$ is $\mathcal{F}_n-$measurable.
    \item Since the $X_i$s are independent, we can distribute the expectation over the product and see that  since everything is non-negative,
    \begin{align*}
        \bbE[|M_n|] &= \bbE[M_n]\\
        &= \bbE\left[\prod_{j=1}^n X_j\right]\\
        &= \prod_{j=1}^n\bbE[X_j]\\
        &= (\bbE[X])^n\\
        &= 1
    \end{align*} where we use the fact that $\bbE[X_j] = \bbE[X] = 1$ for any $j.$ 
    \item For the Martingale property, 
    \begin{align*}
        \bbE[M_n \mid \mathcal{F}_{n-1}]
        &= \bbE\left[\prod_{j=1}^{n-1}X_{j}\cdot  X_n\mid \mathcal{F}_{n-1}\right]\\
        &= \prod_{j=1}^{n-1}X_j \cdot \bbE[X_n \mid \mathcal{F}_{n-1]}]\\
        &= M_{n-1}\bbE[X_n]\\
        &= M_{n-1},
    \end{align*}
    where we use the fact that $X_i$ are $\mathcal{F}_{n-1}$ measurable for $i\leq n-1$ and that $X_n$ is independent of $X_i$ for $i< n.$
\end{itemize}



\end{solution}

(b) Use the optional stopping theorem to show that the probability that the value of \(M_n\) ever gets as high as $3^6$ equals \(3^{-6}\).
\begin{solution}
    Define $T_m = \min\{j : M_j = 3^6 \text{ or } M_j = 3^{-m}\}.$ We see that $T_m$ is a stopping time for $M_n.$ 
    \begin{itemize}
        \item We can reach $3^6$ in $6$ steps, and so 
        \[\bbP\{T_m  \geq 6\} \leq \frac{1}{4^6}\implies \bbP\{T_m \geq 6(2k)\}\leq \frac{1}{4^{6(2k)}} \implies \bbP\{T_m \geq k\} \leq \frac{1}{4^k},\] where we are bounding the probability by the event where $M_n$ keeps bouncing  between $1$ and $\frac{1}{3}$ $2k$ times until it decides to go $6$ times to $3^6$ and so as $k\to \infty,$ we find that 
        \[\bbP\{T_m \geq mk\} \to 0 = \bbP\{T_m = \infty\}.\]
        \item We see that by the previous bullet point
        \[\bbE[T_m] =  \sum_{k=1}^\infty \bbP\{T_m \geq k\} \leq \sum_{k=1}^\infty \frac{1}{4^k} < \infty\]
        \item We can bound
        \[\bbE[M_n \mathbbm{1}_{T_m \geq n}] \leq 3^6 \bbP\{T_m \geq n\} \to 0\]
    \end{itemize}
    Thus, we use the optional the optional stopping theorem that states that 
    \[\bbE[M_{T_m}] = \bbE[M_0] = 1.\] But
    \[1 = \bbE[M_{T_m}]= 3^6 P_{3^6} + 3^{-m}P_{3^{-m}},\] where 
    \[P_{3^6} = \bbP\{M_n = 3^6 \text{ before }M_n = 3^{-m}\}.\] Taking $m\to \infty,$ we see that 
    \[P_{3^6} = \frac{1}{3^6}.\]
\end{solution}

(c) Show that there exists \(M_\infty\) such that, with probability one, \(M_n \to M_\infty\).
\begin{solution}
    It suffices to show that $M_n$ satisfies the conditions of the Martingale Convergence Theorem. However, we showed in the very first step that $\bbE[|M_n|] = 1,$ and so we are done by the MCT.
\end{solution}

(d) Does there exist a \(C < \infty\) such that for all \(n\), \(\mathbb{E}[M_n^2] \leq C\)?
\begin{solution}
    \textbf{No.}
    \begin{align*}
        \bbE[M_n^2] &= \bbE\left[\prod_{j=1}^n X_j^2\right]\\
        &= \prod_{j=1}^n\bbE[X]\\
        &= \bbE[X]^n\\
        &>1^n\\
        &\to \infty
    \end{align*}
\end{solution}

\newpage

\section*{Problem 5}
Define random variables \(\{X_n\}\) recursively by \(X_0 = 1\) and for \(n \geq 1\), \(X_n\) is sampled uniformly from \((0, X_{n-1})\).

(a) Show that \(M_n := 2^n X_n\) is a martingale.
\begin{solution}
    \begin{itemize}
        \item I feel like I get tired of saying `trivial' or `clearly,' or `evidently'. `Obviously' is lowkey condescending. It is left as an exercise to the grader that $M_n$ is $\mathcal{F}_n-$measurable
        \item We claim that 
        \[\bbE[X_n] = \frac{1}{2^n}.\] For $n=1,$ we have that $X_1 \sim U([0,X_0]) = U([0,1]),$ and so $\bbE[X_1] = \frac{1}{2}.$ Supose this holds for a general $n=k.$ For $n = k+1,$ we see that 
        \[\bbE[X_{k+1}] = \frac{\bbE[X_k]}{2} = \frac{1}{2^{k+1}}\]
        
        Thus, 
        \[\bbE[|M_n|] = 2^n\bbE[X_n] = 1.\]
        \item For the martingale property, 
        \begin{align*}
            \bbE[M_n \mid \mathcal{F}_{n-1}] &= \bbE[2^n X_n \mid \mathcal{F}_{n-1}]\\
            &= 2^n \bbE[X_{n} \mid {X}_{n-1}]\\
            &= 2^n \frac{X_{n-1}}{n}\\
            &= 2^{n-1}X_{n-1}
        \end{align*}
    \end{itemize}
\end{solution}

(b) Show that there exists \(M_\infty\) such that, with probability one, \(M_n \to M_\infty\).
\begin{solution}
    We showed above that $\bbE[M_n] = 1$ for all $n,$ and so by the MCT we are done.
\end{solution}

(c) Find \(M_\infty\). (Hint: Consider \(\log M_n\).)
\begin{solution}
    We have that 
    \[\log M_n = n \log 2  +\log X_n\to \infty - \infty =0 \] what 
\end{solution}

\newpage

\section*{Problem 6}
Suppose \(X\) is a standard normal random variable, i.e., \(X \sim \mathcal{N}(0,1)\).

(a) Let \(\Phi(x) := \mathbb{P}(X \leq x)\). Compute \(\mathbb{E}[X \Phi(X)]\).

(b) Let \(Y = |X| + X\). Compute \(\mathbb{E}[Y^3]\).

(c) Show that \(\text{Var}(\sin X) > \text{Var}(\cos X)\).

\newpage

\section*{Problem 7}
Consider the infinite series

\[
\zeta(2k) = \sum_{n=1}^\infty \frac{1}{n^{2k}}, \qquad S(2k) = \sum_{n=0}^\infty \frac{1}{(2n + 1)^{2k}}.
\]

(a) Show that \(\zeta(2k) = \frac{2^{2k}}{2^{2k} - 1} S(2k)\).

\begin{solution}
Computing,
\begin{align*}
\zeta(2k) &= \sum_{n\text{ odd}} \frac{1}{n^{2k}} + \sum_{n \text{ even}} \frac{1}{n^{2k}}\\ &= \sum_{n=0} \frac{1}{(2n + 1)^{2k}} + \sum_{n=0}\frac{1}{(2n)^{2k}}  \\
&= S(2k) + \frac{1}{2^{2k}}\zeta(2k)
\end{align*}
and so rearranging
\[\zeta(2k) = \frac{1}{1-\frac{1}{2^{2k}}}S(2k) = \frac{2^{2k}}{2^{2k} - 1}S(2k)\]

\end{solution}


(b) Suppose \(X\) and \(Y\) are continuous, non-negative, independent random variables with densities \(f_X(x)\) and \(f_Y(y)\). Let \(Z = \frac{Y}{X}\). Show that the density of \(Z\) is given by
\[
f_Z(z) = \int_0^\infty x f_Y(zx) f_X(x) \, dx.
\]

(c) Assume that the random variables \(X\) and \(Y\) obey the Cauchy distribution, i.e.,

\[
f_X(x) = \frac{2}{\pi (1 + x^2)}, \qquad x \geq 0.
\]

Show that

\[
f_Z(z) = \frac{4 \log(z)}{\pi^2 (z^2 - 1)}.
\]

(d) Show that
\[
\int_0^1 \frac{\log z}{z^2 - 1} \, dz = \frac{\pi^2}{8}.
\]

(e) Use the previous part to deduce that \(S(2) = \frac{\pi^2}{8}\).

(f) Conclude that
\[
\zeta(2) = \sum_{n=1}^\infty \frac{1}{n^2} = \frac{\pi^2}{6}.
\]





\end{document}