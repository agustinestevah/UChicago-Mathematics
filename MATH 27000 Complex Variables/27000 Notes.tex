\documentclass[10pt, oneside]{article} 
\usepackage{amsmath, amsthm, amssymb, calrsfs, wasysym, verbatim, bbm, color, graphics, geometry, esint, float}


\geometry{tmargin=.75in, bmargin=.75in, lmargin=.75in, rmargin = .75in}  

\newcommand{\bbR}{\mathbb{R}}
\newcommand{\bbC}{\mathbb{C}}
\newcommand{\bbZ}{\mathbb{Z}}
\newcommand{\bbP}{\mathbb{P}}
\newcommand{\bbN}{\mathbb{N}}
\newcommand{\bbQ}{\mathbb{Q}}
\newcommand{\Cdot}{\boldsymbol{\cdot}}
\newcommand{\scA}{\mathscr{A}}
\newcommand{\curl}{\text{curl}}

\theoremstyle{definition}
\newtheorem{exmp}{Example}[section]
\newtheorem{thm}{Theorem}
\newtheorem{defn}{Definition}
\newtheorem{prop}{Proposition}
\newtheorem{conv}{Convention}
\newtheorem{rem}{Remark}
\newtheorem{lem}{Lemma}
\newtheorem{cor}{Corollary}
\input{paolo-pset.tex}



\title{UChicago Markov Chains, Martingales, and Brownian Motion Analysis Notes: 23500}
\author{Notes by Agustín Esteva, Lectures by Stephen Yearwood, Books by }
\date{Academic Year 2024-2025}

\begin{document}

\maketitle
\tableofcontents

\vspace{.25in}


\newpage
\section{Lectures}

\subsection{Tuesday, Mar 25: $\bbC$ as a field}
Define the Complex numbers:
\[\bbC := \{(x,y) \; | \; x,y \in \bbR\} = \bbR^2\] as a field: 
\[+: \bbC \to \bbC; \quad (a,b) + (c,d) = (a + c, b + d)\]
\[\times: \bbC \to \bbC; \quad (a,b)\times (c,d) = (ac - bd, ad + bc)\] such that for any $z = (a,b) \in \bbC,$ then 
\[z = a + ib,\] where, 
\[a = (a,0) = \Re{z}, \qquad i = (0,1), \qquad b = (0,b) = \Im{z}\]

\begin{rem}
    Why is $\bbC$ not an ordered field? Suppose it is, then there exists some $\mathcal{P}\subseteq \bbC$ of positive elements that is closed under addition and multiplication and that satisfies the trichotomy such that for any $z\in \cal P$, exactly one of the following holds: $z = 0, z\in \mathcal{P}, -z\in \cal P.$ 
    \begin{lem}
        If $z \neq 0,$ then $z^2 \in \cal P.$
    \end{lem}
    \begin{proof}
        If $z\in \cal P,$ then since $\cal P$ is closed under multiplication, $z \times z \in \cal P.$
        If $z\notin \cal P,$ then $-z \in \cal P,$ and thus $(-z)(-z) \in \cal P.$ But then $-z = (-1)(z),$ and so 
        \[(-z)(-z)= (-1)(-1)(z \times z) = z^2 \in \cal P.\]
    \end{proof}
    Consider that since $1$ is a square, then $1 \in \cal P.$ Moreover, since $-1$ is a square (in $\bbC!$), then $-1 \in \cal P.$  
\end{rem}

\begin{defn}
    Let $z\in \bbC$ such that $z = a + ib.$ We say that $\overline{z}$ is the \textbf{complex conjugate} of $z$ if 
    \[ z = a - ib.\]
\end{defn}
That is, we reflect $z$ over the real axis by flipping the sign of the imaginary line. 
\begin{rem}
    \[z \cdot \overline{z} = (a + ib)(a-ib) = (a^2 + b^2, 0) = a^2 + b^2 = |z^2|,\] by the norm defined below in (1). Suppose that $z\neq 0,$ then 
    \[z \cdot \frac{\overline{z}}{|z|} = \frac{|z|^2}{|z|^2} = 1.\] We have found the inverse of any nonzero $z\in \bbC.$ 
\end{rem}
\begin{rem}
It is easy to show the following:
    \[\overline{zw} = \overline{z} \cdot \overline{w}, \qquad  \overline{z + w} = \overline{z} + \overline{w}, \qquad |zw|^2 = |z|^2|w|^2.\]
\end{rem}

\begin{prop}
    $\bbC$ is Banach under the norm, 
    \begin{align}
    \|z\| = \|(a,b)\| = \sqrt{a^2 +b^2}    
    \end{align}
    
\end{prop}
\begin{proof}
    We will first show that the norm satisfies the triangle inequality. It suffices to show that 
    \[|z + w| \leq|z| + |w|,\] and thus we will show that 
    \[|z + w|^2 \leq (|z| + |w|)^2.\] We have by the above remarks that 
    \[|z + w|^2 = (z + w)\overline{(z + w)} = (z + w)(\overline{z} + \overline{w}) = (z\overline{z} + \overline{z}w + \overline{w}z+ w \overline{w}) = (|z|^2 + |w|^2 + z\overline{w} + \overline{z \overline{w}}) = |z|^2 + |w|^2 + 2\Re{z\overline{w}}.\] Thus, we want to show that $\Re{z\overline{w}} \leq |z||w|,$ which comes from the fact that 
    \[\Re{z\overline{w}} \leq |z||\overline{w}| = |z||w|.\]
\end{proof}

\newpage
\subsection{Thursday, Mar 27: The Topology of $\bbC$}
\begin{thm}
    $\bbC$ is complete.
\end{thm}
This theorem is in the sense of Cauchy convergence, since the least upper bound property is meaningless in a non-ordered field, such as $\bbC.$

\begin{rem}
    We denote that disk of radius $r$ around $z_0$ as 
    \[D(z_0, r) = \{z \in \bbC \; | \;|z - z_0| < r\}\]
\end{rem}

\begin{defn}
    We say that $O \subset \bbC$ is \textbf{open} if and only if for all $z_0 \in O,$ there exists some $\epsilon>0$ such that $D(z_0, r) \subset O.$
\end{defn}
It is easy to show that disks are open with the triangle inequality.

\begin{defn}
    Let $O$ be open. We say that $O$ is \textbf{connected} if, whenever $O = O_1 \cup O_2,$ where $O_1, O_2$ are open, disjoint, then at least one of the $O_1,$ $O_2$ are empty. 
\end{defn}


\begin{rem}
    How do we make the real valued $f(x) = x$ continuously differentiable on $[0,1]?$ The endpoints present a problem! We say that $f$ is differentiable at $0$ or at $1$ if the one sided limit exists and agrees with the derivative near the endpoint.
\end{rem}

\begin{defn}
    Let $O$ be an open set. A \textbf{path} is a function $\gamma: [a,b] \to O$ such that $\gamma$ is piecewise continuously differentiable. That is, there exists some finite partition $a = t_0 < t_1 < \dots < t_n = b$ such that $\gamma$ is continuously differentiable on each closed interval $(t_{k-1}, t_k)$ and 
    \[\lim_{t \to t_{k-1}^+} \gamma'(t) = \gamma_+'(t_{k-1}) = \lim_{h\to 0^+} \frac{\gamma(t_{k-1} +h) - \gamma(t_{k-1})}{h}\] and similarly for $t_k.$
\end{defn}

\begin{thm}
    If $O$ is pathwise connected, then it is connected.
\end{thm}
\begin{proof}
    Suppose $O$ is disconnected, then $O = O_1 \sqcup O_2$ and take $[a,b] = \gamma^{-1}(O_1) \sqcup \gamma^{-1}(O_2)$ as a disconnection of the interval, a contradiction!
\end{proof}

\begin{defn}
    Let $O$ be open. We say that a \textbf{polygonal path} inside  of $O$ is a path made up of only horizontal and vertical lines. 
\end{defn}

\begin{thm}
    Suppose $O$ is connected, then $O$ is polygonally connected. 
\end{thm}
\begin{proof}
    Let $z_0 \in O.$ We will show that 
    \[A = \{z \;  | \; \exists \text{ polygonal path between $z_0$ and $z$} \}\] is the same as $O.$ We will first show that $O$ is open. Let $z\in A,$ since $O$ is open, there exists some $r>0$ such that $D(z, r) \subseteq O.$ We will show that for any $z' \in D(z,r),$ $z' \in A.$ It suffices to show that disks are polygonally connected. 

    Since disks are convex, there exists a straight line connecting $z, z'.$ We claim that the imaginary component of this line is in the disk, which is because of the Pythagorean identity (or because $\Im{r} \leq |r|.$) By convexity, the straight line (i.e, the real component) between the imaginary line and $z'$ is in the disk. Thus, we can polygonally connect $z$ and $z'.$

    Thus, $A$ is open. Let $B$ be the set of points in $O$ that cannot be polygonally connected. Evidently, $A \cap B = \emptyset.$ By dichotomy, $A \cup B = O.$ Evidently, $z_0 \in A.$ Clearly, $B$ is open, and so $B$ must be empty and $A = O.$
\end{proof}

\begin{exmp}
Let $t\in [0,1].$ \begin{enumerate}
    \item The straight path from $z_2$ to $z_1$:
    \[\gamma(t) = tz_1 + (1-t)z_2\]
    \item The straight path from $z_1$ to $z_2$ is 
    \[\gamma(t) = (1-t)z_1 + tz_2\]
    \item The circle centered at $z_0$ of radius $r$ counterclockwise. 
    \[\gamma(\theta) = z_0 + r\cos\theta  + ir\sin\theta = z_0 + e^{ir\theta}, \qquad \theta \in [0, 2\pi)\]
\end{enumerate}
\end{exmp}

\begin{defn}
    Suppose $O \subseteq \bbC$ be open, and let $f: O \to \bbC.$ We say that $f$ is \textbf{continuous} at $z_0 \in O$ if 
    \[\lim_{z \to z_0}f(z) = f(z_0), \quad z \in O.\] We say that $f$ is differentiable at $z_0 \in O$ if 
    \[\lim_{h\to 0}\frac{f(z_0 + h) - f(z_0)}{h} = f'(z_0), \quad h \in \bbC\]
\end{defn}
Continuity in $\bbC$ is identical to continuity in $\bbR^2,$ since they are identical as metric spaces. But dividing in $\bbR^2$ makes no sense, so the differentiability is completely different!
\begin{exmp}
    \begin{enumerate}
        \item Suppose $f(z) = z^2,$ then 
        \[\frac{(z + h)^2 - z^2}{h} = 2z + h \to 2z = f'(z)\]
        \item Suppose $f(z) = \overline{z}.$ Then at the origin, 
        \[\lim_{h\to 0}\frac{\overline{(0 + h)} - \overline{0}}{h}= \lim_{h\to 0}\frac{\overline{h}}{h},\] but the looking at a purely real component, the limit approached $1.$ Looking at a purely imaginary components, the limit approaches $-1.$ Oops!
    \end{enumerate}
\end{exmp}


\end{document}