\documentclass[10pt, oneside]{article} 
\usepackage{amsmath, amsthm, amssymb, calrsfs, wasysym, verbatim, bbm, color, graphics, geometry, esint, float}


\geometry{tmargin=.75in, bmargin=.75in, lmargin=.75in, rmargin = .75in}  

\newcommand{\bbR}{\mathbb{R}}
\newcommand{\bbC}{\mathbb{C}}
\newcommand{\bbZ}{\mathbb{Z}}
\newcommand{\bbP}{\mathbb{P}}
\newcommand{\bbN}{\mathbb{N}}
\newcommand{\bbQ}{\mathbb{Q}}
\newcommand{\Cdot}{\boldsymbol{\cdot}}
\newcommand{\scA}{\mathscr{A}}
\newcommand{\curl}{\text{curl}}

\theoremstyle{definition}
\newtheorem{exmp}{Example}[section]
\newtheorem{thm}{Theorem}
\newtheorem{defn}{Definition}
\newtheorem{prop}{Proposition}
\newtheorem{conv}{Convention}
\newtheorem{rem}{Remark}
\newtheorem{lem}{Lemma}
\newtheorem{cor}{Corollary}
% Copyright 2021 Paolo Adajar (padajar.com, paoloadajar@mit.edu)
% 
% Permission is hereby granted, free of charge, to any person obtaining a copy of this software and associated documentation files (the "Software"), to deal in the Software without restriction, including without limitation the rights to use, copy, modify, merge, publish, distribute, sublicense, and/or sell copies of the Software, and to permit persons to whom the Software is furnished to do so, subject to the following conditions:
%
% The above copyright notice and this permission notice shall be included in all copies or substantial portions of the Software.
% 
% THE SOFTWARE IS PROVIDED "AS IS", WITHOUT WARRANTY OF ANY KIND, EXPRESS OR IMPLIED, INCLUDING BUT NOT LIMITED TO THE WARRANTIES OF MERCHANTABILITY, FITNESS FOR A PARTICULAR PURPOSE AND NONINFRINGEMENT. IN NO EVENT SHALL THE AUTHORS OR COPYRIGHT HOLDERS BE LIABLE FOR ANY CLAIM, DAMAGES OR OTHER LIABILITY, WHETHER IN AN ACTION OF CONTRACT, TORT OR OTHERWISE, ARISING FROM, OUT OF OR IN CONNECTION WITH THE SOFTWARE OR THE USE OR OTHER DEALINGS IN THE SOFTWARE.

\usepackage{fullpage}
\usepackage{enumitem}
\usepackage{amsfonts, amssymb, amsmath,amsthm}
\usepackage{mathtools}
\usepackage[pdftex, pdfauthor={\name}, pdftitle={\classnum~\assignment}]{hyperref}
\usepackage[dvipsnames]{xcolor}
\usepackage{bbm}
\usepackage{graphicx}
\usepackage{mathrsfs}
\usepackage{pdfpages}
\usepackage{tabularx}
\usepackage{pdflscape}
\usepackage{makecell}
\usepackage{booktabs}
\usepackage{natbib}
\usepackage{caption}
\usepackage{subcaption}
\usepackage{physics}
\usepackage[many]{tcolorbox}
\usepackage{version}
\usepackage{ifthen}
\usepackage{cancel}
\usepackage{listings}
\usepackage{courier}

\usepackage{tikz}
\usepackage{istgame}

\hypersetup{
	colorlinks=true,
	linkcolor=blue,
	filecolor=magenta,
	urlcolor=blue,
}

\setlength{\parindent}{0mm}
\setlength{\parskip}{2mm}

\setlist[enumerate]{label=({\alph*})}
\setlist[enumerate, 2]{label=({\roman*})}

\allowdisplaybreaks[1]

\newcommand{\psetheader}{
	\ifthenelse{\isundefined{\collaborators}}{
		\begin{center}
			{\setlength{\parindent}{0cm} \setlength{\parskip}{0mm}
				
				{\textbf{\classnum~\semester:~\assignment} \hfill \name}
				
				\subject \hfill \href{mailto:\email}{\tt \email}
				
				Instructor(s):~\instructors \hfill Due Date:~\duedate	
				
				\hrulefill}
		\end{center}
	}{
		\begin{center}
			{\setlength{\parindent}{0cm} \setlength{\parskip}{0mm}
				
				{\textbf{\classnum~\semester:~\assignment} \hfill \name\footnote{Collaborator(s): \collaborators}}
				
				\subject \hfill \href{mailto:\email}{\tt \email}
				
				Instructor(s):~\instructors \hfill Due Date:~\duedate	
				
				\hrulefill}
		\end{center}
	}
}

\renewcommand{\thepage}{\classnum~\assignment \hfill \arabic{page}}

\makeatletter
\def\points{\@ifnextchar[{\@with}{\@without}}
\def\@with[#1]#2{{\ifthenelse{\equal{#2}{1}}{{[1 point, #1]}}{{[#2 points, #1]}}}}
\def\@without#1{\ifthenelse{\equal{#1}{1}}{{[1 point]}}{{[#1 points]}}}
\makeatother

\newtheoremstyle{theorem-custom}%
{}{}%
{}{}%
{\itshape}{.}%
{ }%
{\thmname{#1}\thmnumber{ #2}\thmnote{ (#3)}}

\theoremstyle{theorem-custom}

\newtheorem{theorem}{Theorem}
\newtheorem{lemma}[theorem]{Lemma}
\newtheorem{example}[theorem]{Example}

\newenvironment{problem}[1]{\color{black} #1}{}

\newenvironment{solution}{%
	\leavevmode\begin{tcolorbox}[breakable, colback=green!5!white,colframe=green!75!black, enhanced jigsaw] \proof[\scshape Solution:] \setlength{\parskip}{2mm}%
	}{\renewcommand{\qedsymbol}{$\blacksquare$} \endproof \end{tcolorbox}}

\newenvironment{reflection}{\begin{tcolorbox}[breakable, colback=black!8!white,colframe=black!60!white, enhanced jigsaw, parbox = false]\textsc{Reflections:}}{\end{tcolorbox}}

\newcommand{\qedh}{\renewcommand{\qedsymbol}{$\blacksquare$}\qedhere}

\definecolor{mygreen}{rgb}{0,0.6,0}
\definecolor{mygray}{rgb}{0.5,0.5,0.5}
\definecolor{mymauve}{rgb}{0.58,0,0.82}

% from https://github.com/satejsoman/stata-lstlisting
% language definition
\lstdefinelanguage{Stata}{
	% System commands
	morekeywords=[1]{regress, reg, summarize, sum, display, di, generate, gen, bysort, use, import, delimited, predict, quietly, probit, margins, test},
	% Reserved words
	morekeywords=[2]{aggregate, array, boolean, break, byte, case, catch, class, colvector, complex, const, continue, default, delegate, delete, do, double, else, eltypedef, end, enum, explicit, export, external, float, for, friend, function, global, goto, if, inline, int, local, long, mata, matrix, namespace, new, numeric, NULL, operator, orgtypedef, pointer, polymorphic, pragma, private, protected, public, quad, real, return, rowvector, scalar, short, signed, static, strL, string, struct, super, switch, template, this, throw, transmorphic, try, typedef, typename, union, unsigned, using, vector, version, virtual, void, volatile, while,},
	% Keywords
	morekeywords=[3]{forvalues, foreach, set},
	% Date and time functions
	morekeywords=[4]{bofd, Cdhms, Chms, Clock, clock, Cmdyhms, Cofc, cofC, Cofd, cofd, daily, date, day, dhms, dofb, dofC, dofc, dofh, dofm, dofq, dofw, dofy, dow, doy, halfyear, halfyearly, hh, hhC, hms, hofd, hours, mdy, mdyhms, minutes, mm, mmC, mofd, month, monthly, msofhours, msofminutes, msofseconds, qofd, quarter, quarterly, seconds, ss, ssC, tC, tc, td, th, tm, tq, tw, week, weekly, wofd, year, yearly, yh, ym, yofd, yq, yw,},
	% Mathematical functions
	morekeywords=[5]{abs, ceil, cloglog, comb, digamma, exp, expm1, floor, int, invcloglog, invlogit, ln, ln1m, ln, ln1p, ln, lnfactorial, lngamma, log, log10, log1m, log1p, logit, max, min, mod, reldif, round, sign, sqrt, sum, trigamma, trunc,},
	% Matrix functions
	morekeywords=[6]{cholesky, coleqnumb, colnfreeparms, colnumb, colsof, corr, det, diag, diag0cnt, el, get, hadamard, I, inv, invsym, issymmetric, J, matmissing, matuniform, mreldif, nullmat, roweqnumb, rownfreeparms, rownumb, rowsof, sweep, trace, vec, vecdiag, },
	% Programming functions
	morekeywords=[7]{autocode, byteorder, c, _caller, chop, abs, clip, cond, e, fileexists, fileread, filereaderror, filewrite, float, fmtwidth, has_eprop, inlist, inrange, irecode, matrix, maxbyte, maxdouble, maxfloat, maxint, maxlong, mi, minbyte, mindouble, minfloat, minint, minlong, missing, r, recode, replay, return, s, scalar, smallestdouble,},
	% Random-number functions
	morekeywords=[8]{rbeta, rbinomial, rcauchy, rchi2, rexponential, rgamma, rhypergeometric, rigaussian, rlaplace, rlogistic, rnbinomial, rnormal, rpoisson, rt, runiform, runiformint, rweibull, rweibullph,},
	% Selecting time-span functions
	morekeywords=[9]{tin, twithin,},
	% Statistical functions
	morekeywords=[10]{betaden, binomial, binomialp, binomialtail, binormal, cauchy, cauchyden, cauchytail, chi2, chi2den, chi2tail, dgammapda, dgammapdada, dgammapdadx, dgammapdx, dgammapdxdx, dunnettprob, exponential, exponentialden, exponentialtail, F, Fden, Ftail, gammaden, gammap, gammaptail, hypergeometric, hypergeometricp, ibeta, ibetatail, igaussian, igaussianden, igaussiantail, invbinomial, invbinomialtail, invcauchy, invcauchytail, invchi2, invchi2tail, invdunnettprob, invexponential, invexponentialtail, invF, invFtail, invgammap, invgammaptail, invibeta, invibetatail, invigaussian, invigaussiantail, invlaplace, invlaplacetail, invlogistic, invlogistictail, invnbinomial, invnbinomialtail, invnchi2, invnF, invnFtail, invnibeta, invnormal, invnt, invnttail, invpoisson, invpoissontail, invt, invttail, invtukeyprob, invweibull, invweibullph, invweibullphtail, invweibulltail, laplace, laplaceden, laplacetail, lncauchyden, lnigammaden, lnigaussianden, lniwishartden, lnlaplaceden, lnmvnormalden, lnnormal, lnnormalden, lnwishartden, logistic, logisticden, logistictail, nbetaden, nbinomial, nbinomialp, nbinomialtail, nchi2, nchi2den, nchi2tail, nF, nFden, nFtail, nibeta, normal, normalden, npnchi2, npnF, npnt, nt, ntden, nttail, poisson, poissonp, poissontail, t, tden, ttail, tukeyprob, weibull, weibullden, weibullph, weibullphden, weibullphtail, weibulltail,},
	% String functions 
	morekeywords=[11]{abbrev, char, collatorlocale, collatorversion, indexnot, plural, plural, real, regexm, regexr, regexs, soundex, soundex_nara, strcat, strdup, string, strofreal, string, strofreal, stritrim, strlen, strlower, strltrim, strmatch, strofreal, strofreal, strpos, strproper, strreverse, strrpos, strrtrim, strtoname, strtrim, strupper, subinstr, subinword, substr, tobytes, uchar, udstrlen, udsubstr, uisdigit, uisletter, ustrcompare, ustrcompareex, ustrfix, ustrfrom, ustrinvalidcnt, ustrleft, ustrlen, ustrlower, ustrltrim, ustrnormalize, ustrpos, ustrregexm, ustrregexra, ustrregexrf, ustrregexs, ustrreverse, ustrright, ustrrpos, ustrrtrim, ustrsortkey, ustrsortkeyex, ustrtitle, ustrto, ustrtohex, ustrtoname, ustrtrim, ustrunescape, ustrupper, ustrword, ustrwordcount, usubinstr, usubstr, word, wordbreaklocale, worcount,},
	% Trig functions
	morekeywords=[12]{acos, acosh, asin, asinh, atan, atanh, cos, cosh, sin, sinh, tan, tanh,},
	morecomment=[l]{//},
	% morecomment=[l]{*},  // `*` maybe used as multiply operator. So use `//` as line comment.
	morecomment=[s]{/*}{*/},
	% The following is used by macros, like `lags'.
	morestring=[b]{`}{'},
	% morestring=[d]{'},
	morestring=[b]",
	morestring=[d]",
	% morestring=[d]{\\`},
	% morestring=[b]{'},
	sensitive=true,
}

\lstset{ 
	backgroundcolor=\color{white},   % choose the background color; you must add \usepackage{color} or \usepackage{xcolor}; should come as last argument
	basicstyle=\footnotesize\ttfamily,        % the size of the fonts that are used for the code
	breakatwhitespace=false,         % sets if automatic breaks should only happen at whitespace
	breaklines=true,                 % sets automatic line breaking
	captionpos=b,                    % sets the caption-position to bottom
	commentstyle=\color{mygreen},    % comment style
	deletekeywords={...},            % if you want to delete keywords from the given language
	escapeinside={\%*}{*)},          % if you want to add LaTeX within your code
	extendedchars=true,              % lets you use non-ASCII characters; for 8-bits encodings only, does not work with UTF-8
	firstnumber=0,                % start line enumeration with line 1000
	frame=single,	                   % adds a frame around the code
	keepspaces=true,                 % keeps spaces in text, useful for keeping indentation of code (possibly needs columns=flexible)
	keywordstyle=\color{blue},       % keyword style
	language=Octave,                 % the language of the code
	morekeywords={*,...},            % if you want to add more keywords to the set
	numbers=left,                    % where to put the line-numbers; possible values are (none, left, right)
	numbersep=5pt,                   % how far the line-numbers are from the code
	numberstyle=\tiny\color{mygray}, % the style that is used for the line-numbers
	rulecolor=\color{black},         % if not set, the frame-color may be changed on line-breaks within not-black text (e.g. comments (green here))
	showspaces=false,                % show spaces everywhere adding particular underscores; it overrides 'showstringspaces'
	showstringspaces=false,          % underline spaces within strings only
	showtabs=false,                  % show tabs within strings adding particular underscores
	stepnumber=2,                    % the step between two line-numbers. If it's 1, each line will be numbered
	stringstyle=\color{mymauve},     % string literal style
	tabsize=2,	                   % sets default tabsize to 2 spaces
%	title=\lstname,                   % show the filename of files included with \lstinputlisting; also try caption instead of title
	xleftmargin=0.25cm
}



\title{UChicago Markov Chains, Martingales, and Brownian Motion Analysis Notes: 23500}
\author{Notes by Agustín Esteva, Lectures by Stephen Yearwood, Books by }
\date{Academic Year 2024-2025}

\begin{document}

\maketitle
\tableofcontents

\vspace{.25in}


\newpage
\section{Lectures}

\subsection{Tuesday, Mar 25: $\bbC$ as a field}
Define the Complex numbers:
\[\bbC := \{(x,y) \; | \; x,y \in \bbR\} = \bbR^2\] as a field: 
\[+: \bbC \to \bbC; \quad (a,b) + (c,d) = (a + c, b + d)\]
\[\times: \bbC \to \bbC; \quad (a,b)\times (c,d) = (ac - bd, ad + bc)\] such that for any $z = (a,b) \in \bbC,$ then 
\[z = a + ib,\] where, 
\[a = (a,0) = \Re{z}, \qquad i = (0,1), \qquad b = (0,b) = \Im{z}\]

\begin{rem}
    Why is $\bbC$ not an ordered field? Suppose it is, then there exists some $\mathcal{P}\subseteq \bbC$ of positive elements that is closed under addition and multiplication and that satisfies the trichotomy such that for any $z\in \cal P$, exactly one of the following holds: $z = 0, z\in \mathcal{P}, -z\in \cal P.$ 
    \begin{lem}
        If $z \neq 0,$ then $z^2 \in \cal P.$
    \end{lem}
    \begin{proof}
        If $z\in \cal P,$ then since $\cal P$ is closed under multiplication, $z \times z \in \cal P.$
        If $z\notin \cal P,$ then $-z \in \cal P,$ and thus $(-z)(-z) \in \cal P.$ But then $-z = (-1)(z),$ and so 
        \[(-z)(-z)= (-1)(-1)(z \times z) = z^2 \in \cal P.\]
    \end{proof}
    Consider that since $1$ is a square, then $1 \in \cal P.$ Moreover, since $-1$ is a square (in $\bbC!$), then $-1 \in \cal P.$  
\end{rem}

\begin{defn}
    Let $z\in \bbC$ such that $z = a + ib.$ We say that $\overline{z}$ is the \textbf{complex conjugate} of $z$ if 
    \[ z = a - ib.\]
\end{defn}
That is, we reflect $z$ over the real axis by flipping the sign of the imaginary line. 
\begin{rem}
    \[z \cdot \overline{z} = (a + ib)(a-ib) = (a^2 + b^2, 0) = a^2 + b^2 = |z^2|,\] by the norm defined below in (1). Suppose that $z\neq 0,$ then 
    \[z \cdot \frac{\overline{z}}{|z|} = \frac{|z|^2}{|z|^2} = 1.\] We have found the inverse of any nonzero $z\in \bbC.$ 
\end{rem}
\begin{rem}
It is easy to show the following:
    \[\overline{zw} = \overline{z} \cdot \overline{w}, \qquad  \overline{z + w} = \overline{z} + \overline{w}, \qquad |zw|^2 = |z|^2|w|^2.\]
\end{rem}

\begin{prop}
    $\bbC$ is Banach under the norm, 
    \begin{align}
    \|z\| = \|(a,b)\| = \sqrt{a^2 +b^2}    
    \end{align}
    
\end{prop}
\begin{proof}
    We will first show that the norm satisfies the triangle inequality. It suffices to show that 
    \[|z + w| \leq|z| + |w|,\] and thus we will show that 
    \[|z + w|^2 \leq (|z| + |w|)^2.\] We have by the above remarks that 
    \[|z + w|^2 = (z + w)\overline{(z + w)} = (z + w)(\overline{z} + \overline{w}) = (z\overline{z} + \overline{z}w + \overline{w}z+ w \overline{w}) = (|z|^2 + |w|^2 + z\overline{w} + \overline{z \overline{w}}) = |z|^2 + |w|^2 + 2\Re{z\overline{w}}.\] Thus, we want to show that $\Re{z\overline{w}} \leq |z||w|,$ which comes from the fact that 
    \[\Re{z\overline{w}} \leq |z||\overline{w}| = |z||w|.\]
\end{proof}

\newpage
\subsection{Thursday, Mar 27: The Topology of $\bbC$}
\begin{thm}
    $\bbC$ is complete.
\end{thm}
This theorem is in the sense of Cauchy convergence, since the least upper bound property is meaningless in a non-ordered field, such as $\bbC.$

\begin{rem}
    We denote that disk of radius $r$ around $z_0$ as 
    \[D(z_0, r) = \{z \in \bbC \; | \;|z - z_0| < r\}\]
\end{rem}

\begin{defn}
    We say that $O \subset \bbC$ is \textbf{open} if and only if for all $z_0 \in O,$ there exists some $\epsilon>0$ such that $D(z_0, r) \subset O.$
\end{defn}
It is easy to show that disks are open with the triangle inequality.

\begin{defn}
    Let $O$ be open. We say that $O$ is \textbf{connected} if, whenever $O = O_1 \cup O_2,$ where $O_1, O_2$ are open, disjoint, then at least one of the $O_1,$ $O_2$ are empty. 
\end{defn}


\begin{rem}
    How do we make the real valued $f(x) = x$ continuously differentiable on $[0,1]?$ The endpoints present a problem! We say that $f$ is differentiable at $0$ or at $1$ if the one sided limit exists and agrees with the derivative near the endpoint.
\end{rem}

\begin{defn}
    Let $O$ be an open set. A \textbf{path} is a function $\gamma: [a,b] \to O$ such that $\gamma$ is piecewise continuously differentiable. That is, there exists some finite partition $a = t_0 < t_1 < \dots < t_n = b$ such that $\gamma$ is continuously differentiable on each closed interval $(t_{k-1}, t_k)$ and 
    \[\lim_{t \to t_{k-1}^+} \gamma'(t) = \gamma_+'(t_{k-1}) = \lim_{h\to 0^+} \frac{\gamma(t_{k-1} +h) - \gamma(t_{k-1})}{h}\] and similarly for $t_k.$
\end{defn}

\begin{thm}
    If $O$ is pathwise connected, then it is connected.
\end{thm}
\begin{proof}
    Suppose $O$ is disconnected, then $O = O_1 \sqcup O_2$ and take $[a,b] = \gamma^{-1}(O_1) \sqcup \gamma^{-1}(O_2)$ as a disconnection of the interval, a contradiction!
\end{proof}

\begin{defn}
    Let $O$ be open. We say that a \textbf{polygonal path} inside  of $O$ is a path made up of only horizontal and vertical lines. 
\end{defn}

\begin{thm}
    Suppose $O$ is connected, then $O$ is polygonally connected. 
\end{thm}
\begin{proof}
    Let $z_0 \in O.$ We will show that 
    \[A = \{z \;  | \; \exists \text{ polygonal path between $z_0$ and $z$} \}\] is the same as $O.$ We will first show that $O$ is open. Let $z\in A,$ since $O$ is open, there exists some $r>0$ such that $D(z, r) \subseteq O.$ We will show that for any $z' \in D(z,r),$ $z' \in A.$ It suffices to show that disks are polygonally connected. 

    Since disks are convex, there exists a straight line connecting $z, z'.$ We claim that the imaginary component of this line is in the disk, which is because of the Pythagorean identity (or because $\Im{r} \leq |r|.$) By convexity, the straight line (i.e, the real component) between the imaginary line and $z'$ is in the disk. Thus, we can polygonally connect $z$ and $z'.$

    Thus, $A$ is open. Let $B$ be the set of points in $O$ that cannot be polygonally connected. Evidently, $A \cap B = \emptyset.$ By dichotomy, $A \cup B = O.$ Evidently, $z_0 \in A.$ Clearly, $B$ is open, and so $B$ must be empty and $A = O.$
\end{proof}

\begin{exmp}
Let $t\in [0,1].$ \begin{enumerate}
    \item The straight path from $z_2$ to $z_1$:
    \[\gamma(t) = tz_1 + (1-t)z_2\]
    \item The straight path from $z_1$ to $z_2$ is 
    \[\gamma(t) = (1-t)z_1 + tz_2\]
    \item The circle centered at $z_0$ of radius $r$ counterclockwise. 
    \[\gamma(\theta) = z_0 + r\cos\theta  + ir\sin\theta = z_0 + e^{ir\theta}, \qquad \theta \in [0, 2\pi)\]
\end{enumerate}
\end{exmp}

\begin{defn}
    Suppose $O \subseteq \bbC$ be open, and let $f: O \to \bbC.$ We say that $f$ is \textbf{continuous} at $z_0 \in O$ if 
    \[\lim_{z \to z_0}f(z) = f(z_0), \quad z \in O.\] We say that $f$ is differentiable at $z_0 \in O$ if 
    \[\lim_{h\to 0}\frac{f(z_0 + h) - f(z_0)}{h} = f'(z_0), \quad h \in \bbC\]
\end{defn}
Continuity in $\bbC$ is identical to continuity in $\bbR^2,$ since they are identical as metric spaces. But dividing in $\bbR^2$ makes no sense, so the differentiability is completely different!
\begin{exmp}
    \begin{enumerate}
        \item Suppose $f(z) = z^2,$ then 
        \[\frac{(z + h)^2 - z^2}{h} = 2z + h \to 2z = f'(z)\]
        \item Suppose $f(z) = \overline{z}.$ Then at the origin, 
        \[\lim_{h\to 0}\frac{\overline{(0 + h)} - \overline{0}}{h}= \lim_{h\to 0}\frac{\overline{h}}{h},\] but the looking at a purely real component, the limit approached $1.$ Looking at a purely imaginary components, the limit approaches $-1.$ Oops!
    \end{enumerate}
\end{exmp}


\end{document}