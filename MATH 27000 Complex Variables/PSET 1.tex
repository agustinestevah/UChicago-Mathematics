\documentclass[11pt]{article}

% NOTE: Add in the relevant information to the commands below; or, if you'll be using the same information frequently, add these commands at the top of paolo-pset.tex file. 
\newcommand{\name}{Agustín Esteva}
\newcommand{\email}{aesteva@uchicago.edu}
\newcommand{\classnum}{270}
\newcommand{\subject}{Complex Variables}
\newcommand{\instructors}{Robert Fefferman}
\newcommand{\assignment}{Problem Set 1}
\newcommand{\semester}{Spring 2025}
\newcommand{\duedate}{4-10-2025}
\newcommand{\bA}{\mathbf{A}}
\newcommand{\bB}{\mathbf{B}}
\newcommand{\bC}{\mathbf{C}}
\newcommand{\bD}{\mathbf{D}}
\newcommand{\bE}{\mathbf{E}}
\newcommand{\bF}{\mathbf{F}}
\newcommand{\bG}{\mathbf{G}}
\newcommand{\bH}{\mathbf{H}}
\newcommand{\bI}{\mathbf{I}}
\newcommand{\bJ}{\mathbf{J}}
\newcommand{\bK}{\mathbf{K}}
\newcommand{\bL}{\mathbf{L}}
\newcommand{\bM}{\mathbf{M}}
\newcommand{\bN}{\mathbf{N}}
\newcommand{\bO}{\mathbf{O}}
\newcommand{\bP}{\mathbf{P}}
\newcommand{\bQ}{\mathbf{Q}}
\newcommand{\bR}{\mathbf{R}}
\newcommand{\bS}{\mathbf{S}}
\newcommand{\0bT}{\mathbf{T}}
\newcommand{\bU}{\mathbf{U}}
\newcommand{\bV}{\mathbf{V}}
\newcommand{\bW}{\mathbf{W}}
\newcommand{\bX}{\mathbf{X}}
\newcommand{\bY}{\mathbf{Y}}
\newcommand{\bZ}{\mathbf{Z}}
\newcommand{\Vol}{\text{Vol}}

%% blackboard bold math capitals
\newcommand{\bbA}{\mathbb{A}}
\newcommand{\bbB}{\mathbb{B}}
\newcommand{\bbC}{\mathbb{C}}
\newcommand{\bbD}{\mathbb{D}}
\newcommand{\bbE}{\mathbb{E}}
\newcommand{\bbF}{\mathbb{F}}
\newcommand{\bbG}{\mathbb{G}}
\newcommand{\bbH}{\mathbb{H}}
\newcommand{\bbI}{\mathbb{I}}
\newcommand{\bbJ}{\mathbb{J}}
\newcommand{\bbK}{\mathbb{K}}
\newcommand{\bbL}{\mathbb{L}}
\newcommand{\bbM}{\mathbb{M}}
\newcommand{\bbN}{\mathbb{N}}
\newcommand{\bbO}{\mathbb{O}}
\newcommand{\bbP}{\mathbb{P}}
\newcommand{\bbQ}{\mathbb{Q}}
\newcommand{\bbR}{\mathbb{R}}
\newcommand{\bbS}{\mathbb{S}}
\newcommand{\bbT}{\mathbb{T}}
\newcommand{\bbU}{\mathbb{U}}
\newcommand{\bbV}{\mathbb{V}}
\newcommand{\bbW}{\mathbb{W}}
\newcommand{\bbX}{\mathbb{X}}
\newcommand{\bbY}{\mathbb{Y}}
\newcommand{\bbZ}{\mathbb{Z}}

%% script math capitals
\newcommand{\sA}{\mathscr{A}}
\newcommand{\sB}{\mathscr{B}}
\newcommand{\sC}{\mathscr{C}}
\newcommand{\sD}{\mathscr{D}}
\newcommand{\sE}{\mathscr{E}}
\newcommand{\sF}{\mathscr{F}}
\newcommand{\sG}{\mathscr{G}}
\newcommand{\sH}{\mathscr{H}}
\newcommand{\sI}{\mathscr{I}}
\newcommand{\sJ}{\mathscr{J}}
\newcommand{\sK}{\mathscr{K}}
\newcommand{\sL}{\mathscr{L}}
\newcommand{\sM}{\mathscr{M}}
\newcommand{\sN}{\mathscr{N}}
\newcommand{\sO}{\mathscr{O}}
\newcommand{\sP}{\mathscr{P}}
\newcommand{\sQ}{\mathscr{Q}}
\newcommand{\sR}{\mathscr{R}}
\newcommand{\sS}{\mathscr{S}}
\newcommand{\sT}{\mathscr{T}}
\newcommand{\sU}{\mathscr{U}}
\newcommand{\sV}{\mathscr{V}}
\newcommand{\sW}{\mathscr{W}}
\newcommand{\sX}{\mathscr{X}}
\newcommand{\sY}{\mathscr{Y}}
\newcommand{\sZ}{\mathscr{Z}}


\renewcommand{\emptyset}{\O}

\newcommand{\abs}[1]{\lvert #1 \rvert}
\newcommand{\norm}[1]{\lVert #1 \rVert}
\newcommand{\sm}{\setminus}


\newcommand{\sarr}{\rightarrow}
\newcommand{\arr}{\longrightarrow}

% NOTE: Defining collaborators is optional; to not list collaborators, comment out the line below.
%\newcommand{\collaborators}{Alyssa P. Hacker (\texttt{aphacker}), Ben Bitdiddle (\texttt{bitdiddle})}

\input{paolo-pset.tex}

% NOTE: To compile a version of this pset without problems, solutions, or reflections, uncomment the relevant line below.

%\excludeversion{problem}
%\excludeversion{solution}
%\excludeversion{reflection}

\begin{document}	
	
	% Use the \psetheader command at the beginning of a pset. 
	\psetheader

\section*{Problem 1}
\begin{problem}
    A function $f: \bbC \to \bbC$ is said to satisfy the mean value property if and only for every $z_0 \in \bbC$ and $r>0,$ we have that 
    \[f(z_0) = \frac{1}{2\pi}\int_0^{2\pi}f(z_0 + r\cos(\theta) + i r \sin(\theta))d\theta.\]

    Show that if $f(z) = z^2,$ then $f$ satisfies the mean value property.
\end{problem}
\begin{solution}
    Note that since $f(z) = z^2,$ we have that for any $z_0 \in \bbC,$
    \begin{align*}
    f(z_0 + r\cos(\theta) + i r\sin(\theta)) &= f(z_0 + re^{i \theta})\\ &= z_0^2 + 2z_0e^{i\theta}+ r^2e^{2i\theta}\\
    \end{align*}
    \begin{lemma}
        For any $n \geq 0,$ we have that \[e^{i2\pi n} = 1\]
    \end{lemma}
    \begin{proof}
        \[e^{i2\pi n} = \cos(2\pi n) + i \sin(2\pi n) = 1\]
    \end{proof}
    Evaluating the integral:
    \begin{align*}
        \int_0^{2\pi}(z_0^2 + 2z_0e^{i\theta}+ r^2e^{2i\theta}) d\theta &= z_0^2\int_0^{2\pi}d\theta + 2z_0\int_0^{2\pi}e^{i\theta}d\theta+ r^2\int_0^{2\pi}e^{2i\theta} d\theta\\
        &= 2\pi z_0^2 + 2z_0 \frac{1}{i}[e^{i2\pi} - 1] + r^2 \frac{1}{2i}[e^{4i\pi} - 1]\\
        &= 2\pi z_0^2 + 2z_0 \frac{1}{i}[1 - 1] + r^2 \frac{1}{2i}[1 - 1]\\
        &= 2\pi z_0^2
    \end{align*}
    Thus, dividing by $2\pi$ yields the result.
\end{solution}

\newpage
\section*{Problem 2}
\begin{problem}
    Let $f(z) = z^3,$ and let 
    \[u(x,y) = \Re{f(x + iy)}, \quad v(x,y) = \Im{f(x + iy)}.\] Prove that, for all $x,y \in \bbR,$ 
    \[\triangle u(x,y) = \triangle v(x,y) = 0,\] where $\triangle$ is the Laplacian operator
    \[\nabla = \frac{\partial^2}{\partial x^2} + \frac{\partial^2}{\partial y^2}\]
\end{problem}

\begin{solution}
    Let $x,y \in \bbR,$ then
    \begin{align*}
    f(x + iy) &= (x + iy)^3\\ &= (x^2 + 2ixy  -y^2)(x + iy)\\ &= x^3  + 2ix^2 y - xy^2 +ix^2y - 2xy^2 - iy^3\\ &= x^3 - 3xy^2  + i(3x^2 y - y^3)    
    \end{align*}
    We then have that 
    \[u(x,y) = x^3 - 3xy^2, \qquad v(x,y) = 3x^2y - y^3.\]
    Computing the Laplacian:
    \begin{align*}
        \triangle u(x,y) &= \frac{\partial^2 u}{\partial x^2} + \frac{\partial^2 u}{\partial y^2}\\
        &= 6x + (-6x)\\
        &= 0
    \end{align*}
    and
    \begin{align*}
        \triangle v(x,y) &= \frac{\partial^2 v}{\partial x^2} + \frac{\partial^2 v}{\partial y^2}\\
        &= 6y + (-6y)\\
        &= 0
    \end{align*}
\end{solution}

\newpage
\section*{Problem 3}
\begin{problem}
    Suppose $\Omega \subset \bbC$ is an open connected set (a region) and $u: \Omega \to \bbR$ such that 
    \[\frac{\partial u}{\partial x} =\frac{\partial u}{\partial y} = 0.\] Prove that $u$ must be constant on $\Omega.$
\end{problem}
\begin{solution}
We assume that $\Omega \neq \emptyset$ as otherwise the statement is vacuously true. Let $z_0 \in \Omega.$ 

Since $\Omega$ is connected, then it is polygonally connected. Let $z \in \Omega.$ There exists some polygonal path $\gamma$ that connects $z$ and $z_0.$ Since $\frac{\partial u}{\partial x}: \Re{\Omega} \to \bbR$ is constantly zero, then it is continuous. Similarly, $\frac{\partial u}{\partial y}: \Im{\Omega} \to \bbR$ is continuous. Thus, \[D[u(z)] = \begin{bmatrix}
    \frac{\partial u(z)}{\partial x} &  \frac{\partial u(z)}{\partial y}
\end{bmatrix}: \Omega \to \bbR\] (the total derivative) is continuous on $\Omega.$ Note that $Du$ has a primitive, $u$ on $\Omega,$ since $u' = Du.$ Thus, we have that
\[\int_\gamma D[u(z)] dz = u(z_0) - u(z). \] But we have that $D[u(z)] = 0$ along any path in $\Omega$ since all the components are $0,$ and thus the path integral along $\gamma$ must be $0$ and so
\[u(z_0) = u(z).\] Because $z$ was arbitrary, then $u$ must be constant on $\Omega.$
\end{solution}

\newpage
\section*{Problem 4}
\begin{problem}
    Suppose that $u: \overline{D_r(z_0)}\subseteq \bbC \to \bbR$ is a continuous function which attains its maximum at $z_0.$ Suppose further that $u$ satisfies 
    \[u(z_0) = \frac{1}{2\pi}\int_0^{2\pi}u(z_0 + r\cos \theta + ir\sin\theta)d\theta.\] Prove that if $z\in \overline{D_r(z_0)}$ such that $|z - z_0| = r,$ then $u(z) = u(z_0).$
\end{problem}
\begin{solution}
\begin{lemma}
Let $S_r(z_0)$ denote the sphere of radius $r$ around $z_0.$ We claim that
    \[S_r(z_0) = \{z_0 + re^{i\theta} \mid \theta \in [0, 2\pi)\}\]
\end{lemma}
\begin{proof}
    Let $z \in S_r(z_0).$ Without much loss in generality from a translation, we can assume that $z \in S_r(0).$ Then $z = x + iy.$ Letting $\theta = \tan^{-1}(\frac{y}{x}),$ we see that $x = r\cos\theta$ and $y = r\sin\theta.$ Thus, $x + iy = re^{i\theta}.$  

    Let $z \in \{z_0 + re^{i\theta} \mid \theta \in [0, 2\pi)\},$ then there exists some $\theta_0 \in [0,2\pi)$ such that $z = z_0  + re^{i\theta}.$ Thus, 
    \[|z - z_0| = |re^{i\theta}| = r|e^{i\theta}| = r,\] and so $z_0 \in S_r(z_0).$
\end{proof}
Suppose not. That is, there is some $z'\in S_r(z_0),$ such that $u(z') < u(z_0).$ Let $\epsilon= {u(z_0) - u(z')}.$ 

Then by continuity of $u,$ there exists some $\delta>0$ such that if $z \in \overline{D_r(z_0)}$ with $|z - z'|< \delta,$ then $0\leq |u(z) - u(z')| <\frac{\epsilon}{2}.$ Take $X = S_r(z_0) \cap (z'- \delta, z' + \delta).$ That is, $X$ are the points on the circle which are less than $u(z_0)$ by at least $\frac{\epsilon}{2}.$ That is
\begin{align}
    u(z) < u(z_0) - \frac{\epsilon}{2} \quad \forall z\in X.
\end{align}
Then using (1) in the fourth line and Lemma 2 in the third line, we see that if $|X|$ denotes the area of $X,$ then 
\begin{align*}
    u(z_0) &= \frac{1}{{2\pi}}\int_0^{2\pi} u(z_0 + re^{i\theta})d\theta \\
    &= \frac{1}{{2\pi}}\int_{[0, 2\pi]\sm X} u(z_0 + re^{i\theta})d\theta  + \frac{1}{2\pi}\int_{X} u(z_0 + re^{i\theta}) d\theta\\
    &= \frac{1}{{2\pi}}\int_{[0, 2\pi]\sm X} u(z_0 + re^{i\theta})d\theta  + \frac{1}{2\pi}\int_{X} u(z) d\theta\\
    &< \frac{1}{{2\pi}}\int_{[0, 2\pi]\sm X} u(z_0 + re^{i\theta})d\theta  + \frac{1}{2\pi}\int_{X} (u(z_0) - \frac{\epsilon}{2}) d\theta\\
    &= \frac{1}{2\pi}\int_0^{2\pi}u(z_0 + re^{i\theta})d\theta - \frac{1}{2\pi}\int_X\frac{\epsilon}{2}d\theta\\
    &= u(z_0) - \frac{1}{2\pi}\epsilon|X|
\end{align*}
But then $u(z_0) < u(z_0),$ a contradiction! Thus, we must have that $u(z) \geq u(z_0)$ for all $z \in S_r(z_0).$ However, since $z_0$ is a maximum on the disk, we have that $u(z) = u(z_0).$

\end{solution}

\newpage
\section*{Problem 5}
\begin{problem}
    Suppose $u: \overline{\Omega} \to \bbR$ is a continuous function on the closure of a bounded region $\Omega \subset \bbC.$ Suppose that $u$ has the mean value property in $\Omega.$ That is, whenever $\overline{D_r(z_0)}\subseteq \Omega,$
    \[u(z_0) = \frac{1}{2\pi}\int_0^{2\pi} u(z_0 + re^{i\theta})d\theta.\] Prove that if $u$ takes a maximum value inside of $\Omega,$ then $u$ must be constant.
\end{problem}
\begin{solution}
Since $\overline{\Omega}$ is bounded and closed, then it is compact. Since $u$ is a continuous function over $\overline{\Omega},$ then it achieves its maximum at some $z_0 \in \overline{\Omega}.$ Suppose that $z_0 \in \Omega.$ Define 
    \[A:= \{z \in \Omega \mid u(z) = u(z_0)\}.\] We aim to show that $A$ is clopen,
    since $\Omega$ is a region, then it is an open connected subset, and so the only clopen subsets it contains are itself and the emptyset. But since $z_0 \in A,$ then $A \neq \emptyset,$ and so if we show that $A$ is clopen, then it must necessarily be $\Omega.$

    Let $z \in A$ and $\epsilon>0.$ By continuity of $u,$ there exists some $\delta>0$ such that if $|z - z'| < \delta,$ then $|u(z) - u(z')| < \epsilon.$ Since $\Omega$ is open, then there exists some $r'>0$ such that if $|z - z'| < r',$ then $z' \in \Omega.$ Take $R = \min\{\delta, r'\}.$ Let $z' \in D_R(z).$ There is some $r>0$ with $r \leq R$ such that $z' \in S_r(z').$ By the previous problem, since $u|_{D_r(z)}: D_r(z)\subseteq \overline{\Omega} \to \bbR$ attains its maximum at $z$ and $|z' - z| = r$ and 
    \[u(z) = \frac{1}{2\pi}\int_0^{2\pi}u(z_0  + re^{i\theta})d\theta,\] then $u(z) = u(z').$ Therefore, $z' \in A$ and so we have found an $R>0$ such that if $z' \in D_R(z),$ then $z' \in A.$ Then $A$ is open.

    Let $(z_n) \in A$ with $z_n \to z.$ We claim that $z\in A.$ Since $u$ is continuous, then $u(z_n) \to u(z),$ but since $u(z_n) = u(z_0)$ for all $n$ since each $z_n \in A,$ then $u(z) = u(z_0),$ and thus $z \in A.$ Then we have that $A$ is closed.

    Thus, since $A$ is both closed and open and since $A\neq \emptyset,$ then $A  = \Omega.$ Thus, for all $z\in \Omega,$ $u(z)= u(z_0).$ It remains to see that $u$ is constant on $\partial \Omega.$ 

    Let $z \in \partial \Omega,$ then there exists some $(z_n) \in \Omega$ such that $z_n \to z.$ By continuity of $u,$ we have that $u(z_n) \to u(z),$ but since $u$ is constant in $\Omega,$ we get that $u(z_n) = u(z_0),$ and thus $u(z) = u(z_0),$ and so $u$ is indeed constant on the boundary as well.
\end{solution}

\end{document}