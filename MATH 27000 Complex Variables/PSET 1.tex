\documentclass[11pt]{article}

% NOTE: Add in the relevant information to the commands below; or, if you'll be using the same information frequently, add these commands at the top of paolo-pset.tex file. 
\newcommand{\name}{Agustín Esteva}
\newcommand{\email}{aesteva@uchicago.edu}
\newcommand{\classnum}{208}
\newcommand{\subject}{Accelerated Analysis in $\bbR^n$ III}
\newcommand{\instructors}{Don Stull}
\newcommand{\assignment}{Problem Set 1}
\newcommand{\semester}{Spring 2025}
\newcommand{\duedate}{\today}
\newcommand{\bA}{\mathbf{A}}
\newcommand{\bB}{\mathbf{B}}
\newcommand{\bC}{\mathbf{C}}
\newcommand{\bD}{\mathbf{D}}
\newcommand{\bE}{\mathbf{E}}
\newcommand{\bF}{\mathbf{F}}
\newcommand{\bG}{\mathbf{G}}
\newcommand{\bH}{\mathbf{H}}
\newcommand{\bI}{\mathbf{I}}
\newcommand{\bJ}{\mathbf{J}}
\newcommand{\bK}{\mathbf{K}}
\newcommand{\bL}{\mathbf{L}}
\newcommand{\bM}{\mathbf{M}}
\newcommand{\bN}{\mathbf{N}}
\newcommand{\bO}{\mathbf{O}}
\newcommand{\bP}{\mathbf{P}}
\newcommand{\bQ}{\mathbf{Q}}
\newcommand{\bR}{\mathbf{R}}
\newcommand{\bS}{\mathbf{S}}
\newcommand{\bT}{\mathbf{T}}
\newcommand{\bU}{\mathbf{U}}
\newcommand{\bV}{\mathbf{V}}
\newcommand{\bW}{\mathbf{W}}
\newcommand{\bX}{\mathbf{X}}
\newcommand{\bY}{\mathbf{Y}}
\newcommand{\bZ}{\mathbf{Z}}
\newcommand{\Vol}{\text{Vol}}

%% blackboard bold math capitals
\newcommand{\bbA}{\mathbb{A}}
\newcommand{\bbB}{\mathbb{B}}
\newcommand{\bbC}{\mathbb{C}}
\newcommand{\bbD}{\mathbb{D}}
\newcommand{\bbE}{\mathbb{E}}
\newcommand{\bbF}{\mathbb{F}}
\newcommand{\bbG}{\mathbb{G}}
\newcommand{\bbH}{\mathbb{H}}
\newcommand{\bbI}{\mathbb{I}}
\newcommand{\bbJ}{\mathbb{J}}
\newcommand{\bbK}{\mathbb{K}}
\newcommand{\bbL}{\mathbb{L}}
\newcommand{\bbM}{\mathbb{M}}
\newcommand{\bbN}{\mathbb{N}}
\newcommand{\bbO}{\mathbb{O}}
\newcommand{\bbP}{\mathbb{P}}
\newcommand{\bbQ}{\mathbb{Q}}
\newcommand{\bbR}{\mathbb{R}}
\newcommand{\bbS}{\mathbb{S}}
\newcommand{\bbT}{\mathbb{T}}
\newcommand{\bbU}{\mathbb{U}}
\newcommand{\bbV}{\mathbb{V}}
\newcommand{\bbW}{\mathbb{W}}
\newcommand{\bbX}{\mathbb{X}}
\newcommand{\bbY}{\mathbb{Y}}
\newcommand{\bbZ}{\mathbb{Z}}

%% script math capitals
\newcommand{\sA}{\mathscr{A}}
\newcommand{\sB}{\mathscr{B}}
\newcommand{\sC}{\mathscr{C}}
\newcommand{\sD}{\mathscr{D}}
\newcommand{\sE}{\mathscr{E}}
\newcommand{\sF}{\mathscr{F}}
\newcommand{\sG}{\mathscr{G}}
\newcommand{\sH}{\mathscr{H}}
\newcommand{\sI}{\mathscr{I}}
\newcommand{\sJ}{\mathscr{J}}
\newcommand{\sK}{\mathscr{K}}
\newcommand{\sL}{\mathscr{L}}
\newcommand{\sM}{\mathscr{M}}
\newcommand{\sN}{\mathscr{N}}
\newcommand{\sO}{\mathscr{O}}
\newcommand{\sP}{\mathscr{P}}
\newcommand{\sQ}{\mathscr{Q}}
\newcommand{\sR}{\mathscr{R}}
\newcommand{\sS}{\mathscr{S}}
\newcommand{\sT}{\mathscr{T}}
\newcommand{\sU}{\mathscr{U}}
\newcommand{\sV}{\mathscr{V}}
\newcommand{\sW}{\mathscr{W}}
\newcommand{\sX}{\mathscr{X}}
\newcommand{\sY}{\mathscr{Y}}
\newcommand{\sZ}{\mathscr{Z}}


\renewcommand{\emptyset}{\O}

\newcommand{\abs}[1]{\lvert #1 \rvert}
\newcommand{\norm}[1]{\lVert #1 \rVert}
\newcommand{\sm}{\setminus}


\newcommand{\sarr}{\rightarrow}
\newcommand{\arr}{\longrightarrow}

% NOTE: Defining collaborators is optional; to not list collaborators, comment out the line below.
%\newcommand{\collaborators}{Alyssa P. Hacker (\texttt{aphacker}), Ben Bitdiddle (\texttt{bitdiddle})}

% Copyright 2021 Paolo Adajar (padajar.com, paoloadajar@mit.edu)
% 
% Permission is hereby granted, free of charge, to any person obtaining a copy of this software and associated documentation files (the "Software"), to deal in the Software without restriction, including without limitation the rights to use, copy, modify, merge, publish, distribute, sublicense, and/or sell copies of the Software, and to permit persons to whom the Software is furnished to do so, subject to the following conditions:
%
% The above copyright notice and this permission notice shall be included in all copies or substantial portions of the Software.
% 
% THE SOFTWARE IS PROVIDED "AS IS", WITHOUT WARRANTY OF ANY KIND, EXPRESS OR IMPLIED, INCLUDING BUT NOT LIMITED TO THE WARRANTIES OF MERCHANTABILITY, FITNESS FOR A PARTICULAR PURPOSE AND NONINFRINGEMENT. IN NO EVENT SHALL THE AUTHORS OR COPYRIGHT HOLDERS BE LIABLE FOR ANY CLAIM, DAMAGES OR OTHER LIABILITY, WHETHER IN AN ACTION OF CONTRACT, TORT OR OTHERWISE, ARISING FROM, OUT OF OR IN CONNECTION WITH THE SOFTWARE OR THE USE OR OTHER DEALINGS IN THE SOFTWARE.

\usepackage{fullpage}
\usepackage{enumitem}
\usepackage{amsfonts, amssymb, amsmath,amsthm}
\usepackage{mathtools}
\usepackage[pdftex, pdfauthor={\name}, pdftitle={\classnum~\assignment}]{hyperref}
\usepackage[dvipsnames]{xcolor}
\usepackage{bbm}
\usepackage{graphicx}
\usepackage{mathrsfs}
\usepackage{pdfpages}
\usepackage{tabularx}
\usepackage{pdflscape}
\usepackage{makecell}
\usepackage{booktabs}
\usepackage{natbib}
\usepackage{caption}
\usepackage{subcaption}
\usepackage{physics}
\usepackage[many]{tcolorbox}
\usepackage{version}
\usepackage{ifthen}
\usepackage{cancel}
\usepackage{listings}
\usepackage{courier}

\usepackage{tikz}
\usepackage{istgame}

\hypersetup{
	colorlinks=true,
	linkcolor=blue,
	filecolor=magenta,
	urlcolor=blue,
}

\setlength{\parindent}{0mm}
\setlength{\parskip}{2mm}

\setlist[enumerate]{label=({\alph*})}
\setlist[enumerate, 2]{label=({\roman*})}

\allowdisplaybreaks[1]

\newcommand{\psetheader}{
	\ifthenelse{\isundefined{\collaborators}}{
		\begin{center}
			{\setlength{\parindent}{0cm} \setlength{\parskip}{0mm}
				
				{\textbf{\classnum~\semester:~\assignment} \hfill \name}
				
				\subject \hfill \href{mailto:\email}{\tt \email}
				
				Instructor(s):~\instructors \hfill Due Date:~\duedate	
				
				\hrulefill}
		\end{center}
	}{
		\begin{center}
			{\setlength{\parindent}{0cm} \setlength{\parskip}{0mm}
				
				{\textbf{\classnum~\semester:~\assignment} \hfill \name\footnote{Collaborator(s): \collaborators}}
				
				\subject \hfill \href{mailto:\email}{\tt \email}
				
				Instructor(s):~\instructors \hfill Due Date:~\duedate	
				
				\hrulefill}
		\end{center}
	}
}

\renewcommand{\thepage}{\classnum~\assignment \hfill \arabic{page}}

\makeatletter
\def\points{\@ifnextchar[{\@with}{\@without}}
\def\@with[#1]#2{{\ifthenelse{\equal{#2}{1}}{{[1 point, #1]}}{{[#2 points, #1]}}}}
\def\@without#1{\ifthenelse{\equal{#1}{1}}{{[1 point]}}{{[#1 points]}}}
\makeatother

\newtheoremstyle{theorem-custom}%
{}{}%
{}{}%
{\itshape}{.}%
{ }%
{\thmname{#1}\thmnumber{ #2}\thmnote{ (#3)}}

\theoremstyle{theorem-custom}

\newtheorem{theorem}{Theorem}
\newtheorem{lemma}[theorem]{Lemma}
\newtheorem{example}[theorem]{Example}

\newenvironment{problem}[1]{\color{black} #1}{}

\newenvironment{solution}{%
	\leavevmode\begin{tcolorbox}[breakable, colback=green!5!white,colframe=green!75!black, enhanced jigsaw] \proof[\scshape Solution:] \setlength{\parskip}{2mm}%
	}{\renewcommand{\qedsymbol}{$\blacksquare$} \endproof \end{tcolorbox}}

\newenvironment{reflection}{\begin{tcolorbox}[breakable, colback=black!8!white,colframe=black!60!white, enhanced jigsaw, parbox = false]\textsc{Reflections:}}{\end{tcolorbox}}

\newcommand{\qedh}{\renewcommand{\qedsymbol}{$\blacksquare$}\qedhere}

\definecolor{mygreen}{rgb}{0,0.6,0}
\definecolor{mygray}{rgb}{0.5,0.5,0.5}
\definecolor{mymauve}{rgb}{0.58,0,0.82}

% from https://github.com/satejsoman/stata-lstlisting
% language definition
\lstdefinelanguage{Stata}{
	% System commands
	morekeywords=[1]{regress, reg, summarize, sum, display, di, generate, gen, bysort, use, import, delimited, predict, quietly, probit, margins, test},
	% Reserved words
	morekeywords=[2]{aggregate, array, boolean, break, byte, case, catch, class, colvector, complex, const, continue, default, delegate, delete, do, double, else, eltypedef, end, enum, explicit, export, external, float, for, friend, function, global, goto, if, inline, int, local, long, mata, matrix, namespace, new, numeric, NULL, operator, orgtypedef, pointer, polymorphic, pragma, private, protected, public, quad, real, return, rowvector, scalar, short, signed, static, strL, string, struct, super, switch, template, this, throw, transmorphic, try, typedef, typename, union, unsigned, using, vector, version, virtual, void, volatile, while,},
	% Keywords
	morekeywords=[3]{forvalues, foreach, set},
	% Date and time functions
	morekeywords=[4]{bofd, Cdhms, Chms, Clock, clock, Cmdyhms, Cofc, cofC, Cofd, cofd, daily, date, day, dhms, dofb, dofC, dofc, dofh, dofm, dofq, dofw, dofy, dow, doy, halfyear, halfyearly, hh, hhC, hms, hofd, hours, mdy, mdyhms, minutes, mm, mmC, mofd, month, monthly, msofhours, msofminutes, msofseconds, qofd, quarter, quarterly, seconds, ss, ssC, tC, tc, td, th, tm, tq, tw, week, weekly, wofd, year, yearly, yh, ym, yofd, yq, yw,},
	% Mathematical functions
	morekeywords=[5]{abs, ceil, cloglog, comb, digamma, exp, expm1, floor, int, invcloglog, invlogit, ln, ln1m, ln, ln1p, ln, lnfactorial, lngamma, log, log10, log1m, log1p, logit, max, min, mod, reldif, round, sign, sqrt, sum, trigamma, trunc,},
	% Matrix functions
	morekeywords=[6]{cholesky, coleqnumb, colnfreeparms, colnumb, colsof, corr, det, diag, diag0cnt, el, get, hadamard, I, inv, invsym, issymmetric, J, matmissing, matuniform, mreldif, nullmat, roweqnumb, rownfreeparms, rownumb, rowsof, sweep, trace, vec, vecdiag, },
	% Programming functions
	morekeywords=[7]{autocode, byteorder, c, _caller, chop, abs, clip, cond, e, fileexists, fileread, filereaderror, filewrite, float, fmtwidth, has_eprop, inlist, inrange, irecode, matrix, maxbyte, maxdouble, maxfloat, maxint, maxlong, mi, minbyte, mindouble, minfloat, minint, minlong, missing, r, recode, replay, return, s, scalar, smallestdouble,},
	% Random-number functions
	morekeywords=[8]{rbeta, rbinomial, rcauchy, rchi2, rexponential, rgamma, rhypergeometric, rigaussian, rlaplace, rlogistic, rnbinomial, rnormal, rpoisson, rt, runiform, runiformint, rweibull, rweibullph,},
	% Selecting time-span functions
	morekeywords=[9]{tin, twithin,},
	% Statistical functions
	morekeywords=[10]{betaden, binomial, binomialp, binomialtail, binormal, cauchy, cauchyden, cauchytail, chi2, chi2den, chi2tail, dgammapda, dgammapdada, dgammapdadx, dgammapdx, dgammapdxdx, dunnettprob, exponential, exponentialden, exponentialtail, F, Fden, Ftail, gammaden, gammap, gammaptail, hypergeometric, hypergeometricp, ibeta, ibetatail, igaussian, igaussianden, igaussiantail, invbinomial, invbinomialtail, invcauchy, invcauchytail, invchi2, invchi2tail, invdunnettprob, invexponential, invexponentialtail, invF, invFtail, invgammap, invgammaptail, invibeta, invibetatail, invigaussian, invigaussiantail, invlaplace, invlaplacetail, invlogistic, invlogistictail, invnbinomial, invnbinomialtail, invnchi2, invnF, invnFtail, invnibeta, invnormal, invnt, invnttail, invpoisson, invpoissontail, invt, invttail, invtukeyprob, invweibull, invweibullph, invweibullphtail, invweibulltail, laplace, laplaceden, laplacetail, lncauchyden, lnigammaden, lnigaussianden, lniwishartden, lnlaplaceden, lnmvnormalden, lnnormal, lnnormalden, lnwishartden, logistic, logisticden, logistictail, nbetaden, nbinomial, nbinomialp, nbinomialtail, nchi2, nchi2den, nchi2tail, nF, nFden, nFtail, nibeta, normal, normalden, npnchi2, npnF, npnt, nt, ntden, nttail, poisson, poissonp, poissontail, t, tden, ttail, tukeyprob, weibull, weibullden, weibullph, weibullphden, weibullphtail, weibulltail,},
	% String functions 
	morekeywords=[11]{abbrev, char, collatorlocale, collatorversion, indexnot, plural, plural, real, regexm, regexr, regexs, soundex, soundex_nara, strcat, strdup, string, strofreal, string, strofreal, stritrim, strlen, strlower, strltrim, strmatch, strofreal, strofreal, strpos, strproper, strreverse, strrpos, strrtrim, strtoname, strtrim, strupper, subinstr, subinword, substr, tobytes, uchar, udstrlen, udsubstr, uisdigit, uisletter, ustrcompare, ustrcompareex, ustrfix, ustrfrom, ustrinvalidcnt, ustrleft, ustrlen, ustrlower, ustrltrim, ustrnormalize, ustrpos, ustrregexm, ustrregexra, ustrregexrf, ustrregexs, ustrreverse, ustrright, ustrrpos, ustrrtrim, ustrsortkey, ustrsortkeyex, ustrtitle, ustrto, ustrtohex, ustrtoname, ustrtrim, ustrunescape, ustrupper, ustrword, ustrwordcount, usubinstr, usubstr, word, wordbreaklocale, worcount,},
	% Trig functions
	morekeywords=[12]{acos, acosh, asin, asinh, atan, atanh, cos, cosh, sin, sinh, tan, tanh,},
	morecomment=[l]{//},
	% morecomment=[l]{*},  // `*` maybe used as multiply operator. So use `//` as line comment.
	morecomment=[s]{/*}{*/},
	% The following is used by macros, like `lags'.
	morestring=[b]{`}{'},
	% morestring=[d]{'},
	morestring=[b]",
	morestring=[d]",
	% morestring=[d]{\\`},
	% morestring=[b]{'},
	sensitive=true,
}

\lstset{ 
	backgroundcolor=\color{white},   % choose the background color; you must add \usepackage{color} or \usepackage{xcolor}; should come as last argument
	basicstyle=\footnotesize\ttfamily,        % the size of the fonts that are used for the code
	breakatwhitespace=false,         % sets if automatic breaks should only happen at whitespace
	breaklines=true,                 % sets automatic line breaking
	captionpos=b,                    % sets the caption-position to bottom
	commentstyle=\color{mygreen},    % comment style
	deletekeywords={...},            % if you want to delete keywords from the given language
	escapeinside={\%*}{*)},          % if you want to add LaTeX within your code
	extendedchars=true,              % lets you use non-ASCII characters; for 8-bits encodings only, does not work with UTF-8
	firstnumber=0,                % start line enumeration with line 1000
	frame=single,	                   % adds a frame around the code
	keepspaces=true,                 % keeps spaces in text, useful for keeping indentation of code (possibly needs columns=flexible)
	keywordstyle=\color{blue},       % keyword style
	language=Octave,                 % the language of the code
	morekeywords={*,...},            % if you want to add more keywords to the set
	numbers=left,                    % where to put the line-numbers; possible values are (none, left, right)
	numbersep=5pt,                   % how far the line-numbers are from the code
	numberstyle=\tiny\color{mygray}, % the style that is used for the line-numbers
	rulecolor=\color{black},         % if not set, the frame-color may be changed on line-breaks within not-black text (e.g. comments (green here))
	showspaces=false,                % show spaces everywhere adding particular underscores; it overrides 'showstringspaces'
	showstringspaces=false,          % underline spaces within strings only
	showtabs=false,                  % show tabs within strings adding particular underscores
	stepnumber=2,                    % the step between two line-numbers. If it's 1, each line will be numbered
	stringstyle=\color{mymauve},     % string literal style
	tabsize=2,	                   % sets default tabsize to 2 spaces
%	title=\lstname,                   % show the filename of files included with \lstinputlisting; also try caption instead of title
	xleftmargin=0.25cm
}

% NOTE: To compile a version of this pset without problems, solutions, or reflections, uncomment the relevant line below.

%\excludeversion{problem}
%\excludeversion{solution}
%\excludeversion{reflection}

\begin{document}	
	
	% Use the \psetheader command at the beginning of a pset. 
	\psetheader

\section*{Problem 1}
\begin{problem}
    A function $f: \bbC \to \bbC$ is said to satisfy the mean value property if and only for every $z_0 \in \bbC$ and $r>0,$ we have that 
    \[f(z_0) = \frac{1}{2\pi}\int_0^{2\pi}f(z_0 + r\cos(\theta) + i r \sin(\theta))d\theta.\]

    Show that if $f(z) = z^2,$ then $f$ satisfies the mean value property.
\end{problem}
\begin{solution}
    Note that since $f(z) = z^2,$ we have that for any $z_0 \in \bbC,$
    \begin{align*}
    f(z_0 + r\cos(\theta) + i r\sin(\theta)) &= f(z_0 + re^{i \theta})\\ &= z_0^2 + 2z_0e^{i\theta}+ r^2e^{2i\theta}\\
    \end{align*}
    \begin{lemma}
        For any $n \geq 0,$ we have that \[e^{i2\pi n} = 1\]
    \end{lemma}
    \begin{proof}
        \[e^{i2\pi n} = \cos(2\pi n) + i \sin(2\pi n) = 1\]
    \end{proof}
    Evaluating the integral:
    \begin{align*}
        \int_0^{2\pi}(z_0^2 + 2z_0e^{i\theta}+ r^2e^{2i\theta}) d\theta &= z_0^2\int_0^{2\pi}d\theta + 2z_0\int_0^{2\pi}e^{i\theta}d\theta+ r^2\int_0^{2\pi}e^{2i\theta} d\theta\\
        &= 2\pi z_0^2 + 2z_0 \frac{1}{i}[e^{i2\pi} - 1] + r^2 \frac{1}{2i}[e^{4i\pi} - 1]\\
        &= 2\pi z_0^2 + 2z_0 \frac{1}{i}[1 - 1] + r^2 \frac{1}{2i}[1 - 1]\\
        &= 2\pi z_0^2
    \end{align*}
    Thus, dividing by $2\pi$ yields the result.
\end{solution}

\newpage
\section*{Problem 2}
\begin{problem}
    Let $f(z) = z^3,$ and let 
    \[u(x,y) = \Re{f(x + iy)}, \quad v(x,y) = \Im{f(x + iy)}.\] Prove that, for all $x,y \in \bbR,$ 
    \[\triangle u(x,y) = \triangle v(x,y) = 0,\] where $\triangle$ is the Laplacian operator
    \[\nabla = \frac{\partial^2}{\partial x^2} + \frac{\partial^2}{\partial y^2}\]
\end{problem}

\begin{solution}
    Let $x,y \in \bbR,$ then
    \begin{align*}
    f(x + iy) &= (x + iy)^3\\ &= (x^2 + 2ixy  -y^2)(x + iy)\\ &= x^3  + 2ix^2 y - xy^2 +ix^2y - 2xy^2 - iy^3\\ &= x^3 - 3xy^2  + i(3x^2 y - y^3)    
    \end{align*}
    We then have that 
    \[u(x,y) = x^3 - 3xy^2, \qquad v(x,y) = 3x^2y - y^3.\]
    Computing the Laplacian:
    \begin{align*}
        \triangle u(x,y) &= \frac{\partial^2 u}{\partial x^2} + \frac{\partial^2 u}{\partial y^2}\\
        &= 6x + (-6x)\\
        &= 0
    \end{align*}
    and
    \begin{align*}
        \triangle v(x,y) &= \frac{\partial^2 v}{\partial x^2} + \frac{\partial^2 v}{\partial y^2}\\
        &= 6y + (-6y)\\
        &= 0
    \end{align*}
\end{solution}

\newpage
\section*{Problem 3}
\begin{problem}
    Suppose $\Omega \subset \bbC$ is an open connected set (a region) and $u: \Omega \to \bbR$ such that 
    \[\frac{\partial u}{\partial x} =\frac{\partial u}{\partial y} = 0.\] Prove that $u$ must be constant on $\Omega.$
\end{problem}
\begin{solution}
We assume that $\Omega \neq \emptyset$ as otherwise the statement is vacuously true. Let $z_0 \in \Omega.$ 

Since $\Omega$ is connected, then it is polygonally connected. Let $z \in \Omega.$ There exists some polygonal path $\gamma$ that connects $z$ and $z_0.$ Since $\frac{\partial u}{\partial x}: \Re{\Omega} \to \bbR$ is constantly zero, then it is continuous. Similarly, $\frac{\partial u}{\partial y}: \Im{\Omega} \to \bbR$ is continuous. Thus, \[D[u(z)] = \begin{bmatrix}
    \frac{\partial u(z)}{\partial x} &  \frac{\partial u(z)}{\partial y}
\end{bmatrix}: \Omega \to \bbR\] (the total derivative) is continuous on $\Omega.$ Note that $Du$ has a primitive, $u$ on $\Omega,$ since $u' = Du.$ Thus, we have that
\[\int_\gamma D[u(z)] dz = u(z_0) - u(z). \] But we have that $D[u(z)] = 0$ along any path in $\Omega$ since all the components are $0,$ and thus the path integral along $\gamma$ must be $0$ and so
\[u(z_0) = u(z).\] Because $z$ was arbitrary, then $u$ must be constant on $\Omega.$
\end{solution}

\newpage
\section*{Problem 4}
\begin{problem}
    Suppose that $u: \overline{D_r(z_0)}\subseteq \bbC \to \bbR$ is a continuous function which attains its maximum at $z_0.$ Suppose further that $u$ satisfies 
    \[u(z_0) = \frac{1}{2\pi}\int_0^{2\pi}u(z_0 + r\cos \theta + ir\sin\theta)d\theta.\] Prove that if $z\in \overline{D_r(z_0)}$ such that $|z - z_0| = r,$ then $u(z) = u(z_0).$
\end{problem}
\begin{solution}
\begin{lemma}
Let $S_r(z_0)$ denote the sphere of radius $r$ around $z_0.$ We claim that
    \[S_r(z_0) = \{z_0 + re^{i\theta} \mid \theta \in [0, 2\pi)\}\]
\end{lemma}
\begin{proof}
    Let $z \in S_r(z_0).$ Without much loss in generality from a translation, we can assume that $z \in S_r(0).$ Then $z = x + iy.$ Letting $\theta = \tan^{-1}(\frac{y}{x}),$ we see that $x = r\cos\theta$ and $y = r\sin\theta.$ Thus, $x + iy = re^{i\theta}.$  

    Let $z \in \{z_0 + re^{i\theta} \mid \theta \in [0, 2\pi)\},$ then there exists some $\theta_0 \in [0,2\pi)$ such that $z = z_0  + re^{i\theta}.$ Thus, 
    \[|z - z_0| = |re^{i\theta}| = r|e^{i\theta}| = r,\] and so $z_0 \in S_r(z_0).$
\end{proof}
Suppose not. That is, for all $z\in S_r(z_0),$ we have that $u(z) \neq u(z_0).$
Since $z_0$ is the maximum of $u$ on the disk, then we immediately have that 
    \[u(z) < u(z_0).\] By Lemma 2, for any $z\in S_r(z_0),$ there exists some $\theta \in (0, 2\pi]$ such that $z = z_0 + re^{i\theta}.$  Then 
    \[u(z_0) =\frac{1}{2\pi}\int_0^{2\pi} u(z_0 + re^{i\theta})d\theta = \frac{1}{2\pi}\int_0^{2\pi}u(z(\theta))d\theta < 
    u(z_0) \frac{1}{2\pi}\int_0^{2\pi}d\theta = u(z_0)\] But then $u(z_0) < u(z_0),$ a contradiction!
\end{solution}

\newpage
\section*{Problem 5}
\begin{problem}
    Suppose $u: \overline{\Omega} \to \bbR$ is a continuous function on the closure of a bounded region $\Omega \subset \bbC.$ Suppose that $u$ has the mean value property in $\Omega.$ That is, whenever $\overline{D_r(z_0)}\subseteq \Omega,$
    \[u(z_0) = \frac{1}{2\pi}\int_0^{2\pi} u(z_0 + re^{i\theta})d\theta.\] Prove that if $u$ takes a maximum value inside of $\Omega,$ then $u$ must be constant.
\end{problem}
\begin{solution}
Since $\overline{\Omega}$ is bounded and closed, then it is compact. Since $u$ is a continuous function over $\overline{\Omega},$ then it achieves its maximum at some $z_0 \in \overline{\Omega}.$ Suppose that $z_0 \in \Omega.$ Define 
    \[A:= \{z \in \Omega \mid u(z) = u(z_0)\}.\] We aim to show that $A$ is clopen,
    since $\Omega$ is a region, then it is an open connected subset, and so the only clopen subsets it contains are itself and the emptyset. But since $z_0 \in A,$ then $A \neq \emptyset,$ and so if we show that $A$ is clopen, then it must necessarily be $\Omega.$

    Let $z \in A$ and $\epsilon>0.$ By continuity of $u,$ there exists some $\delta>0$ such that if $|z - z'| < \delta,$ then $|u(z) - u(z')| < \epsilon.$ Since $\Omega$ is open, then there exists some $r'>0$ such that if $|z - z'| < r',$ then $z' \in \Omega.$ Take $R = \min\{\delta, r'\}.$ Let $z' \in D_R(z).$ There is some $r>0$ with $r \leq R$ such that $z' \in S_r(z').$ By the previous problem, since $u|_{D_r(z)}: D_r(z)\subseteq \overline{\Omega} \to \bbR$ attains its maximum at $z$ and $|z' - z| = r$ and 
    \[u(z) = \frac{1}{2\pi}\int_0^{2\pi}u(z_0  + re^{i\theta})d\theta,\] then $u(z) = u(z').$ Therefore, $z' \in A$ and so we have found an $R>0$ such that if $z' \in D_R(z),$ then $z' \in A.$ Then $A$ is open.

    Let $(z_n) \in A$ with $z_n \to z.$ We claim that $z\in A.$ Since $u$ is continuous, then $u(z_n) \to u(z),$ but since $u(z_n) = u(z_0)$ for all $n$ since each $z_n \in A,$ then $u(z) = u(z_0),$ and thus $z \in A.$ Then we have that $A$ is closed.

    Thus, since $A$ is both closed and open and since $A\neq \emptyset,$ then $A  = \Omega.$ Thus, for all $z\in \Omega,$ $u(z)= u(z_0).$ It remains to see that $u$ is constant on $\partial \Omega.$ 

    Let $z \in \partial \Omega,$ then there exists some $(z_n) \in \Omega$ such that $z_n \to z.$ By continuity of $u,$ we have that $u(z_n) \to u(z),$ but since $u$ is constant in $\Omega,$ we get that $u(z_n) = u(z_0),$ and thus $u(z) = u(z_0),$ and so $u$ is indeed constant on the boundary as well.
\end{solution}

\end{document}