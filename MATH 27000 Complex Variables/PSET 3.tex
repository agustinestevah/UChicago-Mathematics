\documentclass[11pt]{article}

% NOTE: Add in the relevant information to the commands below; or, if you'll be using the same information frequently, add these commands at the top of paolo-pset.tex file. 
\newcommand{\name}{Agustín Esteva}
\newcommand{\email}{aesteva@uchicago.edu}
\newcommand{\classnum}{270}
\newcommand{\subject}{Complex Variables}
\newcommand{\instructors}{Robert Fefferman}
\newcommand{\assignment}{Problem Set 3}
\newcommand{\semester}{Spring 2025}
\newcommand{\duedate}{4-24-2025}
\newcommand{\bA}{\mathbf{A}}
\newcommand{\bB}{\mathbf{B}}
\newcommand{\bC}{\mathbf{C}}
\newcommand{\bD}{\mathbf{D}}
\newcommand{\bE}{\mathbf{E}}
\newcommand{\bF}{\mathbf{F}}
\newcommand{\bG}{\mathbf{G}}
\newcommand{\bH}{\mathbf{H}}
\newcommand{\bI}{\mathbf{I}}
\newcommand{\bJ}{\mathbf{J}}
\newcommand{\bK}{\mathbf{K}}
\newcommand{\bL}{\mathbf{L}}
\newcommand{\bM}{\mathbf{M}}
\newcommand{\bN}{\mathbf{N}}
\newcommand{\bO}{\mathbf{O}}
\newcommand{\bP}{\mathbf{P}}
\newcommand{\bQ}{\mathbf{Q}}
\newcommand{\bR}{\mathbf{R}}
\newcommand{\bS}{\mathbf{S}}
\newcommand{\bT}{\mathbf{T}}
\newcommand{\bU}{\mathbf{U}}
\newcommand{\bV}{\mathbf{V}}
\newcommand{\bW}{\mathbf{W}}
\newcommand{\bX}{\mathbf{X}}
\newcommand{\bY}{\mathbf{Y}}
\newcommand{\bZ}{\mathbf{Z}}
\newcommand{\Vol}{\text{Vol}}

%% blackboard bold math capitals
\newcommand{\bbA}{\mathbb{A}}
\newcommand{\bbB}{\mathbb{B}}
\newcommand{\bbC}{\mathbb{C}}
\newcommand{\bbD}{\mathbb{D}}
\newcommand{\bbE}{\mathbb{E}}
\newcommand{\bbF}{\mathbb{F}}
\newcommand{\bbG}{\mathbb{G}}
\newcommand{\bbH}{\mathbb{H}}
\newcommand{\bbI}{\mathbb{I}}
\newcommand{\bbJ}{\mathbb{J}}
\newcommand{\bbK}{\mathbb{K}}
\newcommand{\bbL}{\mathbb{L}}
\newcommand{\bbM}{\mathbb{M}}
\newcommand{\bbN}{\mathbb{N}}
\newcommand{\bbO}{\mathbb{O}}
\newcommand{\bbP}{\mathbb{P}}
\newcommand{\bbQ}{\mathbb{Q}}
\newcommand{\bbR}{\mathbb{R}}
\newcommand{\bbS}{\mathbb{S}}
\newcommand{\bbT}{\mathbb{T}}
\newcommand{\bbU}{\mathbb{U}}
\newcommand{\bbV}{\mathbb{V}}
\newcommand{\bbW}{\mathbb{W}}
\newcommand{\bbX}{\mathbb{X}}
\newcommand{\bbY}{\mathbb{Y}}
\newcommand{\bbZ}{\mathbb{Z}}

%% script math capitals
\newcommand{\sA}{\mathscr{A}}
\newcommand{\sB}{\mathscr{B}}
\newcommand{\sC}{\mathscr{C}}
\newcommand{\sD}{\mathscr{D}}
\newcommand{\sE}{\mathscr{E}}
\newcommand{\sF}{\mathscr{F}}
\newcommand{\sG}{\mathscr{G}}
\newcommand{\sH}{\mathscr{H}}
\newcommand{\sI}{\mathscr{I}}
\newcommand{\sJ}{\mathscr{J}}
\newcommand{\sK}{\mathscr{K}}
\newcommand{\sL}{\mathscr{L}}
\newcommand{\sM}{\mathscr{M}}
\newcommand{\sN}{\mathscr{N}}
\newcommand{\sO}{\mathscr{O}}
\newcommand{\sP}{\mathscr{P}}
\newcommand{\sQ}{\mathscr{Q}}
\newcommand{\sR}{\mathscr{R}}
\newcommand{\sS}{\mathscr{S}}
\newcommand{\sT}{\mathscr{T}}
\newcommand{\sU}{\mathscr{U}}
\newcommand{\sV}{\mathscr{V}}
\newcommand{\sW}{\mathscr{W}}
\newcommand{\sX}{\mathscr{X}}
\newcommand{\sY}{\mathscr{Y}}
\newcommand{\sZ}{\mathscr{Z}}


\renewcommand{\emptyset}{\O}

\newcommand{\abs}[1]{\lvert #1 \rvert}
\newcommand{\norm}[1]{\lVert #1 \rVert}
\newcommand{\sm}{\setminus}


\newcommand{\sarr}{\rightarrow}
\newcommand{\arr}{\longrightarrow}

% NOTE: Defining collaborators is optional; to not list collaborators, comment out the line below.
%\newcommand{\collaborators}{Alyssa P. Hacker (\texttt{aphacker}), Ben Bitdiddle (\texttt{bitdiddle})}

\input{paolo-pset.tex}

% NOTE: To compile a version of this pset without problems, solutions, or reflections, uncomment the relevant line below.

%\excludeversion{problem}
%\excludeversion{solution}
%\excludeversion{reflection}

\begin{document}
	
	% Use the \psetheader command at the beginning of a pset. 
	\psetheader
\section*{Problem 1}

\newpage
\section*{Problem 2}
\begin{problem}
    Let $(a_k) \in \bbC$ and $(b_j) \in \bbC$ and define 
    \[c_n := \sum_{k+j = n} a_kb_j, \quad n \geq 2.\] If $\sum_{k=1}^\infty |a_k| < \infty$ and $\sum_{j=1}^\infty |b_j| < \infty,$ then $\sum_{n=2}^\infty |c_n| <\infty$ and 
    \[\sum_{n=2}^\infty c_n = \sum_{k=1}^\infty a_k \sum_{j=1}^\infty b_j.\]
\end{problem}
\begin{solution}
    Let 
    \[A = \sum_{k=1}^\infty |a_k| < \infty, \quad B = \sum_{j=1}^\infty |b_j| < \infty.\] Then 
    \[\sum_{n=2}^N |c_n| = \sum_{k=1}^{N-1}\sum_{j=1}^{N-k}|a_k||b_j| = \sum_{k=1}^N|a_k| \sum_{j=1}^{N-k}|b_j| \to \sum_{k=1}^\infty |a_k| \sum_{j=1}^\infty |b_k| = AB < \infty.\]
     
     Thus, 
    \begin{align*}
    \left|\sum_{k=1}^{\lfloor\frac{N}{2}\rfloor}a_k \sum_{j=1}^{\lfloor\frac{N}{2}\rfloor}b_j - \sum_{n=2}^N c_n\right| &= 
    \left|\sum_{k=1}^{\lfloor\frac{N}{2}\rfloor}a_k \sum_{j=1}^{\lfloor\frac{N}{2}\rfloor}b_j - \sum_{n=2}^N \sum_{j + k  = n} a_kb_j\right|\\
    &= \left|\sum_{k=1}^{\lfloor\frac{N}{2}\rfloor} \sum_{j=1}^{\lfloor\frac{N}{2}\rfloor} a_k b_j - \sum_{j=1}^{N-1}a_1b_j + \sum_{j=1}^{N-2}a_2 b_j + \cdots + a_{N-1}b_1\right|\\
    &= \left|\sum_{k=1}^{\lfloor\frac{N}{2}\rfloor} \sum_{j=1}^{\lfloor\frac{N}{2}\rfloor} a_k b_j - \sum_{k=1}^{N-1}\sum_{j=1}^{N-k}a_kb_j\right|\\
    &= \left|\sum_{j= \lfloor\frac{N}{2} + 1\rfloor}^{N-1} b_j\sum_{k=1}^{\lfloor \frac{N}{2}\rfloor}a_k-\sum_{k= \lfloor\frac{N}{2} + 1\rfloor}^{N-1} a_k\sum_{j=1}^{\lfloor \frac{N}{2}\rfloor}b_j\right|\\
    &\leq \sum_{j= \lfloor\frac{N}{2} + 1\rfloor}^{N-1} |b_j|\sum_{k=1}^{\lfloor \frac{N}{2}\rfloor}|a_k| + \sum_{k= \lfloor\frac{N}{2} + 1\rfloor}^{N-1} |a_k|\sum_{j=1}^{\lfloor \frac{N}{2}\rfloor}|b_j|\\
    &\leq \sum_{j= \lfloor\frac{N}{2} + 1\rfloor}^{\infty} |b_j|\sum_{k=1}^{\infty}|a_k| + \sum_{k= \lfloor\frac{N}{2} + 1\rfloor}^{\infty} |a_k|\sum_{j=1}^{\infty}|b_j|\\
    &=\sum_{j= \lfloor\frac{N}{2} + 1\rfloor}^{\infty} |b_j| A + \sum_{k= \lfloor\frac{N}{2} + 1\rfloor}^{\infty} |a_k| B\\
    &= A \sum_{j= \lfloor\frac{N}{2} + 1\rfloor}^{\infty} |b_j|  + B\sum_{k= \lfloor\frac{N}{2} + 1\rfloor}^{\infty} |a_k| \\
    &< A\frac{\epsilon}{2A} + B\frac{\epsilon}{2B}\\
    &= \epsilon
    \end{align*}
\end{solution}

\newpage
\section*{Problem 3}
\begin{problem}
    Using the previous problem, prove that 
    \[e^{w + z} = e^{w} e^{z}.\]
\end{problem}
\begin{solution}
For $n\geq 2,$ define 
\[c_n := \sum_{k + j = n}\frac{1}{(n)!}\binom{n}{k}w^k z^{j} = \sum_{k=1}^{n}\frac{1}{(n)!}\binom{n}{k}w^k z^{n-j} = \frac{1}{(n)!}(w + z)^{n}.\] Then note that 
\[\frac{1}{n!}\binom{n}{k} = \frac{1}{n!}\frac{n!}{(n-k)! (k)!} = \frac{1}{(j)! (k)!}.\] Thus, define 
\[a_k := \frac{1}{k!}w^k, \quad b_k := \frac{1}{j!}z^j,\] then 
\[c_n = \sum_{k +j = n} a_k b_j.\] By the previous problem, we have that 
\begin{align*}
e^{w + z} &= \sum_{n=0}^\infty \frac{1}{n!}(w + z)^n\\ &= \sum_{n=0}^\infty \sum_{k=0}^{n} \frac{1}{n!}\binom{n}{k} w^n z^{n-k}\\ &= \sum_{n=0}^\infty c_n\\ &= \sum_{k=0}^\infty a_k \sum_{j=0}^\infty b_j\\
&= \sum_{k=0}^\infty \frac{1}{k!}w^k \sum_{j=0}^\infty \frac{1}{j!}z^j\\
&= e^we^z
\end{align*}

\end{solution}
\newpage
\section*{Problem 4}
\begin{problem}
    Let $O \subseteq \bbC$ be open. Let $(f_n) \in H(O)$ such that $f_n \uconv f$ on every compact subset. Then $f\in H(O).$
\end{problem}
\begin{solution}
    Let $z\in O.$ Since $O$ is open, there exists some closed disk $ \overline{D_r(z)} \subseteq \bbC.$ Note that this disk is compact and connected. Thus, $f_n \uconv f$ on $\overline{D_r(z)}.$ Since each $f_n$ is continuous, then $f$ is continuous on $\overline{D_r(z)}.$ Define \[D_z := D_{\frac{r}{2}}(z) \subset \overline{D_r(z)}\] as an open disk. $D_z$ is open and connected and $f$ is continuous on $D_z.$ Let $\gamma$ be a closed path on $D_z.$ Then $\gamma$ is compact since it is closed and bounded, and thus $f_n \uconv f$ on $\gamma,$ and so 
    \[\int_\gamma f_n(\zeta)d\zeta \to \int_\gamma f(\zeta)d\zeta.\] By Cauchy's theorem, since $\gamma$ is also a closed path on $O$ and each $f_n \in H(O),$ then for every $n,$ 
    \[\int_\gamma f_n(\zeta)d\zeta = 0\implies \int_\gamma f(\zeta)d\zeta = 0.\] Thus, by Problem 7 on the previous PSET, we have that $f\in H(D_z).$ Because this is true for every $z\in O,$ then $f\in H(O).$
\end{solution}

\newpage


\end{document}