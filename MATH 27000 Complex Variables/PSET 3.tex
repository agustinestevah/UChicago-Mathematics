\documentclass[11pt]{article}

% NOTE: Add in the relevant information to the commands below; or, if you'll be using the same information frequently, add these commands at the top of paolo-pset.tex file. 
\newcommand{\name}{Agustín Esteva}
\newcommand{\email}{aesteva@uchicago.edu}
\newcommand{\classnum}{270}
\newcommand{\subject}{Complex Variables}
\newcommand{\instructors}{Robert Fefferman}
\newcommand{\assignment}{Problem Set 3}
\newcommand{\semester}{Spring 2025}
\newcommand{\duedate}{4-24-2025}
\newcommand{\bA}{\mathbf{A}}
\newcommand{\Ind}{\text{Ind}}

\newcommand{\bB}{\mathbf{B}}
\newcommand{\bC}{\mathbf{C}}
\newcommand{\bD}{\mathbf{D}}
\newcommand{\bE}{\mathbf{E}}
\newcommand{\bF}{\mathbf{F}}
\newcommand{\bG}{\mathbf{G}}
\newcommand{\bH}{\mathbf{H}}
\newcommand{\bI}{\mathbf{I}}
\newcommand{\bJ}{\mathbf{J}}
\newcommand{\bK}{\mathbf{K}}
\newcommand{\bL}{\mathbf{L}}
\newcommand{\bM}{\mathbf{M}}
\newcommand{\bN}{\mathbf{N}}
\newcommand{\bO}{\mathbf{O}}
\newcommand{\bP}{\mathbf{P}}
\newcommand{\bQ}{\mathbf{Q}}
\newcommand{\bR}{\mathbf{R}}
\newcommand{\bS}{\mathbf{S}}
\newcommand{\bT}{\mathbf{T}}
\newcommand{\bU}{\mathbf{U}}
\newcommand{\bV}{\mathbf{V}}
\newcommand{\bW}{\mathbf{W}}
\newcommand{\bX}{\mathbf{X}}
\newcommand{\bY}{\mathbf{Y}}
\newcommand{\bZ}{\mathbf{Z}}
\newcommand{\Vol}{\text{Vol}}

%% blackboard bold math capitals
\newcommand{\bbA}{\mathbb{A}}
\newcommand{\bbB}{\mathbb{B}}
\newcommand{\bbC}{\mathbb{C}}
\newcommand{\bbD}{\mathbb{D}}
\newcommand{\bbE}{\mathbb{E}}
\newcommand{\bbF}{\mathbb{F}}
\newcommand{\bbG}{\mathbb{G}}
\newcommand{\bbH}{\mathbb{H}}
\newcommand{\bbI}{\mathbb{I}}
\newcommand{\bbJ}{\mathbb{J}}
\newcommand{\bbK}{\mathbb{K}}
\newcommand{\bbL}{\mathbb{L}}
\newcommand{\bbM}{\mathbb{M}}
\newcommand{\bbN}{\mathbb{N}}
\newcommand{\bbO}{\mathbb{O}}
\newcommand{\bbP}{\mathbb{P}}
\newcommand{\bbQ}{\mathbb{Q}}
\newcommand{\bbR}{\mathbb{R}}
\newcommand{\bbS}{\mathbb{S}}
\newcommand{\bbT}{\mathbb{T}}
\newcommand{\bbU}{\mathbb{U}}
\newcommand{\bbV}{\mathbb{V}}
\newcommand{\bbW}{\mathbb{W}}
\newcommand{\bbX}{\mathbb{X}}
\newcommand{\bbY}{\mathbb{Y}}
\newcommand{\bbZ}{\mathbb{Z}}

%% script math capitals
\newcommand{\sA}{\mathscr{A}}
\newcommand{\sB}{\mathscr{B}}
\newcommand{\sC}{\mathscr{C}}
\newcommand{\sD}{\mathscr{D}}
\newcommand{\sE}{\mathscr{E}}
\newcommand{\sF}{\mathscr{F}}
\newcommand{\sG}{\mathscr{G}}
\newcommand{\sH}{\mathscr{H}}
\newcommand{\sI}{\mathscr{I}}
\newcommand{\sJ}{\mathscr{J}}
\newcommand{\sK}{\mathscr{K}}
\newcommand{\sL}{\mathscr{L}}
\newcommand{\sM}{\mathscr{M}}
\newcommand{\sN}{\mathscr{N}}
\newcommand{\sO}{\mathscr{O}}
\newcommand{\sP}{\mathscr{P}}
\newcommand{\sQ}{\mathscr{Q}}
\newcommand{\sR}{\mathscr{R}}
\newcommand{\sS}{\mathscr{S}}
\newcommand{\sT}{\mathscr{T}}
\newcommand{\sU}{\mathscr{U}}
\newcommand{\sV}{\mathscr{V}}
\newcommand{\sW}{\mathscr{W}}
\newcommand{\sX}{\mathscr{X}}
\newcommand{\sY}{\mathscr{Y}}
\newcommand{\sZ}{\mathscr{Z}}


\renewcommand{\emptyset}{\O}

\newcommand{\abs}[1]{\lvert #1 \rvert}
\newcommand{\norm}[1]{\lVert #1 \rVert}
\newcommand{\sm}{\setminus}


\newcommand{\sarr}{\rightarrow}
\newcommand{\arr}{\longrightarrow}

% NOTE: Defining collaborators is optional; to not list collaborators, comment out the line below.
%\newcommand{\collaborators}{Alyssa P. Hacker (\texttt{aphacker}), Ben Bitdiddle (\texttt{bitdiddle})}

\input{paolo-pset.tex}

% NOTE: To compile a version of this pset without problems, solutions, or reflections, uncomment the relevant line below.

%\excludeversion{problem}
%\excludeversion{solution}
%\excludeversion{reflection}

\begin{document}
	
	% Use the \psetheader command at the beginning of a pset. 
	\psetheader
\section*{Problem 1}
\begin{problem}
    Let $O\subseteq \bbC$ be open and connected and let $f\in H(O).$ If $\{z \in O \mid f(z) = 0\}$ has a limit point, then $f(z) = 0$ for all $z\in O.$
\end{problem}
\begin{solution}
Note that since $f\in H(O),$ $f$ is infinitely differentiable. Suppose, for the sake of contradiction, that $f(z') >0$ for some $z' \in O.$ Then by a problem on a previous PSET, we have that for any $z\in O,$ there exists some $n_z  \in \bbN$ such that $f^{(n_z)}(z) \neq 0.$ Let $z_0 \in O$ such that $z_0$ is a limit point of the vanishing set. Then by Cauchy's theorem, there is some $r>0$ such that if $z \in D_r(z_0),$ then 
\[f(z) = \sum_{n=1}^\infty \frac{1}{n!}f^{(n)}(z_0) (z - z_0)^n = \sum_{n=1}^\infty a_n (z - z_0)^n.\] Let $k$ be the smallest derivative such that $f^{k}(z_0) = a_k \neq 0.$ Then we have that 
\[f(z) = a_k(z - z_0)^k + \sum_{k+1}^\infty a_n (z - z_0)^n = a_k(z - z_0)^k + O((z - z_0)^{k+1}).\] Note that the second term goes to $0$ as $z \to z_0.$ Let $(z_n) \in \{z \in O \mid f(z)  = 0\}$ such that $z_n \to z_0.$ Without loss of generality, we can take $z_n \neq z_0,$ and thus since $|z_0 - z_n| < r$ for large $n,$ we have that 
\[|f(z_n)| = |a_k||(z_n - z_0)|^k \neq 0,\] which is a contradiction to the fact that $f(z_n) = 0.$ Thus, we must have that $f(z) = 0$ for all $z \in O.$
\end{solution}

\newpage
\section*{Problem 2}
\begin{problem}
    Let $(a_k) \in \bbC$ and $(b_j) \in \bbC$ and define 
    \[c_n := \sum_{k+j = n} a_kb_j, \quad n \geq 2.\] If $\sum_{k=1}^\infty |a_k| < \infty$ and $\sum_{j=1}^\infty |b_j| < \infty,$ then $\sum_{n=2}^\infty |c_n| <\infty$ and 
    \[\sum_{n=2}^\infty c_n = \sum_{k=1}^\infty a_k \sum_{j=1}^\infty b_j.\]
\end{problem}
\begin{solution}
    Let 
    \[A = \sum_{k=1}^\infty |a_k| < \infty, \quad B = \sum_{j=1}^\infty |b_j| < \infty.\] Then 
    \[\sum_{n=2}^N |c_n| = \sum_{k=1}^{N-1}\sum_{j=1}^{N-k}|a_k||b_j| = \sum_{k=1}^N|a_k| \sum_{j=1}^{N-k}|b_j| \to \sum_{k=1}^\infty |a_k| \sum_{j=1}^\infty |b_k| = AB < \infty.\]
     
     Thus, $\sum |c_n|$ converges. Moreover, we have that
    \begin{align*}
    \left|\sum_{k=1}^{\lfloor\frac{N}{2}\rfloor}a_k \sum_{j=1}^{\lfloor\frac{N}{2}\rfloor}b_j - \sum_{n=2}^N c_n\right| &= 
    \left|\sum_{k=1}^{\lfloor\frac{N}{2}\rfloor}a_k \sum_{j=1}^{\lfloor\frac{N}{2}\rfloor}b_j - \sum_{n=2}^N \sum_{j + k  = n} a_kb_j\right|\\
    &= \left|\sum_{k=1}^{\lfloor\frac{N}{2}\rfloor} \sum_{j=1}^{\lfloor\frac{N}{2}\rfloor} a_k b_j - \sum_{j=1}^{N-1}a_1b_j + \sum_{j=1}^{N-2}a_2 b_j + \cdots + a_{N-1}b_1\right|\\
    &= \left|\sum_{k=1}^{\lfloor\frac{N}{2}\rfloor} \sum_{j=1}^{\lfloor\frac{N}{2}\rfloor} a_k b_j - \sum_{k=1}^{N-1}\sum_{j=1}^{N-k}a_kb_j\right|\\
    &= \left|\sum_{j= \lfloor\frac{N}{2} + 1\rfloor}^{N-1} b_j\sum_{k=1}^{\lfloor \frac{N}{2}\rfloor}a_k-\sum_{k= \lfloor\frac{N}{2} + 1\rfloor}^{N-1} a_k\sum_{j=1}^{\lfloor \frac{N}{2}\rfloor}b_j\right|\\
    &\leq \sum_{j= \lfloor\frac{N}{2} + 1\rfloor}^{N-1} |b_j|\sum_{k=1}^{\lfloor \frac{N}{2}\rfloor}|a_k| + \sum_{k= \lfloor\frac{N}{2} + 1\rfloor}^{N-1} |a_k|\sum_{j=1}^{\lfloor \frac{N}{2}\rfloor}|b_j|\\
    &\leq \sum_{j= \lfloor\frac{N}{2} + 1\rfloor}^{\infty} |b_j|\sum_{k=1}^{\infty}|a_k| + \sum_{k= \lfloor\frac{N}{2} + 1\rfloor}^{\infty} |a_k|\sum_{j=1}^{\infty}|b_j|\\
    &=\sum_{j= \lfloor\frac{N}{2} + 1\rfloor}^{\infty} |b_j| A + \sum_{k= \lfloor\frac{N}{2} + 1\rfloor}^{\infty} |a_k| B\\
    &= A \sum_{j= \lfloor\frac{N}{2} + 1\rfloor}^{\infty} |b_j|  + B\sum_{k= \lfloor\frac{N}{2} + 1\rfloor}^{\infty} |a_k| \\
    &< A\frac{\epsilon}{2A} + B\frac{\epsilon}{2B}\\
    &= \epsilon
    \end{align*}
where the second to last inequality holds because both series absolutely converge and thus their tail ends can be arbitrarily small. Thus, for $N$ large, 
\[\sum_{n=2}^N c_n = \sum_{k=1}^{\lfloor \frac{N}{2}\rfloor} a_k\sum_{j=1}^{\lfloor \frac{N}{2}\rfloor}b_k \to \sum_{n=2}^\infty c_n = \sum_{k=1}^{\infty} a_k\sum_{j=1}^{\infty}b_k\]
\end{solution}

\newpage
\section*{Problem 3}
\begin{problem}
    Using the previous problem, prove that 
    \[e^{w + z} = e^{w} e^{z}.\]
\end{problem}
\begin{solution}
For $n\geq 1,$ define
\[c_n := \sum_{k + j = n}\frac{1}{(n)!}\binom{n}{k}w^k z^{j} = \sum_{k=0}^{n}\frac{1}{(n)!}\binom{n}{k}w^k z^{n-j} = \frac{1}{(n)!}(w + z)^{n}.\] Then note that 
\[\frac{1}{n!}\binom{n}{k} = \frac{1}{n!}\frac{n!}{(n-k)! (k)!} = \frac{1}{(j)! (k)!}.\] Thus, define 
\[a_k := \frac{1}{k!}w^k, \quad b_k := \frac{1}{j!}z^j,\] then 
\[c_n = \sum_{k +j = n} a_k b_j.\] By the previous problem, we have that 

\[e^{z + w} = \sum_{n=0}^\infty \frac{1}{n!}(w + z)^n =  \sum_{n=0}^\infty c_n = \sum_{k=0}^\infty a_n \sum_{j=0}^\infty b_n = \sum_{k=0}^\infty\frac{w^k}{k!}  \sum_{j=0}^\infty \frac{z^j}{j!} = e^w e^z\]

\end{solution}
\newpage
\section*{Problem 4}
\begin{problem}
    Let $O \subseteq \bbC$ be open. Let $(f_n) \in H(O)$ such that $f_n \uconv f$ on every compact subset. Then $f\in H(O).$
\end{problem}
\begin{solution}
    Let $z\in O.$ Since $O$ is open, there exists some closed disk $ \overline{D_r(z)} \subseteq \bbC.$ Note that this disk is compact and connected. Thus, $f_n \uconv f$ on $\overline{D_r(z)}.$ Since each $f_n$ is continuous, then $f$ is continuous on $\overline{D_r(z)}.$ Define \[D_z := D_{\frac{r}{2}}(z) \subset \overline{D_r(z)}\] as an open disk. $D_z$ is open and connected and $f$ is continuous on $D_z.$ Let $\gamma$ be a closed path on $D_z.$ Then $\gamma$ is compact since it is closed and bounded, and thus $f_n \uconv f$ on $\gamma,$ and so 
    \[\int_\gamma f_n(\zeta)d\zeta \to \int_\gamma f(\zeta)d\zeta.\] By Cauchy's theorem, since $\gamma$ is also a closed path on $O$ and each $f_n \in H(O),$ then for every $n,$ 
    \[\int_\gamma f_n(\zeta)d\zeta = 0\implies \int_\gamma f(\zeta)d\zeta = 0.\] Thus, by Problem 7 on the previous PSET, we have that $f\in H(D_z).$ Because this is true for every $z\in O,$ then $f\in H(O).$
\end{solution}

\newpage
\section*{Problem 5}
\begin{enumerate}
    \item 
\begin{problem}
Suppose $\gamma$ is a closed path in $\bbC.$ Let $O$ be the open, unbounded complement of $\gamma.$ Prove the winding number is $0$ for any $z\in O.$
\end{problem}

\begin{solution}
Without loss of generality up to a translation, we can assume that $z = 0.$ Also assume $\gamma(t): [a,b] \to O.$ Then because we are in $\bbC,$ we can write
\[\gamma(t) = \rho(t)e^{i\phi(t)},\] where $\rho(t) = |\gamma(t) - 0|$ and $\phi(t)$ is the angle between $0$ and $\varphi(t).$ Note that since $\gamma (t)$ is piecewise differentiable by assumption, then so are $\rho(t)$ and $\phi(t).$
Thus, 
\begin{align*}
    \int_\gamma \frac{d\zeta}{\zeta - 0}d\zeta &= \int_a^b \frac{\gamma'(t)}{\gamma(t) }dt\\
    &= \int_a^b \frac{\rho'(t)e^{i\phi(t)} + i\rho(t)\phi'(t)e^{i\phi (t)}}{\rho(t)e^{i\phi(t)}}dt\\
    &= \int_a^b \frac{\rho'(t)}{\rho(t)} dt + i \int_a^b\phi'(t)dt
\end{align*}
Since both have primitives, we can evaluate them using FTC. Moreover, we know that since $\gamma$ is closed, we must necessarily have $\rho(a) = \rho (b)$ and $\phi(a) = \phi(b)+2\pi n$ for some $n \in \bbZ$ Thus, 
\[\int_\gamma \frac{d\zeta}{\zeta - 0}d\zeta = \log(\rho(t))\bigg|_a^b + i\phi(t)\bigg|_a^b = i2\pi n.\] Thus, 
\[\Ind_\gamma (0) = n.\] Because it is clearly invariant over a translation, we have that 
\[\Ind_\gamma(z) = n.\] By problem 2 on PSET 2, since $\Ind_\gamma(z) \in \bbZ$ for all $z \in O,$ we have that $\Ind_\gamma$ is constant over $O.$ It suffices to show that $n = 0.$ To see this, we note that as $O$ is unbounded, we can see that 
\[\lim_{z\to \infty} \Ind_\gamma(z) = \lim_{z\to \infty}\int_\gamma \frac{d\zeta}{\zeta - z}  d\zeta = \lim_{z\to \infty} \int_a^b \frac{\gamma'(t)}{\gamma(t) - z}dt = \int_a^b \lim_{z\to \infty} \frac{\gamma'(t)}{\gamma(t) - z} =  0.\] Note that we can exchange limits because of the uniform convergence to $0$ of the function within the integrand. Thus, $\Ind_\gamma$ is constant over $O,$ we have that  
\[\Ind_\gamma(z) = 0, \quad \forall\, z\in O.\]
\end{solution}
\item

\begin{problem}
Let $\gamma_{d}$ be the path as pictured below. 
\[
    \begin{tikzpicture}[scale=1.5]
    % Axis
    \draw[->] (-2.2,0) -- (2.2,0) node[right] {$\Re$};
    \draw[->] (0,-1.6) -- (0,1.6) node[above] {$\Im$};

    % Points on real axis
    \filldraw (-2,0) circle (0.03) node[below] {$-R$};
    \filldraw (2,0) circle (0.03) node[below] {$R$};

    % Semicircle in lower half-plane
    \draw[thick,->] (2,0) arc (0:-180:2);
    
    % Arrow on the line segment
    \draw[thick,<-] (-0.5,0) -- (-0.49,0);

    % Label and dot for i
    \filldraw (0,1) circle (0.03);
    \node at (0.25,1.05) {$i$};

\end{tikzpicture}
\]
What is $\Ind_\gamma(i).$
\end{problem}
\begin{solution}
    Clearly, $i \in O,$ where $O$ is the unbounded connected complement of $\gamma.$ By part (a), we know that $\Ind_{\gamma_d}(i) = 0.$
\end{solution}
\item 
\begin{problem}
Let $R>1$ and define $\gamma_u$ as pictured below. 
\[
\begin{tikzpicture}[scale=1.5]
    % Axis
    \draw[->] (-2.2,0) -- (2.2,0) node[right] {$\Re$};
    \draw[->] (0,-1.6) -- (0,1.6) node[above] {$\Im$};

    % Points on real axis
    \filldraw (-2,0) circle (0.03) node[below] {$-R$};
    \filldraw (2,0) circle (0.03) node[below] {$R$};

    % Semicircle in upper half-plane (counterclockwise)
    \draw[thick,->] (-2,0) arc (180:0:2);

    % Arrow on the real axis segment
    \draw[thick,->] (0.5,0) -- (0.51,0);

    % Label and dot for i
    \filldraw (0,1) circle (0.03);
    \node at (0.25,1.05) {$i$};

\end{tikzpicture}\]
What is $\gamma_u(i)$?
\end{problem}
\begin{solution}
    Consider that since the orientations cancel out the middle interval, we have that $\gamma_d + \gamma_u$ is the closed circle around $0$ of radius $R.$ We have seen in class that $\Ind_{C_R}(0) = 1,$ where $C_R = \gamma_d + \gamma_u$ is the circle of radius $R$ around $0.$ By a corollary of what we proved in the proof of part (a), we know that the winding number is also constant in the bounded complement of $C_R,$ and since $i$ is in this connected open bounded complement, then 
    \[1 = \Ind_{C_r}(i) = \Ind_{\gamma_u}(i) + \Ind_{\gamma_d}(i) = \Ind_{\gamma_u}(i)\]
\end{solution}

\end{enumerate}

\newpage
\section*{Problem 6}
 \begin{enumerate}
            \item[(a)] Prove that
    \[
        \lim_{z \to i} \left( \frac{1}{z^2 + 1} - \frac{1}{2i(z - i)} \right)
    \]
    exists.
\begin{solution}
    Since $z^2 + 1 = (z -i)(z + i),$ we can write 
    \[\lim_{z \to i} \left( \frac{1}{z^2 + 1} - \frac{1}{2i(z - i)} \right) = \lim_{z\to i}\left[\frac{1}{z-i} \left( \frac{1}{z + i} - \frac{1}{2i}\right) \right] = \lim_{z\to i}\frac{\frac{1}{z + i}- \frac{1}{2i}}{z-i} \to \frac{0}{0}.\] We apply L'Hopital's rule:
    \begin{align*}
        \lim_{z \to i} \left( \frac{1}{z^2 + 1} - \frac{1}{2i(z - i)} \right) &= \lim_{z\to i}  \frac{\left(\frac{1}{z + i}- \frac{1}{2i}\right)'}{(z-i)'}\\ &= \lim_{z\to i} \frac{-\frac{1}{(z + i)^2}}{1}\\ &= \frac{1}{4}
    \end{align*}
\end{solution}

    \item[(b)] Call this limit \( L \). Then, define
    \[
    f(z) = 
    \begin{cases}
        \frac{1}{z^2 + 1} - \frac{1}{2i(z - i)}, & z \neq i \\
        L, & z = i
    \end{cases}
    \]
    What is 
    \[
    \int_{\gamma_\text{up}} f(z)\,dz?
    \]
    \begin{solution}
        Note that $\gamma_u$ is a closed path in $O: =D_{R + 1}(0) \subseteq \bbC,$ which is open and convex in $\bbC.$ Also observe that $f \in H(O)$ except for one point, $z = i.$ Note that by part (a), $f(z)$ is continuous at $z = i.$
        
        We use Goursat's and Cauchy's theorem,  which forgive functions which are holmorphic at all but one point, at which they are continuous, that states that since $\gamma_u$ is closed in an open and convex set, 
        then 
        \[\int_{\gamma_u}f(\zeta)d\zeta= 0\]
    \end{solution}

    \item[(c)] Taking the limit of the integral in part (b) as \( R \to \infty \), evaluate
    \[
    \int_{-\infty}^\infty \frac{1}{1 + x^2} \, dx.
    \]
    \begin{solution}
We have that 
\[\int_{-R}^R \frac{1}{1 + x^2}dx = \int_{-R}^R f(x) + \frac{1}{2i(x - i)}dx = \int_{-R}^R f(x)dx + \int_{-R}^R \frac{1}{2i(x-i)}dx\]
Denote the top arc of the half circle by $\gamma_{\text{arc}}.$ Then we have by part (a) that
\[0 = \int_{\gamma_u}f(\zeta)d\zeta = \int_{[-R, R]} f(\zeta)d\zeta + \int_{\gamma_{\text{arc}}}f(\zeta)d\zeta = \int_{-R}^R f(x)dx \implies \int_{-R}^R f(x)dx= - \int_{\gamma_{\text{arc}}}f(\zeta)d\zeta\]
But we know that 
\begin{align*}
    \left|\int_{\gamma_{\text{arc}}}f(\zeta)d\zeta\right| &\leq \text{arclength}(\gamma_{\text{arc}})\max_{z \in \gamma_{arc}}(|\frac{1}{z^2  + 1} - \frac{1}{2i(z-i)}|)\\
    &\leq \pi R (\frac{1}{R^2 + 1} + \frac{1}{|2(R -1)|})\\
    &\to 0
\end{align*}
as $R\to \infty.$ Thus, it suffices to compute 
\[\frac{1}{2i}\int_{-R}^R \frac{1}{(x-i)}dx.\] To do this, we will use Cauchy's residue formula, which states that 
\[\int_{-R}^R \frac{1}{(x-i)}dx = 2\pi i \,\text{res}_i(f) = 2\pi i\, \lim_{z\to i}\left[(x-i)\frac{1}{x-i}\right] = 2\pi i.\] Thus, 
\[\int_{-R}^R \frac{1}{1 + x^2}dx = \int_{-R}^R \frac{1}{2i(x-i)}dx = 2\pi i \frac{1}{2i } = \pi.\]
    \end{solution}
\end{enumerate}


















\end{document}