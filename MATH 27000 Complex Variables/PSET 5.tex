\documentclass[11pt]{article}
\usepackage{float}
% NOTE: Add in the relevant information to the commands below; or, if you'll be using the same information frequently, add these commands at the top of paolo-pset.tex file. 
\newcommand{\name}{Agustín Esteva}
\newcommand{\email}{aesteva@uchicago.edu}
\newcommand{\classnum}{270}
\newcommand{\subject}{Complex Variables}
\newcommand{\instructors}{Robert Fefferman}
\newcommand{\assignment}{Problem Set 5}
\newcommand{\semester}{Spring 2025}
\newcommand{\duedate}{5-15-2025}
\newcommand{\bA}{\mathbf{A}}
\newcommand{\Ind}{\text{Ind}}

\newcommand{\bB}{\mathbf{B}}
\newcommand{\bC}{\mathbf{C}}
\newcommand{\bD}{\mathbf{D}}
\newcommand{\bE}{\mathbf{E}}
\newcommand{\bF}{\mathbf{F}}
\newcommand{\bG}{\mathbf{G}}
\newcommand{\bH}{\mathbf{H}}
\newcommand{\bI}{\mathbf{I}}
\newcommand{\bJ}{\mathbf{J}}
\newcommand{\bK}{\mathbf{K}}
\newcommand{\bL}{\mathbf{L}}
\newcommand{\bM}{\mathbf{M}}
\newcommand{\bN}{\mathbf{N}}
\newcommand{\bO}{\mathbf{O}}
\newcommand{\bP}{\mathbf{P}}
\newcommand{\bQ}{\mathbf{Q}}
\newcommand{\bR}{\mathbf{R}}
\newcommand{\bS}{\mathbf{S}}
\newcommand{\bT}{\mathbf{T}}
\newcommand{\bU}{\mathbf{U}}
\newcommand{\bV}{\mathbf{V}}
\newcommand{\bW}{\mathbf{W}}
\newcommand{\bX}{\mathbf{X}}
\newcommand{\bY}{\mathbf{Y}}
\newcommand{\bZ}{\mathbf{Z}}
\newcommand{\Vol}{\text{Vol}}

%% blackboard bold math capitals
\newcommand{\bbA}{\mathbb{A}}
\newcommand{\bbB}{\mathbb{B}}
\newcommand{\bbC}{\mathbb{C}}
\newcommand{\bbD}{\mathbb{D}}
\newcommand{\bbE}{\mathbb{E}}
\newcommand{\bbF}{\mathbb{F}}
\newcommand{\bbG}{\mathbb{G}}
\newcommand{\bbH}{\mathbb{H}}
\newcommand{\bbI}{\mathbb{I}}
\newcommand{\bbJ}{\mathbb{J}}
\newcommand{\bbK}{\mathbb{K}}
\newcommand{\bbL}{\mathbb{L}}
\newcommand{\bbM}{\mathbb{M}}
\newcommand{\bbN}{\mathbb{N}}
\newcommand{\bbO}{\mathbb{O}}
\newcommand{\bbP}{\mathbb{P}}
\newcommand{\bbQ}{\mathbb{Q}}
\newcommand{\bbR}{\mathbb{R}}
\newcommand{\bbS}{\mathbb{S}}
\newcommand{\bbT}{\mathbb{T}}
\newcommand{\bbU}{\mathbb{U}}
\newcommand{\bbV}{\mathbb{V}}
\newcommand{\bbW}{\mathbb{W}}
\newcommand{\bbX}{\mathbb{X}}
\newcommand{\bbY}{\mathbb{Y}}
\newcommand{\bbZ}{\mathbb{Z}}

%% script math capitals
\newcommand{\sA}{\mathscr{A}}
\newcommand{\sB}{\mathscr{B}}
\newcommand{\sC}{\mathscr{C}}
\newcommand{\sD}{\mathscr{D}}
\newcommand{\sE}{\mathscr{E}}
\newcommand{\sF}{\mathscr{F}}
\newcommand{\sG}{\mathscr{G}}
\newcommand{\sH}{\mathscr{H}}
\newcommand{\sI}{\mathscr{I}}
\newcommand{\sJ}{\mathscr{J}}
\newcommand{\sK}{\mathscr{K}}
\newcommand{\sL}{\mathscr{L}}
\newcommand{\sM}{\mathscr{M}}
\newcommand{\sN}{\mathscr{N}}
\newcommand{\sO}{\mathscr{O}}
\newcommand{\sP}{\mathscr{P}}
\newcommand{\sQ}{\mathscr{Q}}
\newcommand{\sR}{\mathscr{R}}
\newcommand{\sS}{\mathscr{S}}
\newcommand{\sT}{\mathscr{T}}
\newcommand{\sU}{\mathscr{U}}
\newcommand{\sV}{\mathscr{V}}
\newcommand{\sW}{\mathscr{W}}
\newcommand{\sX}{\mathscr{X}}
\newcommand{\sY}{\mathscr{Y}}
\newcommand{\sZ}{\mathscr{Z}}


\renewcommand{\emptyset}{\O}

\newcommand{\abs}[1]{\lvert #1 \rvert}
\newcommand{\norm}[1]{\lVert #1 \rVert}
\newcommand{\sm}{\setminus}


\newcommand{\sarr}{\rightarrow}
\newcommand{\arr}{\longrightarrow}

% NOTE: Defining collaborators is optional; to not list collaborators, comment out the line below.
%\newcommand{\collaborators}{Alyssa P. Hacker (\texttt{aphacker}), Ben Bitdiddle (\texttt{bitdiddle})}

\input{paolo-pset.tex}

% NOTE: To compile a version of this pset without problems, solutions, or reflections, uncomment the relevant line below.

%\excludeversion{problem}
%\excludeversion{solution}
%\excludeversion{reflection}

\begin{document}
	
	% Use the \psetheader command at the beginning of a pset. 
	\psetheader
\section*{Problem 1}
\begin{problem}
    Prove that 
    \[\int_{-\infty}^\infty e^{-x^2}\,dx= \sqrt{\pi}\]
\end{problem}
\begin{solution}
If we let 
\[I = \int_{-\infty}^\infty e^{-x^2}\,dx,\] then if we let $r= x^2 + y^2$ and $\theta = \tan \frac{y}{x}$ we get the change of variables
\begin{align*}
I^2 &= \int_{-\infty}^\infty e^{-x^2}\,dx \int_{-\infty}^\infty e^{-y^2}\,dy\\
&= \int_{-\infty}^\infty\int_{-\infty}^\infty e^{-(x^2 + y^2)}\,dxdy\\
&= \int_0^{2\pi}\int_0^\infty e^{-r^2}r\,dr\,d\theta\\
&= \pi \int_0^\infty e^{-u}\,du\\
&= \pi \bigg[-e^{-u} + e^{-u}\bigg]_0^\infty\\
&= \pi,
\end{align*}
and so $I = \sqrt{\pi}$

\end{solution}

\newpage
\section*{Problem 2}
Let $t\in \bbR,$ what is 
\[\int_{-\infty}^\infty e^{-x^2}e^{itx}\,dx.\]
\begin{solution}
    We compute:
    \begin{align*}
\int_{-\infty}^\infty e^{-x^2}e^{itx}\,dx &= \int_{-\infty}^\infty e^{-x^2 + itx}\,dx   \\
&= \int_{-\infty}^\infty e^{-(x^2 - itx)}\,dx\\
&= \int_{-\infty}^\infty e^{-(x^2- itx - \frac{t^2}{4}) - \frac{t^2}{4}}\,dx\\
&= \int_{-\infty}^\infty e^{-(x - \frac{it}{2})^2 - \frac{t^2}{4}}\,dx\\
&= e^{-\frac{t^2}{4}}\int_{-\infty}^\infty e^{-(x - \frac{it}{2})^2}\,dx
    \end{align*}
Let $\gamma_R$ be the the following path: 
\[\begin{tikzpicture}
    % Define points
    \coordinate (A) at (-4, 0);
    \coordinate (B) at (4, 0);
    \coordinate (C) at (4, 2);
    \coordinate (D) at (-4, 2);
    \coordinate (M) at (0, 2);

    % Draw axes
    \draw[->] (-5, 0) -- (5, 0) node[right] {Re};
    \draw[->] (0, -1) -- (0, 3) node[above] {Im};

    % Draw path
    \draw[->, thick] (A) -- (B) node[below right] {$R$};
    \draw[->, thick] (B) -- (C);
    \draw[->, thick] (C) -- (D);
    \draw[->, thick] (D) -- (A) node[below left] {$-R$};

    % Vertical line

    % Mark -t/2
    \node[above] at (0, 2) {$-\frac{t}{2}$};

    % Arrows for orientation
    \node at (-3, 0.1) {};
    \node at (3, 0.1) {};
    \node at (0, 2.1) {\small $\uparrow$};
    \node at (4.1, 1) {};
    \node at (-4.1, 1) {};


\end{tikzpicture}.\]
Since $e^{-z^2}$ is holomorphic and $\gamma_R$ is closed, we know that 
\[\int_{\gamma_R}e^{-z^2}\,dz = 0.\] Moreover, if let let $U_\gamma, D_\gamma, L_\gamma, R_\gamma$ be the obvious choices for the segments of the paths, we also have that 
\begin{align*}
    0 &= \int_{\gamma_R}e^{-z^2}\,dz\\ 
    &= \int_{U_\gamma}e^{-z^2}\,dz  + \int_{L_\gamma}e^{-(x-\frac{t}{2})^2}\,dx + \int_{D_\gamma}e^{-z^2}\,dz + \int_{R_\gamma}e^{-x^2}\,dx\\
    &= \int_{0}^{-\frac{t}{2}}e^{-(R + y)^2}\,dy  - \int_{-R}^Re^{-(x-\frac{t}{2})^2}\,dx + \int_{-\frac{t}{2}}^0e^{-(-R + y)^2}\,dy + \int_{-R}^Re^{-x^2}\,dx \\
    &= -\int_{-R}^Re^{-(x-\frac{t}{2})^2}\,dx  + \int_{-R}^Re^{-x^2}\,dx\\
    &= -\int_{-R}^Re^{-(x-\frac{t}{2})^2}\,dx +\sqrt{\pi}
\end{align*}
Thus, our integral computes to 
\[\boxed{\sqrt{\pi}e^{-\frac{t^2}{4}}}\]
\end{solution}

\newpage
\section*{Problem 3}
\begin{problem}
    Let $n \in \bbN.$ Show that there are exactly $n$ $n$th roots of unity. What are they?
\end{problem}
\begin{solution}
As a consequence of the fundamental theorem of algebra, the polynomial $P(z) = z^n - 1$ has exactly $n$ roots. Thus, there exist $n$ different $z\in \bbC$ such that $z^n = 1.$ To characterize such $z,$ we have that
\[z^n = 1 \iff n\text{Log}(z) = 0 \iff e^{n\text{Log}(z)} = 1 \iff n\text{Log}(z) = 2\pi i k \geq 0 \iff z = e^{\frac{2\pi i k}{n}}.\] For only $k \in \{0,1,2,\dots, n-1,\}$ we have by periodicity that $\frac{2\pi i k}{n}$ takes on distinct values and so the $n$th roots of unity are given by 
\[\boxed{\{e^0, e^{\frac{2\pi i}{n}}, \dots, e^{\frac{2\pi i(n-1)}{n}}\}}\]
\end{solution}

\newpage
\section*{Problem 4}
\begin{problem}
    Find the following residues:
    \begin{enumerate}
        \item Let $a,b \in \bbC$ with $a\neq b,$ what is 
        \[\Res_a \left(\frac{1}{z - a}\frac{1}{z-b}\right).\]
        \begin{solution}
We calculate the power series of $g(z ) = \frac{1}{z-b}$ around $a$ to be 
\[g(z) = \frac{1}{a-b} + \sum_{n=1}^\infty a_n(z-a)^n\]
\[f(z) = \frac{1}{z-a} g(z) = \frac{1}{z-a}\left(\frac{1}{a-b} + \sum_{n=1}^\infty a_n(a - z)^n\right),\] and so 
\[a_{-1} = \boxed{\frac{1}{a-b}},\] which is the residue.
\end{solution}
\item 
\[\Res_i (\frac{1}{z}e^{iz})\]
\begin{solution}
The expansion of $f$ around $i$ has no negative powers since $f(z)$ is holomorphic in a disk not containing $z=  0$, and so $f(z)$ is analytic around $z = i,$ implying that \[a_{-1} = \boxed{0}\] is our residue.
\end{solution}
\item 
\[\Res_0 (\frac{1}{z}e^{iz})\]
\begin{solution}
    We have that 
    \[\frac{1}{z}e^{iz} = \frac{1}{z}(1 + \frac{1}{1!} (iz)+ \frac{1}{2!}(iz)^2 + \dots + ),\] and so 
    \[a_{-1} = \boxed{1}\] which is the residue. 
    \end{solution}
    \item 
\[\Res_1 (\frac{1}{z^3 - 1}\]
\begin{solution}
We can factor and consider expanding about $z = 1$ 
\[f(z) = \frac{1}{z^3 - 1} = \frac{1}{z-1}\frac{1}{z^2 + 1 + z} =: \frac{1}{z-1}g(z)= \frac{1}{z-1}(b_0 + b_1 z + \cdots )\] where the last equality comes due to $g(z)$ being perfectly analytic/holomorphic about $z = 1.$ Thus, we look for \[b_0 = g(1)= \boxed{\frac{1}{3}},\] which is $a_{-1}$ in the Laurent series, and thus our residual. 
\end{solution}

    \end{enumerate}
\end{problem}

\newpage
\section*{Problem 5}
\begin{problem}
    Are complex polynomials dense in the set of continuous complex functions $f(z): \overline{D_1(z)} \to \bbC$?
\end{problem}
\begin{solution}
\textbf{No}, we need the added assumption that $f$ is holomorphic. 

Suppose we could approximate $f(z)$ though! Then 
\[P_n\uconv f\] for polynomials $P_n.$ Let $K \subseteq \overline{D_1(z)}$ be a compact subset. It should be clear that the convergence is uniform on $K$ as well. Since $P_n \in H(\overline{D_1(z)})$ for all $n,$ then by Problem 4 on PSET 3, $f \in H(\overline{D_1(z)}).$ Thus, it suffices to show there exists a function merely continuous on the unit disk. Take 
\[f(z) = \overline{z}.\] We have shown that $f$ is not differentiable anywhere. To check continuity, note that for $|z-w|< \epsilon$
\[|f(z) - f(w)| = |\overline{z} - \overline{w}| = |\overline{z - w}| = |z - w|< \epsilon\]
\end{solution}

\end{document}