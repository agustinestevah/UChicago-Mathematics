\documentclass[11pt]{article}
\usepackage{float}
% NOTE: Add in the relevant information to the commands below; or, if you'll be using the same information frequently, add these commands at the top of paolo-pset.tex file. 
\newcommand{\name}{Agustín Esteva}
\newcommand{\email}{aesteva@uchicago.edu}
\newcommand{\classnum}{270}
\newcommand{\subject}{Complex Variables}
\newcommand{\instructors}{Robert Fefferman}
\newcommand{\assignment}{Problem Set 6}
\newcommand{\semester}{Spring 2025}
\newcommand{\duedate}{5-22-2025}
\newcommand{\bA}{\mathbf{A}}
\newcommand{\Ind}{\text{Ind}}

\newcommand{\bB}{\mathbf{B}}
\newcommand{\bC}{\mathbf{C}}
\newcommand{\bD}{\mathbf{D}}
\newcommand{\bE}{\mathbf{E}}
\newcommand{\bF}{\mathbf{F}}
\newcommand{\bG}{\mathbf{G}}
\newcommand{\bH}{\mathbf{H}}
\newcommand{\bI}{\mathbf{I}}
\newcommand{\bJ}{\mathbf{J}}
\newcommand{\bK}{\mathbf{K}}
\newcommand{\bL}{\mathbf{L}}
\newcommand{\bM}{\mathbf{M}}
\newcommand{\bN}{\mathbf{N}}
\newcommand{\bO}{\mathbf{O}}
\newcommand{\bP}{\mathbf{P}}
\newcommand{\bQ}{\mathbf{Q}}
\newcommand{\bR}{\mathbf{R}}
\newcommand{\bS}{\mathbf{S}}
\newcommand{\bT}{\mathbf{T}}
\newcommand{\bU}{\mathbf{U}}
\newcommand{\bV}{\mathbf{V}}
\newcommand{\bW}{\mathbf{W}}
\newcommand{\bX}{\mathbf{X}}
\newcommand{\bY}{\mathbf{Y}}
\newcommand{\bZ}{\mathbf{Z}}
\newcommand{\Vol}{\text{Vol}}

%% blackboard bold math capitals
\newcommand{\bbA}{\mathbb{A}}
\newcommand{\bbB}{\mathbb{B}}
\newcommand{\bbC}{\mathbb{C}}
\newcommand{\bbD}{\mathbb{D}}
\newcommand{\bbE}{\mathbb{E}}
\newcommand{\bbF}{\mathbb{F}}
\newcommand{\bbG}{\mathbb{G}}
\newcommand{\bbH}{\mathbb{H}}
\newcommand{\bbI}{\mathbb{I}}
\newcommand{\bbJ}{\mathbb{J}}
\newcommand{\bbK}{\mathbb{K}}
\newcommand{\bbL}{\mathbb{L}}
\newcommand{\bbM}{\mathbb{M}}
\newcommand{\bbN}{\mathbb{N}}
\newcommand{\bbO}{\mathbb{O}}
\newcommand{\bbP}{\mathbb{P}}
\newcommand{\bbQ}{\mathbb{Q}}
\newcommand{\bbR}{\mathbb{R}}
\newcommand{\bbS}{\mathbb{S}}
\newcommand{\bbT}{\mathbb{T}}
\newcommand{\bbU}{\mathbb{U}}
\newcommand{\bbV}{\mathbb{V}}
\newcommand{\bbW}{\mathbb{W}}
\newcommand{\bbX}{\mathbb{X}}
\newcommand{\bbY}{\mathbb{Y}}
\newcommand{\bbZ}{\mathbb{Z}}

%% script math capitals
\newcommand{\sA}{\mathscr{A}}
\newcommand{\sB}{\mathscr{B}}
\newcommand{\sC}{\mathscr{C}}
\newcommand{\sD}{\mathscr{D}}
\newcommand{\sE}{\mathscr{E}}
\newcommand{\sF}{\mathscr{F}}
\newcommand{\sG}{\mathscr{G}}
\newcommand{\sH}{\mathscr{H}}
\newcommand{\sI}{\mathscr{I}}
\newcommand{\sJ}{\mathscr{J}}
\newcommand{\sK}{\mathscr{K}}
\newcommand{\sL}{\mathscr{L}}
\newcommand{\sM}{\mathscr{M}}
\newcommand{\sN}{\mathscr{N}}
\newcommand{\sO}{\mathscr{O}}
\newcommand{\sP}{\mathscr{P}}
\newcommand{\sQ}{\mathscr{Q}}
\newcommand{\sR}{\mathscr{R}}
\newcommand{\sS}{\mathscr{S}}
\newcommand{\sT}{\mathscr{T}}
\newcommand{\sU}{\mathscr{U}}
\newcommand{\sV}{\mathscr{V}}
\newcommand{\sW}{\mathscr{W}}
\newcommand{\sX}{\mathscr{X}}
\newcommand{\sY}{\mathscr{Y}}
\newcommand{\sZ}{\mathscr{Z}}


\renewcommand{\emptyset}{\O}

\newcommand{\abs}[1]{\lvert #1 \rvert}
\newcommand{\norm}[1]{\lVert #1 \rVert}
\newcommand{\sm}{\setminus}


\newcommand{\sarr}{\rightarrow}
\newcommand{\arr}{\longrightarrow}

% NOTE: Defining collaborators is optional; to not list collaborators, comment out the line below.
%\newcommand{\collaborators}{Alyssa P. Hacker (\texttt{aphacker}), Ben Bitdiddle (\texttt{bitdiddle})}

\input{paolo-pset.tex}

% NOTE: To compile a version of this pset without problems, solutions, or reflections, uncomment the relevant line below.

%\excludeversion{problem}
%\excludeversion{solution}
%\excludeversion{reflection}

\begin{document}
	
	% Use the \psetheader command at the beginning of a pset. 
	\psetheader
\section*{Problem 1}
\begin{problem}
    Suppose $f \in H(O),$ where $O\subseteq \bbC$ is an open connected region. If there is some $z_0 \in O$ such that $|f(z_0)| \geq |f(z)|$ for all $z\in O,$ then $f$ is constant on $O.$
\end{problem}
\begin{solution}
Define 
\[A = \{z\in O \mid |f(z)|= |f(z_0)|\}.\] It suffices to show that $A$ is clopen in $O.$ Note that $A \neq \emptyset$ since $z_0 \in A.$

Let $z\in A.$ Since $O$ is open, there exists some $R>0$ such that $\overline{D_R(z)} \subseteq O.$ Let $z' \in D_R(z),$ then consider the circle $C_r(z)$ where $r:= |z -z'|.$ As a consequence of the Cauchy integral equation, we have seen in class that 
\[f(z_0) = \frac{1}{2\pi i}\int_0^{2\pi}f(z_0 + re^{i\theta})\,d\theta.\] By PSET 1 problem 4, we have that since $|f|$ achieves its max at $z$ and $|f|$ is continuous since $f$ is holomorphic, then $|f(z)| = |f(z')|,$ and so $z' \in A.$

Let $z \in O$ such that $(z_n) \in A$ such that $z_n \to z.$ Then since $z_n \in A,$ then $f(z_n) = f(z_0),$ and so by continuity, $f(z) = f(z_0),$ and thus $z \in A.$ 

Since $O$ is connected and $A$ is a nonempty clopen set, we are dnoe.

\end{solution}

\newpage
\section*{Problem 2}
\begin{problem}
    Suppose $f \in H(O)$ where $O\subseteq \bbC$ is an open connected region. If $|f|$ is constant, then $f$ is constant.
\end{problem}
\begin{solution}
    Let $f = u + iv.$ Then since $|f|$ is constant, we have that 
    \[|f| = |u + iv| = \sqrt{u^2 + v^2} \implies u^2 + v^2 \equiv C.\]
    Thus, differentiating the above with respect to $x$ and then $y,$
\[\frac{\partial u^2}{\partial x} + \frac{\partial v^2}{\partial x} = 0\]
\[\frac{\partial u^2}{\partial y} + \frac{\partial v^2}{\partial y} = 0\]

    Thus,
    \begin{align}
\begin{pmatrix}
        \frac{\partial u}{\partial x} & \frac{\partial v}{\partial x}\\
        \frac{\partial u}{\partial y} & \frac{\partial v}{\partial y}
    \end{pmatrix}\begin{pmatrix}
        u\\v
    \end{pmatrix} = \begin{pmatrix}
        \frac{\partial u}{\partial x}u + \frac{\partial v}{\partial x}v\\
        \frac{\partial u}{\partial y}u + \frac{\partial v}{\partial y}v\\
    \end{pmatrix}
    = \begin{pmatrix}
        0\\0
    \end{pmatrix}        
    \end{align}

Consider that using the Riemann-Cauchy equations, we find that the above implies that every component in the Jacobian is zero.
\begin{align*}
\det \begin{pmatrix}
        \frac{\partial u}{\partial x} & \frac{\partial v}{\partial x}\\
        \frac{\partial u}{\partial y} & \frac{\partial v}{\partial y}
    \end{pmatrix}
    &= \frac{\partial u}{\partial x}\frac{\partial v}{\partial y} - 
    \frac{\partial v}{\partial x}\frac{\partial u}{\partial y}\\ &= (\frac{\partial u}{\partial x})^2 + (\frac{\partial u}{\partial y})^2\\
    &= |\nabla u|^2\\
    &= 0
    \end{align*}
    
    but also using similar logic,
    \[\det \begin{pmatrix}
        \frac{\partial u}{\partial x} & \frac{\partial v}{\partial x}\\
        \frac{\partial u}{\partial y} & \frac{\partial v}{\partial y}
    \end{pmatrix}
    = |\nabla v|^2 = 0\]
This then implies that $\nabla u = \nabla v =0,$ showing that $f$ is constant.

\end{solution}


\newpage
\section*{Problem 3}
\begin{problem}
    Give another proof of the fundamental theorem of algebra. 
\end{problem}
\begin{solution}
    Let $P(z)$ be a non-constant polynomial that doesn't vanish. Since $P(0) \neq 0,$ we know that $f(z):= \frac{1}{|P(0)|} \neq 0.$ Let $\epsilon>0$ such that $\frac{1}{|P(0)|} >\epsilon.$
    
    We know that as $z\to \infty,$ $|P(z)| \to \infty.$ Thus, $\frac{1}{|P(z)|}\to 0$ as $z\to \infty.$ Thus, there exists some $R>0$ such that for $z\notin \overline{D_R(0)},$ $\frac{1}{|P(z)|} < \frac{\epsilon}{2}.$ Since $\frac{1}{{|P(z)|}}$ is continuous and $\overline{D_R(0)}$ is compact, then $\frac{1}{|P(z)|}$ achieves its maximum on some $z_0 \in \overline{D_R(0)}.$ Necessarily, $\frac{1}{|P(z_0)|} \geq\epsilon > \frac{1}{|P(z)|}$ for any $z\notin \overline{D_R(0)}.$ Thus, $\frac{1}{|P(z)|}$ attains its maximum on $\bbC.$ This is a contradiction by the Maximum Modulus Principle, and thus $\frac{1}{|P(z)|}$ is constant. By Problem 2, this implies that $\frac{1}{P(z)}$ is constant. Hence, $P(z)$ is constant, which is a contradiction. 
\end{solution}

\newpage

\section*{Problem 4}
Solve the following using residue theorem
\begin{enumerate}
    \item What is 
    \[\int_{-\infty}^\infty \frac{1}{1 + x^2}\,dx\]
    \begin{solution}
We use $\gamma_R = \gamma_1 \circ \gamma_{\text{arc}}$ to be the upper semircle of radius $R$ about the origin. We note that if $f(z) = \frac{1}{1 + z^2},$ then $f$ has a pole at $z = i$. We have shown in a previous PSET that $\Ind_\gamma (i)= 1,$ and thus
\[\int_{\gamma_R} f(z)\,dz=2\pi i \Res_{i}f(z).\] We see that since $\frac{1}{z+ i}$ is a perfectly analytic function about $z = i,$ we can write
\begin{align*}
    f(z) &=\frac{1}{(z-i)(z + i)}\\
    &= \frac{1}{z-i} (a_0 + a_1(z-i)+ \dots)
\end{align*}
and it becomes clear that the residue is $a_0 = \frac{1}{2i}.$ Hence
\begin{align*}
    \pi &= \int_{\gamma_R} f(z)\,dz\\
    &= \int_{\gamma_1}f(z)\,dz + \int_{\gamma_{\text{arc}}} f(z)\,dz\\
    &= \int_{-R}^R \frac{1}{1 + x^2}\,dx + \int_{\gamma_{\text{arc}}} f(z)\,dz\\
    &\to \int_{-\infty}^\infty \frac{1}{1 + x^2}\,dx
\end{align*}
\[\left|\int_{\gamma_{\text{arc}}} f(z)\,dz\right| \leq \pi R \frac{1}{1 + R^2}\to 0\]
    \end{solution}
    \item What is 
\[\int_{C_r(0)}\frac{1}{(z-a)(z-b)}\] where $|a| \leq r \leq |b|.$
\begin{solution}
    The function $f(z)$ only has a single pole about $z = a,$ and we have seen in a previous PSET that this residue is simply $\frac{1}{a - b}.$ Similarly, the winding number of $z = a$ is $1$ since it is in the bounded region of $C_r(0)$ which we have shown in a previous PSET is constantly one. Thus, we use the residue theorem
    \[\int_{C_r(0)} \frac{1}{(z-a)(z-b)}\,dz = 2\pi i \Res_{z=a} f(z) = \frac{2\pi i}{a-b}\]
\end{solution}
    \item What is 
    \[\int_0^\infty \frac{1}{1 + x^3}\,dx.\]
    \begin{solution}
     Let 
    \[f(z)= \frac{1}{1 + z^3}\]


Let $\gamma_1(t) = t$ for $t \in [0,R]$ be the straight line. Then 
\[\int_{\gamma_1} f(z)\,dz = \int_0^R f(\gamma(t)\gamma'(t)\,dt = \int_0^R \frac{1}{1 + t^3}\,dt\]
Let $\gamma_2(t) = -t$ for $t\in [0,R]$ be the straight line in the opposite direction. Then 
\[\int_{\gamma_2} f(z)\,dz = - \int_0^R \frac{1}{1 - t^3}\,dt\]
This doesn't work!
Let $\omega^3 = 1$ and let 
\[\gamma_2(t) = t\omega\] Then 
\[\int_{\gamma_2}f(z)\,dz = \int_{0}^R f(\gamma_2(t))\gamma_2'(t)\,dt = \int_0^R \frac{1}{1 + t^3} \omega \,dt = \omega \int_0^R \frac{1}{1 + t^3}\,dt\]
We know that 
\[\omega = e^{\frac{2}{3}\pi i}.\] We also know that $f$ has a single pole in the closed path $\gamma_R := \gamma_1 \circ \gamma_{arc} \circ - \gamma_2.$ To see this, we need to find $z$ such that 
\[z^3 = -1\] We know that $z = -e^{\frac{2\pi i}{3} \cdot 0}, -e^{\frac{2\pi i}{3}}, -e^{\frac{4\pi i}{3}} = z_1, z_2, z_3.$ The only pole within our arc is $-e^{\frac{4\pi i}{3}}.$ Thus, 
\[\int_{\gamma_R} f(z)\,dz = 2\pi i \Res_{-e^{\frac{4\pi i}{3}}}f(z)\,dz\] To find this residue, note the Laurent expansion
\[\frac{1}{1 + z^3} = \frac{1}{(z - z_1)(z - z_2)(z - z_3)}= \frac{1}{z-z_3}(a_0 + a_1 (z -z_3) + \cdots)\] since $\frac{1}{(z-z_1)(z - z_2)}$ is a perfectly analytical function about $z_3,$ and thus 
$a_0 = \frac{1}{(z_3 - z_1)(z_3 - z_2)} = \frac{1}{(-e^{\frac{4\pi i}{3}}- (-1))(-e^{\frac{4\pi i}{3}}-(-e^{\frac{2\pi i}{3}}))} = \frac{1}{(-e^{\frac{4\pi i}{3}} + 1)(-e^{\frac{4\pi i}{3}} + e^{\frac{2\pi i}{3}})}= \frac{1}{e^{\frac{8\pi i}{3}} -e^{\frac{4\pi i}{3}} - e^{\frac{6\pi i}{3}}+ e^{\frac{2\pi i}{3}}} = \frac{1}{-1 + 2e^{\frac{2\pi i}{3}} - e^{\frac{4\pi i}{3}}} = -\frac{1}{6} - \frac{i}{2\sqrt{3}}$

\[\int_{\gamma_R} f(z)\,dz = 2\pi i \Res_{-e^{\frac{4\pi i}{3}}}f(z)\,dz = 2\pi i \,a_0 = -\frac{\pi i}{3} + \frac{\pi}{\sqrt{3}}\]

To see that the arc portion of the integral goes to zero at infinity, we estimate the size of the integral, noting that the angles $-z_1, -z_2 = \omega, -z_3$ cut the circle into three parts.   
\begin{align*}
    \left|\int_{\gamma_{\text{arc}}} f(z)\right| \leq \frac{2}{3}\pi R \max_{z \in \gamma_{\text{arc}}}|\frac{1}{1+ z^3}| \propto \frac{1}{R^2}\to 0.
\end{align*}
Thus, 
\begin{align*}
    \frac{2\pi i}{3}e^{\frac{2}{3}\pi i}&=\int_{\gamma_R}f(z)\,dz\\ &= \int_{\gamma_{1}} f(z)\,dz - \int_{\gamma_2} f(z)\,dz + \int_{\gamma_{\text{arc}}}f(z)\,dz\\
    &\to \int_0^R \frac{1}{1 + x^3}\,dx - \omega\int_{0}^R \frac{1}{1 + t^3}\,dt\\
    &= \int_0^R \frac{1}{1 + x^3}\,dx\left(1 - e^{\frac{2}{3}\pi i}\right)\\
    &= \left(1 - e^{\frac{2}{3}\pi i}\right)\int_0^R \frac{1}{1 + x^3}\,dx
\end{align*}

Dividing both sides yields that 
\[\int_0^\infty \frac{1}{1 + x^3}\,dx= \frac{2\pi }{3\sqrt{3}}.\]


Note that in this proof, we used the assumption that $\Ind_{\gamma_R}(z_k) = 1.$ To prove this, note that you can add a curve until $\gamma_R \circ \gamma_R' = C_R,$ and we know that $\Ind_{C_R}(z_k) = 1.$ But $z_k$ is in the unbounded portion of $\gamma'_R,$ implying that $\Ind_{\gamma'_R}(z_k) = 0.$   
\end{solution}

\end{enumerate}

\newpage
\section*{Problem 5}
\begin{problem}
    Use the Residue theorem to evaluate 
    \[\sum_{n=1}^\infty \frac{1}{n^4}\]
\end{problem}
\begin{solution}
    We use a lot of facts from class:
    \begin{enumerate}
        \item $\cot z$ is $2\pi-$periodic
        \item $\frac{1}{z^2}\cot z$ is bounded on $[-\frac{\pi}{2}, \frac{\pi }{2}]\sm D_{r}(0)$ for small $r>0.$
        \item $\Res_{n\pi} \cot z = 1$ for any $n\in \bbZ,$ and $n\pi$ is a simple pole.
    \end{enumerate}
We need one further fact which was utilized without proof in class:
\begin{lemma}
    Suppose $f$ has a simple pole at $z = z_0.$ Then if $g \in H(D_r(z_0))$ for $r>0,$ we have that 
    \[\Res_{z_0}fg = g(z_0)\Res_{z_0}f.\]
\end{lemma}
\begin{proof}
    Since $f$ has a simple pole at $z_0,$ we know that 
    $\Res_{z_0} f = a_{-1},$ where 
    \[f(z) = a_{-1}(z - z_0)^{-1} + a_0 + a_1(z-z_0)+ \cdots\] hence
    \[(z-z_0) f(z)= a_{-1} + a_0(z-z_0) + a_1(z-z_0)^2 + \cdots\] and thus in the limit,
    \[\lim_{z\to z_0}f(z) = \Res_{z_0}f\] We claim that $fg$ has a simple pole at $z_0.$ To see this, we know that $g$ is analytic about $z_0,$ so we can express it as 
    \begin{align*}
    (fg)(z) &= (a_{-1}(z-z_0)^{-1} + a_0 + a_1(z-z_0) + \cdots )(b_0 + b_1(z-z_0) + \cdots ) \\
    &= a_{-1}b_0 (z-z_0)^{-1} + a_{-1}b_1 + \cdots
    \end{align*}
    Since the Laurent series only has a zero of order $1$ at $z_0,$ we know that $fg$ has a simple pole at $z = z_0.$ Thus, we have shown that 
    \[\Res_{z_0} fg = \lim_{z \to z_0}(z - z_0)f(z)g(z) = \lim_{z\to z_0}[(z-z_0)f(z)]\lim_{z\to z_0} g(z) = g(z_0)\Res_{z_0}f\]
   
\end{proof}

    First, we claim that $\frac{1}{z^4}\cot z$ is bounded on $[-\frac{\pi}{2}, \frac{\pi }{2}]\sm D_{r}(0)$ for small $r>0.$ We have that $\frac{1}{z^2}  \cot z$ is bounded on this compact set, and since $\frac{1}{z^2}$ is continuous on this set, then it is bounded as well. Hence $\frac{1}{z^2}\frac{1}{z^2}\cot z$ is bounded on this set by some $M.$ We aim to calculate $\Res_{0}f(z),$ where $f(z) = \frac{1}{z^4}\cot z.$ To do this, we note from class that 
    \begin{align*}
        \frac{1}{z^4}\cot z &= \frac{1}{z^5}\left[(1 - \frac{z^2}{2!} + \frac{z^4}{4!} + \cdots)(1 + (\frac{z^2}{3!} - \frac{z^4}{5!}+\cdots)+ (\frac{z^2}{3!} - \frac{z^4}{5!}+\cdots)^2 + \cdots)\right]\\
    \end{align*}
    Hence, we are looking for powers of $z^4$ in the bracketed term. Considering first the $1$ term in the first sum, we see that using the distributive property, we have the following coefficients of $z^4$
    \[1 \mapsto -\frac{1}{5!} + (\frac{1}{3!})^2\]
    \[-\frac{z^2}{2!}\mapsto -\frac{1}{2!}\frac{1}{3!}\]
    \[\frac{z^4}{4!}\mapsto \frac{1}{4!}\] 
    and so $a_0 = -\frac{1}{120} + \frac{1}{36} - \frac{1}{12} + \frac{1}{24}= -\frac{1 }{45}$ is the residual at $0$ For $n\pi$ where $n\in \bbZ\sm \{0\}.$ We claim that the residual of $f$ is $\frac{1}{n^4\pi^4}.$ To see this, we note that $\frac{1}{z^4}$ does not have a pole at $z = n\pi$ and thus 
    \[\Res_{n\pi} \frac{1}{z^4} \cot z = \frac{1}{(n\pi)^4}\Res_{n\pi}\cot z = \frac{1}{n^4\pi^4}.\] Here we used our Lemma 1. Thus, we see that if $\mathcal{D} = \{D_r(n\pi)\}_{n \in \bbZ,}$ then integrating $f(z)$ in $\bbC\sm \mathcal{D}$ over the curve $C_{(n + \frac{1}{2})\pi}(0),$ we estimate the integral (using the fact that its bounded) by
    \[\left|\int_{C_{(n + \frac{1}{2})\pi}(0)} f(z)\,dz\right| \leq 2\pi (n+\frac{1}{2})\pi \max_{z\in C_{(n + \frac{1}{2})\pi}(0)}|\frac{\cot z}{z^4}| \leq 2\pi (n+\frac{1}{2})\pi \frac{M}{(n\pi)^4}|\to 0.\] Using the Cauchy Residue Theorem, 
    \[\int_{C_{(n + \frac{1}{2})\pi}(0)} f(z)\,dz = 2\pi i \sum_{z_k \in \Res} \Res_{z_k}f = 2\pi i\left(-\frac{1}{45}+ \sum_{|k|\leq n, k\neq 0} \frac{1}{k^4\pi^4}\right)\] Thus, we see that in the limit,
    \begin{align*}
        0 &= -\frac{1}{45}+ \sum_{n\neq 0} \frac{1}{n^4\pi^4}\\
        &= -\frac{1}{45} + \frac{2}{\pi^4}\sum_{n=1}^\infty \frac{1}{n^4}
    \end{align*}
    Rearranging, we see that 
    \[\sum_{n=1}^\infty \frac{1}{n^4} = \frac{\pi^4}{90}\]
    
\end{solution}







\end{document}